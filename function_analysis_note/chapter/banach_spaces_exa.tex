\chapter{Banach Spaces Example}\label{chp:banach_spaces_exa}


Notions such as convergent sequence and Cauchy sequence make sense for any normed
space.


\begin{definition}{}{}
    Let $X$ be a normed space, and let {$x_n$} be a sequence of points in $X$.\\
    (i) We say that $\{x_n\}$ if for every $\epsilon>0$, there exists an $N\in \N$ so that
    \begin{align*}
        i,j\geq N\Rightarrow ||x_i-x_j||<\epsilon.
    \end{align*}
    (ii) We say that $\{x_n\}$ converges to a point $x\in X$ if
    \begin{align*}
        lim_{n\rightarrow +\infty} ||x_n-x|| = 0.
    \end{align*}
\end{definition}



\begin{definition}{}{}
    The normed vector space $(X, ||\cdot||)$ is said to be complete if every Cauchy sequence is convergent in X. A Banach space is a complete normed vector space.

\end{definition}

\section{Reference}

\begin{itemize}
    \item \href{https://people.math.sc.edu/schep/Banach.pdf}{lecutre note from south carolina}
    \item \href{https://assets.press.princeton.edu/chapters/s9627.pdf}{lecture notes from princeton}
\end{itemize}