\chapter{Elementary Properties and Examples}\label{chp:1_1}
\section{Elementary Properties and Examples}
\begin{remark}
    $\F$ will mean either $\R$ or $\C$.
\end{remark}

\begin{definition}{}{the inner product space def}
    If $\mathcal{X}$ is a vector space over $\F$. An inner product on $\mathcal{X}$ is a mapping $u:\mathcal{X}\times \mathcal{X}\rightarrow \F$ 
    such that $\forall x,y,z\in X,\alpha,\beta \in F$,\\
    (1) $u(\alpha x+\beta y,z)=\alpha u(x,z)+\beta u(y,z)$\\
    (2) $u(x,y)=\overline{u(y,x)}$\\
    (3) $u(x,x)\geqs 0$,  $<x,x>=0$ iff $x=0$\\
    $\mathcal{X}$ together with such a funtion $u$ is called an inner product space.
\end{definition}

\begin{remark}
    We always denote our innner product by $<x,y>:=u(x,y)$. 
\end{remark}
\begin{remark}
    By def\ref{def:the inner product space def}(1)(2), we can get $u(x,\alpha y + \beta z)=\overline{\alpha} u(x,y) +\overline{\beta} u(x,z)$.
\end{remark}

\begin{definition}{}{}
    $\mathcal{X}$ is an inner product space.
    The mapping
    \begin{align*}
        ||\cdot||: \mathcal{X}&\rightarrow [0,+\infty]\\
        x &\mapsto \sqrt{<x,x>}
    \end{align*} 
    defines a norm on $\mathcal{X}$. This norm is called norm induced by the inner space or simply induced norm.
\end{definition}

Before checking the induced norm well definited, we introduce the Cauchy-Schwarz inequality.

\begin{theorem}{}{}
    If $\mathcal{X}$ is an inner product space. Then
    \begin{align*}
        |<x,y>|\leqs ||x||\cdot ||y||.
    \end{align*}
    equality occurs iff $x=ky(k\neq 0)$.
\end{theorem}
\begin{proofsolution}
    write when available.
\end{proofsolution}

Now let's check the three properties of induced norm.\\
(a) $||x|| = \sqrt{<x,x>}\geqs 0$; $||x||=0\Leftrightarrow <x,x>=0\Leftrightarrow x=0$.\\
(b) $||\alpha x|| = \sqrt{<\alpha x,\alpha x>} = \sqrt{\alpha^2<x,x>}= |\alpha| ||x||$.\\
(c) \begin{align*}
    ||x+y||^2 &= <x+y,x+y> \\
    &= <x,x>+<x,y>+<y,x>+<y,y> \\
    &= ||x||^2 + \Re<x,y> + ||y||^2.
\end{align*}
    Since $\Re<x,y>\leqs |<x,y>|\leqs ||x||||y||$, $||x+y||^2\leqs ||x||^2 + 2||x||||y||+||y||^2$. Then $||x+y||^2\leqs (||x||+||y||)^2$. Hence, $||x+y||\leqs ||x||+||y||$.

    

\begin{proposition}{Continuity of the Inner Product}{}
    Let $\mathcal{X}$ be an inner product space with induced norm $||\cdot||$. 
    Then $<\cdot,\cdot>:\mathcal{X}\times\mathcal{X}\rightarrow \F$ is continuous.  
\end{proposition}


\begin{proposition}{Parallelogram Law}{parallelogram law}
    Let $\mathcal{X}$ be an inner product space. Then $\forall x,y\in \mathcal{X}$
    \begin{align*}
        ||x+y||^2+||x-y||^2 = 2(||x||^2+||y||^2).
    \end{align*}
\end{proposition}

\begin{proposition}{Polarization Identity}{}
    Let $\mathcal{X}$ be an inner product space. Then $\forall x,y\in\mathcal{X}$
    \begin{align*}
        <x,y> = \frac{1}{4}(||x+y||^2-||x-y||^2- \ci ||y+\ci x||^2+\ci ||y-\ci x||^2).
    \end{align*}
\end{proposition}

\begin{remark}
    In an inner product space, the inner product determines the norm. The polarization identity shows
    that the norm determines the inner product. But not every norm on a vector space $X$ is induced by an inner product, 
    for example $(\ell^{\infty},||\cdot||_{\infty})$ is not an inner space.
\end{remark}

\begin{theorem}{}{}
    Suppose $(\mathcal{X},||\cdot||)$ is a normed linear space. The norm $||\cdot||$ is induced by an inner product 
    iff the parallelogram law holds in $(\mathcal{X},||\cdot||)$.
\end{theorem}
\begin{proofsolution}
    ($\Rightarrow$):See the proof of prop\ref{prop:parallelogram law}.\\
    ($\Leftarrow$): We need to show that the inner product determined by the norm is well definited. \
    Use the polarization identity to define $<\cdot,\cdot>$. Then immediately 
    \begin{align*}
        <x,x> &= \frac{1}{4}(||x+x||^2)=||x||^2 \Rightarrow <x,x>\geqs 0, <x,x>=0 \iff x=0,\\
        <x,y> &= \overline{<y,x>}
    \end{align*}
    Use parallelogram law, we can get
    \begin{align*}
        <x,y+z> &= <x,y>+<x,z>,\\
        <\alpha x,y>&=\alpha<x,y>
    \end{align*}
    The inner product satisfys three properties in def\ref{def:the inner product space def} and so well definited.
\end{proofsolution}

\begin{definition}{Cauchy Sequence, Convergent Sequence}{}
    Let $\mathcal{X}$ be a normed space, and let $\{x_n\}$ be a sequence of points in $X$. \\
    (1) We say that $\{x_n\}$ is a Cauchy sequence if for every $\epsilon>0$, there exists an $N\in \N$ so that
    \begin{align*}
        i,j\geqs N\Rightarrow ||x_i-x_j||<\epsilon.
    \end{align*}
    (2) We say that $\{x_n\}$ converges to a point $x\in \mathcal{X}$ if
    \begin{align*}
        \lim_{n\rightarrow \infty} ||x_n-x||=0.
    \end{align*}
\end{definition}

\begin{remark}
    In a non empty space $\mathcal{X}$, any constant sequence is a cauchy sequence, so you can always find a cauchy sequence.
\end{remark}

\begin{proposition}{}{}
    Cauchy sequence is always bounded.
\end{proposition}
\begin{proof}
    Suppose $\{x_n\}$ is a Cauchy sequence. Then for $\epsilon=1$, $\exists N$, $\mathrm{s.t.}$, when $n>N$: $||x_n-x_N||<1$. 
    Then by  triangle inequality, we have
    \begin{align*}
        ||x_n|| = ||x_n-x_N+x_N||\leqs ||x_n-x_N||+||x_N|| <1+||x_N||.
    \end{align*}
    Let $M=\max\{||x_1||,||x_2||,...,||x_N||,1+||x_N||\}$, then $\forall n$, $||x_n||\leqs M$. Hence, $\{x_n\}$ is bounded.
\end{proof}

\begin{proposition}{Convergent Sequence are Cauchy}{}
    If $\mathcal{X}$ is a normed space, then every convergent sequene in $\mathcal{X}$ is a Cauchy sequence.
\end{proposition}
\begin{proof}
    Let $\{x_n\}$ be a sequence that converges to a point $x\in X$, and let $\epsilon>0$. Since $\lim_{n\rightarrow \infty}||x_n-x||=0$, there exists an $N\in \N$ so that
    \begin{align*}
        n\geqs N\Rightarrow ||x_n-x||<\frac{\epsilon}{2}.
    \end{align*}
    If $i,j\geqs N$, it follows that
    \begin{align*}
        ||x_i-x_j||\leqs ||x_i-x||+||x_j-x||<\frac{\epsilon}{2} + \frac{\epsilon}{2}=\epsilon,
    \end{align*}
    which proves that $\{x_n\}$ is a Cauchy sequence.
\end{proof}
Though every convergent sequence is Cauchy, it is not necessarily the case that
every Cauchy sequence in a normed space converges. For example, let $\Q$ be the normed
space of all rational numbers under the usual norm: $||q_1-q_2||=|q_1-q_2|$. 
Then there are many Cauchy sequences in $\Q$ that do not converge to any point in $\Q$.
For example, the sequence
\begin{align*}
    3, 3.1, 3.14, 3.141, 3.1415, 3.14159, ...
\end{align*}
is a Cauchy sequence in $\Q$, but it does not converge to any point in $\Q$.

\begin{definition}{Complete Normed Space}{}
    A normed space $\mathcal{X}$ is said to be complete if every Cauchy sequence in $\mathcal{X}$ converges to a point in $\mathcal{X}$.    
\end{definition}


\begin{definition}{Hilbert Space}{}
    An inner product space which is complete with respect to the norm induced by 
    the inner product is called a Hilbert space.
\end{definition}


\begin{proposition}{}{}
    If $\mathcal{X}$ is a vector space and $<\cdot,\cdot>_{\mathcal{X}}$ is an inner product on $\mathcal{X}$.
    The completion of $\mathcal{X}$ is a Hilbert space. 
\end{proposition}
\begin{proof}
    You can construct the completion of $\mathcal{X}$, denoted by $\overline{X}$, by referring to \href{https://sites.math.northwestern.edu/~scanez/courses/320/notes/completion.pdf}{lecture notes from nw}. 
    Then You can construct the inner product in $\overline{X}$ by referring to \href{https://math.stackexchange.com/questions/1234209/if-x-is-an-inner-product-space-then-is-its-completion-a-hilbert-space}{answer in mathexchage}.
\end{proof}

\begin{definition}{$L^p$ function}{}
    Let $(X,\mu)$ be a measure space, and let $p\in [1,\infty)$. An $L^p$ function on $X$ is a measurable function $f$ on $X$ for which
    \begin{align*}
        \int_{X}^{}|f|^p d\mu < \infty.
    \end{align*}
\end{definition}
An important special case of $L^p$ function is for the measure space $(\N,\mu)$, where
$\mu$ is counting measure on $\N$. In this case, a measure function $f$ on $\N$ is just a sequence
\begin{align*}
    f(1),f(2),f(3),...
\end{align*}
and the Lebesgue integral is the same as the sum of the series
\begin{align*}
    \int_{\N}^{} f d\mu = \sum\limits_{n\in \N} f(n).
\end{align*}
The definition of an $L^p$ function on $\N$ takes the following form.
\begin{definition}{$\ell^p$ sequence}{}
    An $\ell^p$ sequence is a sequence $\{x_n\}$ of real numbers for which
    \begin{align*}
        \sum\limits_{n\in \N}|x_n|^p<\infty.
    \end{align*}
\end{definition}

\begin{proposition}{}{}
    For any measure space $(X, \mu)$, the set 
    \begin{align*}
        L^p = \{f|f \ is\  L^p\  function\}
    \end{align*}
    forms a Hilbert space under the inner product
    \begin{align*}
        <f,g> = \int_{X}^{}f\overline{g}d\mu.
    \end{align*}
\end{proposition}


\begin{proposition}{}{}
    The set 
    \begin{align*}
        \ell^p = \{x=(x_1,x_2,x_3,...)| x \ is\  \ell^p\  sequence\}
    \end{align*}
    forms a Hilbert space under the inner product
    \begin{align*}
        <x,y> = \sum\limits_{i}^{}x_i\overline{y_i}.
    \end{align*}
\end{proposition}


\begin{definition}{}{}
    Let $I$ be a nonempty index set. For each $\alpha\in I$, let $y_{\alpha}$ be a nonnegative real number. 
    Define $\sum\limits_{\alpha\in I}^{}y_{\alpha} = \sup\{\sum\limits_{\alpha\in J}y_{\alpha}| J\subset I$ and $J$ is finite$\}$.  
\end{definition}

\begin{proposition}{}{}
    If $\sum\limits_{\alpha\in I}y_{\alpha} < \infty$, then $y_{\alpha}= 0$ for at at most countably many $\alpha$.
\end{proposition}

\begin{proof}
    refer to \href{https://math.stackexchange.com/questions/20661/the-sum-of-an-uncountable-number-of-positive-numbers}{answer in mathexchage}.
    Let $l\in\N$. We claim that the subset $S_l$ of $S$
    \begin{align*}
        S_l = \{\alpha|\ \frac{1}{l}\leqs|y_{\alpha}|\}
    \end{align*}
    is a finite set. Sine $\sum\limits_{\alpha\in I}y_{\alpha} < \infty$ ,there exist $M\in \N$
    \begin{align*}
        M\geqs \sum\limits_{\alpha\in S_l}|y_{\alpha}|\geqs \sum\limits_{i\in S_l}\frac{1}{l} = \frac{N}{l}
    \end{align*}
    where $N$ is the number of elements in $S_l$. 
    Thus $S_l$ has at most $||x||^2l$  elements.\\
    Hence $\{\alpha|\  0< |y_{\alpha}|\}= \cup_{l\in\N}S_l$ is countable as the countable union of finite sets.
\end{proof}

\section{Homework}
\begin{exercise}{}{}
    Let $A$ be any set and let $l^2(A)$ denote the set of all functions $x:A\rightarrow \F$ such that $x(i)=0$ for all but a countable number of $i$ and 
    $\sum\limits_{i\in A}|x(i)|^2<\infty$. For $x,y\in l^2(A)$ define
    \begin{align*}
        <x,y> = \sum\limits_{i}^{}x(i)\overline{y(i)}.
    \end{align*}
    Then $l^2(A)$ is a Hilbert space.
\end{exercise}
\begin{proof}
    Note that the induced norm in $l^2(A)$ is
    \begin{align*}
        ||x|| = (\sum\limits_{i}^{}{|x(i)|}^{2})^{\frac{1}{2}}.
    \end{align*}
    Let $\{x_n\}$ be Cauchy sequence in $l^2(A)$ and 
    Then, 
    \begin{align*}
        |x_m(i)-x_n(i)|\leqs (\sum\limits_{i}^{}|x_m(i)-x_n(i)|^2)^{\frac{1}{2}}=||x_n-x_m||\rightarrow 0.
    \end{align*}
    Thus, $x_n(i)$ is a Cauchy sequence over the real(or complex) numbers. By the completness of $\R$(ro $\C$), it must converge to a limit, denoted by $xl(i)$.
    Now we show that $x_n\rightarrow x$($x:i\mapsto xl(i))$ when $n\rightarrow \infty$. 
    \par
    Since $\{x_n\}$ is Cauchy, $\forall \epsilon$, $\exists N$, $\mathrm{s.t.}$, when $m,n>N, \forall k: $
    \begin{align*}
        \sum\limits_{i}^{k}|x_m(i)-x_n(i)|^2\leqs ||x_m-x_n||^2<\sqrt{\frac{\epsilon}{2}}.
    \end{align*} 
    Let $m\rightarrow \infty$ and $n>N$, we have
    \begin{align*}
        \sum\limits_{i}^{k}|x(i)-x_n(i)|^2\leqs ||x-x_n||^2\leqs\sqrt{\frac{\epsilon}{2}}.
    \end{align*}
    Let $k\rightarrow \infty$ and $n>N$, we have
    \begin{align*}
        (\sum\limits_{i}^{\infty}|x(i)-x_n(i)|^2)^{\frac{1}{2}}\leqs ||x-x_n||\leqs\frac{\epsilon}{2}<\epsilon.
    \end{align*}
    Hence, $x_n\rightarrow x$ when $n\rightarrow \infty$.Then, we show that $x\in l^2(A)$. 
    \par
    We first show that $x(i)=0$ for all but a countable number of $i$. Suppose the contrary. 
    Then for any $x_n$, $||x-x_n||^2$ being a sum of uncountably many nonzero elements, which cannot be bounded. This is a contradiction as $||x-x_n||\rightarrow 0$ ($n\rightarrow \infty$).
    \par
    Finally, we show that $\sum\limits_{i\in A}^{}|x(i)|^2<\infty$. Since $\{x_n\}$ is Cauchy, $\{x_n\}$ is bounded. Then $\exists M>0$, $\forall n$, we have $||x_n||<M$. Then $\forall n,k$, we have
    \begin{align*}
        \sum\limits_{i}^{k}|x_n(i)|^2\leqs ||x_n||^2\leqs M^2.
    \end{align*}
    Let $n\rightarrow \infty$, we have
    \begin{align*}
        \sum\limits_{i}^{k}|x_n(i)|^2\leqs ||x||^2\leqs M^2.
    \end{align*}
    Let $k\rightarrow \infty$, we have
    \begin{align*}
        \sum\limits_{i}^{}|x_n(i)|^2\leqs M^2.
    \end{align*}
    Hence, $x\in l^2(A)$.
    Thus, every Cauchy sequence in $l^2(A)$ is convergent and so $l^2(A)$ is a Hilbert space.
\end{proof}

If $A=\N$, $l^2(A)$ is usually denoted by $l^2$. Let $(X,\Omega, \mu)$ be a measure space consisting of a set $X$, a 
$\sigma$-algebra $\Omega$ of subsets of $X$, and a countably additive measure $\mu$ defined 
on $\Omega$ with values in the non-negative extended real numbers.
Note that if $\Omega$= the set of all subsets of $A$ and for $E$ in $\Omega$, $\mu(E):=\infty$ if $E$ is infinite and $\mu(E)$=the cardinality of $E$ if $E$ is finite, 
then $l^2(A)$ and $L^2(A,\Omega,\mu)$ are equal.





\section{Reference}
\begin{itemize}
    \item \href{https://web.mat.bham.ac.uk/~malevao/MSM3P21/l10.pdf}{lecture notes from brmh}
    \item \href{https://sites.math.washington.edu/~burke/crs/555/555_notes/hilbert.pdf}{lecture notes from washington university}
    \item \href{https://e.math.cornell.edu/people/belk/measuretheory/BanachSpaces.pdf}{lecture notes from cornell}
    \item the completness of $L^p$ space
    \begin{itemize}
        \item[*] \href{https://assets.press.princeton.edu/chapters/s9627.pdf}{lectrue notes from princeton}
        \item[*] \href{https://e.math.cornell.edu/people/belk/measuretheory/LpFunctions.pdf}{lecture notes from cornell}
        \item[*] \href{https://www.ndsu.edu/pubweb/~littmann/Topics15/class8-28.pdf}{lecture notes from ndsu} 
    \end{itemize}
    \item \href{https://math.stackexchange.com/questions/3886899/prove-that-l2i-is-a-hilbert-space}{homework1}
\end{itemize}