\chapter{The Adjoint of an Operator}\label{chp:2_2}

\begin{definition}{}{}
    If $\hH$ and $\hK$ are Hilbert spaces, a function $u:\hH\times\hK\rightarrow \F$ is a sesquilinear form if for 
    $h,g\in\hH$, $k,l\in \hK$,and $\alpha,\beta\in\F$,\\
    (1) $u(\alpha h+\beta g,k)=\alpha u(h,k)+\beta u(g,k)$;\\
    (2) $u(h,\alpha k+\beta l)=\overline{\alpha}u(h,k)+\overline{\beta}u(h,l)$.
\end{definition}

The prefix "sesqui" is used because the function is linear in one variable 
but (for $\F = \C$) only conjugate linear in the other. ("Sesqui" means 
"one-and -a-half.")

A sesquilinear form is bounded if there is a constant $M$ such that 
$|u(h,k)|\leqs M||h||\cdot ||k|||$ for all $h$ in $\hH$ and $k$ in $\hK$. 
The constant $M$ is called a bound for $u$.


\begin{proposition}
    If $A \in \mathcal{B}(\mathscr{H},\mathscr{K})$, 
then $u(h, k):=  \inner{Ah}{k}$ is a bounded sesquilinear form.
\end{proposition}

\begin{proof}{}{}
    \begin{align*}
        u(\alpha h+\beta g,k) &=  \inner{A(\alpha h+\beta g)}{k}\\
                            &= \inner{\alpha Ah+\beta Ag}{k}\\
                            &= \alpha \inner{Ah}{k} + \beta\inner{Ag}{k}\\
                            &= \alpha u(h,k) + \beta u(g,k),
    \end{align*}
    \begin{align*}
        u(h,\alpha k+\beta l) &=  \inner{Ah}{\alpha k+\beta l}\\
                            &= \overline{\alpha} \inner{Ah}{k} + 
                            \overline{\beta}\inner{Ag}{l}\\
                            &= \overline{\alpha} u(h,k) + \overline{\beta} u(g,l),
    \end{align*}
    \begin{align*}
        |u(h,k)| = |\inner{Ah}{k}|\leqs ||Ah||\cdot ||k||\leqs ||h||\cdot||k||.
    \end{align*}
\end{proof}

Also, if $B\in \mathcal{B}(\mathscr{H},\mathscr{H})$, 
$u(h, k) = \inner{h}{Bk}$ is a bounded sesquilinear form. 
Are there any more? Are these two forms related?

\begin{theorem}{}{}
    If $u:\mathscr{H}\times\mathscr{H}\rightarrow \F$ is a bounded sesquilinear form with bound $M$, 
    then there are unique operators $A$ in $\mathscr{B}(\mathscr{H},\mathscr{K})$ and $B$ in $\mathscr{B}(\mathscr{K},\mathscr{H})$
    such that 
    \begin{align}
        u(h,k) = \inner{Ah}{k}=\inner{h}{Bk}
        \label{eq:adjont equation}
    \end{align}
    for all $h$ in $\mathscr{H}$ and $k$ in $\mathscr{K}$ and $||A||,||B||\leqs M$. 
\end{theorem}

\begin{proofsolution}
    
    
\end{proofsolution}


\begin{definition}{}{}
    If $A\in \mathcal{B}(\hH,\hK)$, then the unique operator $B$ in $\mathscr{B}(\hK,\hH)$ satisfying (\ref{eq:adjont equation})
    is call the adjoint of $A$ and is denote by $B=A^*$.
\end{definition}

\begin{proposition}{}{}
    If $U\in \mathcal{B}(\hH,\hK)$, then $U$ is an isomorphism iff $U$ is invertible and $U^{-1}=U^*$.
\end{proposition}

\begin{proof}
    $U$ is isomorphism, then for $h,g\in \hH$, 
    \begin{align*}
        <h,g>=<Uh,Ug> = <h,U^*Ug>.
    \end{align*} 
    So, $U^*U=I$. Since, $U$ is surjection, $U$ is invertible and $U^{-1}=U^*$.

    Conversely, let $U$ be invertible with $U^{-1}=U^*$. Then, $u$ is a surjection, and 
    \begin{align*}
        <Ux,Uy> = <x,U^*Uy>=<x,Iy>=<x,y>.
    \end{align*}
\end{proof}

From now on we will examine and prove results for the adjoint of operators 
in $\mathcal{B}(\hH)$. Often, as in the next proposition, there are analogous results for 
the adjoint of operators in $\mathcal{B}(\hH,\hK)$.

\begin{proposition}{}{}
    If $A,B\in \mathscr{B}(\mathcal{\hH})$ and $\alpha\in\F$, then:\\
    (1) $(\alpha A+B)^*=\overline{\alpha}A^*+B^*$.\\
    (2) $(AB)^*=B^*A^*$.\\
    (3) $A^{**}=(A^*)^*=A$.\\
    (4) If $A$ is invertible in $\mathcal{B}(\hH)$ and $A^{-1}$ is its inverse, then $A^*$ is invertible and $(A^*)^{-1} = (A^{-1})^*$.
\end{proposition}
\begin{proof}
    (1) $<(\alpha A+B)h,g>=\alpha <Ah,g> + <Bh,g> = \alpha<h,A^*g> + <h,B^*g>=<h,(\overline{\alpha}A^*+B^*)g>$.
    \\
    (2)\\
(3)\\
(4)
\end{proof}

\begin{proposition}{}{}
    If $A\in \mathcal{B}(\hH)$ , $||A||=||A^*||=\sqrt{||A^*A||}$.
\end{proposition}

\begin{definition}
    If $A\in \mathcal{B}(\hH)$, then\\
    (1) $A$ is hermitian or self-adjoint if $A^*=A$.\\
    (2) $A$ is normal if $AA^*=A^*A$.
\end{definition}

In the analogy between the adjoint and the complex conjugate, hermitian 
operators become the analogues of real numbers and, unitaries are 
the analogues of complex numbers of modulus 1. Normal operators, as we 
shall see, are the true analogues of complex numbers. Notice that hermitian 
and unitary operators are normal.

\begin{proposition}{}{}
    If $\mathscr{H}$ is a $\C$-Hilbert space and $A\in \mathcal{B}(\hH)$, then $A$ is self-adjoint iff
    $\inner{Ah}{h}\in\R$ for all $h$ in $\hH$. 
\end{proposition}

\begin{proposition}{}{}
    If $A=A^*$, then 
    \begin{align*}
        ||A||=\sup_{||h||=1}|\inner{Ah}{h}|. 
    \end{align*}
\end{proposition}

\begin{corollary}{}{}
    If $A=A^*$ and $\inner{Ah}{h}=0$ for all $h$, then $A=0$.
\end{corollary}

\begin{proposition}{}{}
    If $\hH$ is $\C$-Hilbert space and $A\in \mathcal{B}(\hH)$ such that $\inner{Ah}{h}=0$ for all $h$ in $\hH$, then $A=0$.
\end{proposition}

\begin{exercise}{II2 T12}{}
    Let $\sum\limits_{n=0}^{\infty}\alpha_nz^n$ be a power series with radius of convergence $R$, $0<R\leqs \infty$.
    If $A\in \mathscr{B}(\mathscr{H})$ and $||A||<R$,
    show that there is an operator $T$ in $\mathscr{B}(\mathscr{H})$ such that for any $h,g\in \mathscr{H}$,
    $\inner{Th}{g}=\sum\limits_{n=0}^{\infty}\alpha_n\inner{A^nh}{g}$.
    [If $f(z)=\sum\alpha_nz^n$, the operator $T$ is usually denoted by $f(A)$.]
\end{exercise}

\begin{exercise}{II2 T15}{}
    If $A$ is a normal operator on $\mathscr{H}$, show that $A$ is injective if and only if $A$ has dense range.
    Give an example of an operator $B$ such that ker$B=(0)$ but ran$B$ is not dense.
    Give an example of an operator $C$ such that $C$ is surjective but ker$C\neq (0)$.
\end{exercise}