\chapter{The Hahn-Banach Theorem}\label{chp:3_6}

If there is a "fundamental theorem of functional analysis," it is the Hahn-Banach
theorem. The theorem is somewhat abstract-looking at first, but its importance will be
clear after studying some of its corollaries.

To state and prove the Hahn-Banach
Extension Theorem, we first work in the setting
$\F=\R$, then extend the results to the
complex case.

\begin{definition}{}{}
    If $\mathscr{X}$ is a vector space, 
    a sublinear functional is a function $q:\mathscr{X}\rightarrow \R$ such that\\
    (1) $q(x+y)\leqs q(x)+q(y)$ for all $x,y$ in $\mathscr{X}$;\\
    (2) $q(\alpha x)=\alpha q(x)$ for $x$ in $\mathscr{X}$ and $\alpha\geqs 0$. 
\end{definition}
\begin{remark}
    Note that every seminorm is a sublinear functional, but not conversely.    
\end{remark}

\begin{theorem}{The Hahn-Banach Theorem on $\R$}{The Hahn-Banach Theorem on R}
        Let $\mathscr{X}$ be a vector space over $\R$ and let $q$ be a sublinear functional on $\mathscr{X}$.
        If $\mathscr{M}$ is a linear mainfold in $\mathscr{X}$ and $f:\mathscr{M}\rightarrow \R$ is a linear funcional
        such that $f(x)\leqs q(x)$ for all $x$ in $\mathscr{M}$, 
        then there is a linear functional $F:\mathscr{X}\rightarrow \R$ 
        such that \\
        (1) $F|\mathscr{M}=f$ ($F$ extends $f$)\\
        (2) $F(x)\leqs q(x)$ for all $x$ in $\mathscr{X}$ ($F$ is dominated by $q$).
\end{theorem}

The proof will invoke Zorns Lemma,
a result that is equivalent to the axiom of choice (as well as ordering principal and the Hausdorff maximality principal).
A partial order $\preceq$ on a set is a relation that is reflexive, symmetric and transitive; that is\\
(1) $x\preceq x$ for all $x\in S$;\\
(2) for $x,y\in S$, if $x\preceq y$ and $y\preceq x$, then $x=y$;\\
(3) for $x,y,z\in S$, if $x\preceq y$ and $y\preceq z$, then $x\preceq z$.

We call $S$, or more precisely, $(S,\preceq)$ a patially ordered set or poset.
A subset $T$ of $S$ is totally ordered, if for each $x,y\in T$ either $x\preceq y$ or $y\preceq x$.
A totally ordered subset $T$ is often called a chain.
An upper bound $z$ for a chain $T$ is an element $z\in S$ such that $t\preceq z$ for all $t\in T$.
A maximal element for $S$ is $w\in S$ that has no successor; 
that is there does not exist an $s\in S$ such that $s\neq w$ and $w\preceq s$. 


\begin{theorem}{Zorn's Lemma}{Zorn's Lemma}
    Suppose $S$ is a partially ordered set.
    If every chain in $S$ has an upper bound, then $S$ has a maximal element.
\end{theorem}

Now, let us prove the Hahn-Banach Theorem on $\R$.
\begin{proofsolution}
    The idea is to show that the extension can be done one dimension at a time and
    then infer the existence of an extension to the whole space by appeal to Zorn's lemma.
    We may of course assume $\mathscr{M}\neq \mathscr{X}$. 
    So, fix a vector $x\in\mathscr{X}\setminus \mathscr{M}$
    and consider the subspace $\mathscr{M}+\R x\subset \mathscr{X}$.
    For any $m_1,m_2\in\mathscr{M}$, by hypothesis,
    \begin{align*}
        f(m_1) + f(m_2)=f(m_1+m_2)\leqs q(m_1+m_2)\leqs q(m_1-x)+q(m_2+x).
    \end{align*}
    Rearranging gives, for $m_1,m_2\in\mathscr{M}$,
    \begin{align*}
        f(m_1)-q(m_1-x)\leqs q(m_2+x)-f(m_2)
    \end{align*}
    and thus
    \begin{align*}
        \sup_{m\in\mathscr{M}} \{f(m)-q(m-x)\}\leqs \inf_{m\in\mathscr{M}} \{q(m+x)-f(m)\}.
    \end{align*}
    Now choose any real number $\lambda$ satisfying
    \begin{align*}
        \sup_{m\in\mathscr{M}} \{f(m)-q(m-x)\}\leqs \lambda \leqs \inf_{m\in\mathscr{M}} \{q(m+x)-f(m)\}.
    \end{align*}
    In particular, for $m\in\mathscr{M}$,
    \begin{align}
        \begin{aligned}
        f(m)-\lambda \leqs q(m-x)\\
        f(m)+\lambda \leqs q(m+x)
        \end{aligned}
        \label{eq:dominated inequation}
    \end{align}
    Let $\mathscr{N}=\mathscr{M}+\R x$ and define $F:\mathscr{N}\rightarrow \R$ by $F(m+tx)=f(m)+t\lambda$ for $m\in\mathscr{M}$
    and $t\in\R$. Thus $F$ is linear and agrees with $F$ on $\mathscr{M}$ by definition.
    We now check that $F(y)\leqs q(y)$ for all $y\in\mathscr{M}+\R x$.
    Accordingly, suppose $m\in\mathscr{M},t\in\R$ and let $y=m+tx$.
    If $t=0$ there is nothing to prove. If $t>0$, then, in view of inequation (\ref{eq:dominated inequation}),
    \begin{align}
        F(y) = F(m+tx) = t(f(\frac{m}{t})+\lambda)\leqs t(q(\frac{m}{t}+x))=q(m+tx)=q(y).
    \end{align}
    and a similar estimate shows that $F(m+tx)\leqs q(m+tx)$for $t < 0$.
    \par
    We have thus successfully extended $f$ to $\mathscr{M}+\R x$.
    To finish the proof, let $\mathscr{S}$ be the collcetion of all pairs $(\mathscr{M}_1,f_1)$, 
    where $\mathscr{M}_1$ is a linear manifold in $\mathscr{X}$ such that $\mathscr{M}_1\supset \mathscr{M}$ 
    and $f_1:\mathscr{M}\rightarrow \R$ is a linear functional with $f_1|\mathscr{M}=f$ and $f_1\leqs q$ on $\mathscr{M}_1$.
    If $(\mathscr{M}_1,f_1)$ and $(\mathscr{M}_2,f_2)\in\mathscr{S}$, define $(\mathscr{M}_1,f_1)\preceq \mathscr{M}_2,f_2$ 
    to mean that $\mathscr{M}_1\subseteq \mathscr{M}_2$ and $f_2|\mathscr{M}_1=f_1$.
    So $(\mathscr{S},\preceq)$ is a partially ordered set.
    Suppose $\mathscr{C}=\{(\mathscr{M}_i,f_i):i\in I\}$ is a chain in $\mathscr{S}$.
    If $\mathscr{N}\equiv \cup \{\mathscr{M}_i:i\in I\}$, then the fact that $\mathscr{C}$ is a chain implies that $\mathscr{N}$
    is a linear mainfold. Define $F:\mathscr{N}\rightarrow\R$ by setting $F(x)=f_i(x)$ if $x\in \mathscr{M}_i$. 
    Then $F$ is well defined, linear, and satisfies $F\leqs q$ on $\mathscr{N}$.
    So $(\mathscr{N}, F)\in \mathscr{S}$ and $(\mathscr{N},F)$ is an upper bound for $\mathscr{C}$.
    By Zorn's Lemma, $\mathscr{S}$ has a maximal element $(\mathscr{Y},F)$.
    Since it always possible to extend to a strictly larger subspace, the maximal element must have $\mathscr{N}=\mathscr{X}$,
    and the proof is finished. 
\end{proofsolution}

The proof is a typical application of Zorn's lemma - one knows how to carry out a
construction one step a time, but there is no clear way to do it all at once.


Before obtaining further corollaries, we extend these results to the complex case.
First, if $X$ is a vector space over $\C$, 
then trivially it is also a vector space over $\R$, and
there is a simple relationship between the
$\R$- and $\C$-linear functionals.

\begin{lemma}{}{R- and C-linear functionals relation}
    Let $\mathscr{X}$ be a vector space over $\C$.
    If $g:\mathscr{X}\rightarrow \C$ is a $\C$-linear functional, 
    then $f(x)=\text{Re }g(x)$ defines an $\R$-linear functional on $\mathscr{X}$
    and $g(x)=f(x)-if(ix)$.
    Conversely, if $f:\mathscr{X}\rightarrow \R$ is $\R$-linear then $g(x)=f(x)-if(ix)$ 
    is $\C$-linear.
    If in addition $p:\mathscr{X}\rightarrow \R$ is a seminorm, then $|f(x)|\leqs p(x)$ for all $x\in\mathscr{X}$
    if and only if $|g(x)|\leqs p(x)$ for all $x\in\mathscr{X}$.
\end{lemma}


\begin{theorem}{The Hahn-Banach Theorem on $\F$}{}
    Let $\mathscr{X}$ be a vector space over $\C$ and let $p:\mathscr{X}\rightarrow [0,\infty)$ be a seminorm.
    If $\mathscr{M}$ is a linear mainfold in $\mathscr{X}$ and $f:\mathscr{M}\rightarrow \C$ is a linear funcional
    such that $|f(x)|\leqs p(x)$ for all $x$ in $\mathscr{M}$, 
    then there is a linear functional $F:\mathscr{X}\rightarrow \C$ 
    such that \\
    (1) $F|\mathscr{M}=f$ ($F$ extends $f$)\\
    (2) $|F(x)|\leqs p(x)$ for all $x$ in $\mathscr{X}$ ($F$ is dominated by $p$).
\end{theorem}

\begin{proofsolution}
    Case 1: Note that $f(x)\leqs |f(x)|\leqs p(x)$, by theorem\ref{thm:The Hahn-Banach Theorem on R},
    there exists $F:\mathscr{X}\rightarrow \R$ such that $F|\mathscr{M}=f$ and $F(x)\leqs p(x)$ for all $x\in\mathscr{X}$.
    Hence, $-F(x)=F(-x)\leqs p(-x)=p(x)$ and so $|F(x)|\leqs p(x)$.
    \\
    Case 2:
    Let $f_1=\text{Re }f$. By lemma\ref{lem:R- and C-linear functionals relation}, $|f_1|\leqs p$, 
    by Case 1, there exist $F_1$ such $F_1|\mathscr{M}=f_1$ and $|F_1(x)|\leqs p(x)$ for $x$ in $\mathscr{X}$.
    Let $F(x)=F_1(x)-iF_1(ix)$, by lemma\ref{lem:R- and C-linear functionals relation}, 
    $F|\mathscr{M}=f$ and $|F(x)|\leqs p(x)$ for all $x$ in $\mathscr{X}$.
\end{proofsolution}

The following corollaries are quite important, and when the Hahn-Banach theorem
is applied it is usually in one of the following forms:

\begin{corollary}{}{}
    Let $\mathscr{X}$ be a normed vector space.\\
    (1) If $\mathscr{M}$ is a linear manifold in $\mathscr{X}$, 
    and $f:\mathscr{M}\rightarrow \F$ is a bounded linear functional, then there is an $F$ in $\mathscr{X}^*$
    such that $F|\mathscr{M}=f$ and $||F||=||f||$.\\
    (2) (Linear functionals detect norms)
    If $x\in\mathscr{X}$ is nonzero, there exists $F\in\mathscr{X}^*$ with $||F||=1$ such that $F(x)=||x||$.\\
    (3) (Linear functionals separate points)
    If $x\neq y$ in $\mathscr{X}$, there exists $F\in\mathscr{X}^*$ such that $F(x)\neq F(y)$.\\
\end{corollary}

\begin{proof}
    (1) Consider the seminorm $p(x)=||f||\cdot ||x||$. 
    Then $|f(x)|\leqs ||f||\cdot ||x||=p(x)$ for $x\in\mathscr{M}$.
    Hence, by Hahn-Banach Theorem,
    there exists a linear funcional $F$ on $\mathscr{X}$ such that 
    $F|\mathscr{M}=f$ and $|F(x)|\leqs p(x)$ for all $x\in\mathscr{X}$.
    Then $\frac{|F(x)|}{||x||}\leqs ||f||$ for $x\neq 0$ and so $||F||\leqs ||f||$.
    On the other hand, $||F||\geqs ||f||$ since $F$ agree with $f$ on $\mathscr{M}$.
    Hence, $||F||=||f||$.\\
    (2) Let $\mathscr{M}$ be a one-dimensional linear mainfold of $\mathscr{X}$ spanned by $x$.
    Define a functional $f:\mathscr{M}\rightarrow \F$ by $f(t\frac{x}{||x||})=t$. Then $|f(y)|=||y||$ for $y\in\mathscr{M}$
    and thus $||f||=\sup_{y\neq 0} \frac{|f(y)|}{||y||}=1$. 
    By (1), the funcional $f$ can extend to a funcional $F$ on $\mathscr{X}$ such that $F|\mathscr{M}=f$, $||F||=||f||$ and so $F(x)=||x||$.
    \\
    (3) Let $F$ be as in $(2)$, then $F(x)-F(y)=F(x-y)=||x-y||\neq 0$ and so $F(x)\neq F(y)$. 
\end{proof}


\begin{corollary}{}{}
    If $\mathscr{X}$ is a normed space, $\{x_1,x_2,...,x_d\}$ is a linearly independent subset of $\mathscr{X}$, 
    and $\alpha_1,\alpha_2,...,\alpha_d$ are arbitrary scalars, then there is an $f$ in $\mathscr{X}$ such that $f(x_j)=\alpha_j$ for $1\leqs j\leqs d$.
\end{corollary}

\begin{proof}
    Let $\mathscr{M}$ be the linear span of $x_1,...,x_d$ and 
    define $g:\mathscr{M}\rightarrow \F$ by $g(\sum\limits_{j}\beta_jx_j)=\sum\limits_{j}\beta_j\alpha_j$.
    Then $g$ is linear. Since $\mathscr{M}$ is finite dimensional, $g$ is bounded. Then by Hahn-Banach Theorem,
    there is $f:\mathscr{X}\rightarrow \F$ such that $f|\mathscr{M}=g$ and so $f(x_j)=\alpha_j, 1\leqs j\leqs d$.
\end{proof}

\begin{corollary}{}{}
    If $x$ is a normed space and $x\in\mathscr{X}$, then
    \begin{align*}
        ||x||=\sup \{|f(x)|:f\in\mathscr{X}^* \text{ and } ||f||\leqs 1\}.
    \end{align*}
\end{corollary}

\begin{proof}
    Let $\alpha = \sup \{|f(x)|:f\in\mathscr{X}^* \text{ and }||f||\leqs 1\}$.
    Since $|f(x)|\leqs ||f||\cdot ||x||$  and $||f||\leqs 1$, it follows that $|f(x)|\leqs ||x||$. Then $\alpha\leqs ||x||$.
    On the other hand, there is $f\in \mathscr{X}^*$ such that $||f||=1$ and $f(x)=||x||$. Then $\alpha\geqs ||x||$.
    Hence, $\alpha=||x||$.
\end{proof}

\begin{corollary}
    If $\mathscr{X}$ is normed space, $\mathscr{M}\leqs \mathscr{X}$, $x_0\in \mathscr{X}\setminus \mathscr{M}$, and $d=\text{dist}(x_0,\mathscr{M})$,
    then there is an $f$ in $\mathscr{X}$ such that $f(x_0)=1$, $f|\mathscr{M}=0$, and $||f||=d^{-1}$.
\end{corollary}

\begin{proof}
    Since $||x_0+\mathscr{M}||=d$, then there is $g\in (\mathscr{X}\setminus \mathscr{M})^*$ such that $g(x_0+\mathscr{M})=d$ and $||g||=1$.
    Let $Q:\mathscr{X}\rightarrow \mathscr{X}/\mathscr{M}$ be the natural map and $f=d^{-1}g\circ Q:\mathscr{X}\rightarrow \F$.
    Then $f$ is continuous, $f|\mathscr{M}=0$ since $\text{ker}f=\mathscr{M}$ and $f(x_0)=1$.
    Also, $|f(x)|=d^{-1}|g(Q(x))|\leqs d^{-1}||g||\cdot ||Q(x)||\leqs d^{-1}||x||$;
    hence $||f||\leqs d^{-1}$.
    On the other hand, $||g||=1$ so there is a sequence $\{x_n\}$ such that $|g(x_n+\mathscr{M})|\rightarrow 1$ and $||x_n+\mathscr{M}||<1$ for all $n$.
    Let $y_n\in\mathscr{M}$ such that $||x_n+y_n||<1$. 
    Then $|f(x_n+y_n)|=d^{-1}|g(x_n+\mathscr{M})|\rightarrow d^{-1}$, so $||f||\geqs d^{-1}$.
    So $||f||=d^{-1}$.
\end{proof}



\section{Homework}

\begin{exercise}{III6 T1}{}
    Let $\mathscr{X}$ be a vector space over $\C$.\\
    (a) If $f:\mathscr{X}\rightarrow \R$ is an $\R$-linear Functional, then
    $\tilde{f}(x)=f(x)-if(ix)$ is a $\C$-linear functional and $f=\text{Re }\tilde{f}$.\\
    (b) If $g:\mathscr{X}\rightarrow \C$ is $\C$-linear, $f=\text{Re }g$, and $\tilde{f}$ is defined as in (1), then $\tilde{f}=g$.\\
    (c) If $p$ is a seminorm on $\mathscr{X}$ and $f$ and $\tilde{f}$ are as in $(a)$,
    then $|f(x)|\leqs p(x)$ for all $x$ if and only if $|\tilde{f}(x)|\leqs p(x)$ for all $x$.\\
    (d) If $\mathscr{X}$ is a normed space and $f$ and $\tilde{f}$ are as in (a), then $||f||=||\tilde{f}||$.
\end{exercise}

\begin{proof}
    (a) By the construction of $\tilde{f}$, it is clearly that $\text{Re }\tilde{f}=f$.
    Define $h:\mathscr{X}\rightarrow \C$ given by $h(x)=ix$. Then $h$ is $\R$-linear.
    Since composition of two linear functions is a linear function,
    it follows that $\tilde{f}=f(x)-g\circ f \circ g(x)$ is $\R$-linear.
    So we only need to check $\tilde{f}(ix)=i\tilde{f}(x)$.
    \begin{align*}
        \tilde{f}(ix)&= f(ix)-if(iix)\\
                    &= f(ix)-if(-x)=f(ix)+if(x)\\
                    &=i(f(x)-if(ix))=i\tilde{f}(x).
    \end{align*}
    Hence, $\tilde{f}$ is $\C$-linear.\\
    (b) If $g$ is $\C$-linear, then $f=\text{Re }g$ is $\R$-linear.
    Since $\text{Im }z=-\text{Re }(iz), \forall z\in\C$, we have
    \begin{align*}
        g(x)&=\text{Re }g(x) + i\text{Im }g(x)\\
            &= \text{Re }g(x)-i\text{Re}(ig(x))\\
            &= \text{Re }g(x)-i\text{Re}(g(ix))\\
            &= f(x)-if(ix)=\tilde{f}(x).
    \end{align*}
    (c) ($\Rightarrow$):
    Now assume that $|f(x)|\leqs p(x)$ and
    fix $x\in\mathscr{X}$ and choose $\theta$ such that 
    $\tilde{f}(x)=e^{i\theta}|\tilde{f}(x)|$.
    Hence,
    \begin{align*}
        |\tilde{f}(x)|&= \text{Re }|\tilde{f}(x)|\\
                      &= \text{Re}(e^{-i\theta}\tilde{f}(x)) =\text{Re }\tilde{f}(e^{-i\theta}x)= f(e^{-i\theta}x) \\
                      &\leqs |f(e^{-i\theta}x)|=|f(x)|\leqs p(x).
    \end{align*} 
    ($\Leftarrow$):
    Suppose $|\tilde{f}(x)|\leqs p(x)$. 
    Then $f(x)=\text{Re }\tilde{f}(x)\leqs |\tilde{f}(x)|\leqs p(x)$.
    Also, $-f(x)=\text{Re }\tilde{f}(-x)\leqs |\tilde{f}(-x)|\leqs p(x)$.
    Hence $|f(x)|\leqs p(x)$.\\
    (d) For all $x\in\mathscr{X}$, by (c), $|\tilde{f}(x)|\leqs |f(x)|$ 
    and $|f(x)|=|\text{Re }\tilde{f}(x)|\leqs |\tilde{f}(x)|$. 
    Hence, $||f||\leqs ||\tilde{f}||$ and $||\tilde{f}||\leqs ||f||$. So $||f||=||\tilde{f}||$.
\end{proof}


\begin{exercise}{III6 T2}{}
    If $\mathscr{X}$ is a normed space, $\mathscr{M}$ is a linear manifold in $\mathscr{X}$, 
    and $f:\mathscr{M}\rightarrow \F$ is a bounded linear functional, then there is an $F$ in $\mathscr{X}^*$
    such that $F|\mathscr{M}=f$ and $||F||=||f||$.
\end{exercise}
\begin{proof}
    Consider the seminorm $p(x)=||f||\cdot ||x||$. 
    Then $|f(x)|\leqs ||f||\cdot ||x||=p(x)$ for $x\in\mathscr{M}$.
    Hence, there exists a linear funcional $F$ on $\mathscr{X}$ such that 
    $F|\mathscr{M}=f$ and $|F(x)|\leqs p(x)$ for all $x\in\mathscr{X}$.
    Then $\frac{|F(x)|}{||x||}\leqs ||f||$ for $x\neq 0$ and so $||F||\leqs ||f||$.
    On the other hand, $||F||\geqs ||f||$ since $F$ agree with $f$ on $\mathscr{M}$.
    Hence, $||F||=||f||$.
\end{proof}

\section{Reference}

\begin{itemize}
    \item \href{https://people.clas.ufl.edu/pascoej/files/6617notes25april2018.pdf}{The Hahn-Banach Theorem}
    \item \href{https://www.math.kit.edu/iana1/lehre/funcana2012w/media/fa-lecturenotes.pdf}{III6 T1; P32}
    \item \href{https://people.clas.ufl.edu/pascoej/files/6617notes25april2018.pdf}{III6 T2; P31}
\end{itemize}