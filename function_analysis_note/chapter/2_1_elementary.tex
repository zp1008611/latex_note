\chapter{Elementary Properties and Examples}\label{chp:2_1}


Let $\hH$ and $\hK$ be two Hilbert space over $\F$. Recall that a map $A:\hH\rightarrow \hK$ is a linear transformation if for all $x_1,x_2\in \hH$ and $\alpha,\beta\in \F$,
\begin{align*}
    A(\alpha x_1+\beta x_2) = \alpha A(x_1)+\beta A(x_2).
\end{align*}
Then $N(A)=\{x\in\hH: Ax=0\}$ and $R(A)$ are subspaces of $\hH$ and $\hK$ respectively. 

The collection of all linear operators from $\hH$ to $\hK$ forms a vector space $\mathcal{L}(\hH, \hK)$ under pointwise
addition and scalar multiplication of functions.

Let's recall the definition of bounded linear transformation and the norm of bounded linear transformation. The proof of the next proposition is similar to the proofs of the corresponding results for linear funtionals in proposition\ref{prop:bouded operator equivalent}. 

\begin{proposition}{}{}
    Let $\mathcal{H}$ and $\mathcal{K}$ be Hilbert spaces and $A:\mathcal{H}\rightarrow \mathcal{K}$ be a linear transformation. Then the following are equivalent\\
    (1) $A$ is 
\end{proposition}

As in definition\ref{def:norm bounded linear funcional}, if $A$ is a bounded linear transformation, the norm of $A$ defined as
\begin{align*}
    ||A||=\sup_{x\in \hH,||x||\leqs 1}||Ax||.
\end{align*}
And then
\begin{align*}
    ||A||&=\sup_{||x||=1}||Ax||\\
        &= \sup_{x\neq 0} \frac{||Ax||}{||x||}\\
        &= \inf\{c>0:||Ax||\leqs C||x||,x\in \hH\}.
\end{align*}
Also, 
\begin{align*}
    ||Ax||\leqs ||x||. 
\end{align*}

We denote the collection of all bounded linear operators from $\hH$ to $\hK$ by $\mathcal{B}(\hH, \hK)$. It is a subspace
of $\mathcal{L}(\hH, \hK)$. They coincide when $\hH$ and $\hK$ are of finite dimension, of course.
For $\hK=\hH$, $\mathcal{B}(\hH):=\mathcal{B}(\hH,\hH)$. Note that $\mathcal{B}(\hH,\F)$= all the bounded linear functionals on $\hH$.

Now we introduce some example of bounded linear transformation.