\chapter{Orthonormal Sets of Vectors and Bases}\label{chp:1_4}



It will be shown in this chapter that, as in Euclidean space, each Hilbert 
space can be coordinatized. The vehicle for introducing the coordinates is 
an orthonormal basis. The corresponding vectors in $\F^d$ are the vectors 
$\{e_1, e_2, ... , e_d\}$, where $e_k$ is the $d$-tuple having a $1$ in the $k$th place and zeros elsewhere.

Subsets $\mathscr{E}$ mentioned below can be finite, countably infinite and uncoutably infinite. 

\begin{definition}{}{}
    A subset $\mathscr{E}\subset \hH$ is said to orthogonal if $\inner{e_1}{e_2}=0$, for all $e_1\neq e_2\in \mathscr{E}$. 
    We say it is orthonormal if we further regive $||e||=1$ for all $e\in \mathscr{E}$. 
\end{definition}

\begin{remark}
    Orthogonal set can contain zero vector, but orthonormal set can not. 
\end{remark}


\begin{definition}{}{}
    $\{h_1,..., h_k\}$ is said to be linearly dependent, if there exist scalars $a_1,a_2,...,a_k$, not all zero, such that
    \begin{align*}
        a_1h_1+...+a_kh_k=0
    \end{align*}
    where $0$ denotes the zero vector.
    An infinite set of vectors is linearly independent if every nonempty finite subset is linearly independent.
\end{definition}


\begin{corollary}{}{}
    Every orthonormal set is linearly independent.
\end{corollary}
\begin{proof}
    If the set is finite. Suppose $\{h_1,...,h_k\}$ is an orthonormal set. Assume $b_1h_1+...b_kh_k=0$, then $b_1u_1=-\sum\limits_{i=2}^{k}b_iu_i$. 
    When $2\leqs i\leqs n$, $0=\inner{h_i}{-b_1h_1}=\sum\limits_{j=2}^{k}b_i\inner{h_i}{h_j}
    =b_i\inner{h_i}{h_i}=b_i$. Then $b_1h_1=0$.
    Since $h_1\neq 0$, it follows that $b_1=0$. Hence, $b_i=0$ for all $1\leqs i\leqs k$ and so $\{h_1,...,h_k\}$ is linearly independent.
    If the set is infinite, the above proof can show that every nonempty finite subset is linearly independent and so the set is linearly independent.
\end{proof}


Now we consider the question: Does the orthonormal set always exist? The following theorem tell us we can always construct an orthonormal set.

\begin{theorem}{The Gram-Schmidt Orthogonalization Process}{}
    If $\{h_1,...,h_k\}$ is a linearly independent subset of $\hH$, 
    then there is an orthonormal set $\{e_1,...,e_k\}$ such that $span(\{e_1,...,e_k\})= span(\{h_1,...,h_k\})$.
\end{theorem}
\begin{proofsolution}
    Define $\{e_k\}$ inductivey. Start with $e_1=\frac{h_1}{||h_1||}$. 
    Suppose for $k-1$, $e_1,...,e_{k-1}$ exist. Let $v_k = h_k-\sum\limits_{j=1}^{k-1}<e_j,h_k>e_j$ and $e_k = \frac{v_k}{||v_k||}$.
    Then $\inner{e_k}{e_j}=\frac{1}{||v_k||}(\inner{h_k}{e_j}-\inner{e_j}{h_k}\inner{e_j}{e_j})=0$ and $||e_k||=1$.
    Hence, $\{e_1,...,e_k\}$ is an orthonormal set. 
    from the construct process of $e_j(1\leqs j \leqs k)$, we can know $e_j\in span(\{h_1,...,h_k\})$ and so $span(\{e_1,...,e_k\})\subset span(\{h_1,...,h_k\})$. 
    Since $\{e_1,...,e_k\}$ is linearly independent, $dim(span(\{e_1,...,e_k\}))=n$. Hence, $span(\{e_1,...,e_k\})= span(\{h_1,...,h_k\})$.
\end{proofsolution}

\begin{proposition}{}{}
    If $\mathscr{H}$ is a Hilbert space and $\{h_n:n\in \N\}$ is a linearly independent subset of $\mathscr{H}$,
    then there is an orthonormal set $\{e_n:n\in\N\}$ such that for every $n$, the linear span of $\{e_1,...,e_n\}$ equals the linear span of $\{h_1,...,h_n\}$.
\end{proposition}


Remember that $\vee A$ is the closed linear span of $A$.
Now we consider the following question: How can we determine $Ph$ when $h$ and the subspace
$\mathscr{M}$ are given? When $\mathscr{M}$ is finite, we have

\begin{proposition}{}{}
    Let $\{e_1,...,e_n\}$ be an orthonormal set in $\mathscr{H}$ and let $\mathscr{M}=\vee \{e_1,...,e_n\}$.
    If $P$ is the orthogonal projection of $\mathscr{H}$ onto $\mathscr{M}$, then 
    \begin{align*}
        Ph=\sum\limits_{k=1}^{n}\inner{h}{e_k}e_k
    \end{align*}
    for all $h$ in $\mathscr{H}$.
\end{proposition}
\begin{proof}
    Let $Qh=\sum\limits_{k=1}^{n}\inner{h}{e_k}e_k$.
    it suffices to show that $Qh$ is the unique element in $\mathscr{M}$ such that $h-Qh\perp \mathscr{M}$.
    For $1\leqs j\leqs n$, $\inner{Qh}{e_j}=\sum\limits_{k=1}^{n}\inner{h}{e_k}\inner{e_k}{e_j}=\inner{h}{e_j}$ as $e_k\perp e_j$ when $k\neq j$.
    Then $\inner{h-Qh}{e_j}=0$ and so $h-Qh\perp \mathscr{M}$. Since $Qh\in \mathscr{M}$, it follows that $Qh$ is the unique element in $\mathscr{M}$
    such that $h-Qh\perp \mathscr{M}$. Hence, by the property of $Ph$, $Ph=Qh$ for every $h$ in $\mathscr{H}$.
\end{proof}

\begin{proposition}{Bessel's inequality}{Bessel's inequality countable}
    If $\{e_n:n\in\N\}$ is an orthonormal set 
    and $h\in\mathscr{H}$, then 
    \begin{align*}
        \sum\limits_{n=1}^{\infty}|\inner{h}{e_n}|^2\leqs ||h||^2.
    \end{align*}
\end{proposition}

\begin{proof}
    Let $h_n=h-\sum\limits_{k=1}^{n}\inner{h}{e_k}e_k$.
    Then $\inner{h_n}{e_k}=\inner{h}{e_k}-\inner{h}{e_k}=0$.
    By the Pythagorean Theorem,
    \begin{align*}
        ||h||^2 &= ||h_n||^2 +||\sum\limits_{k=1}^{n}\inner{h}{e_k}e_k||^2\\
                &=||h_n||^2+\sum\limits_{k=1}^{n}|\inner{h}{e_k}|^2\\
                &\geqs \sum\limits_{k=1}^{n} |\inner{h}{e_k}|^2.
    \end{align*}
    Since $n$ was arbitrary, the result is proved.
\end{proof}


\begin{corollary}{}{}
    If $\mathscr{E}$ is an orthonormal set in $\mathscr{H}$ and $h\in\mathscr{H}$,
    then $\inner{h}{e}\neq 0$ for at most countable number of vectors $e$ in $\mathscr{E}$.
\end{corollary}

\begin{proof}
    Let $n\in\N_+$. We claim that the subset $\mathscr{E}_n$
    \begin{align*}
        \mathscr{E}_n = \{e\in \mathscr{E}: \frac{1}{n}\leqs|\inner{h}{e}|\}
    \end{align*} 
    of $\mathscr{E}$ is a finite set. Pick $e_1,...,e_N$ from $\mathscr{E}_n$, then by Bessel's inequality,
    \begin{align*}
        ||h||^2\geqs \sum\limits_{k=1}^{N}|\inner{h}{e_k}|^2\geqs \frac{N}{n^2},
    \end{align*}
    It follows that the cardinality of $\mathscr{E}_n$ cannot exceed $n^2||h||^2$.
    Let $\mathscr{E}_x$ be the subset of $\mathscr{E}$ consisting of all $e$ such that 
    $\inner{h}{e}$ is non-zero.
    Since $\mathscr{E}_x=\cup_{n\in\N_+} \mathscr{E}_n$, 
    it follows that $\mathscr{E}_x$ is a countable union of finite sets and so $\mathscr{E}_x$ is countable. 
\end{proof}

\begin{corollary}{}{Bessel's inequality infinite}
    If $\mathscr{E}$ is an orthonormal set and $h\in\mathscr{H}$, then
    \begin{align*}
        \sum\limits_{e\in\mathscr{E}}|\inner{h}{e}|^2\leqs ||h||^2.
    \end{align*}
\end{corollary}

\begin{proof}
    Restrict our attention to the $\{e \in\mathscr{E} : \inner{h}{e}\neq 0\}$ which is countable
    by the last corollary, and now it is just a straight up use of Bessel's ineq in proposition\ref{prop:Bessel's inequality countable}
\end{proof}

Let $\mathscr{F}$ be the collection of all finite subsets of $I$ and order 
by inclusion, so $\mathscr{F}$ becomes a directed set. 
For each $F$ in $\mathscr{F}$, define 
$h_F=\sum \{h_i:i\in F\}$. 
Since this is a finite sum, $h_F$ is a well-defined element of $\mathscr{H}$. 
Now $\{h_F: F\in\mathscr{F}\}$ is a net in $\mathscr{H}$.

\begin{definition}{}{}
    With the notation above, the sum $\sum\{h_i:i\in I\}$ converges if 
the net $\{ h_F: F \in\mathscr{F}\}$ converges; the value of the sum is the limit of the net.
\end{definition}

Now Corollary\ref{cor:Bessel's inequality infinite} 
can be given its precise meaning; namely, 
$\sum\limits_{e\in\mathscr{E}}|\inner{h}{e}|^2$ converges 
and the value $\leqs ||h||^2$.

\begin{lemma}{}{}
    If $\mathscr{E}$ is an orthonormal set and $h\in\mathscr{H}$,
    then $\sum\{\inner{h}{e}e:e\in\mathscr{E}\}$ converges in $\mathscr{H}$.
\end{lemma}

Now we can determine $Ph$ even if $\mathscr{M}$ is infinite or possibly uncountable.
\begin{corollary}{}{}
    Let $\mathscr{E}$ be an orthonormal subset of $\mathscr{H}$ and let $\mathscr{M}=\vee \mathscr{E}$. 
    If $P$ is the orthogonal projection of $\mathscr{H}$ onto $\mathscr{M}$,
    show that $Ph=\sum\{\inner{h}{e}e:e\in\mathscr{E}\}$ for every $h$ in $\mathscr{H}$.
\end{corollary}


In particular, if $\vee \mathscr{E}=\hH$, then every $h\in \hH$ 
may be expanded in terms of elements of $\mathscr{E}$. 
The following theorem gives equivalent conditions for 
this property of $\vee \mathscr{E}$.

\begin{theorem}{}{orthonormal basis equivalent}
    If $\mathscr{E}$ is an orthonormal subset of $\hH$, then the following conditions are equivalent:\\
    (1) If $h\in\mathscr{H}$ and $h\perp \mathscr{E}$, then $h=0$.($\mathscr{E}$ is a maximal orthonormal set).\\
    (2) $\vee \mathscr{E}=\hH$. \\
    (3) If $h\in\mathscr{H}$, then $h=\sum \{\inner{h}{e}e:e\in\mathscr{E}\}$.\\
    (4) If $g,h\in\mathscr{H}$, then
        $\inner{g}{h}=\sum \{\inner{g}{e}\inner{e}{h}:e\in\mathscr{E}\}$.\\
    (5) If $h\in\mathscr{H}$, then $||h||^2=\sum \{|\inner{h}{e}|^2:e\in\mathscr{E}\}$ (Parseval's Identity)
\end{theorem}
\begin{proofsolution}
    (1)$\Rightarrow$ (2): Since $\{u_n\}$ is maximal, $span(\{u_n\})^{\perp} = \{0\}$. 
    Let $\mathscr{M} = \overline{span(\{u_n\})}$, then $\mathscr{M}^{\perp}=\{0\}$. Since $\mathscr{M}^{\perp}\bigoplus \mathscr{M} = \hH$ , $\mathscr{M} = \overline{span(\{u_n\})}=\hH$.
    \\
    (2)$\Rightarrow$ (3): 
\end{proofsolution}

% \begin{proposition}{}{}
%     Let $V$ be an orthogonal subset of $\hH$. Then
%     \begin{align*}
%         \sum\limits_{v} \{v:v\in V\}
%     \end{align*}
%     converges iff 
%     \begin{align*}
%         \sum\limits_{v} ||v||^2<\infty.
%     \end{align*}
% \end{proposition}
% \begin{proof}
%     Let $s_k$ be the sequence of partial sums of the given series. By the Pythagorean theorem, 
%     \begin{align*}
%         ||s_i-s_j||^2 = ||\sum\limits_{n=i+1}^{j}v_n||^2 = \sum\limits_{n=i+1}^{j}||v_n||^2.
%     \end{align*}
%     for all $i\leqs j$. Then
%     \begin{align*}
%         & \sum\limits_{n=1}^{\infty} v_n\ converges\\
%         & \Leftrightarrow \{s_k\} \ is\  a\  \ Cauchy\  sequence\\
%         & \Leftrightarrow \sum\limits_{n=i+1}^{j}||v_n||^2\rightarrow 0,\ i,j\rightarrow \infty\\
%         & \Leftrightarrow \sum\limits_{n=1}^{\infty} ||v_n||^2<\infty.
%     \end{align*}
% \end{proof}

% \begin{corollary}{}{$ell^2$-linear combinations converges}
%     Let $\{u_n\}$ be an orthonormal sequence of vectors in a Hilbert space, and let {an}
%     be a sequence of real(complex) numbers. Then the series 
%     \begin{align*}
%         \sum\limits_{n=1}^{\infty} a_nu_n.
%     \end{align*}
%     converges iff $\{a_n\}$ lies in $\ell^2$.
% \end{corollary}

% \begin{proof}
%     Let $s_k$ be the sequence of partial sums of the given series. By the Pythagorean theorem, 
%     \begin{align*}
%         ||s_i-s_j||^2 = ||\sum\limits_{n=i+1}^{j}||a_nu_n||^2 = \sum\limits_{n=i+1}^{j}|a_n|^2||u_n||^2= \sum\limits_{n=i+1}^{j}|a_n|^2.
%     \end{align*}
%     for all $i\leqs j$. Then
%     \begin{align*}
%         & \sum\limits_{n=1}^{\infty} a_nu_n\ converges\\
%         & \Leftrightarrow \sum\limits_{n=1}^{\infty} |a_n|^2<\infty.
%     \end{align*}
% \end{proof}

% In general, if $\{a_n\}$ is an $\ell^2$ sequence set, then the sum
% \begin{align*}
%     \sum\limits_{n=1}^{\infty} a_nu_n
% \end{align*}
% is called a $ell^2$-linear combintion of the vectors $\{u_n\}$. By corollary\ref{cor:$ell^2$-linear combinations converges}, 
% every $\ell^2$-linear combination orthonormal vectors in a Hilbert space converges.

% \begin{definition}{Fourier Coefficients}{}
%     If $\{v_n\}$ is an orthogonal set of $\hH$ and $x\in \hH$. 
%     Then $<u_n,x>$ is called the Fourier coefficients of $x$ 
%     with respect to the orthonormal set $\{v_n\}$.
% \end{definition}

% When an orthonormal set is finite in $\hH$, we have

% \begin{theorem}{}{fourier coefficients theorem}
%     Let $\{u_1,u_2,...,u_k\}$ is an orthonormal set in $\hH$ and $x=\sum\limits_{i=1}^{k}a_iu_i$, then $a_i=<u_i,x>$ ($1\leqs j\leqs k$) and
%     $||x||^2=\sum\limits_{i=1}^{k}|a_i|^2$. 
% \end{theorem}

% \begin{proofsolution}
%     $<u_i,x>=\sum\limits_{j=1}^{k}<u_i,u_j>=a_i<u_i,u_i>=a_i$. Then use Pythagorean Theorem, we get
%     \begin{align*}
%         ||x||^2=||\sum\limits_{j=1}^{k} a_iu_i||^2=\sum\limits_{j=1}^{k}||a_iu_i||^2=\sum\limits_{j=1}^{k}|a_i|^2.
%     \end{align*}
% \end{proofsolution}




% Now we consider the following question: How can we determine $Px_0$ when $x_0$ and the subspace
% $\mathscr{M}$ are given? When $\mathscr{M}$ is finite, we have

% \begin{proposition}{}{finite projection representaion}
%     Let $\mathscr{M}=\overline{span(\{u_1,...,u_k\})}$. Then $\forall x\in \hH$, the vector
%     \begin{align*}
%         y = \sum\limits_{i=1}^{k} <u_i,x> u_i
%     \end{align*}
%     is the projection of $x$ onto $\mathscr{M}$. 
% \end{proposition}

% \begin{proof}
%     Observe that $<u_i,y>=<u_i,x>$ for each $i$, and hence $<u_i,x-y> = 0$ for each $i$. Hence, $x-y\perp \mathscr{M}$. 
%     By theorem\ref{thm:orthogonal distance} and definition\ref{def:projection}, $y$ is the projection of $x$ onto $\mathscr{M}$. 
% \end{proof}

% Now we generalize this proposition when $\mathscr{M}$ is infinite.

% \begin{lemma}{}{}
%     Let $\{u_n\}$ be an orthonormal set in $\hH$ and $x\in \hH$. Then 
%     \begin{align*}
%         \sum\limits_{n=1}^{\infty} <u_n,x>^2\leqs ||x||^2, 
%     \end{align*} 
%     which implies $\{<u_n,x>\}$ is a $\ell^2$ sequence.\\
% \end{lemma}

% \begin{proof}
%     Let $N\in\N$, then by proposition\ref{prop:finite projection representaion}, 
%     \begin{align*}
%         y_N = \sum\limits_{n=1}^{N}<u_n,x>u_n
%     \end{align*}
%     be the projection fo $x$ onto $span(\{u_1,...,u_N\})$. Then $x-y_N\perp y_N$, so by the Pythagorean theorem
%     \begin{align*}
%         ||x||^2 = ||y_N+x-y_N||^2 = ||y_N||^2 + ||x-y_N||^2 \geqs ||y_N||^2= \sum\limits_{n=1}^{N} <u_n,x>^2.
%     \end{align*}
%     This holds for all $N \in \N$, so the desired inequality follows. 
% \end{proof}

% \begin{theorem}{}{}
%     Let $\{u_n\}$ be an orthonormal set in $\hH$ and $x\in\hH$. $\mathscr{M}$ is $\overline{span(\{u_n\})}$. Then
%     \begin{align*}
%         y=\sum\limits_{n=1}^{\infty} <u_n,x>u_n
%     \end{align*}
%     is the projection of $x$ onto the $\mathscr{M}$.
% \end{theorem}

% \begin{proofsolution}
%     Since $\{u_n,x\}$ is a $ell^2$ sequence, and thus the sum for $y$ converges. Then,
%     \begin{align*}
%         <u_i,
%     \end{align*}
% \end{proofsolution}




\begin{definition}{}{}
    An orthonormal set $\mathscr{E}$ in $\hH$ satisfying any of the equivalent conditions (1)-(5) in theorem\ref{thm:orthonormal basis equivalent} 
    is called a complete orthonormal set(or a complete orthonormal system) or an orthonormal basis in $\hH$. 
\end{definition}

\begin{remark}
    If $\hH$ is infinite dimensional, an orthonormal basis is not a basis in the usual
    definition of a basis for a vector space (i.e., each $h\in\hH$ has a unique representation as a
    finite linear combination of basis elements). Such a basis in this context is called a Hamel
    basis.
\end{remark}

The following theorem shows that orthonormal basis always exists in Hilbert space.
\begin{theorem}{}{}
    Every Hilbert space $\hH$ has an orthonormal basis. If $\mathscr{E}$ is an orthonormal set, then $\mathscr{E}$ has an orthonormal basis containing $\mathscr{E}$. 
\end{theorem}

\begin{proofsolution}
    referring to \href{https://www.math.ucdavis.edu/~hunter/book/ch6.pdf}{lecture notes from ucdavis}
\end{proofsolution}

Just as in finite dimensional spaces, a basis in Hilbert space can be used 
to define a concept of dimension. For this purpose the next result is pivotal.

\begin{proposition}{}{}
    If $\mathscr{H}$ is a Hilbert space, any two bases have the same cardinality.
\end{proposition}

\begin{definition}{}{}
    The dimension of a Hilbert space is the cardinality of a basis 
and is denoted by dim$\mathscr{H}$.
\end{definition}

\begin{proposition}{}{}
    If $\mathscr{H}$ is an infinite dimensional Hilbert space, 
    then $\mathscr{H}$ is separable if and only if $\mathscr{H}$ has a countable basis.
\end{proposition}

\begin{exercise}{I4 T13}{}
    Let $\mathscr{E}$ be an orthonormal subset of $\mathscr{H}$ and let $\mathscr{M}=\vee \mathscr{E}$. 
    If $P$ is the orthogonal projection of $\mathscr{H}$ onto $\mathscr{M}$,
    show that $Ph=\sum\{\inner{h}{e}e:e\in\mathscr{E}\}$ for every $h$ in $\mathscr{H}$.
\end{exercise}
\begin{proof}
    Let $Qh=\sum\{\inner{h}{e}e:e\in\mathscr{E}\}$.
    it suffices to show that $Qh$ is the unique element in $\mathscr{M}$ such that $h-Qh\perp \mathscr{M}$.
    By Lemma 4.12, there are vectors $e_1,e_2,...$ in $\mathscr{E}$ such that $\{e\in\mathscr{E}:\inner{h}{e}\neq 0\}={e_1,e_2,...}$
    and $Qh=\sum\limits_{n=1}^{\infty}\inner{h}{e_n}e_n$.
    By Bessel's Inequality, $\sum\limits_{n=1}^{\infty}\inner{h}{e_n}^2\leqs ||h||^2<\infty$.
    and thus $\{\inner{h}{e_n}\}$ is $l^2$ and the sum for $Qh$ converges. Since $\inner{Qh}{e_n}=\inner{h}{e_n}$ for all $n$, 
    it follows that $\inner{h-Qh}{e_n}=0$ for all $n$. Then by the continuity of inner product, 
    $\inner{s}{h-Qh}= 0$ for any $s$ in $l^2$-span of $\{e_n\}$. Hence, $h-Qh\perp \mathscr{M}$.
    Since $Qh\in \mathscr{M}$, it follows that $Qh$ is the unique element in $\mathscr{M}$
    such that $h-Qh\perp \mathscr{M}$. Hence, by the property of $Ph$, $Ph=Qh$ for every $h$ in $\mathscr{H}$.
\end{proof}


\begin{exercise}{I4 T19}{}
    If $\{h\in \mathscr{H}: ||h||\leqs 1\}$ is compact, show that dim$\mathscr{H}<\infty$.
\end{exercise}

\begin{proof}
    Suppose that $\mathscr{H}$ is not finite dimensional.
    We want to show that $\overline{B(0;1)}$ is not compact. 
    It suffices to show that $\overline{B(0,1)}$ is not sequentially compact.
    To do this, we construct a sequence in $\overline{B(0,1)}$ which have no convergent subsequence.
    We will uset the following fact usually known as Riesz's Lemma:
    Let $\mathscr{M}$ be a closed subspace of a Banach space $\mathscr{X}$. Given any $r\in (0,1)$, 
    there exists an $x\in\mathscr{X}$ such that $||x||=1$ and $d(x,\mathscr{M})\geqs r$.
    \\
    Pick $x_1\in\mathscr{X}$ such that $||x_1||=1$. Let $\mathscr{M}_1=\vee \{x_1\}$. Then $\mathscr{M}_1$ is closed.
    Then accroding to Riesz's Lemma, there exists $x_2\in \mathscr{H}$ such that $||x_2||=1$ and $d(x_2,\mathscr{M}_1)\geqs \frac{1}{2}$.
    Now consider the subspace $\mathscr{M}_2=\vee\{x_1, x_2\}$. Since $\mathscr{H}$ is infinite dimensional,
    $\mathscr{M}_2$ is a proper closed subspace of $\mathscr{H}$, and we can apply the Riesz's Lemma to find
    an $x_3 \in \mathscr{H}$ such that $||x_3||=1$ and $d(x_3,\mathscr{M}_2)\geqs \frac{1}{2}$.
    If we continue to proceed this way, we will have a sequence $(x_n)$ and a sequence of
    closed subspaces $(\mathscr{M}_n)$ such that for all $n \in \N$: $||x_n||=1$ and $d(x_{n+1},\mathscr{M}_n)\geqs \frac{1}{2}$.
    It is clear that the sequence $(x_n)$ is in $\overline{B(0,1)}$, 
    and for $m>n$ we have $x_m\in \mathscr{M}_m\subset \mathscr{M}_{n-1}$ and $||x_n-x_m||\geqs d(x_n,\mathscr{M}_{n-1})\geqs \frac{1}{2}$.
    Therefore, no subsequence of $(x_n)$ can form a Cauchy sequence. 
    Thus, $\overline{B(0, 1)}$ is not compact.
\end{proof}


\section{Reference}
\begin{itemize}
    \item \href{https://web.mat.bham.ac.uk/~malevao/MSM3P21/l14.pdf}{lecture notes from brmh}
    \item \href{https://ocw.mit.edu/courses/18-102-introduction-to-functional-analysis-spring-2021/resources/mit18_102s21_lec14/}{lecture notes from mit}
    \item \href{https://www.math.ucdavis.edu/%7Ehunter/book/ch6.pdf}{lecture notes from ucdavis}
    \item \href{https://sites.math.washington.edu/~burke/crs/555/555_notes/hilbert.pdf}{lecture notes from washington}
    \item \href{https://e.math.cornell.edu/people/belk/measuretheory/HilbertSpaces.pdf}{lecture notes from cornell}
    \item \href{https://www.math.cuhk.edu.hk/course_builder/1415/math5011/functional%20Analysis%202014.pdf}{lecture notes from cuhk}
    \item \href{https://personal.math.ubc.ca/~malabika/teaching/ubc/spring18/math421-510/HW3-Solution.pdf}{I4 T13 P5}
    \item \href{https://thichchaytron.files.wordpress.com/2013/10/functional-problems-anhle-full-www-mathvn-com.pdf}{I4 T19: P17}
\end{itemize}