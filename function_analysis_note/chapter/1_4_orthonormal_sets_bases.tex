\chapter{Orthonormal Sets of Vectors and Bases}\label{chp:1_4}



It will be shown in this chapter that, as in Euclidean space, each Hilbert 
space can be coordinatized. The vehicle for introducing the coordinates is 
an orthonormal basis. The corresponding vectors in $\F^d$ are the vectors 
$\{e_1, e_2, ... , e_d\}$, where $e_k$ is the $d$-tuple having a $1$ in the $k$th place and zeros elsewhere.

\begin{definition}{}{}
    A subset $A\subset \hH$ is said to orthogonal if $<x,y>=0$, for all $x,y\in A$ such that $x\neq y$. 
    We say it is orthonormal if we further regive $||x||=1$ for all $x\in A$. 
\end{definition}

\begin{remark}
    Orthogonal set can contain zero vector, but orthonormal set can not. 
\end{remark}



Subsets $\mathscr{E}, V$ mentioned below can be finite, countably infinite and uncoutably infinite. 

\begin{proposition}{}{}
    Let $V$ be an orthogonal subset of $\hH$. Then
    \begin{align*}
        \sum\limits_{v} \{v:v\in V\}
    \end{align*}
    converges iff 
    \begin{align*}
        \sum\limits_{v} ||v||^2<\infty.
    \end{align*}
\end{proposition}
\begin{proof}
    Let $s_k$ be the sequence of partial sums of the given series. By the Pythagorean theorem, 
    \begin{align*}
        ||s_i-s_j||^2 = ||\sum\limits_{n=i+1}^{j}v_n||^2 = \sum\limits_{n=i+1}^{j}||v_n||^2.
    \end{align*}
    for all $i\leqs j$. Then
    \begin{align*}
        & \sum\limits_{n=1}^{\infty} v_n\ converges\\
        & \Leftrightarrow \{s_k\} \ is\  a\  \ Cauchy\  sequence\\
        & \Leftrightarrow \sum\limits_{n=i+1}^{j}||v_n||^2\rightarrow 0,\ i,j\rightarrow \infty\\
        & \Leftrightarrow \sum\limits_{n=1}^{\infty} ||v_n||^2<\infty.
    \end{align*}
\end{proof}

\begin{corollary}{}{$ell^2$-linear combinations converges}
    Let $\{u_n\}$ be an orthonormal sequence of vectors in a Hilbert space, and let {an}
    be a sequence of real(complex) numbers. Then the series 
    \begin{align*}
        \sum\limits_{n=1}^{\infty} a_nu_n.
    \end{align*}
    converges iff $\{a_n\}$ lies in $\ell^2$.
\end{corollary}

\begin{proof}
    Let $s_k$ be the sequence of partial sums of the given series. By the Pythagorean theorem, 
    \begin{align*}
        ||s_i-s_j||^2 = ||\sum\limits_{n=i+1}^{j}||a_nu_n||^2 = \sum\limits_{n=i+1}^{j}|a_n|^2||u_n||^2= \sum\limits_{n=i+1}^{j}|a_n|^2.
    \end{align*}
    for all $i\leqs j$. Then
    \begin{align*}
        & \sum\limits_{n=1}^{\infty} a_nu_n\ converges\\
        & \Leftrightarrow \sum\limits_{n=1}^{\infty} |a_n|^2<\infty.
    \end{align*}
\end{proof}

In general, if $\{a_n\}$ is an $\ell^2$ sequence set, then the sum
\begin{align*}
    \sum\limits_{n=1}^{\infty} a_nu_n
\end{align*}
is called a $ell^2$-linear combintion of the vectors $\{u_n\}$. By corollary\ref{cor:$ell^2$-linear combinations converges}, 
every $\ell^2$-linear combination orthonormal vectors in a Hilbert space converges.

\begin{definition}{Fourier Coefficients}{}
    If $\{v_n\}$ is an orthogonal set of $\hH$ and $x\in \hH$. 
    Then $<u_n,x>$ is called the Fourier coefficients of $x$ 
    with respect to the orthonormal set $\{v_n\}$.
\end{definition}

When an orthonormal set is finite in $\hH$, we have

\begin{theorem}{}{fourier coefficients theorem}
    Let $\{u_1,u_2,...,u_k\}$ is an orthonormal set in $\hH$ and $x=\sum\limits_{i=1}^{k}a_iu_i$, then $a_i=<u_i,x>$ ($1\leqs j\leqs k$) and
    $||x||^2=\sum\limits_{i=1}^{k}|a_i|^2$. 
\end{theorem}

\begin{proofsolution}
    $<u_i,x>=\sum\limits_{j=1}^{k}<u_i,u_j>=a_i<u_i,u_i>=a_i$. Then use Pythagorean Theorem, we get
    \begin{align*}
        ||x||^2=||\sum\limits_{j=1}^{k} a_iu_i||^2=\sum\limits_{j=1}^{k}||a_iu_i||^2=\sum\limits_{j=1}^{k}|a_i|^2.
    \end{align*}
\end{proofsolution}

\begin{definition}{}{}
    $\{x_1,..., x_k\}$ is said to be linearly dependent, if there exist scalars $a_1,a_2,...,a_k$, not all zero, such that
    \begin{align*}
        a_1x_1+...+a_kx_k=0
    \end{align*}
    where $0$ denotes the zero vector.
    An infinite set of vectors is linearly independent if every nonempty finite subset is linearly independent.
\end{definition}


\begin{corollary}{}{}
    Every orthonormal set is linearly independent.
\end{corollary}
\begin{proof}
    If the set is finite. Suppose $\{u_1,...,u_k\}$ is an orthonormal set. Assume $b_1u_1+...b_ku_k=0$, then $b_1u_1=-\sum\limits_{i=2}^{k}b_iu_i$. 
    By theorem\ref{thm:fourier coefficients theorem}, $b_i=<u_i,u_1>=0$($2\leqs i\leqs k$). Then $b_1u_1=0$.
    Since $u_1\neq 0$, $b_1=0$. Hence, $b_i=0$ for all $1\leqs i\leqs k$ and so $\{u_1,...,u_k\}$ is linearly independent.
    If the set is infinite, the above proof can show that every nonempty finite subset is linearly independent and so the set is linearly independent.
\end{proof}


Now we consider the question: Does the orthonormal set always exist? The following theorem tell us we can always construct an orthonormal set.

\begin{theorem}{The Gram-Schmidt Orthogonalization Process}{}
    If $\{x_1,...,x_k\}$ is a linearly independent subset of $\hH$, 
    then there is an orthonormal set $\{u_1,...,u_k\}$ such that $span(\{u_1,...,u_k\})= span(\{x_1,...,x_k\})$.
\end{theorem}
\begin{proofsolution}
    Define $\{u_k\}$ inductivey. Start with $u_1=\frac{x_1}{||x_1||}$. 
    Suppose for $k-1$, $u_1,...,u_{k-1}$ exist. Let $v_k = x_k-\sum\limits_{j=1}^{k-1}<u_j,x_k>u_j$ and $u_k = \frac{v_k}{||v_k||}$.
    Then $<u_k,u_j>=\frac{1}{||v_k||}(<x_k,u_j>-<u_j,x_k><u_j,u_j>)=0$ and $||u_k||=1$.
    Hence, we can $\{u_1,...,u_k\}$ is an orthonormal set. 
    from the construct process of $u_j(1\leqs j \leqs k)$, we can know $u_j\in span(\{x_1,...,x_k\})$ and so $span(\{u_1,...,u_k\})\subset span(\{x_1,...,x_k\})$. 
    Since $\{u_1,...,u_k\}$ is linearly independent, $dim(span(\{u_1,...,u_k\}))=n$. Hence, $span(\{u_1,...,u_k\})= span(\{x_1,...,x_k\})$.
\end{proofsolution}


Now we consider the following question: How can we determine $Px_0$ when $x_0$ and the subspace
$\mathcal{M}$ are given? When $\mathcal{M}$ is finite, we have

\begin{proposition}{}{finite projection representaion}
    Let $\mathcal{M}=\overline{span(\{u_1,...,u_k\})}$. Then $\forall x\in \hH$, the vector
    \begin{align*}
        y = \sum\limits_{i=1}^{k} <u_i,x> u_i
    \end{align*}
    is the projection of $x$ onto $\mathcal{M}$. 
\end{proposition}

\begin{proof}
    Observe that $<u_i,y>=<u_i,x>$ for each $i$, and hence $<u_i,x-y> = 0$ for each $i$. Hence, $x-y\perp \mathcal{M}$. 
    By theorem\ref{thm:orthogonal distance} and definition\ref{def:projection}, $y$ is the projection of $x$ onto $\mathcal{M}$. 
\end{proof}

Now we generalize this proposition when $\mathcal{M}$ is infinite.

\begin{lemma}{}{}
    Let $\{u_n\}$ be an orthonormal set in $\hH$ and $x\in \hH$. Then 
    \begin{align*}
        \sum\limits_{n=1}^{\infty} <u_n,x>^2\leqs ||x||^2, 
    \end{align*} 
    which implies $\{<u_n,x>\}$ is a $\ell^2$ sequence.\\
\end{lemma}

\begin{proof}
    Let $N\in\N$, then by proposition\ref{prop:finite projection representaion}, 
    \begin{align*}
        y_N = \sum\limits_{n=1}^{N}<u_n,x>u_n
    \end{align*}
    be the projection fo $x$ onto $span(\{u_1,...,u_N\})$. Then $x-y_N\perp y_N$, so by the Pythagorean theorem
    \begin{align*}
        ||x||^2 = ||y_N+x-y_N||^2 = ||y_N||^2 + ||x-y_N||^2 \geqs ||y_N||^2= \sum\limits_{n=1}^{N} <u_n,x>^2.
    \end{align*}
    This holds for all $N \in \N$, so the desired inequality follows. 
\end{proof}

\begin{theorem}{}{}
    Let $\{u_n\}$ be an orthonormal set in $\hH$ and $x\in\hH$. $\mathcal{M}$ is $\overline{span(\{u_n\})}$. Then
    \begin{align*}
        y=\sum\limits_{n=1}^{\infty} <u_n,x>u_n
    \end{align*}
    is the projection of $x$ onto the $\mathcal{M}$.
\end{theorem}

\begin{proofsolution}
    Since $\{u_n,x\}$ is a $ell^2$ sequence, and thus the sum for $y$ converges. Then,
    \begin{align*}
        <u_i,
    \end{align*}
\end{proofsolution}

In particular, if $\overline{span(\{u_n\})}=\hH$, then every $x\in \hH$ may be expanded in terms of elements of $\{u_n\}$. The following theorem gives equivalent conditions for this property of $\{u_n\}$.

\begin{theorem}{}{orthonormal basis equivalent}
    If $\{u_n\}$ is an orthonormal subset of $\hH$, then the following conditions are equivalent:\\
    (1) $\{u_n\}$ is a maximal orthonormal set.\\
    (2) $\overline{span(\{u_n\})}=\hH$;\\
    (3) $||x||=\sum\limits_{n=1}^{\infty}|<u_n,x>|^2$ for all $x\in\hH$ (Parsevars Identity);\\
    (4) $x=\sum\limits_{n=1}^{\infty}<u_n,x>u_n$ for all $x\in \hH$;\\
    (5) $<u_n,x>=0$ for all $n$ implies $x=0$;
\end{theorem}
\begin{proofsolution}
    (1)$\Rightarrow$ (2): Since $\{u_n\}$ is maximal, $span(\{u_n\})^{\perp} = \{0\}$. 
    Let $\mathcal{M} = \overline{span(\{u_n\})}$, then $\mathcal{M}^{\perp}=\{0\}$. Since $\mathcal{M}^{\perp}\bigoplus \mathcal{M} = \hH$ , $\mathcal{M} = \overline{span(\{u_n\})}=\hH$.
    \\
    (2)$\Rightarrow$ (3): 
\end{proofsolution}


\begin{definition}{}{}
    An orthonormal set $\{u_n\}$ in $\hH$ satisfying any of the equivalent conditions (1)-(5) in theorem\ref{thm:orthonormal basis equivalent} 
    is called a complete orthonormal set(or a complete orthonormal system) or an orthonormal basis in $\hH$. 
\end{definition}

\begin{remark}
    If $\hH$ is infinite dimensional, an orthonormal basis is not a basis in the usual
    definition of a basis for a vector space (i.e., each $x\in\hH$ has a unique representation as a
    finite linear combination of basis elements). Such a basis in this context is called a Hamel
    basis.
\end{remark}

\begin{theorem}{}{}
    Every Hilbert space $\hH$ has an orthonormal basis. If $\{u_n\}$ is an orthonormal set, then $\hH$ has an orthonormal basis containing $\{u_n\}$. 
\end{theorem}

\begin{proofsolution}
    referring to \href{https://www.math.ucdavis.edu/~hunter/book/ch6.pdf}{lecture notes from ucdavis}
\end{proofsolution}


\section{Reference}
\begin{itemize}
    \item \href{https://web.mat.bham.ac.uk/~malevao/MSM3P21/l14.pdf}{lecture notes from brmh}
    \item \href{https://ocw.mit.edu/courses/18-102-introduction-to-functional-analysis-spring-2021/resources/mit18_102s21_lec14/}{lecture notes from mit}
    \item \href{https://www.math.ucdavis.edu/%7Ehunter/book/ch6.pdf}{lecture notes from ucdavis}
    \item \href{https://sites.math.washington.edu/~burke/crs/555/555_notes/hilbert.pdf}{lecture notes from washington}
    \item \href{https://e.math.cornell.edu/people/belk/measuretheory/HilbertSpaces.pdf}{lecture notes from cornell}
    \item \href{https://www.math.cuhk.edu.hk/course_builder/1415/math5011/functional%20Analysis%202014.pdf}{lecture notes from cuhk}
\end{itemize}