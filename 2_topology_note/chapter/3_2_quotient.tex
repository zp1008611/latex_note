\chapter{Quotient Space and Quotient Mapping}\label{chp:3_2}


We have seen how to create new topological spaces from given topological spaces
using the operation of forming a subspace and the operation of forming a product of a set of topological spaces.
In this chapter we introduce a third operation, namely that of forming a quotient space(of a topological space).
As examples we shall see the Klein bottle and M\"obius strip.

\section{Quotient spaces}

\begin{definition}{}{}
    Let ($X$,$\tau$) be a topological space and let $X^*$ be a partition of $X$ into
    disjoint subsets whose union is $X$. Let $p:X$ → $X^*$ be the surjective (onto) map
    that carries each point of $X$ to the element of $X^*$
    containing it. $p$ is called the
    projection map from $X$ to $X^*$
    . In the topology on $X^*$
    induced by $p$ ($\tau^*=\{U\subset X^*: p^{-1}(U)\in\tau\}$), the space $X^*$ under this topology is the quotient space of $X$.
\end{definition}

Note. Recall that we have a partition of a set if and only if we have an equivalence
relation on the set. So the approach in terms
of partitions can be replaced with an approach based on equivalence relations.
The idea of the quotient space is that points of the subsets in the partition (or
the equivalent points under the equivalence relation) are “identified” with each
other. For this reason, quotient spaces are sometimes called “identifying spaces” or
“decomposition spaces.” You will notice the parallel between quotient spaces and
quotient groups in which all elements of a coset are identified.

Then we can get the definition of quotient space by the equivalence relation.

\begin{definition}{}{}
    Let ($X$,$\tau$) be a topological space and and $\sim$ any equivalence relation on $X$.
    Let $X/\sim$ be the set of all equivalence classes of $\sim$ and $p:X$ → $X/\sim$ given by $x\mapsto [x]$. 
    In the topology on $X/\sim$
    induced by $p$ ($\tilde{\tau}=\{U\subset X/\sim: p^{-1}(U)\in\tau\}$), the space $X/\sim$ under this topology is the quotient space of $X$.
\end{definition}

\begin{definition}{}{}
    Let $(X,\tau)$ and $(Y,\tau_1)$ be topological spaces.
    Then $(Y,\tau_1)$ is said to be a quotient space of $(X,\tau)$ if there exists a mapping $f:(X,\tau)\rightarrow (Y,\tau_1)$ satisfying the following properties: \\
    (1) $f$ is surjective.\\
    (2) For each subset $U$ of $Y$, $U\in \tau_1\Leftrightarrow f^{-1}(U)\in \tau$.
    And $f$ is said to be a quotient mapping.
\end{definition}

\begin{remark}
    From the definition, it is clear that every quotient mapping is continous.
\end{remark}

\begin{remark}
    Property (2) is equivalent to: For each subset $A$ of $Y$, 
    $A$ is closed in $(Y,\tau_1)\Leftrightarrow f^{-1}(A)$ is closed in $(X,\tau)$.
\end{remark}

\begin{proposition}{}{}
    If $f:(X,\tau)\rightarrow (Y,\tau_1)$ is a surjective, continous and open mapping, then $f$ is a quotient mapping.
    If $f$ is a surjective, continous and closed mapping, then it is a quotient mapping.
\end{proposition}
\begin{proof}
    If $f$ is continous and surjective, then (1) and necessary condition of (2) hold.
    If $f$ is open, then if $f^{-1}(U)\in \tau$ ,then $U=f(f^{-1}(U))\tau$. 
\end{proof}

\begin{proposition}{}{}
    $f:(X,\tau)\rightarrow (Y,\tau_1)$ is injective. Then $f$ is a homeomorphism iff it is a quotient mapping.
\end{proposition}
\begin{proof}
    ($\Rightarrow$): $f$ is homeomorphism, then $f$ is continous and surjective.
    And for $U\in Y$, if $f^{-1}(U)\in \tau$, then $U=f(f^{-1}(U))\in \tau_1$.
    \\
    ($\Leftarrow$): $f$ is quotient mapping then $f$ is continous and surjective. Then $f$ has an inverse $f^{-1}$.
    For $K\in \tau$, $K=f^{-1}(f(K))\in\tau$ as $f$ is bijective. Then, $f(K)\in \tau_1$. Hence, $f^{-1}$ is continous.
\end{proof}

\begin{proposition}{}{}
    Let $X$ be compact and $Y$ be Hausdorff space. If $f:X\rightarrow Y$ is continous and surjective, then
    $f$ is a quotient mapping.
\end{proposition}

\begin{proof}
    Let $A$ be a closed set in $X$. Since $X$ is compact, it follows that $A$ is compact. 
    Then $f(A)$ is compact as $f$ is continous. Since $Y$ is Hausdorff, it follows that $f(A)$ is closed.
    Then $f$ is continous, surjective and closed mapping. Hence, $f$ is a quotient mapping. 
\end{proof}

\begin{proposition}{Universality}{}
    Let $f:X\rightarrow X'$ be a quotient mapping and 
    $g:X'\rightarrow Y$ be a map. Then $g$ is continous iff $g\circ f$ is continous.
\end{proposition}

\begin{proof}
    ($\Rightarrow$): $f$ is a quotient mapping, then $f$ is continous. Then $g\circ f$ is continous as $g$ is continous.
    \\
    ($\Leftarrow$): For open set $U$ in $Y$, then $(gf)^{-1}(U)=f^{-1}(g^{-1}(U))$ is open in $X$.
    Then $g^{-1}(U)$ is open in $X'$ as $f$ is a quotient mapping. Hence, $g$ is continous. 
\end{proof}

Let $f:X\rightarrow Y$ be a mapping, we define a equivalence relation $\overset{f}{\sim}$:
$\forall x,x'\in X, x\sim x'\Leftrightarrow f(x)=f(x')$.

\begin{proposition}{}{}
    If $f:X\rightarrow Y$ is a quotient mapping, then $X/\overset{f}{\sim}\cong Y$.
\end{proposition}




\section{Exercise}

\begin{exercise}{P86 T1}{}
    Let $f:X\rightarrow Y$ and $g:Y\rightarrow Z$ be continuous mapping such that $g\circ f$ is a quotient mapping. 
    Then $g$ is quotient mapping.
\end{exercise}

\begin{proof}
    Since $g\circ f$ is a quotient mapping, it follows that $g\circ f$ is surjective and 
    for $U\subset Z$, $U$ is open in $Z$ $\Leftrightarrow$ $f^{-1}(g^{-1}(U)$) is open in $X$.
    Then for any $z\in Z$, $\exists x\in X$ such that $z=g(f(x))$. Let $y=f(x)$, then $y\in Y$ and $z=g(y)$.
    Hence, $g$ is surjective. If $U$ is open in $Z$, then $g^{-1}(U)$ is open in $Y$ as $g$ is continous.
    If $g^{-1}(U)$ is open in $Y$, then $f^{-1}(g^{-1}(U))$ is open in $X$ as $f$ is continous, then $U$ is open in $Z$.
    Hence, $g$ is a quotient mapping.
\end{proof}

\begin{exercise}{P86 T2}{}
    Let $f:X\rightarrow Y$ be a quotient mapping , $B$ be open(or closed) set of $Y$
    and $A=f^{-1}(B)$. Then $f_A:A\rightarrow B$ is a quotient mapping.
\end{exercise}

\begin{proof}
    Since $f$ is a quotient mapping, it follows that $f$ is surjective and for $U\subset Y$, $U$ is open in $Y$ $\Leftrightarrow$ $f^{-1}(U)$ is open in $X$.
    Then $A=f^{-1}(B)$ is open in $X$ as $B$ is open in $Y$ and $f$ is continous.
    Since $f$ is surjective, it follows that $f_A(A)=f(A)=f(f^{-1})(B)=B$. Then $f_A$ is surjective.
    For $K\subset B$, if $K$ is open in $B$, then $K=O_Y\cap B$ ($O_Y$ is open in $Y$) is open in $Y$, then $f_A^{-1}(K)=f^{-1}(K)\subset A$ is open in $X$ , then $f_A^{-1}(K)=f_A^{-1}(K)\cap A$ and so open in $A$.
    If $f_A^{-1}(K)$ is open in $A$, then $f^{-1}(K)=f_A^{-1}(K)=O_X\cap A$ ($O_X$ is open in $X$) is open in $X$, then $K\subset B$ is open in $Y$, then $K=K\cap B$ and so open in $B$. 
    Hence, $f_A$ is a quotient mapping.
\end{proof}

\section{Reference}

\begin{itemize}
    \item \href{http://staff.ustc.edu.cn/~wangzuoq/Courses/21S-Topology/Notes/Lec06.pdf}{THE QUOTIENT TOPOLOGY}
    \item \href{}{topology without tears ch11}
\end{itemize}

