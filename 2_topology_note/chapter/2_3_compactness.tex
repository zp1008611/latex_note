\chapter{Compactness}\label{chp:compactness}

\section{Definitions of various compactness}


\begin{definition}{}{}
    Let $(X,\tau)$ be a topological space, and $A\subset X$ be a subset.\\
    (1) A family of subsets $\mathscr{U}=\{U_{\alpha}\}$ is called a covering of $A$ if $A\subset \cup_{\alpha} U_{\alpha}$.\\
    (2) A covering $\mathscr{U}$ is called a finite covering if it is a finite collection.\\
    (3) A covering $\mathscr{U}$ is called an open covering if each $U_{\alpha}$ is open.\\
    (4) A covering $\mathscr{V}$ is a sub-covering of $\mathscr{U}$ if $\mathscr{V}\subset \mathscr{U}$ and $A\subset \cup_{v\in\mathscr{V}}V$
\end{definition}

\begin{definition}{}{}
    Let $(X,\tau)$ be a topological space. \\
    (1) We say $X$ is compact in $X$ if any open covering $\mathscr{U}=\{U_{\alpha}\}$ of $X$ admits a finite sub-covering, 
    i.e. there exists $\{U_{\alpha_1},U_{\alpha_2},...,U_{\alpha_k}\}\subset \mathcal{U}$ s.t. $X=\cup_{i=1}^{k}U_{\alpha_i}$.\\
    (2) We say $X$ is sequentially compact if any sequence $x_1,x_2,...\in X$ admits a
    convergent subsequence $x_{n_1},x_{n_2},...\rightarrow x_0\in X$.
\end{definition}

\begin{remark}
    Suppose $A\subset X$ be a subset, then we say $A$ is compact/sequentially compact if, 
    when endowed with the subspace topology, $(A,\tau_A)$ is compact/sequentially compact. 
\end{remark}

\begin{proposition}{}{subset compact condition}
    Let $(X,\tau)$ be a topological space, then
    \begin{align*}
        &A\subset X \text{ is compact }\Leftrightarrow \\
        &\text{ for any family of open sets $\mathcal{U}=\{U_{\alpha}\}$ in $X$ satisfying $A\subset \cup_{\alpha}U_{\alpha}$}, \\
        & \text{ one can find } U_{\alpha_1},...,U_{\alpha_k}\in \mathcal{U} \text{ s.t. } A\subset \cup_{j=1}^k U_{\alpha_j}.
    \end{align*}
    In a word, $A\subset X$ is compact $\Leftrightarrow$ any open covering of $A$ in X admits finite sub-covering. 
\end{proposition}

\begin{proof}
    For any family of open sets $\mathcal{U}=\{U_{\alpha}\}$ in $X$ satisfying $A\subset \cup_{\alpha}U_{\alpha}$, 
    then $A=A\cap A\subset A\cap (\cup_{\alpha}U_{\alpha})= \cup_{\alpha} (A\cap U_{\alpha})$. 
    Then $\mathscr{U}_A=\{U_\alpha\cap A:U_\alpha\in \mathcal{U}\}$ is a open covering of $(A,\tau_A)$.\\
    ($\Rightarrow$): Since $(A,\tau_A)$ is compact, one can find $U_{\alpha_1}, ..., U_{\alpha_k}$ such that $A=\cup_{i=1}^k(U_{\alpha_i}\cap A)$
    and so $A\subset \cup_{i=1}^k U_{\alpha_i}$. \\
    ($\Leftarrow$): Since $A\subset \cup_{j=1}^k U_{\alpha_j}$, it follows that $A=A\cap A\subset A\cap \cup_{j=1}^k U_{\alpha_j}= \cup_{j=1}^k (A\cap U_{\alpha_j})$.
    Also $\cup_{j=1}^k (A\cap U_{\alpha_j})\subset A$ and so $A=\cup_{j=1}^k (A\cap U_{\alpha_j})$.
    Then any open covering of $(A,\tau_A)$ adimits finite open sub-covering and so $A$ is compact.
\end{proof}

\section{Examples of compactness}

\begin{example}{}{}
    Any finite topological space, including the empty set, is compact. 
    More generally, any space with a finite topology (only finitely many open sets) is compact; 
    this includes in particular the trivial topology.
\end{example}

\begin{example}{}{}
    Any space carrying the cofinite topology is compact and sequentially compact.
\end{example}

\begin{proof}
    1
\end{proof}

\begin{example}{}{}
    In the cocountable topology on an uncountable set, no infinite set is compact.
\end{example}

\begin{example}{}{}
    No discrete space with an infinite number of points is compact. 
    The collection of all singletons of the space is an open cover which admits no finite subcover. 
    Finite discrete spaces are compact.
\end{example}

\begin{example}{}{}
    The closed unit interval $[0, 1]$ is compact. 
    This follows from the Heine-Borel theorem.
    The open interval $(0, 1)$ is not compact: 
    the open cover $(\frac{1}{n},1-\frac{1}{n})$ for $n=3,4,...$ does not have a finite subcover. 
\end{example}

\begin{example}{}{}
    The set $\R$ of all real numbers is not compact 
    as there is a cover of open intervals that 
    does not have a finite subcover. 
    For example, intervals $(n-1,n+1)$, where $n$ takes all integer values in $\Z$, cover $\R$ but there is no finite subcover.
\end{example}


\begin{remark}
    (1) We will see later: for topological spaces,
    \begin{itemize}
        \item compact $\centernot\implies$ sequentially compact;
        \item sequentially compact $\centernot\implies$ compact.
    \end{itemize}
    \hspace{1cm} (2) We will prove: for metric spaces, compact $\Leftrightarrow$ sequentially compact 
\end{remark}


\section{Characterization of compactness via closed sets or basis}

\section{Compactness in metric space}
Now, we prove that for metric spaces, compact $\Leftrightarrow$ sequentially compact 

\begin{proposition}{}{}
    Compact $C_1$ space is sequentially compact.  
\end{proposition}

\begin{proof}
    Suppose $X$ is a compact $C_1$ space. 
    We need to show that for any sequence ${x_n}\in X$, 
    there exists convergent subsequence $x_{n_k}\underset{k\rightarrow \infty}{\longrightarrow} x_0\in X$.
    Firstly, we claim that there exists $x_0\in X$ such that any neighborhood of $x$ has infinite elements of $\{x_n\}$. 
    Suppose not, then $\forall x\in X$, there exists a open neighborhood $U_x$ of $x$ such that $U_x$ has finite elements of $\{x_n\}$.
    Then, $\mathscr{U}=\{U_x:x\in X\}$ is open covering of $X$, but $\{x_n\}$ can not be covered by any finite covering of $\mathscr{U}$, 
    contradicting the fact that $X$ is compact.
    Secondly, we will construct subsequence $\{x_{n_k}\}$ of $\{x_n\}$ such that $x_{n_k}\underset{k\rightarrow \infty}{\longrightarrow} x_0$.
    By $X$ is $C_1$ space, we can take a countable neighborhood basis $\{U_n\}$ of $x$ such that $U_m\subset U_n$ when $m>n$.
    Then for any neighborhood $U$ of $x_0$ and $x_i\in \{x_n\}$, $\exists U_n\in \{U_n\}$ such that $x_i\in U_n\subset U$.
    Let $x_{n_i}\in U_i$, then we get a subsequence $\{x_{n_k}\}$ of $\{x_n\}$, and $U$ has infinite elements of $\{x_{n_k}\}$. 
    Then, $x_{n_k}\underset{k\rightarrow \infty}{\longrightarrow} x_0$.
\end{proof}

\begin{remark}
    Metric space is $C_1$ space so in metric space, compact $\Rightarrow$ sequentially compact.
\end{remark}

The proof for the converse is a bit difficult. We need to use a few lemmas.

\begin{lemma}{}{}
    Suppose $K$ is a subset of a metric space $X$ and 
\end{lemma}





\section{Proposition of compactness}
\subsection{Compactness v.s. continuous map}

compactness and sequentially compactness
are preserved under continuous maps:

\begin{proposition}{}{direct image of compact set is compact}
    Let $f:X\rightarrow Y$ be continous. \\
    (1) If $A\subset X$ is compact, then $f(A)$ is compact in $Y$.\\
    (2) If $A\subset X$ is sequentially compact, then $f(A)$ is sequentially compact in $Y$. 
\end{proposition}
\begin{proof}
    (1) Suppose $A$ is compact. Given any open covering $\mathscr{V}=\{V_{\alpha}\}$ of $f(A)$ in $Y$, 
    then $\mathscr{U}=\{f^{-1}(V_{\alpha})\}$ is an open convering of $A$ in $X$ (Since $A\subset f^{-1}(f(A))=f^{-1}(\cup_{\alpha}V_{\alpha})=\cup_{\alpha} f^{-1}(V_{\alpha})$). 
    By compactness of $A$, there exists $\alpha_1,...,\alpha_k$ such that $A\subset U_{i=1}^{k} f^{-1}(V_{\alpha_i})$. 
    It follows that $f(A)\subset f(U_{i=1}^{k} f^{-1}(V_{\alpha_i}))=\cup_{i=1}^{k}f(f^{-1}(V_{\alpha_i}))\subset U_{i=1}^{k}V_{\alpha_i}$, 
    i.e. $f(A)$ is compact. 
\end{proof}

\begin{proposition}{}{R compact closed bounded}
    Let $A$ be a closed subset of $\R$. Then the following statements are equivalent.\\
    (1) $A$ is compact.\\
    (2) $A$ is closed and bounded.
\end{proposition}

\begin{proof}
    referring to \href{https://www.math.cuhk.edu.hk/course_builder/1920/math2060b/compact%20in%20R.pdf}{ Compacts sets in $\R$} Th1.7
\end{proof}


\begin{corollary}{}{}
    Let $f:X\rightarrow \R$ be any continuous map. If $X$ is compact or sequentially compact,
    then $f(X)$ is bounded in $\R$. Moreover, there exists $a,b\in X$ s.t. 
    $f(a)=\inf_{x\in X}f(x)$ and $f(b)=\sup_{x\in X}f(x)$. 
\end{corollary}

\begin{proof}
    By proposition\ref{prop:direct image of compact set is compact}, $f(X)$ is compact. 
    By proposition\ref{prop:R compact closed bounded}, $f(X)$ is closed and bounded. 
    Then $f(X)$ has a least upper bound and a greatest lower bound. By definition of bound, $\inf_{x\in X}f(x)$ and $\sup_{x\in X}f(x)$ 
    are limit points of sequences in $f(X)$. Since $f(X)$ is closed, it follows that $\inf_{x\in X}f(x)$, 
    $sup_{x\in X}f(x)\in f(X)$ and so $a,b\in X$ exist.
\end{proof}


\subsection{Subspace of a compact space}

As usual, we would like to construct new compact spaces from old compact spaces,
or even non-compact spaces. The first candidates one can look at is: subspaces of
a compact space. Unfortunately, it is easy to see that a compact space could have
non-compact subspace, e.g. $(0,1)$ is a subspace of $[0,1]$. 
\par
We can take a closer look at the problem: which subsets of $[0,1]$ remain to be compcat? 
We know that a set in $\R$ is compact iff it is bounded and closed. 
\par
If $A$ is a subset of $[0,1]$, it is automatically bounded. 
So far a subset $A\subset [0,1]$ to be compact, it is enough to require $A$ to be closed.
\par
It turns out that for more general topological spaces, it is also enough to require closedness 
for a subset to be compact. 

\begin{proposition}{}{}
    Let $A\subset X$ be a closed subset.\\
    (1) If $X$ is compact, then $A$ compact.\\
    (2) If $X$ is sequentially compact, then $A$ is a sequentially compact.
\end{proposition}

\begin{proof}
    (1) For any open covering $\mathscr{U}_A$ of $A$ in $X$ (i.e. $A\subset \cup_{U\in\mathscr{U}} U$), 
    $\mathscr{U}\cup \{A^c\}$ is an open covering of $X$, which adimits a finite sub-covering $U_1,...,U_m,A^c$ as $X$ is compact. 
    It follows $A\subset \cup_{i=1}^{m}U_i$, then by proposition\ref{prop:subset compact condition}, 
    $A$ is compact.
\end{proof}

\begin{proposition}{}{}
    Suppose $K\subset A \subset X$. Then $K$ is compact relative to $A$ if and only if 
    it is compact relative to $X$.
\end{proposition}

\begin{proof}
    referring to \href{https://math.iisc.ac.in/~vamsipingali/teaching/um204analysis2017spring/25Jan.pdf}{proof of transitive compactness}
\end{proof}


\subsection{Compact v.s. Hausdorff}
Although it seems that compactness and Hausdorff property are very different, it
turns out that they are “the dual” to each other in the following sense:

\begin{proposition}{}{}
    (1) If $(X,\tau)$ is compact, then\\
    (a) Every closed subset in $X$ is compact.\\
    (b) If $\tau'\subset \tau$, then $(X,\tau')$ is compact.\\
    (2) If $(X,\tau)$ is Hausdorff, then\\
    (a) Every compact subset in $X$ is closed.\\
    (b) If $\tau'\supset \tau$, then $(X,\tau')$ is Hausdorff.
\end{proposition}

\begin{proof}
    (2)(a): Let $A\subset X$ be compact, $x_0\in X\setminus A$. Since $X$ is Hausdorff, it follows that for any $y\in A$ 
    we can find $U_y$ and $V_y$ such that $x_0\in U,y\in V_y$ and $U_y\cap V_y=\O$. Since $A\subset \cup_{y\in A} V_y$, 
    one can find $y_1,...,y_m$ s.t. $A\subset \cup_{i=1}^{m} V_{y_i}$.
    Then $\cup_{i}^{m}U_{y_i}\subset (\cup_{i}^{m} V_{y_i})^c\subset A^c$. 
    Hence, $A^c$ is open and so $A$ is closed.
\end{proof}

\begin{theorem}{}{}
    Let $X$ be compact and $Y$ be Hausdorff space. 
    If $f:X\rightarrow Y$ is continous and bijective, then
    $f$ is homeomorphism.
\end{theorem}

\begin{proof}
    
\end{proof}

\subsection{Compactness “enhances” the separation axioms $T_2$ and $T_3$}

\begin{theorem}{}{}
    Any compact Hausdorff space is $T_4$.
\end{theorem}
This a consequence of the following proposition, which show how compactness "enhance"
the separation axioms (via a simple “local-to-global” argument):

\begin{proposition}{}{}
    For topological spaces, we have\\
    (1) Compact + $T_2$ $\Rightarrow$ $T_3$.\\
    (2) Compact + $T_3$ $\Rightarrow$ $T_4$.
\end{proposition}

\begin{proof}
    (1) Let $x\in X, A\subset X$ be closed(and thus compact as $X$ is Hausdorff), and $x\notin A$.
    Then for any $y\in A$, there exists open sets $U_{x,y},V_y$ such that $x\in U_{x,y}, y\in V_y$ and $U_{x,y}\cap V_y=\O$.
    Since $A\subset \cup_{y\in A}V_y$, by compactness of $A$, one can find $V_{y_1},...,V_{y_n}$ covering $A$. It follows that 
    \begin{align*}
        U:= U_{x,y_1}\cap ...\cap U_{x,y_n}\text{ and } V:=V_{y_1}\cup ...\cup V_{y_n}
    \end{align*} 
    are open neighborhood of $x$ and $A$, and $U\cap V=\O$( $U\cap V=\cap_{i=1}^{n}U_{x,y_i}\cap \cup_{i=1}^{n}V_{y_i}$
    $=\cap_{i=1}^{n}(U_{x,y_i}\cap \cup_{i=1}^{n}V_{y_i})$$=\cap_{i=1}^{n}\cup_{i=1}^{n}(U_{x,y_i}\cap V_{y_i})$).
    \\
    (2) Let $A,B\subset X$ be closed and $A\cap B=\O$, then for any $y\in A$, there exists open sets $U_y$ and $V_y$ such that
    $y\in V_y,B\subset U_y$ and $V_y\cap U_y=\O$. Since $A\subset \cup_{y\in A}V_y$, by compactness of $A$, 
    one can find $V_{y_1},...,V_{y_n}$ covering $A$. It follows that
    \begin{align*}
        U:= U_{y_1}\cap ...\cap U_{y_n} \text{ and } V:=V_{y_1}\cup...\cup V_{y_n}
    \end{align*}
    are open neighborhood of $B$ and $A$, and $U\cap V=\O$.
\end{proof}



\section{Compactness of product space}



\section{Exercise}
\begin{exercise}{P59 T3}{}
    Let $(X,\tau)$ be a topological space. 
    Then finite union of compact subset of $X$ is compact.
\end{exercise}

\begin{proof}
    Suppose $A_1,...,A_n$ are compact subset of $X$. Then for any family of open sets $\mathcal{U}$ in $X$ satisfying $\cup_{i=1}^{n}A_i\subset \cup_{U\in \mathcal{U}} U$,
    then $A_i\subset \cup_{U\in \mathcal{U}} U$. Since $A_i$ is compact, one can find $U_{i}^{\alpha_1},...,U_{i}^{\alpha_{n_i}}\in \mathcal{U}$ such that $A_i\subset \cup_{j=1}^{n_i} U_{i}^{\alpha_j}$.
    Then $\cup_{i=1}^n A_i\subset\cup_{i=1}^{n}(\cup_{j=1}^{n_i}U_i^{\alpha_j})$. Hence, $\cup_{i=1}^{n} A_i$ is compact.
\end{proof}


\begin{exercise}{P59 T5}{}
    If $A$ is an infinite subset of a compact space $X$, then $A$
    has a limit point in $X$.
\end{exercise}


\begin{proof}
    It suffices to show that there exists $x_0\in X$ such that any open neighborhood $U$ of $x_0$ has infinite points of $A$.
    Suppose not, then $\forall x\in X$, there exists an open neighborhood $U_x$ of $x$ such that $U_x\cap A$ is finite. 
    Since $X\subset \cup_{x\in X} U_x$ and $X$ is compact, one can find $U_{x_1},...,U_{x_n}$ such that $X=\cup_{i=1}^{n}U_{x_i}$.
    Then $A=A\cap X=\cup_{i=1}^{n}(A\cap U_{x_i})$ and so $A$ is finite. This is a contradiction as $A$ is finite.
\end{proof}

\begin{exercise}{P59 T12}{}
    Let $X$ be a Hausdorff space and $(A_{\alpha})_{\alpha\in J}$ is a family of compact subsets of $X$.
    Then $\cap_{\alpha\in J}A_{\alpha}$ is compact.
\end{exercise}

\begin{proof}
    Since $X$ is Hausdorff, it follows that $A_{\alpha}$($\forall \alpha\in J$) is closed in $X$. 
    Then $K=\cap_{\alpha\in J}A_{\alpha}$ is closed in $X$. For all $\alpha\in J$, since $K\subset A_{\alpha} \subset X$, 
    it follows that $K$ is closed in $A_{\alpha}$. Since $A_{\alpha}$ is compact, one can get $K$ is compact relative to $A_{\alpha}$.
    Then $K$ is compact relative to $X$.
\end{proof}

\begin{exercise}{P59 T13}{}
    Let $X$ be a $T_3$ space , $A$ be a compact subset of $X$
    and $U$ be a neighborhood of $A$. 
    Then there exists a neighborhood $V$ of $A$ such that $\overline{V}\subset U$.
\end{exercise}

\begin{proof}
    Since $U$ is a neighborhood of $A$, it follows that $\exists$ open set $K$ such that $A\subset K\subset U$. 
    Then $A\cap K^c=\O$. Since $X$ is $T_3$, $\forall a\in A$, 
    one can find open sets $U_a,K_a$ such that $a\in U_a, K^c\subset K_a$ and $U_a\cap K_a=\O$. 
    Then $A\subset \cup_{a\in A}U_a$. Since $A$ is compact in $X$, one can find $U_{a_1},...,U_{a_n}$ such that $A\subset \cup_{i=1}^{n} U_{a_i}$.
    Let $V=\cup_{i=1}^{n}U_{a_i}$. 
    Since $U_{a}\subset K_a^c$, it follows that $V\subset \cup_{i=1}^{n} K_{a_i}^c\subset (\cap_{i=1}^{n} K_{a_i})^c$.
    Let $O=\cap_{i=1}^{n} K_{a_i}$. Then $V\subset O^c$ and $O$ is open. So $V\subset \overline{V}\subset O^c$.
    Since $K^c\subset O$, it follows that $A\subset V\subset \overline{V}\subset O^c\subset K\subset U$.
\end{proof}



\section{Reference}

\begin{itemize}
    \item \href{http://staff.ustc.edu.cn/~wangzuoq/Courses/21S-Topology/Notes/Lec08.pdf}{Compactness: various definitions and examples}
    \item \href{https://people.clas.ufl.edu/mjury/files/sequential_compactness_notes.pdf}{Sequentially compact metric spaces}
    \item \href{https://www.umsl.edu/~siegelj/SetTheoryandTopology/Compact2.html}{The Lebesgue Number of a Covering}
    \item \href{http://staff.ustc.edu.cn/~wangzuoq/Courses/21S-Topology/Notes/Lec10.pdf}{COMPACTNESS OF PRODUCT SPACE}
\end{itemize}