\chapter{Hadamard, Kronecker and Khatri-Rao Products}

\section{Partitioned matrices}

Let $\{\alpha_{m_1},...,\alpha_{m_R}\}$
and $\{\beta_{n_1},...,\beta_{n_S}\}$
be partitions of the sets $\{1,...,m\}$
and $\{1,...,n\}$, respectively, with 
$m_r\in \langle m \rangle$ and $n_s\in\langle n\rangle$,
such that $\sum\limits_{r=1}^{R}m_r=m$ and $\sum\limits_{s=1}^{S}n_s=n$.
It is said that matrices $\bold{A}_{rs}$ of dimensions $(m_r,n_s)$
form a partition of the matrix $\bold{A}\in \K^{m\times n}$ into $(R,S)$ blocks,
or that $\bold{A}$ is partitioned into $(R,S)$ blocks, if $\bold{A}$ can be written as:
\begin{align*}
    \bold{A}=\begin{pmatrix}
        \bold{A}_{11}&\bold{A}_{12}&...&\bold{A}_{1S}\\
        \bold{A}_{21}&\bold{A}_{22}&...&\bold{A}_{2S}\\
        ...&...&...&...\\
        \bold{A}_{R1}&\bold{A}_{R2}&...&\bold{A}_{RS}\\
    \end{pmatrix}=(\bold{A}_{rs}), r\in\langle R\rangle, s\in\langle S\rangle.
\end{align*}
All submatrices of the same row-block $(r)$ contain the same number $(m_r)$ of rows.
Similarly, all submatrices of the same column-block $(s)$ contain the same number $(n_s)$ of columns,
that is:
\begin{align*}
    \begin{pmatrix}
        \bold{A}_{r1}&\bold{A}_{r2}&...&\bold{A}_{rS}
    \end{pmatrix}\in \K^{m_r\times n},
    \begin{pmatrix}
        \bold{A}_{1s}&\bold{A}_{2s}&...&\bold{A}_{Rs}
    \end{pmatrix}\in \K^{m\times n_s}.
\end{align*}
It is then said that the submatrices $\bold{A}_{rs}$ are of compatible dimensions.
\par
In the particular case where $n=1$,
the partitioned matrix becomes a block-column vector:
\begin{align*}
    \bold{a}=\begin{pmatrix}
        \bold{a}_1&\bold{a}_2&...&\bold{a}_{R}
    \end{pmatrix}\in \K^{m\times 1},
    \bold{a}_r\in \K^{m_r\times 1},r\in \langle R\rangle.
\end{align*}
Similarly, when $m=1$, the partitioned matrix becomes a block-row vector:
\begin{align*}
    \bold{a}^{T}=\begin{pmatrix}
        \bold{a}_1^T&\bold{a}_2^T&...&\bold{a}_S^T
    \end{pmatrix}\in \K^{1\times n}, \bold{a}_s\in\K^{n_s\times 1},s\in\langle S\rangle.
\end{align*}

\section{Notation}
We will write $\bold{A}^*,\bold{A}^T,\bold{A}^H,\bold{A}^{\dagger},\bold{A}_{i\cdot},\bold{A}_{\cdot j},r(\bold{A})$ 
and $\text{det}(\bold{A})$, for the conjugate, the transpose,
the transconjugate (also known as conjugate transpose or Hermitian transpose),
the Moore-Penrose pseudo-inverse, the $i$th row, the $j$th column,
the rank and the determinant of $\bold{A}\in\K^{I\times J}$, respectively.
In the following literature, the Hadamard product is denoted by $*$.
The symbols $\otimes$ and $\odot$ are used for the Kronecker and Khatri-Rao product, respectively.
\par
The symbol $\bold{1}_I$ denotes a column vector of size $I$
whose elements are all equal to $1$. 
The elements of the matrices $\bold{0}_{I\times J}$ and
$\bold{1}_{I\times J}$ of size $(I\times J)$ are all equal to $0$
and $1$, respectively. The symbols $\bold{I}_N$ and $\bold{e}_n^{(N)}$
denote the identity matrix of order $N$ and the $n$th vector of 
the canonical basis of the vector space $\R^N$, respectively.


\section{Hadamard product}

\subsection{Definition and identities}
\begin{definition}{}{}
    Let $\bold{A}$ and $\bold{B}\in \K^{I\times J}$ be two matrices of the same size.
    The Hadamard product of $\bold{A}$ and $\bold{B}$ is the
    matrix $\bold{C}\in\K^{I\times J}$ defined as follows:
    \begin{align*}
        \bold{C}=\bold{A}*\bold{B}=\begin{pmatrix}
            a_{11}b_{11}& a_{12}b_{12}& ... & a_{1J}b_{1J}\\
            a_{21}b_{21}& a_{22}b_{22}& ... & a_{2J}b_{2J}\\
            ...& ... & ... & ...\\
            a_{I1}b_{I1}& a_{I2}b_{I2} & ...& a_{IJ}b_{IJ}
        \end{pmatrix},
    \end{align*}
    i.e. $c_{ij}=a_{ij}b_{ij}$, and therefore $\bold{C}=(a_{ij}b_{ij})$, with $i\in\langle I\rangle$, $j\in\langle J\rangle$.
\end{definition}

\section{Kronecker product}

\subsection{Kronecker product of vectors}
\subsubsection{Definition}
\begin{definition}{}{}
    (a) For $\bold{u}\in \K^I$ and $\bold{v}\in \K^J$, we have:
    \begin{align*}
        \bold{x}=\bold{u}\otimes \bold{v}\in \K^{IJ}\Leftrightarrow x_{j+(i-1)J}=u_iv_j
    \end{align*}
    or equivalently:
    \begin{align*}
        \bold{u}\otimes \bold{v}=\begin{pmatrix}
            u_1v_1& ... & u_1v_J& u_2v_1&...& u_2v_J&...& u_Iv_1&...& u_Iv_J
        \end{pmatrix}^T=\begin{pmatrix}
            u_1\bold{v}&...&u_I\bold{v}
        \end{pmatrix}.
    \end{align*}
    (b) Similarly, for $\bold{u}\in \K^{I}, \bold{v}\in \K^J$, and $\bold{w}\in\K^K$, we have:
    \begin{align*}
        \bold{x}=\bold{u}\otimes \bold{v}\otimes \bold{w}\in \K^{IJK}\Leftrightarrow x_{k+(j-1)K+(i-1)JK}=u_iv_jw_k.
    \end{align*}
\end{definition}
\begin{remark}
    By convention, the order of the dimensions in a product $IJK$ follows
    the order of variation of the corresponding indices $(i,j,k)$. For example,
    $\K^{IJK}$ means that the index $i$ varies more slowly than $j$, which itself varies more slowly than $k$.
\end{remark}

\subsection{Kronecker product of matrices}

\subsubsection{Definitions and identities}
\begin{definition}{}{}
    Given two matrices $\bold{A}\in\K^{I\times J}$ and $\bold{B}\in \K^{M\times N}$
    of arbitrary size, the right Kronecker product of $\bold{A}$ by $\bold{B}$ is the matrix $\bold{C}\in \K^{IM\times JN}$
    defined as follows:
    \begin{align*}
        \bold{C}=\bold{A}\otimes \bold{B}=\begin{pmatrix}
            a_{11}\bold{B} & a_{12}\bold{B} & ...& a_{1J}\bold{B}\\
            a_{21} \bold{B} & a_{22}\bold{B} & ... & a_{2J}\bold{B}\\
            ...&...&...&...\\
            a_{I1}\bold{B}&a_{I2}\bold{B}&...&a_{IJ}\bold{B}
        \end{pmatrix}=(a_{ij}\bold{B}).
    \end{align*}
\end{definition}
\begin{remark}
    The Kronecker product is a matrix partitioned into $(I,J)$ blocks, where the block $(i,j)$ is given by the matrix $a_{ij}\bold{B}\in \K^{M\times N}$.
    The element $a_{ij}b_{mn}$ is located at the position $((i-1)M+m,(j-1)N+n)$ in $\bold{A}\otimes \bold{B}$.
\end{remark}

% \begin{example}{}{}
%     For $\bold{A}=\begin{pmatrix}
%         a_{11}&a_{12}\\
%         a_{21}&a_{22}
%     \end{pmatrix}$,
%     $\bold{B}=\begin{pmatrix}
%         b_{11}&b_{12}\\
%         b_{21}&b_{22}
%     \end{pmatrix}$, we have:
%     \begin{align*}
%         \bold{A}\otimes \bold{B}=\begin{pmatrix}
            
%         \end{pmatrix}
%     \end{align*}
% \end{example}


\section{Reference}
\begin{itemize}
    \item \href{}{From Algebraic Structures to Tensors ch5}
    \item \href{}{Matrix and Tensor Decompositions in Signal Processing ch2}
    \item \href{https://www.ime.unicamp.br/~cnaber/Kronecker.pdf}{Kronecker Product and the vec Operator}
    \item \href{https://www.refaad.com/Files/GLM/GLM-4-1-3.pdf}{On Hadamard and Kronecker Products Over Matrix of Matrices}
    \item \href{https://citeseerx.ist.psu.edu/document?repid=rep1&type=pdf&doi=f087fb761b13f83f14c2409c0f7a0b072812458d}{HADAMARD, KHATRI-RAO, KRONECKER AND OTHER MATRIX PRODUCTS}
\end{itemize}