% \documentclass[11pt,twoside]{book} %纸质版用twoside
\documentclass[12pt,oneside]{book} %电子版用oneside
\usepackage{setspace}

% \documentclass{article}

%%%%%%%%%%%%%%%%%%%%%%%%%%%%%%%%%%%%%%%%%%%%%%%%%%
%%%%%%%%%%%%%%%%%%%%% preamble %%%%%%%%%%%%%%%%%%%
%%%%%%%%%%%%%%%%%%%%%%%%%%%%%%%%%%%%%%%%%%%%%%%%%%

\usepackage[mono=false]{libertine} % new linux font, ignore mono

\usepackage{luatex85}

%\renewcommand{\baselinestretch}{1.05}
\usepackage{amsmath,amsthm,amssymb,mathrsfs,amsfonts,dsfont}
\usepackage{epsfig,graphicx}
\usepackage{tabularx}
\usepackage{blkarray}
\usepackage{slashed}
\usepackage{color}
\usepackage{listings}
\usepackage{caption}
% \usepackage{fullpage}
\usepackage{lipsum} % provides dummy text for testing
\usepackage[toc,title,titletoc,header]{appendix}
\usepackage{minitoc}
\usepackage{color}
\usepackage{multicol} % two-col ToC
\usepackage{bm}
\usepackage{imakeidx} % before hyperref
\usepackage{hyperref}
\usepackage{indentfirst}
\setlength{\parindent}{2em}


% link colors settings
\hypersetup{
    colorlinks=true,
    citecolor=magenta,
    linkcolor=blue,
    filecolor=green,      
    urlcolor=cyan,
    % hypertexnames=false,
}
\usepackage[capitalise]{cleveref}
\usepackage{subcaption}
\usepackage{enumitem}
\usepackage{mathtools}
\usepackage{physics}
\usepackage[linesnumbered,ruled,vlined,algosection]{algorithm2e}
\SetCommentSty{textsf}
\usepackage{epigraph}
\epigraphwidth=1.0\linewidth
\epigraphrule=0pt

% adjust margin
\usepackage[margin=2.3cm]{geometry}
\headheight13.6pt


\usepackage{graphicx}
\usepackage[justification=centering]{caption} % 图注居中
\usepackage{setspace}
\usepackage{geometry}
\usepackage{float}
\usepackage{hyperref}
\usepackage[utf8]{inputenc}
\usepackage[english]{babel}
\usepackage{framed}


\newcommand{\HRule}[1]{\rule{\linewidth}{#1}}





\setstretch{1.2}
% \geometry{
%     textheight=9in,
%     textwidth=5.5in,
%     top=1in,
%     headheight=12pt,
%     headsep=25pt,
%     footskip=30pt
% }





%%%%%%%%%%%%%%%% thmtools %%%%%%%%%%%%%%%%%%%%%

\usepackage{thmtools}
\usepackage[dvipsnames]{xcolor}
\usepackage[most]{tcolorbox}
\usepackage{enumerate}

\colorlet{LightGreen}{Green!15} %def
\colorlet{LightBlue}{Blue!15} %thm
\colorlet{LightOrange}{Orange!15} %lem
\colorlet{LightGray}{Gray!15}  %prop
\colorlet{LightRed}{Red!40} %cor
\colorlet{LightYellow}{Yellow!15} %exa


% \newtcbtheorem[
%   number within = chapter % 按每个 chapter 分别编号
% ]{definition% 环境名
% }{Definition% 这个参数可以设成“定理”“引理”“推论”等,编号就会变成“定理 1.1”“引理 1.1”“推论 1.1”等
% }{
%   attach title to upper = \par\vspace{1ex}, % 不要单独的标题栏,定理名完了之后分段,加上适量空白
%   separator sign = \quad, % 定理编号和定理名字之间用什么分隔;默认是冒号
%   sharp corners, % 直角;默认是圆角
%   enhanced jigsaw, frame hidden, % 隐藏 tcb 边框
%   colback = LightGreen, % 背景色
%   coltitle = blue!20!cyan!80!black, % 标题(定理编号和名字)的颜色
%   fonttitle = \sffamily\small, % 标题(定理编号和名字)的字体
%   description font = \normalsize, % 定理名字的字体
%   fontupper = \normalfont, % box 内的字体
% }{def% label 前缀
% }

\newtcbtheorem[
  auto counter,number within = chapter % 按每个 chapter 分别编号
]{definition% 环境名
}{Definition% 这个参数可以设成“定理”“引理”“推论”等,编号就会变成“定理 1.1”“引理 1.1”“推论 1.1”等
}{
  sharp corners, % 直角;默认是圆角
  colback=Green!5,
  colframe=Green!50!black,
  fonttitle=\sffamily\small
}{def% label 前缀
}


% 计数器设置
\makeatletter
\renewcommand\theHtcb@cnt@definition{\thechapter.\arabic{tcb@cnt@definition}}
\makeatother

\newtcbtheorem[
  auto counter,number within = chapter % 按每个 chapter 分别编号
]{theorem% 环境名
}{Theorem% 这个参数可以设成“定理”“引理”“推论”等,编号就会变成“定理 1.1”“引理 1.1”“推论 1.1”等
}{
  sharp corners, % 直角;默认是圆角
  colback=yellow!10,
  colframe=yellow!50!black,
  fonttitle=\sffamily\small
}{thm% label 前缀
}
% 计数器设置
\makeatletter
\renewcommand\theHtcb@cnt@theorem{\thechapter.\arabic{tcb@cnt@theorem}}
\makeatother

\newtcbtheorem[
  auto counter,number within = chapter % 按每个 chapter 分别编号
]{proposition% 环境名
}{Proposition% 这个参数可以设成“定理”“引理”“推论”等,编号就会变成“定理 1.1”“引理 1.1”“推论 1.1”等
}{
  sharp corners, % 直角;默认是圆角
  colback=Red!5,
  colframe=Red!50!black,
  fonttitle=\sffamily\small
}{prop% label 前缀
}
% 计数器设置
\makeatletter
\renewcommand\theHtcb@cnt@proposition{\thechapter.\arabic{tcb@cnt@proposition}}
\makeatother

\newtcbtheorem[
  auto counter,number within = chapter % 按每个 chapter 分别编号
]{corollary% 环境名
}{Corollary% 这个参数可以设成“定理”“引理”“推论”等,编号就会变成“定理 1.1”“引理 1.1”“推论 1.1”等
}{
  sharp corners, % 直角;默认是圆角
  colback=Blue!5,
  colframe=Blue!50!black,
  fonttitle=\sffamily\small
}{cor% label 前缀
}

% 计数器设置
\makeatletter
\renewcommand\theHtcb@cnt@corollary{\thechapter.\arabic{tcb@cnt@corollary}}
\makeatother

\newtcbtheorem[
  auto counter,number within = chapter % 按每个 chapter 分别编号
]{lemma% 环境名
}{Lemma% 这个参数可以设成“定理”“引理”“推论”等,编号就会变成“定理 1.1”“引理 1.1”“推论 1.1”等
}{
  sharp corners, % 直角;默认是圆角
  colback=Gray!10,
  colframe=Gray!50!black,
  fonttitle=\sffamily\small
}{lem% label 前缀
}

% 计数器设置
\makeatletter
\renewcommand\theHtcb@cnt@lemma{\thechapter.\arabic{tcb@cnt@lemma}}
\makeatother


\newtcbtheorem[
  auto counter,number within = chapter % 按每个 chapter 分别编号
]{example}
{Example}%
  {
    enhanced, breakable,
    colback = white, colframe = purple, colbacktitle = purple,
    attach boxed title to top left = {yshift = -2mm, xshift = 5mm},
    boxed title style = {sharp corners},
    fonttitle=\sffamily\small
  }
{exa}

% 计数器设置
\makeatletter
\renewcommand\theHtcb@cnt@example{\thechapter.\arabic{tcb@cnt@example}}
\makeatother


\newtcbtheorem[
  auto counter,number within = chapter % 按每个 chapter 分别编号
]{exercise}
{Exercise}%
  {
    enhanced, breakable,
    colback = white, colframe = cyan, colbacktitle = cyan,
    attach boxed title to top left = {yshift = -2mm, xshift = 5mm},
    boxed title style = {sharp corners},
    fonttitle=\sffamily\small
  }
{exer}

% 计数器设置
\makeatletter
\renewcommand\theHtcb@cnt@exercise{\thechapter.\arabic{tcb@cnt@exercise}}
\makeatother


% \declaretheorem[numberwithin=chapter,shaded={rulecolor=LightGreen,
% rulewidth=2pt,bgcolor=LightGreen,
% textwidth=12em}]{definition}

\usepackage{changepage}
\newenvironment{remark}{\underline{\textbf{Remark.}}}{\par}

\newenvironment{proofsolution}
    {\renewcommand\qedsymbol{$\square$}\color{blue}\begin{adjustwidth}{0em}{2em}\begin{proof}[\textit Proof.~]}
    {\end{proof}\end{adjustwidth}}


%%%%%%%%%%%%%%%% index %%%%%%%%%%%%%%%%%%%%%
\begin{filecontents}{index.ist}
% https://tex.stackexchange.com/questions/65247/index-with-an-initial-letter-of-the-group
headings_flag 1
heading_prefix "{\\centering\\large \\textbf{"
heading_suffix "}}\\nopagebreak\n"
delim_0 "\\nobreak\\dotfill"
\end{filecontents}
\newcommand{\myindex}[1]{\index{#1} \emph{#1}}
\makeindex[columns=3, intoc, title=Alphabetical Index, options= -s index.ist]
%%%%%%%%%%%%%%%% index %%%%%%%%%%%%%%%%%%%%%

%%%%%%%%%%%%%%%% ToC %%%%%%%%%%%%%%%%%%%%%
% Link Chapter title to ToC: https://tex.stackexchange.com/questions/32495/linking-the-section-text-to-the-toc
\usepackage[explicit]{titlesec}
\titleformat{\chapter}[display]
  {\normalfont\huge\bfseries}{\chaptertitlename\ {\thechapter}}{20pt}{\hyperlink{chap-\thechapter}{\Huge#1}
\addtocontents{toc}{\protect\hypertarget{chap-\thechapter}{}}}
\titleformat{name=\chapter,numberless}
  {\normalfont\huge\bfseries}{}{-20pt}{\Huge#1}

%%%%%%%%%%%%%%%%%%% fancyhdr %%%%%%%%%%%%%%%%%
\usepackage{fancyhdr}
\pagestyle{fancy} % enable fancy page style
\renewcommand{\headrulewidth}{0.0pt} % comment if you want the rule
\fancyhf{} % clear header and footer
\fancyhead[lo,le]{\leftmark}
\fancyhead[re,ro]{\rightmark}
\fancyfoot[CE,CO]{\hyperref[toc-contents]{\thepage}}

% https://tex.stackexchange.com/questions/550520/making-each-page-number-link-back-to-beginning-of-chapter-or-section
\makeatletter
\def\chaptermark#1{\markboth{\protect\hyper@linkstart{link}{\@currentHref}{Chapter \thechapter ~ #1}\protect\hyper@linkend}{}}
\def\sectionmark#1{\markright{\protect\hyper@linkstart{link}{\@currentHref}{\thesection ~ #1}\protect\hyper@linkend}}
\makeatother
%%%%%%%%%%%%%%%%%%% fancyhdr %%%%%%%%%%%%%%%%%


%%%%%%%%%%%%%%%%%%% biblatex %%%%%%%%%%%%%%%%%
\usepackage[doi=false,url=false,isbn=false,style=alphabetic,backend=biber,backref=true]{biblatex}
\addbibresource{bib.bib}

\newbibmacro{string+doiurlisbn}[1]{%
  \iffieldundef{doi}{%
    \iffieldundef{url}{%
      \iffieldundef{isbn}{%
        \iffieldundef{issn}{%
          #1%
        }{%
          \href{http://books.google.com/books?vid=ISSN\thefield{issn}}{#1}%
        }%
      }{%
        \href{http://books.google.com/books?vid=ISBN\thefield{isbn}}{#1}%
      }%
    }{%
      \href{\thefield{url}}{#1}%
    }%
  }{%
    \href{http://dx.doi.org/\thefield{doi}}{#1}%
  }%
}

% https://tex.stackexchange.com/questions/94089/remove-quotes-from-inbook-reference-title-with-biblatex
\DeclareFieldFormat[article,incollection,inproceedings,book,misc]{title}{\usebibmacro{string+doiurlisbn}{\mkbibemph{#1}}}
% https://tex.stackexchange.com/questions/454672/biblatex-journal-name-non-italic
\DeclareFieldFormat{journaltitle}{#1\isdot}
\DeclareFieldFormat{booktitle}{#1\isdot}
% https://tex.stackexchange.com/questions/10682/suppress-in-biblatex
\renewbibmacro{in:}{}
% add video field: https://tex.stackexchange.com/questions/111846/biblatex-2-custom-fields-only-one-is-working
\DeclareSourcemap{
    \maps[datatype=bibtex]{
      \map{
        \step[fieldsource=video]
        \step[fieldset=usera,origfieldval]
    }
  }
}
\DeclareFieldFormat{usera}{\href{#1}{\textsc{Online video}}}
\AtEveryBibitem{
    \csappto{blx@bbx@\thefield{entrytype}}{% put at end of entry
        \iffieldundef{usera}{}{\space \printfield{usera}}
    }
}


%%%%%%%%%%%%%%%%%%%%%%%notations%%%%%%%%%%%%%%%%%%%%%%%%%%%%%%
\newcommand{\F}{\ensuremath{\mathbb{F}}}
\newcommand{\C}{\ensuremath{\mathbb{C}}} 
\newcommand{\R}{\ensuremath{\mathbb{R}}}
\newcommand{\J}{\ensuremath{\mathbb{J}}}
\newcommand{\Q}{\ensuremath{\mathbb{Q}}}
\newcommand{\Z}{\ensuremath{\mathbb{Z}}}
\newcommand{\N}{\ensuremath{\mathbb{N}}}
\newcommand{\K}{\ensuremath{\mathbb{K}}}
\newcommand{\Zo}{\ensuremath{\mathbb{Z}_{\geqslant 0}}} % 非负整数集
\newcommand{\Zi}{\ensuremath{\mathbb{Z}_{\geqslant 1}}} % 正整数集
\newcommand{\id}{\mathrm{id}}
\newcommand{\im}{\mathrm{im}\,}                         % 映射的像
\newcommand{\leqs}{\leqslant}
\newcommand{\geqs}{\geqslant}
\newcommand{\ci}{\mathrm{i}}
\newcommand{\hH}{\mathscr{H}}
\newcommand{\hK}{\mathscr{K}}
\newcommand{\inner}[2]{\langle#1,#2\rangle}

%%%%%%%%%%%%%%%%%%% biblatex %%%%%%%%%%%%%%%%%

%%%%%%%%%%%%%%%%%%%%% glossaries %%%%%%%%%%%%%%%%%
% !TEX root = ./notes_template.tex
% \usepackage[style=super]{glossaries}
% https://www.overleaf.com/learn/latex/Glossaries
\usepackage[style=super,toc,acronym]{glossaries}
\setlength{\glsdescwidth}{1\linewidth}
\makeglossaries

\renewcommand\glossaryname{List of Abbreviations and Symbols}

\newglossaryentry{Q2}{name={$Q_2(f)$},
%sort=Q2,
description={Two-side (bounded) error quantum query complexity}}

\newglossaryentry{real_number}{name={$\mathbb{R}$},description={Real number}}

% \newglossaryentry{gcd}{name={gcd},description={greatest common divisor}}

\newacronym{gcd}{GCD}{Greatest Common Divisor}


\newglossaryentry{svm}{name={SVM},description={Support Vector Machine}}

\newglossaryentry{gd}{name={GD},description={Gradient Descent}}

\newglossaryentry{qft}{name={QFT},description={Quantum Field Theory}}

\newglossaryentry{qm}{name={QM},description={Quantum Mechanics}}

\newglossaryentry{v}{name={$\vec{v}$},description={a vector}}

% physics
\newglossaryentry{hamiltonian}{name={$\hat{H}$},description={Hamiltonian}}

\newglossaryentry{lagrangian}{name={$L$},description={Lagrangian}}
%%%%%%%%%%%%%%%%%%%%% glossaries %%%%%%%%%%%%%%%%%

%%%%%%%%%%%%%%%%%%%%% glossaries-extra %%%%%%%%%%%%%%%%%
% \usepackage[record,abbreviations,symbols,stylemods={list,tree,mcols}]{glossaries-extra}
%%%%%%%%%%%%%%%%%%%%% glossaries-extra %%%%%%%%%%%%%%%%%


% !TEX root = ./notes_template.tex

%%%%%%%%%%%%%%%%%%%%%%%%%%%%%%%%%%%%
%%%%%%%%%%%%%%%%%%%%%%%%%%%%%%%%%%%%
% math
\let\iff\relax
\newcommand{\iff}{\text{ iff }}
\newcommand{\OPT}{\textup{OPT}}

% physics
\newcommand{\acreation}{a^\dagger}



%%%%%%%%%%%%%%%%%%%%%%%%%%%%%%%%%%%%%%%%%%%%%%%%%%
%%%%%%%%%%%%%%%% begin of document %%%%%%%%%%%%%%%
%%%%%%%%%%%%%%%%%%%%%%%%%%%%%%%%%%%%%%%%%%%%%%%%%%

\begin{document}

\title{\bf \huge Study Notes of Matrix and Tensor}
% \title{\bf \huge Homework of Functional Analysis}
\author{Pei Zhong}
% \date{Update on \today}

\maketitle

% \newpage
% \let\cleardoublepag\clearpage

\tableofcontents

\begin{spacing}{1}

%%%%%%%%%%%%%%update progress%%%%%%%%%%



%%%%%%%%%%%%%%update progress end%%%%%%%%




%%%%%%%%%%%%%%%preface%%%%%%%%%%%%%
\chapter*{Preface}

Notes mainly refer to following materials:


\begin{itemize}
    \item[*] Machine learning
    \begin{itemize}
        \item \href{https://www.cs.cornell.edu/courses/cs4780/2023sp/}{lecture notes from cornell}
        \item \href{https://www.cs.cmu.edu/~hn1/documents/machine-learning/notes.pdf}{lecture notes from cmu}
        \item \href{https://cs229.stanford.edu/main_notes.pdf}{lecture notes of CS229}
    \end{itemize}
    \item[*] Deep learning
    \begin{itemize}
        \item \href{https://udlbook.github.io/udlbook/}{understanding deep learning}
        \item \href{https://www.bilibili.com/video/BV1Wv411h7kN/?spm_id_from=333.337.search-card.all.click}{lecture video from Hongyi Lee}
        \item \href{https://cs231n.github.io/}{lecture notes from Stanford}
    \end{itemize}
    \item[*] Reinforcement learning
    \begin{itemize}
        \item \href{https://web.stanford.edu/class/cs234/modules.html}{lecture notes from stanford}
        \item \href{https://people.cs.umass.edu/~bsilva/courses/CMPSCI_687/Fall2022/Lecture_Notes_v1.0_687_F22.pdf}{lecture notes from umass}
    \end{itemize}
\end{itemize}







%%%%%%%%%%%%%preface end%%%%%%%%%%%%%



\chapter{Matrix Algebra}

\section{Notations and definitions}

Scalars, column vectors, matrices, and hypermatrices/tensors of order higher than two will be 
denoted by lowercase letters $(a,b,...)$, 
bold lowercase letters $(\bold{a,b},...)$,
bold uppercase letters $(\bold{A},\bold{B},...)$,
and calligraphic letters $(\mathcal{A},\mathcal{B},...)$ respectively.

\par
A matrix $\bold{A}$ of dimensions $I\times J$, with $I$ and $J\in \N^*$,
denoted by $\bold{A}(I,J)$, is an array of $IJ$ elements stored in $I$ rows and 
$J$ columns; the elements belong to a field $\K$. Its $i$th row and
$j$th column, denoted by $A_{i\cdot}$ and $A_{\cdot j}$, respectively, are called $i$th row vector and
$j$th column vector. The element located at the intersection of $A_{i\cdot}$ and $A_{\cdot j}$ is 
designated by $a_{ij}$. We will use the notation $\bold{A}=(a_{ij})$, 
with $a_{ij}\in\K$, $i\in \langle {I} \rangle=\{1,2,...,I\}$ and 
$j\in \langle{J} \rangle=\{1,2,...,J\}$.
\par
A matrix $A\in\K^{I\times J}$ is written in the form:
\begin{align*}
    \begin{pmatrix}
        a_{11}& a_{12} & ... & a_{1J} \\
        a_{12}& a_{22} & ... & a_{2J} \\
        ...& ...  & ... & ...\\
        a_{I1}& a_{I2}  & ... & a_{IJ}
      \end{pmatrix}
\end{align*} 
The special cases $I=1$ and $J=1$ correspond respectively to row vectors of dimension $J$ 
and to column vectors of dimension $I$:
\begin{align*}
    \bold{v}=\begin{pmatrix}
        v_1& v_{2} & ... & v_{J} 
      \end{pmatrix}\in \K^{1\times J},
    \bold{u}=\begin{pmatrix}
        u_1\\ u_{2} \\ ... \\ u_{I} 
      \end{pmatrix}\in \K^{I\times 1}.
\end{align*}
In the following, for column vectors, $\K^{I}$ will be used instead of $\K^{I\times 1}$.
\par
Identity matrix (i.e., a matrix with all 1's on the diagonal
and 0's everywhere else) is denoted by $\bold{E}$.
$\bold{e}_{i}^{(I)}$ is the column vector of dimension $I$,
in which element is equal to $1$ at position $i$ and $0s$ elsewhere.
$\bold{E}_{ij}^{I\times J}$ is the matrix of dimension $I\times J$
in which element is equal to $1$ at position $(i,j)$ and $0s$ elsewhere.

\section{Transposition and conjugate transposition}
\begin{definition}{transpose and the conjugate transpose of a column vector}{transpose and the conjugate transpose of a column vector}
    The transpose and the conjugate transpose (also called transconjugate) of a column
    vector $\bold{u}=\begin{pmatrix}u_1\\ u_{2} \\ ... \\ u_{I} \end{pmatrix}\in \C^{I}$, 
    denoted by $\bold{u}^{T}$ and $\bold{u}^{H}$, respectively, are the row vectors defined as:
    \begin{align*}
        \bold{u}^T=\begin{pmatrix}u_1\\ u_{2} \\ ... \\ u_{I} \end{pmatrix}\text{ and } 
        \bold{u}^H=\begin{pmatrix}u_1^*& u_{2}^* & ... & u_{I}^* \end{pmatrix},
    \end{align*}
    where $u_i^*$ is the conjugate of $u_i$ also denoted by $\overline{u}_i$.
\end{definition}

\begin{definition}{transpose and the conjugate transpose of a matrix}{transpose and the conjugate transpose of a matrix}
    The transpose of $\bold{A}\in\K^{I\times J}$ is the matrix denoted by $\bold{A}^T$, of dimensions $J\times I$, such that $A^T=(a_{ji})$, 
    with $i\in\langle {I}\rangle$ and $j\in\langle {J}\rangle$.
    In the case of a complex matrix, the conjugate transpose, also known as Hermitian transpose and denoted by $\bold{A}^H$,
    is defined as: $\bold{A}^{H}=(\bold{A}^*)^T=(\bold{A}^T)^*=(a_{ji}^*)$,
    where $\bold{A}^*=(a_{ij}^*)$ is the conjugate of $\bold{A}$.
\end{definition}

\begin{remark}
    In mathematics, the complex conjugate of a complex number 
    is the number with an equal real part and 
    an imaginary part equal in magnitude but opposite in sign. 
    That is, if $a$ and $b$ are real numbers 
    then the complex conjugate of $a+\text{i}b$ is $a-\text{i}b$.
    The complex conjugate of $z$ is often denoted as $\bar{z}$ or $z^*$.
\end{remark}


\begin{proposition}{}{}
    The operations of transposition and conjugate transposition satisfy:
    \begin{align*}
        (\bold{A}^T)^T=\bold{A}&, (\bold{A}^H)^H=\bold{A}\\
        (\bold{A}+\bold{B})=\bold{A}^T+\bold{B}^T&, (\bold{A}+\bold{B})^{H}=\bold{A}^H+\bold{B}^H,\\
        (\alpha \bold{A})^T=\alpha \bold{A}^T&, (\alpha \bold{A})^H=\alpha^*\bold{A}^H, 
    \end{align*}
    for any matrix $\bold{A},\bold{B}\in\C^{I\times J}$ and any scalar $\alpha\in\C$.
\end{proposition}

\begin{remark}
    By decomposing $\bold{A}$ using its real and imaginary parts, we have:
    \begin{align*}
        \bold{A}=\text{Re}(\bold{A}) +\text{i} \text{Im} (\bold{A}) \Rightarrow 
        \left\{\begin{matrix}
           \bold{A}^T = (\text{Re}(\bold{A}))^T+\text{i} (\text{Im}(A))^T \\
           \bold{A}^H = (\text{Re}(\bold{A}))^H-\text{i} (\text{Im}(A))^H
          \end{matrix}\right.
    \end{align*}.
\end{remark}


\section{Vector outer product and vectorization}
\subsection{Vector outer product}
The outer product of two vectors $\bold{u}\in\K^{I}$ and $\bold{v}\in \K^{J}$,
denoted $\bold{u}\circ \bold{v}$,
gives a matrix $\bold{A}\in \K^{I\times J}$
such that $a_{ij}=(\bold{u}\circ\bold{v})_{ij}=u_iv_j$,
and therefore, $\bold{u}\circ\bold{v}=\bold{u}\bold{v}^T=(u_iv_j)$, 
with $i\in \langle I\rangle,j\in\langle J\rangle$.

\begin{example}{}{}
    For $I=2,J=3$, we have:
    \begin{align*}
        \begin{pmatrix}
            u_1\\ u_{2}
        \end{pmatrix}\circ
        \begin{pmatrix}
            v_1\\ v_{2} \\v_{3}
        \end{pmatrix}
        = \begin{pmatrix}
            u_1\\ u_{2}
        \end{pmatrix} 
        \begin{pmatrix}
            v_1& v_{2} &v_{3}
        \end{pmatrix}
        = \begin{pmatrix}
            u_1v_1& u_1v_{2} & u_1v_{3}\\
            u_2v_1& u_2v_2 & u_2v_3
        \end{pmatrix}.
    \end{align*}
\end{example}

\subsection{Vectorization}
A very widely used operation in matrix computation is vectorization
which consists of stacking the columns of a matrix $\bold{A}\in \K^{I\times J}$
on top of each other to form a column vector of dimension $JI$:
\begin{align*}
    \bold{A}=\begin{pmatrix}
        \bold{A}_{\cdot 1}& \bold{A}_{\cdot 2} & ... & \bold{A}_{\cdot J}
    \end{pmatrix}\in \K^{I\times J}
    \Rightarrow \text{vec}(\bold{A}) = \begin{pmatrix}
        \bold{A}_{\cdot 1} \\ \bold{A}_{\cdot 2}\\ ...\\ \bold{A}
        _{\cdot J}
    \end{pmatrix}\in \K^{JI}.
\end{align*}
This operation defines an isomorphism between the space $\K^{JI}$ of vectors
of dimension $JI$ and the space $\K^{I\times J}$ of matrices $I\times J$.
Indeed, the canonical basis of $\K^{JI}$, denoted by $\{\bold{e}_{(j-1)I+i}^{(JI)}\}$, 
allows us to write $\text{vec}(\bold{A})$ as:
\begin{align*}
    \bold{A}=\sum\limits_{i=1}^{I}\sum\limits_{j=1}^{J}a_{ij}\bold{e}_i^{(I)}\circ \bold{e}_j^{(J)}
    \Rightarrow
    \text{vec}(\bold{A})=\sum\limits_{i=1}^{I}\sum\limits_{j=1}^{J}a_{ij}\bold{e}_{(j-1)I+i}^{(JI)},
\end{align*}
with $\bold{e}_{(j-1)I+i}^{(JI)}=\text{vec}(\bold{e}_i^{(I)}\circ \bold{e}_j^{(J)})=\text{vec}(\bold{e}_i^{(I)}(\bold{e}_j^{(J)})^T)$.

\begin{remark}
    Since the operator $\text{vec}$ satisfies $\text{vec}(\alpha \bold{A}+\beta \bold{B})=\alpha \text{vec}(\bold{A})+\beta \text{vec}(\bold{B})$
    for all $\alpha,\beta\in\K$, it is linear.
\end{remark}


\section{Vector inner product and orthogonality}
\subsection{Inner product}

In this section, we recall the definition of the inner product(also called dot product)
of two vectors $\bold{a},\bold{b}\in\K^{I}$.
\begin{definition}{}{}
    If $\K=\R$, the vector inner product is defined as:
    \begin{align*}
        \inner{\cdot}{\cdot}:\R^I\times \R^I&\rightarrow \R\\
        (\bold{a},\bold{b})&\mapsto \inner{\bold{a}}{\bold{b}} = \bold{a}^T\bold{b}=\sum\limits_{i=1}^{I}a_ib_i.
    \end{align*}
    In $\C^I$, the definition of the vector inner product is given by:
    \begin{align*}
        \inner{\cdot}{\cdot}:\C^I\times \C^I&\rightarrow \C\\
        (\bold{a},\bold{b})&\mapsto \inner{\bold{a}}{\bold{b}} = \bold{a}^H\bold{b}=\sum\limits_{i=1}^{I}a_i^*b_i.
    \end{align*}
\end{definition}
\begin{remark}
    Whether $\K=\C$ or $\R$, $\inner{\bold{a}}{\bold{a}}\in \R$. 
\end{remark}


\subsection{Orthogonality}
\begin{definition}{}{}
    Two vectors $\bold{a}$ and $\bold{b}$ of $\K^{I}$ are said to be 
    orthogonal if and only if $\inner{\bold{a}}{\bold{b}}=0$. 
\end{definition}


\section{Vector Norms}
\begin{definition}{}{}
    Let $v:\C^n\rightarrow \R$. 
    Then $v$ is a norm if for all $\bold{x},\bold{y}\in\C^n$
    \begin{itemize}
        \item $\bold{x}\neq 0\Rightarrow v(\bold{x})>0$,
        \item $v(\alpha \bold{x})=|\alpha|v(\bold{x})$, and
        \item $v(\bold{x}+\bold{y})\leqs v(\bold{x})+v(\bold{y})$ 
    \end{itemize}
\end{definition}
\begin{remark}
    often we will use $||\cdot||$ to denote a vector norm.
\end{remark}


\subsection{Vector 2-norm}
\begin{definition}{}{}
    The vector 2-norm $||\cdot||_2:\C^n\rightarrow \R$ is defined by
    \begin{align*}
        ||\bold{x}||_2=\sqrt{\inner{\bold{x}}{\bold{x}}}=\sqrt{\bold{x}^H\bold{x}}=\sqrt{x_1^*x_1+x_2^*x_2+...+x_n^*x_n}=\sqrt{|x_1|^2+...+|x_n|^2}.
    \end{align*}
\end{definition}

\begin{theorem}{}{}
    Let $\bold{x},\bold{y}\in\C^n$. Then $|\inner{\bold{x}}{\bold{y}}|=|\bold{x}^H\bold{y}|\leqs ||\bold{x}||_2\cdot ||\bold{y}||_2$.
\end{theorem}

\begin{proposition}{}{}
    The vector 2-norm is a norm.
\end{proposition}

\subsection{Vector 1-norm}
\begin{definition}{}{}
    The vector 1-norm $||\cdot||_1:\C^n\rightarrow \R$ is defined by
    \begin{align*}
        ||\bold{x}||_1=|x_1|+|x_2|+...+|x_{n}|.
    \end{align*}
\end{definition}

\begin{proposition}{}{}
    The vector 1-norm is a norm.
\end{proposition}

\subsection{Vector $\infty$-norm}
\begin{definition}{}{}
    The vector $\infty$-norm $||\cdot||_{\infty}:\C^n\rightarrow \R$ is defined by $||\bold{x}||_{\infty}=\max_{i}|x_i|$.
\end{definition}

\begin{proposition}{}{}
    The vector $\infty$-norm is a norm.
\end{proposition}

\subsection{Vector $p$-norm}
\begin{definition}{}{}
    The vector $p$-norm $||\cdot||_p:\C^n\rightarrow \R$ is defined by
    \begin{align*}
        ||\bold{x}||_p=\sqrt[p]{|x_1|^p+|x_2|^p+...+|x_n|^p}.
    \end{align*}
\end{definition}

\begin{proposition}{}{}
    The vector $p$-norm is a norm.
\end{proposition}

\section{Matrix Norms}
It is not hard to see that vector norms are
all measures of how "big" the vectors are. Similarly,
we want to have measures for how "big" matrices are.
We will start with one that are somewhat artificial and then move on to the important class of induced matrix norms.

\subsection{Frobenius norm}
\begin{definition}{}{}
    The Frobenius norm $||\cdot||_F:\C^{m\times n}\rightarrow \R$ is defined by
    \begin{align*}
        ||\bold{A}||_F=\sqrt{\sum\limits_{i=1}^{n}\sum\limits_{j=1}^{n}|a_{ij}|^2}.
    \end{align*}
\end{definition}

\begin{remark}
    $||\bold{A}||_F=||\text{vec}(\bold{A})||_2$.
\end{remark}

\begin{proposition}{}{}
    The Frobenius norm is a norm.
\end{proposition}

\subsection{Induced matrix norms}
\begin{definition}{}{}
    Let $||\cdot||_{\mu}:\C^m\rightarrow \R$ and $||\cdot||_{v}:\C^n\rightarrow \R$ be vector norms.
    Define $||\cdot||_{\mu,v}:\C^{m\times n}\rightarrow \R$ by 
    \begin{align*}
        ||\bold{A}||_{\mu,v}=\sup_{\bold{x}\in\C^n,\bold{x}\neq 0}\frac{||\bold{A\bold{x}}||_{\mu}}{||\bold{x}||_v}.
    \end{align*}
\end{definition}

\begin{remark}
    How "big" $\bold{A}$ is, as measured by $||\bold{A}||_{\mu,v}$, 
    is defined as the most that $\bold{A}$ magnifies the length of nonzero vectors,
    where the length of the vector $\bold{x}$ is measured with norm $||\cdot||_v$ and
    the length of the transformed vector $\bold{A}\bold{x}$ is measured with norm $||\cdot||_{\mu}$.
\end{remark}

\begin{proposition}{}{}
    Let $||\cdot||_{\mu}:\C^m\rightarrow \R$ and $||\cdot||_{v}:\C^n\rightarrow \R$ be vector norms.
    \begin{align*}
        ||\bold{A}||_{\mu,v}&=\sup_{\bold{x}\in\C^n,\bold{x}\neq 0}\frac{||\bold{A\bold{x}}||_{\mu}}{||\bold{x}||_v}\\
                            &=\max_{\bold{x}\in\C^n,\bold{x}\neq 0}\frac{||\bold{A\bold{x}}||_{\mu}}{||\bold{x}||_v}\\
                            &=\max_{||\bold{x}||_v=1}||\bold{A}\bold{x}||_{\mu}.
    \end{align*}
\end{proposition}

\begin{proposition}{}{}
    $||\cdot||_{\mu,v}:\C^{m\times n}\rightarrow \R$ is a norm. 
\end{proposition}

\begin{definition}{}{}
    Define $||\cdot||_p:\C^{m\times n}\rightarrow \R$ by
    \begin{align*}
        ||\bold{A}||_{p}&=\max_{\bold{x}\in\C^n,\bold{x}\neq 0}\frac{||\bold{A\bold{x}}||_{p}}{||\bold{x}||_p}\\
                            &=\max_{||\bold{x}||_p=1}||\bold{A}\bold{x}||_{p}.
    \end{align*}
\end{definition}

\begin{proposition}{}{}
    For all $\bold{A}\in \C^{m\times n}, \bold{x}\in \C^n$,
    \begin{align*}
        ||\bold{A}\bold{x}||\leqs ||\bold{A}||_p\cdot ||\bold{x}||_p.
    \end{align*}
\end{proposition}
\begin{proof}
    By the definition of $||\bold{A}||_p$,
    \begin{align*}
        \frac{||\bold{A}||_p}{||\bold{x}||_p}\leqs ||\bold{A}||_p.
    \end{align*}
\end{proof}


\begin{proposition}{}{}
    For any $\bold{A}\in \C^{m\times k}$ and $\bold{B}\in\C^{k\times n}$, 
    \begin{align*}
        ||\bold{A}\bold{B}||_p&\leqs ||\bold{A}||_p||\bold{B}||_p\\
        ||\bold{A}\bold{B}||_F&\leqs ||\bold{A}||_F||\bold{B}||_F.
    \end{align*}
\end{proposition}

\section{Matrix multiplication}
Suppose $\bold{A}\in\R^{m\times r}$ and $\bold{B}\in \R^{r\times n}$,
the matrix multiplication $\bold{C}=\bold{A}\bold{B}$ can be viewed from
three different perspectives as follows:
\par
\underline{Dot Product Matrix Multiply}. 
Every element $c_{ij}$ of $\bold{C}$ is the 
dot product of row vector $\bold{A}_{i\cdot}$ and column vector $\bold{B}_{\cdot j}$.
\begin{align*}
    \bold{A}&=\begin{pmatrix}
        \bold{A}_{1\cdot}^T\\ ...\\\bold{A}_{m\cdot}^T
    \end{pmatrix},\bold{A}_{k\cdot}\in\R^r\\
    \bold{B}&=\begin{pmatrix}
        \bold{B}_{\cdot 1}&...\bold{B}_{\cdot n}
    \end{pmatrix},\bold{B}_{\cdot k}\in\R^r\\
    \bold{C}&=(c_{ij}),c_{ij}=\bold{A}_{i\cdot}^T\bold{B}_{\cdot j}=\sum\limits_{k=1}^{r}a_{ik}b_{kj}.
\end{align*}
\par
\underline{Column Combination Matrix Multiply}.
Every column $\bold{C}_{\cdot j}$ of $\bold{C}$ is a linear combination of column vector
$\bold{A}_{\cdot k}$ of $\bold{A}$ with columns $b_{kj}$ as the weight coefficients.






\section{Matrix trace, Matrix inner product}

\subsection{Definition and properties of the trace}
\begin{definition}{}{}
    The trace of a square matrix $\bold{A}$ of order $I$ is defined as the sum of its diagonal elements:
    \begin{align*}
        \text{tr}(\bold{A})=\sum\limits_{i=1}^{I}a_{ii}.
    \end{align*}
\end{definition}

\begin{proposition}{}{}
    The trace satisfies the following properties:
    \begin{align*}
        \text{tr}(\alpha \bold{A}+\beta \bold{B}) &= \alpha \text{tr}(\bold{A}) + \beta \text{tr}(\bold{B}),\\
        \text{tr}(\bold{A}^T)&=\text{tr}(\bold{A}),\\
        \text{tr}(\bold{A}^*)&=\text{tr}(\bold{A}^H)=(\text{tr}(\bold{A}))^*,\\
    \end{align*}
\end{proposition}


\section{Eigenvalues and eigenvectors}
\begin{definition}{}{}
    A real matrix $\bold{A}$ is a symmetric matrix 
    if it equals to its own transpose, 
    that is $\bold{A} = \bold{A}^T$.
\end{definition}

\begin{definition}{}{}
    A complex matrix $\bold{A}$ is a hermitian matrix 
    if it equals to its own complex conjugate transpose, 
    that is $\bold{A} = \bold{A}^H$.
\end{definition}

\begin{definition}{}{}
    A real matrix $\bold{Q}$ is an orthogonal matrix 
    if the inverse of $\bold{Q}$ equals to 
    the transpose of $\bold{Q}$, $\bold{Q}^{-1} = \bold{Q}^T$
    , that is $\bold{Q}\bold{Q}^T = \bold{Q}^T\bold{Q} = I$.
\end{definition}

\begin{definition}{}{}
    A complex matrix $\bold{U}$ is a unitary matrix 
    if the inverse of $\bold{U}$ equals the complex conjugate
    transpose of $\bold{U}$, $\bold{U}^{-1} = \bold{U}^H$, 
    that is $\bold{U}\bold{U}^H = \bold{U}^H\bold{U} = \bold{I}$.
\end{definition}

\begin{definition}{}{}
    A hermitian matrix $\bold{Q}$ is positive semidefinite (abbreviated SPSD and denoted by $\bold{Q}\succeq 0$) if 
    \begin{align*}
        \bold{x}^H\bold{Q}\bold{x}\geqs 0 \text{ for all } \bold{x}\in \C^n.
    \end{align*}
\end{definition}

\begin{definition}{}{}
    A hermitian matrix $\bold{Q}$ is positive semidefinite (abbreviated SPD and denoted by $\bold{Q}\succ 0$) if 
    \begin{align*}
        \bold{x}^H\bold{Q}\bold{x}> 0 \text{ for all } \bold{x}\in \C^n, \bold{x}\neq 0.
    \end{align*}
\end{definition}

A number $\lambda\in\C$ is an eigenvalue of $\bold{M}$ if there exists 
a vector $\bold{\bar{x}}\neq 0$ such that $\bold{M}\bold{\bar{x}}=\lambda \bold{\bar{x}}$.
$\bold{\bar{x}}$ is called an eigenvector of $\bold{M}$ (and is called an eigenvector corresponding to $\lambda$).
Note that $\lambda$ is an eigenvalue of $\bold{M}$ if and only if there exists $\bold{\bar{x}}\neq 0$ such that 
$(\bold{M}-\lambda \bold{E})\bold{\bar{x}}=0$ or, equivalently, if and only if $\text{det}(\bold{M}-\lambda \bold{E})=0$.

Let $g(\lambda)=\text{det}(\bold{M}-\lambda \bold{E})$. Then $g(\lambda)$ is 
a polynomial of degree $n$, and so will have $n$ roots 
that will solve the equation $g(\lambda)=\text{det}(\bold{M}-\lambda \bold{E})=0$, 
including multiplicities. These roots are the eigenvalues of $\bold{M}$.


\begin{proposition}{}{}
    If $\bold{Q}$ is a real symmetric matrix, all of its eigenvalues are real numbers.
\end{proposition}

\begin{proposition}{}{}
    If $\bold{Q}$ is a complex hermitian matrix, all of its eigenvalues are real numbers.
\end{proposition}

\begin{proposition}{}{}
    If $\bold{Q}$ is a hermitian matrix, 
    its eigenvectors corresponding to different eigenvalues are orthogonal.
\end{proposition}

\begin{proposition}{}{}
    If $\bold{Q}$ is SPSD, the eigenvalues of $\bold{Q}$ are nonnegative.
\end{proposition}

\begin{theorem}{}{}
    If $\bold{A}$ is a real symmetric matrix, then $\bold{A}=\bold{Q}\bold{D}\bold{Q}^T$,
    where $\bold{Q}$ is an orthonormal matrix, the columns of $\bold{Q}$ are an orthonormal basis of eigenvectors of $\bold{A}$,
    and $\bold{D}$ is a diagonal matrix of the corresponding eigenvalues of $\bold{A}$.
\end{theorem}

\begin{theorem}{}{}
    If $\bold{A}$ is a complex hermitian matrix, then $\bold{A}=\bold{U}\bold{D}\bold{U}^H$,
    where $\bold{U}$ is an unitary matrix, the columns of $\bold{U}$ are an orthonormal basis of eigenvectors of $\bold{A}$,
    and $\bold{D}$ is a diagonal matrix of the corresponding eigenvalues of $\bold{A}$.
\end{theorem}

\begin{proposition}{}{}
    If $\bold{D}$ is SPSD, the $\bold{Q}=\bold{M}^H\bold{M}$ for some matrix $\bold{M}$.
\end{proposition}

\begin{definition}{}{}
    The Rayleigh quotient of the matrix $\bold{A}\in \C^{n\times n}$ at the nonzero vector $\bold{x}\in \C^{n}$
    is the scalar
    \begin{align*}
        \frac{\bold{x}^H\bold{A}\bold{x}}{\bold{x}^H\bold{x}}\in \C.
    \end{align*}
\end{definition}
\begin{remark}
    If $(\lambda,\bold{u})$ is an eigenpair for $\bold{A}$, then notice that
    \begin{align*}
        \frac{\bold{u}^H\bold{A}\bold{u}}{\bold{u}^H\bold{u}}=\frac{\bold{u}^H(\lambda \bold{u})}{\bold{u}^H\bold{u}}=\lambda,
    \end{align*}
    so Rayleigh quotients generalize eigenvalues.
\end{remark}
\begin{remark}
    For Hermitian $\bold{A}\in \C^{n\times n}$, 
    then there exists unitary matrix $\bold{U}$ such that $A=\bold{U}\Lambda \bold{U}^H$, 
    for any $\bold{x}\neq \bold{0}$,  it can be represented by $\bold{x}=\bold{U}\bold{c}=\sum\limits_{j=1}^{n}c_j\bold{u}_j$, then 
    \begin{align*}
        \frac{\bold{x}^H\bold{A}\bold{x}}{\bold{x}^H\bold{x}}
        =\frac{\bold{c}^H\bold{U}^H\bold{U}\Lambda \bold{U}^H\bold{U}\bold{c}}{\bold{c}^H\bold{U}^H\bold{U}\bold{c}}
        =\frac{c^H\Lambda c}{c^Hc}
    \end{align*}
    The diagonal
    structure of $\Lambda$ allows for an illuminating refinement,
    \begin{align*}
        \frac{\bold{x}^H\bold{A}\bold{x}}{\bold{x}^H\bold{x}}=\frac{\lambda_1 |c_1|^2+...+\lambda_n |c_n|^2}{|c_1|^2+...+|c_n|^2}.
    \end{align*}
    As the numerator and denominator are both real, notice that the Rayleigh
    quotients for a Hermitian matrix is always real. We can say more: if the eigenvalues are ordered, $\lambda_1\leqs ... \leqs \lambda_n$,
    \begin{align*}
        \frac{\bold{x}^H\bold{A}\bold{x}}{\bold{x}^H\bold{x}}=\frac{\lambda_1 |c_1|^2+...+\lambda_n |c_n|^2}{|c_1|^2+...+|c_n|^2}\geqs \frac{\lambda_1 (|c_1|^2+...+|c_n|^2)}{|c_1|^2+...+|c_n|^2}=\lambda_1,
    \end{align*}
    and similarly,
    \begin{align*}
        \frac{\bold{x}^H\bold{A}\bold{x}}{\bold{x}^H\bold{x}}=\frac{\lambda_1 |c_1|^2+...+\lambda_n |c_n|^2}{|c_1|^2+...+|c_n|^2}\leqs \frac{\lambda_n (|c_1|^2+...+|c_n|^2)}{|c_1|^2+...+|c_n|^2}=\lambda_n.
    \end{align*}
\end{remark}

\begin{proposition}{}{the range of the Rayleigh quotient}
    For a Hermitian matrix $\bold{A}\in \C^{n\times n}$ with eigenvalues $\lambda_1\leqs ...\leqs \lambda_n$,
    the Rayleigh quotient for nonzero $\bold{x}\in \C^{n\times n}$ satisfies
    \begin{align*}
        \frac{\bold{x}^H\bold{A}\bold{x}}{\bold{x}^H\bold{x}}\in [\lambda_1,\lambda_n].
    \end{align*}
\end{proposition}

\begin{proposition}{}{}
    If $\bold{A}\in \C^{m\times n}$ with $m<n$ and $\bold{A}$ has rank $m$, then 
    \begin{align*}
        ||\bold{A}||_2=\sqrt{\lambda_{\max}(\bold{A}^H\bold{A})}=\sqrt{\lambda_{\max}(\bold{A}\bold{A}^H)},
    \end{align*}
    where $\lambda_{\max}(\bold{M})$ denotes the largest eigenvalue of a matrix $\bold{M}$.
\end{proposition}

\begin{proof}
    \begin{align*}
        ||\bold{A}||_2^2=\sup_{\bold{x}\in\C^n,||\bold{x}||\neq 0} \frac{||\bold{Ax}||_2^2}{||\bold{x}||_2^2}
        =\sup_{\bold{x}\in\C^n,||\bold{x}||\neq 0}\frac{\inner{\bold{Ax}}{\bold{Ax}}}{\inner{\bold{x}}{\bold{x}}}
        =\sup_{\bold{x}\in\C^n,||\bold{x}||\neq 0}\frac{\bold{x}\bold{A}^H\bold{A}\bold{x}}{\bold{x}^H\bold{x}}
        =\lambda_{\text{max}}(\bold{A}^H\bold{A}).
    \end{align*}
\end{proof}

\begin{proposition}{}{}
    If $\bold{A}\in \C^{n\times n}$ is a hermitian matrix, then 
    \begin{align*}
        ||\bold{A}||_2= |\lambda_{\max}(\bold{A})|.
    \end{align*}
\end{proposition}

\begin{proof}
    Since $\bold{A}$ is a hermitian matrix, we have $\bold{A}=\bold{U}^H\Lambda \bold{U}$ where $\bold{U}$ is an unitary matrix and $\Lambda$ is a diagonal matrix containing the eigenvalue of $\bold{A}$.
    Let $\bold{y}=\bold{Ux}$, then $||\bold{y}||_2=||\bold{Ux}||_2$. Hence 
    \begin{align*}
        ||\bold{A}||_2^2=\max_{||\bold{x}||_2=1} ||\bold{Ax}||_2^2 = \max_{||\bold{x}||_2=1} \inner{\bold{Ax}}{\bold{Ax}}
        = \max_{||\bold{x}||_2=1} \bold{x}^H\bold{A}^H\bold{A}\bold{x} =\max_{||\bold{y}||_2=1} \bold{y}^H\Lambda\bold{y}=(\lambda_{\max}(\bold{A}))^2.
    \end{align*}
\end{proof}


\begin{proposition}{}{}
    Suppose that $\bold{A}\in\R^{n\times n}$ is a SPSD. Then the following are equivalent:
    \begin{align*}
        &\text{(a) } h>0 \text{ satisfies } ||\bold{A}^{-1}||_2\leqs \frac{1}{h}.\\
        &\text{(b) } h>0 \text{ satisfies } ||\bold{A}\bold{x}||_2 \geqs h\cdot ||\bold{x}||_2 \text{ for any vector } \bold{x}\\
        &\text{(c) } h>0 \text{ satisfies } |\lambda_i(\bold{A})|\geqs h \text{ for every eigenvalue } \lambda_i(\bold{A}) \text{ of } \bold{A}, i=1,...,m.
    \end{align*}
\end{proposition}
\begin{proof}
    By proposition \ref{prop:the range of the Rayleigh quotient}, 
    we have for all $\bold{x}\neq 0$,
    \begin{align*}
        \frac{||\bold{Ax}||_2}{||\bold{x}||_2}\in [\lambda_{\min}(\bold{A}),\lambda_{\max}(\bold{A})].
    \end{align*}
    Firstly, we claim that 
    \begin{align*}
        ||\bold{A}^{-1}||_2 =  \frac{1}{\lambda_{\min}(\bold{A})}.
    \end{align*}
    In fact, 
    \begin{align*}
        ||\bold{A}^{-1}||_2 &=  \max_{\bold{x}\neq 0} \frac{||\bold{A}^{-1}\bold{x}||_2}{||\bold{x}||_2}
        =\max_{\bold{Ax}\neq 0} \frac{||\bold{A}^{-1}\bold{A}x||_2}{||\bold{Ax}||_2} \\
        &=\max_{\bold{Ax}\neq 0} \frac{||\bold{x}||_2}{||\bold{Ax}||_2}
        = \max_{\bold{x}\neq 0} \frac{||\bold{x}||_2}{||\bold{Ax}||_2}\\
        &=\frac{1}{\min_{\bold{x}\neq 0} \frac{||\bold{Ax}||_2}{||\bold{x}||_2}}
        =\frac{1}{\lambda_{\min}(\bold{A})},
    \end{align*}
    the second and forth equality follows from the fact that $\bold{x}\neq 0$
    if and only if $\bold{Ax}\neq 0$ since $\bold{A}$ is nonsingular. 
    \par
    (a) $\Rightarrow$ (b) \quad $\frac{1}{\lambda_{\min}(\bold{A})}=||\bold{A}^{-1}||_2\leqs \frac{1}{h}$
    $\Rightarrow \frac{||\bold{Ax}||_2}{||\bold{x}||_2}\geqs \lambda_{\min}(\bold{A})\geqs h$, $\forall \bold{x}\neq 0$.
    \par
    (b) $\Rightarrow$ (a) \quad $||\bold{Ax}||_2\geqs h\cdot ||x||_2$, $\forall \bold{x}$ $\Rightarrow$ $\lambda_{\min}(\bold{A})\geqs h$
    $\Rightarrow$ $\frac{1}{\lambda_{\min}(\bold{A})} = ||\bold{A}^{-1}||_2\leqs \frac{1}{h}$.
    \par
    (b) $\Leftrightarrow$ (c) \quad $||\bold{Ax}||_2\geqs h\cdot ||x||_2$, $\forall \bold{x}$ $\Leftrightarrow$ $\lambda_{\min}(\bold{A})\geqs h$
    $\Leftrightarrow$ $\lambda_i(\bold{A})\geqs h$, $i=1,...,m$.
    \par
\end{proof}


\section{Generalized inverses}

\section{Reference}
\begin{itemize}
    \item \href{}{From Algebraic Structures to Tensors ch4 matrix algebra}
    \item \href{https://www.cs.utexas.edu/users/flame/Notes/NotesOnNorms.pdf}{Notes on Vector and Matrix Norms}
    \item \href{https://ocw.mit.edu/courses/15-084j-nonlinear-programming-spring-2004/resources/lec4_quad_form/}{symmetric, positive definted, eigenvalues and eigenvectors}
    \item \href{https://www.cmor-faculty.rice.edu/~caam440/chapter2.pdf}{Rayleigh quotient}
    \item \href{https://www.math.uwaterloo.ca/~jmckinno/Math225/Week7/Lecture2m.pdf}{The Principal Axis Theorem}
    \item \href{https://staff.polito.it/ada.boralevi/didattica/Dispense_ENG.pdf}{linear algebra and geometry note}
    \item \href{http://jde27.uk/la/36_eigenapplications2.html}{eigenvalues application: ellipses}
    \item \href{https://www.sjsu.edu/faculty/guangliang.chen/Math253S20/lec6ginverse.pdf}{Generalized inverses}
\end{itemize}

\chapter{Matrix Decompositions}
\chapter{Hadamard, Kronecker and Khatri–Rao Products}

\section{Partitioned matrices}

Let $\{\alpha_{m_1},...,\alpha_{m_R}\}$
and $\{\beta_{n_1},...,\beta_{n_S}\}$
be partitions of the sets $\{1,...,m\}$
and $\{1,...,n\}$, respectively, with 
$m_r\in \langle m \rangle$ and $n_s\in\langle n\rangle$,
such that $\sum\limits_{r=1}^{R}m_r=m$ and $\sum\limits_{s=1}^{S}n_s=n$.
It is said that matrices $\bold{A}_{rs}$ of dimensions $(m_r,n_s)$
form a partition of the matrix $\bold{A}\in \K^{m\times n}$ into $(R,S)$ blocks,
or that $\bold{A}$ is partitioned into $(R,S)$ blocks, if $\bold{A}$ can be written as:
\begin{align*}
    \bold{A}=\begin{pmatrix}
        \bold{A}_{11}&\bold{A}_{12}&...&\bold{A}_{1S}\\
        \bold{A}_{21}&\bold{A}_{22}&...&\bold{A}_{2S}\\
        ...&...&...&...\\
        \bold{A}_{R1}&\bold{A}_{R2}&...&\bold{A}_{RS}\\
    \end{pmatrix}=(\bold{A}_{rs}), r\in\langle R\rangle, s\in\langle S\rangle.
\end{align*}
All submatrices of the same row-block $(r)$ contain the same number $(m_r)$ of rows.
Similarly, all submatrices of the same column-block $(s)$ contain the same number $(n_s)$ of columns,
that is:
\begin{align*}
    \begin{pmatrix}
        \bold{A}_{r1}&\bold{A}_{r2}&...&\bold{A}_{rS}
    \end{pmatrix}\in \K^{m_r\times n},
    \begin{pmatrix}
        \bold{A}_{1s}&\bold{A}_{2s}&...&\bold{A}_{Rs}
    \end{pmatrix}\in \K^{m\times n_s}.
\end{align*}
It is then said that the submatrices $\bold{A}_{rs}$ are of compatible dimensions.
\par
In the particular case where $n=1$,
the partitioned matrix becomes a block-column vector:
\begin{align*}
    \bold{a}=\begin{pmatrix}
        \bold{a}_1&\bold{a}_2&...&\bold{a}_{R}
    \end{pmatrix}\in \K^{m\times 1},
    \bold{a}_r\in \K^{m_r\times 1},r\in \langle R\rangle.
\end{align*}
Similarly, when $m=1$, the partitioned matrix becomes a block-row vector:
\begin{align*}
    \bold{a}^{T}=\begin{pmatrix}
        \bold{a}_1^T&\bold{a}_2^T&...&\bold{a}_S^T
    \end{pmatrix}\in \K^{1\times n}, \bold{a}_s\in\K^{n_s\times 1},s\in\langle S\rangle.
\end{align*}

\section{Notation}
We will write $\bold{A}^*,\bold{A}^T,\bold{A}^H,\bold{A}^{\dagger},\bold{A}_{i\cdot},\bold{A}_{\cdot j},r(\bold{A})$ 
and $\text{det}(\bold{A})$, for the conjugate, the transpose,
the transconjugate (also known as conjugate transpose or Hermitian transpose),
the Moore-Penrose pseudo-inverse, the $i$th row, the $j$th column,
the rank and the determinant of $\bold{A}\in\K^{I\times J}$, respectively.
In the following literature, the Hadamard product is denoted by $*$.
The symbols $\otimes$ and $\odot$ are used for the Kronecker and Khatri-Rao product, respectively.
\par
The symbol $\bold{1}_I$ denotes a column vector of size $I$
whose elements are all equal to $1$. 
The elements of the matrices $\bold{0}_{I\times J}$ and
$\bold{1}_{I\times J}$ of size $(I\times J)$ are all equal to $0$
and $1$, respectively. The symbols $\bold{I}_N$ and $\bold{e}_n^{(N)}$
denote the identity matrix of order $N$ and the $n$th vector of 
the canonical basis of the vector space $\R^N$, respectively.


\section{Hadamard product}

\subsection{Definition and identities}
\begin{definition}{}{}
    Let $\bold{A}$ and $\bold{B}\in \K^{I\times J}$ be two matrices of the same size.
    The Hadamard product of $\bold{A}$ and $\bold{B}$ is the
    matrix $\bold{C}\in\K^{I\times J}$ defined as follows:
    \begin{align*}
        \bold{C}=\bold{A}*\bold{B}=\begin{pmatrix}
            a_{11}b_{11}& a_{12}b_{12}& ... & a_{1J}b_{1J}\\
            a_{21}b_{21}& a_{22}b_{22}& ... & a_{2J}b_{2J}\\
            ...& ... & ... & ...\\
            a_{I1}b_{I1}& a_{I2}b_{I2} & ...& a_{IJ}b_{IJ}
        \end{pmatrix},
    \end{align*}
    i.e. $c_{ij}=a_{ij}b_{ij}$, and therefore $\bold{C}=(a_{ij}b_{ij})$, with $i\in\langle I\rangle$, $j\in\langle J\rangle$.
\end{definition}

\section{Kronecker product}

\subsection{Kronecker product of vectors}
\subsubsection{Definition}
\begin{definition}{}{}
    (a) For $\bold{u}\in \K^I$ and $\bold{v}\in \K^J$, we have:
    \begin{align*}
        \bold{x}=\bold{u}\otimes \bold{v}\in \K^{IJ}\Leftrightarrow x_{j+(i-1)J}=u_iv_j
    \end{align*}
    or equivalently:
    \begin{align*}
        \bold{u}\otimes \bold{v}=\begin{pmatrix}
            u_1v_1& ... & u_1v_J& u_2v_1&...& u_2v_J&...& u_Iv_1&...& u_Iv_J
        \end{pmatrix}^T=\begin{pmatrix}
            u_1\bold{v}&...&u_I\bold{v}
        \end{pmatrix}.
    \end{align*}
    (b) Similarly, for $\bold{u}\in \K^{I}, \bold{v}\in \K^J$, and $\bold{w}\in\K^K$, we have:
    \begin{align*}
        \bold{x}=\bold{u}\otimes \bold{v}\otimes \bold{w}\in \K^{IJK}\Leftrightarrow x_{k+(j-1)K+(i-1)JK}=u_iv_jw_k.
    \end{align*}
\end{definition}
\begin{remark}
    By convention, the order of the dimensions in a product $IJK$ follows
    the order of variation of the corresponding indices $(i,j,k)$. For example,
    $\K^{IJK}$ means that the index $i$ varies more slowly than $j$, which itself varies more slowly than $k$.
\end{remark}

\subsection{Kronecker product of matrices}

\subsubsection{Definitions and identities}
\begin{definition}{}{}
    Given two matrices $\bold{A}\in\K^{I\times J}$ and $\bold{B}\in \K^{M\times N}$
    of arbitrary size, the right Kronecker product of $\bold{A}$ by $\bold{B}$ is the matrix $\bold{C}\in \K^{IM\times JN}$
    defined as follows:
    \begin{align*}
        \bold{C}=\bold{A}\otimes \bold{B}=\begin{pmatrix}
            a_{11}\bold{B} & a_{12}\bold{B} & ...& a_{1J}\bold{B}\\
            a_{21} \bold{B} & a_{22}\bold{B} & ... & a_{2J}\bold{B}\\
            ...&...&...&...\\
            a_{I1}\bold{B}&a_{I2}\bold{B}&...&a_{IJ}\bold{B}
        \end{pmatrix}=(a_{ij}\bold{B}).
    \end{align*}
\end{definition}
\begin{remark}
    The Kronecker product is a matrix partitioned into $(I,J)$ blocks, where the block $(i,j)$ is given by the matrix $a_{ij}\bold{B}\in \K^{M\times N}$.
    The element $a_{ij}b_{mn}$ is located at the position $((i-1)M+m,(j-1)N+n)$ in $\bold{A}\otimes \bold{B}$.
\end{remark}

\begin{example}{}{}
    For $\bold{A}=\begin{pmatrix}
        a_{11}&a_{12}\\
        a_{21}&a_{22}
    \end{pmatrix}$,
    $\bold{B}=\begin{pmatrix}
        b_{11}&b_{12}\\
        b_{21}&b_{22}
    \end{pmatrix}$, we have:
    \begin{align*}
        \bold{A}\otimes \bold{B}=\begin{pmatrix}
            
        \end{pmatrix}
    \end{align*}
\end{example}


\section{Reference}
\begin{itemize}
    \item \href{}{From Algebraic Structures to Tensors ch5}
    \item \href{}{Matrix and Tensor Decompositions in Signal Processing ch2}
\end{itemize}
\chapter{Tensor Operations}

\section{Notation}
Let $\chi\in \K^{I_1}$


\section{Notion of slice}

\section{Mode combination}

\section{Matricization}

\section{Multiplication operations}






%%%%%%%%%%%%%%%Content%%%%%%%%%%%%%%%
% % \mainmatter % separat the number of toc and mainmatter
% \chapter*{Preface}

Notes mainly refer to following materials:


\begin{itemize}
    \item[*] Machine learning
    \begin{itemize}
        \item \href{https://www.cs.cornell.edu/courses/cs4780/2023sp/}{lecture notes from cornell}
        \item \href{https://www.cs.cmu.edu/~hn1/documents/machine-learning/notes.pdf}{lecture notes from cmu}
        \item \href{https://cs229.stanford.edu/main_notes.pdf}{lecture notes of CS229}
    \end{itemize}
    \item[*] Deep learning
    \begin{itemize}
        \item \href{https://udlbook.github.io/udlbook/}{understanding deep learning}
        \item \href{https://www.bilibili.com/video/BV1Wv411h7kN/?spm_id_from=333.337.search-card.all.click}{lecture video from Hongyi Lee}
        \item \href{https://cs231n.github.io/}{lecture notes from Stanford}
    \end{itemize}
    \item[*] Reinforcement learning
    \begin{itemize}
        \item \href{https://web.stanford.edu/class/cs234/modules.html}{lecture notes from stanford}
        \item \href{https://people.cs.umass.edu/~bsilva/courses/CMPSCI_687/Fall2022/Lecture_Notes_v1.0_687_F22.pdf}{lecture notes from umass}
    \end{itemize}
\end{itemize}







% \part{Mathematics}

% % !TEX root = ../notes_template.tex
\chapter{Discrete Math}\label{chp:discrete_math}


\section{Proof}

\begin{theorem}
\end{theorem}

\begin{solution}
By induction:
\end{solution}



\section{Quantifier}
\lipsum % dummy text - remove from real document

\section{Graph}
\citetitle{babaiGraphIsomorphismQuasipolynomial2016}
\cite{babaiGraphIsomorphismQuasipolynomial2016}

\section{Number theory}
Figure example
\begin{figure}[!ht]
    \centering
    \includegraphics[width=1\linewidth]{./figure/elliptic_curves.pdf}
    \caption{Elliptic curves \cite{childsUniversalComputationQuantum2009} }
\end{figure}


\section{Algorithm}
% \begin{center}
% \begin{minipage}{.9\linewidth}
% algorithm2e
% https://www.overleaf.com/learn/latex/Algorithms#The_algorithm2e_package
\begin{algorithm}[H]
    \SetKwInOut{Input}{input}
    \SetKwInOut{Output}{output}
    \Input{Integer $N$ and parameter $1^t$}
    \Output{A decision as to whether $N$ is prime or composite}
    \BlankLine
    \For{ $i = 1,2, \ldots, t$} {
        $a\leftarrow \qty{1,\dots,N_1}$\;
        \If{$a^{N-1} \neq 1 \mod{N}$}
    {\Return "composite"}
    }
    \Return "prime"
    \caption{Primality testing - first attempt}
    \label{alg:miller_rabin}
\end{algorithm}
% \end{minipage}
% \end{center}

% \part{Computer Science}
% % \input{./chapter/complexity.tex}
% % !TEX root = ../notes_template.tex
\chapter{Machine Learning}\label{chp:machine_learning}
\minitoc

\section{Regression}
% \gls{algorithm};
\subsection{Gradient descent}\label{sec:gradient_descent}
\gls{gd};
% \glsxtrshort{gd}

\section{Support Vector Machine}
\gls{svm};
% % \input{./chapter/algorithms.tex}

% \part{Physics}
% % !TEX root = ../notes_template.tex
\chapter{Quantum Mechanics}\label{chp:quantum_mechanics}
\minitoc

\section{Hamiltonian}
\gls{hamiltonian};
% \glsxtrshort{qm};

\section{Path Integral}
\gls{lagrangian}

\section{Quantum Field Theory}
\gls{qft};
% % \input{./chapter/quantum_field_theory.tex}

% % \begin{appendices}
% % % !TEX root = ../notes_template.tex
\chapter{Formulas}

\section{Gaussian distribution}\label{sec:gaussian_distribution}
\begin{Definition}[Gaussian distribution]\label{def:gaussian_distribution}
    \myindex{Gaussian distribution}
\end{Definition}

\begin{theorem}[Central limit theorem]\label{thm:central_limit_theorem}
\end{theorem}
% % \end{appendices}

% \backmatter

% %%%%%%%%%%%%%%% Reference %%%%%%%%%%%%%%%

% \printbibliography[heading=bibintoc]
% \printindex
\end{spacing}
\end{document}