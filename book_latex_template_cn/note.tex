\documentclass[12pt,oneside,UTF8]{ctexbook} %电子版用oneside
\CTEXsetup[format={\Large\bfseries}]{section} % 小标题靠左

%%%%%%%%%%%%%%%%%%%%%%%
% – text font –
% compile using Xelatex
%%%%%%%%%%%%%%%%%%%%%%%
% – 中文字体 –
%\setmainfont{Microsoft YaHei} % 微软雅黑
%\setmainfont{YouYuan} % 幼圆
% \setmainfont{NSimSun} % 新宋体
% \setmainfont{KaiTi} % 楷体
% \setmainfont{SimSun} % 宋体
%\setmainfont{SimHei} % 黑体

\usepackage{setspace}

% \documentclass{article}


\usepackage[center]{titlesec}%chapter1修改为第1章
\titleformat{\chapter}{\raggedright\Huge\bfseries}{第\,\thechapter\,章}{1em}{}
\titleformat{\section}{\raggedright\Large\bfseries}{\,\thesection\,}{1em}{}
\titleformat{\subsection}{\raggedright\large\bfseries}{\,\thesubsection\,}{1em}{}


\usepackage{caption}
\captionsetup[figure]{name=图}

\newcommand\figref{\textbf{图}~\ref}
\newcommand\thmref{\textbf{定理}~\ref}
\newcommand\equref{\textbf{式}~\ref}
\newcommand\exaref{\textbf{例}~\ref}
\input{notes_template.tex}

%%%%%%%%%%%%%%%%%%% biblatex %%%%%%%%%%%%%%%%%

%%%%%%%%%%%%%%%%%%%%% glossaries %%%%%%%%%%%%%%%%%
\input{./glossaries.tex}
%%%%%%%%%%%%%%%%%%%%% glossaries %%%%%%%%%%%%%%%%%

%%%%%%%%%%%%%%%%%%%%% glossaries-extra %%%%%%%%%%%%%%%%%
% \usepackage[record,abbreviations,symbols,stylemods={list,tree,mcols}]{glossaries-extra}
%%%%%%%%%%%%%%%%%%%%% glossaries-extra %%%%%%%%%%%%%%%%%


\input{./macros.tex}

%%%%%%%%%%%%%%%%%%%%%%%%%%%%%%%%%%%%%%%%%%%%%%%%%%
%%%%%%%%%%%%%%%% begin of document %%%%%%%%%%%%%%%
%%%%%%%%%%%%%%%%%%%%%%%%%%%%%%%%%%%%%%%%%%%%%%%%%%

\begin{document}

\title{\bf \huge Study Notes of linear programming}
\author{钟沛,汪智笑,费爱跃}
\date{Update on \today}

\maketitle



\tableofcontents
\begin{spacing}{1.5}
% 这里是1.5倍行距
%%%%%%%%%%%%%%%%%%%preface%%%%%%%%%%%%%%%%%%

\chapter*{前言}


该笔记主要参考书籍为:
\begin{itemize}
    \item Model Building in Mathematical Programming 5th Edition
    \item introduction to linear programming
\end{itemize}

\par
笔记中所有代码用python编写,代码可在以下网址访问:
\begin{itemize}
    \item -
\end{itemize}


%%%%%%%%%%%%%%%%%%preface end%%%%%%%%%%%%%%%%%


\part{线性规划理论}
%%%%%%%%%%%%%%%%%%linear programming theory%%%%%%%%%%%
\chapter{线性规划}\label{chp:线性规划}

\section{线性规划简述}

\section{标准型}

\section{图解法}
考虑以下具有两个决策变量的线性规划问题:


\begin{equation}
  \begin{aligned}
\min -x_1-x_2\\
s.t.\     x_1+2x_2 &\leq 3 & (a)\\
          2x_1+x_2 &\leq 3& (b)\\
		x_1,x_2&\geq 0&(c)
  \end{aligned}
  \label{equ:example1}
\end{equation}

该问题的可行解区域如\figref{fig:feasible_region1}所示.

\begin{figure}[htbp]
    \centering
    \includegraphics[width=0.6\textwidth]{figure/ch1/feasible_region1.png}
    \caption{}
    \label{fig:feasible_region1}
\end{figure}

利用图解法直观地说明最优解的候选解由有限个角点给出. 

下面对线性问题的几何特性进行更加深入的研究,首先给出超平面,半平面和多面体的定义.

\begin{definition}{多面体}{}
    多面体是由$\{\rm x\in \R^n|\rm Ax\geq b\}$中的点组成的点集,$\rm A$是一个$m\times n$矩阵,$\rm b$是一个$\R^m$向量.
    
\end{definition}

\begin{definition}{超平面和半平面}{}
    
    $\rm a$是$\rm\R^n$中的非零向量,$b$是一个常数.
    集合$\{x\in R^n|\rm a^Tx=b\}$称为超平面.
    集合$\{x\in R^n|\rm a^Tx\geq b\}$称为半平面.
    
\end{definition}

\begin{remark}
    可以注意到,超平面是半平面的边界. 
    另外,超平面定义中的向量$\rm a$与超平面垂直. 对于超平面上的两个$\R^n$向量$x, y$,
    有$a^Tx=a^Ty$,则$a^T(x-y)=0$,则向量$a$与向量$x-y$的内积为0,即$a$垂直于$x-y$,
    由于超平面的任意向量均可以由两个向量相减得到,
    则$a$垂直于超平面上的任意向量,则向量$a$垂直于超平面.
\end{remark}

%%%%%%%%%%%%%%%%%%%linear programming end%%%%%%%%%%%%%%%

\begin{theorem}{Bbbbb}{bbbbb}
    blalalal
\end{theorem}

\begin{thmproof}
    xxx
\end{thmproof}

\begin{example}{}{example1}
    ppppp
\end{example}

\begin{exasolution}
    xxx
\end{exasolution}

交叉引用\thmref{thm:bbbbb}

交叉引用\equref{equ:example1}

交叉引用\exaref{exa:example1}

\part{线性规划应用选讲}
%%%%%%%%%%%%%%%%%%%%linear programming application%%%%%%%%%%%%%%





%%%%%%%%%%%%%%%%%%%%linear programming application end%%%%%%%%%%%%%


%%%%%%%%%%%%%%%Content%%%%%%%%%%%%%%%
% % \mainmatter % separat the number of toc and mainmatter
% \chapter*{Preface}

The notes mainly refer to:
\begin{itemize}
    \item \href{}{A Course in Functional Analysis 2nd}
    \item \href{https://users.math.msu.edu/users/banelson/teaching/920/chI_notes.pdf}{lecture notes from Michigan university}
    \item \href{https://www.math.cuhk.edu.hk/course_builder/1415/math5011/functional%20Analysis%202014.pdf}{lecture notes from cuhk}
    \item \href{https://thichchaytron.files.wordpress.com/2013/10/functional-problems-anhle-full-www-mathvn-com.pdf}{Functional Analysis Problems with Solutions}
    \item \href{https://www.math.utoronto.ca/mnica/oral/func_notes.pdf}{lecture notes by Mihai Nica}
    \item \href{https://users.math.msu.edu/users/banelson/conferences/GOALS/notes/KU%20Leuven.pdf}{Functional Analysis}
    \item \href{https://web.pdx.edu/~erdman/CFA/functional_analysis_pdf.pdf}{Companion to Functional Analysis}
    \item \href{https://www.ucl.ac.uk/~ucahad0/3103_handout_1.pdf}{(Functional Analysis}
\end{itemize}


Homework:\\
$I_1$: T2, T6\\
$I_2$: T2, T3\\
$I_3$: T3, T5\\
$I_4$: T13, T19\\
$I_5$: T6, T9\\
$II_1$: T9, T11\\
$II_2$: T12, T15\\
$II_3$: T4, T6\\
$II_4$: T4, T8\\
$III_1$: T4, T5\\
$III_2$: T1, T4\\
$III_3$: T2\\
$III_4$: T1, T3\\
$III_5$: T2\\
$III_6$: T1, T2\\


% \part{Mathematics}

% \input{./chapter/discrete_math.tex}

% \part{Computer Science}
% % \input{./chapter/complexity.tex}
% \input{./chapter/machine_learning.tex}
% % \input{./chapter/algorithms.tex}

% \part{Physics}
% \input{./chapter/quantum_mechanics.tex}
% % \input{./chapter/quantum_field_theory.tex}

% % \begin{appendices}
% % \input{./chapter/appendix_formula.tex}
% % \end{appendices}

% \backmatter

% %%%%%%%%%%%%%%% Reference %%%%%%%%%%%%%%%

% \printbibliography[heading=bibintoc]
% \printindex

\end{spacing}
\end{document}