% !TEX root = ../notes_template.tex
\chapter*{Preface}


Notes mainly refer to the following resources:
Convex optimization and duality:
\begin{itemize}
    \item \href{https://web.stanford.edu/~boyd/cvxbook/}{Convex Optimizaiton. Stephen Boyd, Lieven Vandenberghe}
    \item \href{https://link.springer.com/book/10.1007/978-0-387-40065-5}{Numerical Optimization(2nd). Jorge Nocedal, Stephen J.Wright} 
    \item \href{https://sites.gatech.edu/ece-6270-fall-2022/}{Convex Optimization: Theory, Algorithms, and Applications}
    \item \href{https://www.math.cuhk.edu.hk/course_builder/2324/math4230/}{cuhk optimization course reference}
    \item \href{https://cdn.syscop.de/publications/Diehl2016.pdf}{lecture notes from University of Freiburg}
    \item \href{https://faculty.ucmerced.edu/mcarreira-perpinan/teaching/EECS260/lecture-notes.pdf}{lecture notes from University of California}
    \item \href{https://www.uio.no/studier/emner/matnat/math/nedlagte-emner/MAT-INF2360/v13/matinf2360part3.pdf}{Lecture notes for the course MAT-INF2360}
    \item \href{https://www.numerical.rl.ac.uk/people/nimg/msc/lectures/paper.pdf}{An introduction to algorithms for nonlinear optimization} 
    \item \href{https://www.princeton.edu/~aaa/Public/Teaching/ORF523/S16/}{ORF 523 Convex and Conic Optimization}
    \item \href{}{lecture notes from School of Electrical Engineering at Korea Advanced Institute of Science and Technology}
    \item \href{https://www.stat.cmu.edu/~ryantibs/convexopt-F18/}{Convex Optimization: Fall 2018}
\end{itemize}


application:

\begin{itemize}
    \item \href{https://web.eecs.umich.edu/~fessler/course/598/}{Optimization methods for signal and image processing and machine learning}
\end{itemize}



% In Part I, we will study the basic concepts and several
% mathematical definitions required to understand what convex optimization is as well as how to
% translate an interested problem into a convex problem. We will then explore five instances of
% convex optimization problems: LP, least-squares, QP, SOCP and SDP. Specifically we will focus
% on techniques which serve recognizing (and translating to) such problems. We will also study
% some prominent algorithms for solving such problems. 
% \par
% In Part II, we will study one of the key
% theories in the optimization field, called duality. There are two types of dualities: (1) strong
% duality; (2) weak duality. It turns out that the strong duality is quite useful for gaining some
% algorithmic insights for convex problems. 
% The weal duality helps dealing with difficult non-convex problems, by providing an approximated solution.
% \par 
% In part III, our focus is on continuous optimization (rather than discrete optimization) with special emphasis on nonlinear programming.
% We will Utilize the knowledge of PartI and PartII to mine the optimal solution conditions and 
% solving methods for nonlinear optimization problems. 






