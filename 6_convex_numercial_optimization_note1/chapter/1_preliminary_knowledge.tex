\chapter{Preliminary Knowledge}

The goal of this chapter is to refresh your memory on some
topics in linear algebra and multivariable calculus that will be
relevant to the following content. You can use this as a reference
throughout the semester. 

\section{Inner Products and Norms}
\subsection{Inner Products}
Without hypotheses, we use $x\in \R^n$ to denote vectors of $\R^n$ with $x=(x_1,x_2,\cdots,x_n)$ and $x\geqs y$ to denote point-wise order $x_i\geqs y_i,1\leqs i \leqs n$. Then, $\R^n_+$ and $\R^n_{++}$ are defined by $\R^n_+:=\{x\in \R^n\mid x\geqs 0\}$ and  $\R^n_+:=\{x\in \R^n\mid x> 0\}$. Besides, $e_i$ is the $i$-th 
standard basis $e_i=(0,\cdot,0,\mathop{1}\limits^{i},0,\cdots,0)^T$.
\begin{definition}{Inner product and vector norm}{Inner product}
    (i) For given vectors $x,y\in \R^n$, their \textbf{inner product} and the induce \textbf{Euclidean norm}(or, \textbf{$l^2$-norm}) are defined by
    \begin{align*}
        \inner{x}{y} &:=x^Ty = \sum_{i=1}^nx_iy_i\\
        \norm{x}_2 &:=(\inner{x}{x})^{\frac{1}{2}} = \left(\sum\limits_{i=1}^nx_i^2\right)^{\frac{1}{2}}. 
    \end{align*}

    (ii) For a given vector $x\in \R^n$, the \textbf{$l^1$-norm} and \textbf{$l^{\infty}$-norm} are defined by
    \begin{align*}
        \norm{x}_1=\sum_{i=1}^n|x_i|,\norm{x}_{\infty}=\max_{1\leqs i \leqs n}|x_i|.
    \end{align*}
\end{definition}
\begin{proposition}{Fundamental properties of norm}{}
    For $l^p$-norm $\norm{\cdot}_p,p=1,2,\infty$ of the vectors $x,y\in \R^n$, one have

    (i) Positive defininteness: $\norm{x}_p\geqs 0$ and $\norm{x}_p=0\Leftrightarrow x = 0$.

    (ii) Positive Homogeneity: $\norm{kx}_p = k\norm{x}_p$ for $k\geqs 0$.

    (iii) Subadditivity: $\norm{x+y}_p\leqs \norm{x}_p + \norm{y}_p$.
\end{proposition}
\begin{remark}
    Essentially, any functional $p:\R^n\to \R_+$ satisfying the above three properties is a norm of $\R^n$.
\end{remark}

The following inequality plays a important role in theoretical analysis.
\begin{theorem}{Cauchy-Schwarz Inequality}{}
    \begin{equation*}
        \inner{x}{y}=x^Ty\leqs \norm{x}_2\norm{y}_2, x,y\in \R^n,
    \end{equation*}
    the equality holds if and only if $x = ky,k\in \R$ i.e. $x,y$ are parallel.
\end{theorem}

\subsection{Vector Norms}

\subsection{Matrix Norms}
Analogous to vector norm, we could define norm for matrices, here are some frequently used matrix norms:
\begin{definition}{matrix norms}{}
    For a given matrix $A\in \R^{n\times n},$ there are some norms as following:

    (i) \textbf{Frobenius norm}: $\norm{A}_F:=\left(\sum\limits_{i=1}^n\sum\limits_{j=1}^na_{ij}^2\right)^{\frac{1}{2}}$.

  (ii) \textbf{$2$-norm}, or \textbf{spectral norm}: $\norm{A}_2:=\max\limits_{\norm{x}_2=1}\norm{Ax}_2$.
\end{definition}

\begin{definition}{condition number}{}
    For a given matrix $A\in \R^{n\times n}$, the condition of $A$ is defined as
    \begin{align*}
        \kappa(A) = ||A||_2\cdot ||A^{-1}||_2,
    \end{align*}
    If $A$ is singular then $\kappa(A)=\infty$. 
    In numerical analysis the condition number of a matrix $A$ is is a way of describing how well or badly the system
    $Ax=b$ could be approximated. If $\kappa(A)$ is small the problem is well-conditioned and if
    $\kappa(A)$ is large the problem is rather ill-conditioned.
\end{definition}
\begin{remark}
    If $A$ is a symmetric matrix, another expression for the condition number is $\kappa(A) = |\frac{\lambda_{max}}{\lambda_{min}}|$, where $\lambda_{max}$ and $\lambda_{min}$
    denote the largest and smallest eigenvalues of $A$.
\end{remark}

\begin{theorem}{Spectral norm characterization}{equivalent characterization of 2-norm of matrix}
    For a given matrix $A\in \R^{n\times n}$,
    \begin{equation*}
        \norm{A}_2=\max_{\norm{x}=1}\norm{Ax}_2=\left(\lambda_{\max}(A^TA)\right)^{\frac{1}{2}},
    \end{equation*}
    in which $\lambda_{\max}(A)$ is the maximal eigenvalues of $A$. Furthermore, by notation $\lambda(A)$ denoted the set of all eigenvalues of $A$, we have $\norm{A}_2=\max(|\lambda(A)|)$ when $A$ is symmetric.
\end{theorem}
\begin{proof}
    By Cauchy-Schwarz inequality one have
    \begin{equation*}
        \norm{Ax}_2^2=\inner{Ax}{Ax}=x^T(A^TAx)\leqs \norm{x}_2\norm{A^TAx}_2=\norm{A^TAx}_2,\forall \ x\in \R^n,\norm{x}=1,
    \end{equation*}
    and the equality holds if and only if $x,A^TAx$ are parallel, that is $A^TAx =\lambda x$ for some $\lambda\in \R$ i.e. $x$ is an eigenvector of $A^TA$ associated with eigenvalue $\lambda$. Hence
    \begin{equation*}
        \max_{\norm{x}_2=1}\norm{Ax}_2\leqs \max_{\norm{x}_2=1}\left(\norm{x}_2\norm{A^TAx}_2\right)^{\frac{1}{2}}\leqs \max_{\norm{x}_2=1}\left(\norm{x}_2(\lambda\norm{x}_2)\right)^{\frac{1}{2}}\leqs (\lambda_{\max}(A^TA))^{\frac{1}{2}},
    \end{equation*}
    and there is eigenvector $x^*$ associated with maximal eigenvalue $\lambda_{\max}$ of $A^TA$ such that $\norm{Ax^*}_2=(\lambda_{\max}(A^TA))^{\frac{1}{2}}$, therefore $\max\limits_{\norm{x}_2=1}\norm{Ax}_2=(\lambda_{\max}(A^TA))^{\frac{1}{2}}$.

    When $A$ is symmetric, $\lambda_{\max}(A^TA)=\lambda_{\max}(A^2)=(\max(|\lambda(A)|))^2$, then the conclusion is obvious.
\end{proof}

The next property of $2$-norm is common in theoretical analysis. Its proof is similar to theorem \ref{thm:equivalent characterization of 2-norm of matrix} hence it is remained to an exercise.

\begin{corollary}{Compatibility of $2$-norm}{}
For a given $A\in \R^{n\times n},x\in \R^n$,

(i) $\norm{Ax}_2\leqs \norm{A}_2\norm{x}_2$.

(ii) When $A$ is symmetric, the quadratic form has the attained upper bound and lower bound
\begin{equation}
    \lambda_{\min}(A)\norm{x}_2\leqs x^TAx\leqs \lambda_{\max}(A)\norm{x}_2.
\end{equation}

(iii) As a corollary of (ii), $A$ is positive (semi-)definited if and only if $\lambda_{\min}(A)>0(\geqs 0)$.
\end{corollary}


\section{Quadratic Form}

The theoretical analysis and algorithm design always involve matrix, especially symmetric positive definited matrix. We briefly introduction the concept of quadratic form.
\begin{definition}{Quadratic form}{}
    Give a multivariate quadratic function $f:\R^n\to \R,$
    \begin{equation*}
        f(x_1,x_2,\cdots,x_n) = a_{11}x_1^2+\cdots+a_{nn}x^2_n+2a_{12}x_1x_2+\cdots + 2a_{n-1,n}x_{n-1}x_n =\sum_{i=1}^n\sum_{j=1}^na_{ij}x_ix_j,
    \end{equation*}
    one rewrite $f$ as 
    \begin{equation*}
        f(x) = \frac{1}{2}x^TAx,x=(x_1,\cdots,x_n)^T,A=\begin{bmatrix}
        a_{11}&a_{12}&\cdots & a_{1n}\\
        a_{12}&a_{22}&\cdots & a_{2n}\\
        \vdots&\vdots&\ddots&\vdots\\
        a_{1n}&a_{2n}&\cdots & a_{n n}\\
        \end{bmatrix},
    \end{equation*}
    such form is called \textbf{quadratic form}, the symmetric matrix $A$ is called coefficient matrix for $f$. Besides, a symmetric matrix $A$ is said to be \textbf{positive (semi-)definited} if quadratic form $\frac{1}{2}x^TAx>0(\geqs 0)$ for every nonzero vector $x\in \R^n,x\ne 0$.
\end{definition}
\begin{remark}
    Quadratic form also could be rewritten to inner product form $\frac{1}{2}x^TAx = \frac{1}{2}\inner{x}{Ax}$. Note that since $A$ is symmetric then $\inner{x}{Ax} = \inner{Ax}{x}$. Actually, by definition $\inner{x}{Ay}=\inner{A^Tx}{y}$ holds for every $x\in \R^n,y\in \R^m,A\in \R^{n\times m}$.
\end{remark}


\section{Differentiation}
\begin{definition}{Gradient and Hessian}{Gradient and Hessian}
    Let $f:\R^n\to \R$ be a multivariate function, the partially derivative of component $x_i$, is defined by 
    \begin{equation*}
        \frac{\partial f}{\partial x_i}:=\lim\limits_{t\downarrow 0}\dfrac{f(x + te_i)-f(x)}{t} = f_{i}'(x_i),
    \end{equation*}
    that is derivative of function $f(x)$ as a function of $x_i$ alone. By notation $f_{i}'(x)$ denoted derivative function $\frac{\partial f}{\partial x_i}$, the definition of second-order partially derivative is defined natrually,
    \begin{equation*}
        \dfrac{\partial^2 f}{\partial x_i\partial x_j} := \dfrac{\partial f_i'}{\partial x_j} = \dfrac{\partial }{\partial x_j}\left(\dfrac{\partial f}{\partial x_i}\right).
    \end{equation*}

    (i) The \textbf{gradient} of $f$, is vector consists of partially derivatives $\nabla f(x):= (\frac{\partial f}{\partial x_1},\cdots,\frac{\partial f}{\partial x_n})^T$.

    (ii) The \textbf{Hessian} of $f$, is matrix consists of second-order partially derivatives
    \begin{equation*}
        \nabla^2f(x):=\begin{bmatrix}
            \frac{\partial^2 f}{\partial x_1^2}&\frac{\partial^2 f}{\partial x_1\partial x_2}&\cdots&\frac{\partial^2 f}{\partial x_1\partial x_n}\\
            \frac{\partial^2 f}{\partial x_1\partial x_2}&\frac{\partial^2 f}{\partial x_2^2}&\cdots&\frac{\partial^2 f}{\partial x_2\partial x_n}\\
            \vdots&\vdots&\ddots&\vdots\\
            \frac{\partial^2 f}{\partial x_1\partial x_n}&\frac{\partial^2 f}{\partial x_2\partial x_n}&\cdots&\frac{\partial^2 f}{\partial x_n^2}
        \end{bmatrix}.
    \end{equation*}
\end{definition}
\begin{remark}
    Almost every case we considered, $\nabla^2 f(x)$ is symmetric i.e. $\frac{\partial^2 f}{\partial x_i\partial x_j}=\frac{\partial^2 f}{\partial x_j\partial x_i}$.
\end{remark}

\begin{definition}{Jacobian}{}
    Consider vector-valued function $f:\R^n\to \R^m,f=(f_1(x),\cdots,f_m(x))^T$, the \textbf{Jacobian} of $f$ is matrix consists of partially derivatives with dimension $m\times n$,
 \begin{equation*}
     J(f) = \begin{bmatrix}
            \frac{\partial f_1}{\partial x_1}&\frac{\partial f_1}{\partial x_2}&\cdots&\frac{\partial f_1}{\partial x_n}\\
             \frac{\partial f_2}{\partial x_1}&\frac{\partial f_2}{\partial x_2}&\cdots&\frac{\partial f_2}{\partial x_n}\\
            \vdots&\vdots&\ddots&\vdots\\
             \frac{\partial f_m}{\partial x_1}&\frac{\partial f_m}{\partial x_2}&\cdots&\frac{\partial f_m}{\partial x_n}
        \end{bmatrix}.
 \end{equation*}
\end{definition}
\begin{remark}
    One could see that $\nabla^2 f(x) = J(\nabla f(x))$.
\end{remark}



\begin{definition}{vector}{}
    A vector is a matrix with only one column. Thus, all vectors are inherently column vectors. 
\end{definition}

\begin{remark}
    We follows the convention of the gradient being a column vector, while the derivative is a row vector. 
\end{remark}

\begin{definition}{Jacobian Matrix}{Jacobian Matrix}
    Let $y=f(x)$ where $y$ is an $m$-element vector, and $x$ is an $n$-element vector. The symbol
    \begin{equation}\label{eq:Jacobian Matrix}
        \frac{\partial y}{\partial x} 
        = \begin{pmatrix}
            \frac{\partial y_1}{\partial x_1}& \frac{\partial y_1}{\partial x_2} & ... & \frac{\partial y_1}{\partial x_n}\\
            \frac{\partial y_2}{\partial x_1}& \frac{\partial y_2}{\partial x_2} & ... & \frac{\partial y_2}{\partial x_n}\\
            ...& ... & ... & ...\\
            \frac{\partial y_m}{\partial x_1}& \frac{\partial y_m}{\partial x_2} & ... & \frac{\partial y_m}{\partial x_n}
        \end{pmatrix}
    \end{equation}
    will denote the $m\times n$ matrix of first-order partial derivatives of the function from $x$ to $y$. Such a matrix is called the Jacobian martix of the function $f$. ($J_{ij}=\frac{\partial y_i}{\partial x_j}$)
\end{definition}

\begin{remark}
    If $x$ is actually a scalar in \eqref{eq:Jacobian Matrix}, then the resulting Jacobian matrix is a $m\times 1$ matrix;
    that is, a single column(a vector). 
    On the other hand, if $y$ is actually a scalar in \eqref{eq:Jacobian Matrix} , then the resulting Jacobian matrix is a $1\times n$ matrix; 
    that is, a single row (the transpose of a vector).
\end{remark}
\begin{remark}
    We follows the convention of\ $\frac{\partial y^T}{\partial x}=(\frac{\partial y}{\partial x})^T$
\end{remark}

\section{Approximation Method}
\subsection{First-order Approximation}

\subsection{Second-order Approximation}

\subsection{Approximation with Integral}



\section{Reference}
\begin{itemize}
    \item matrix norm: \href{https://www.sjsu.edu/faculty/guangliang.chen/Math253S20/lec7matrixnorm.pdf}{lecture notes from sjsu}
    \item quadratic form: \href{https://ocw.mit.edu/courses/15-084j-nonlinear-programming-spring-2004/resources/lec4_quad_form/}{lecture notes from mit}
    \item understand matrix norm using function analysis : \href{https://sites.math.washington.edu/~greenbau/Math_554/Course_Notes/ch1.3new.pdf}{lecture notes from }
    \item matrix differentiation: \href{https://atmos.washington.edu/~dennis/MatrixCalculus.pdf}{matrix differentiation}
    \item matrix derivative: \href{https://www.doc.ic.ac.uk/~ahanda/referencepdfs/MatrixCalculus.pdf}{matrix derivative}
\end{itemize}

% \begin{itemize}
%     \item $x=(x_1,x_2,...,x_n)^T$, $x_i\in \R$
%     \item $F(\alpha)=(h_1(\alpha),h_2(\alpha),...,h_n(\alpha))^T$, $h_i:\R\rightarrow \R$, $F:\R^n\rightarrow \R$
%         \begin{itemize}
%             \item[*] $\frac{dF}{d\alpha}=(\frac{dh_1}{d\alpha},\frac{dh_2}{d\alpha},...,\frac{dh_n}{d\alpha})^T$ 
%         \end{itemize}
%     \item $f(x),f:\R^n\rightarrow \R$
%         \begin{itemize}
%             \item[*] $\frac{\partial f}{\partial x}=(\nabla f(x))^T=(\frac{\partial f}{\partial x_1},\frac{\partial f}{\partial x_2},...,\frac{\partial f}{\partial x_n})$
%             \item[*] 
%             \begin{align*}
%                 \frac{\partial^2 f}{\partial x^2}=\frac{\frac{\partial f}{\partial x}}{\partial x}
%                                 &= \begin{pmatrix}
%                                     \frac{\partial f}{\partial x_1\partial x_1}& \frac{\partial f}{\partial x_1\partial x_2} & ... & \frac{\partial f}{\partial x_1\partial x_n}\\
%                                     \frac{\partial f}{\partial x_2\partial x_1}& \frac{\partial f}{\partial x_2\partial x_2} & ... & \frac{\partial f}{\partial x_2\partial x_n}\\
%                                     ...& ... & ... & ...\\
%                                     \frac{\partial f}{\partial x_n\partial x_1}& \frac{\partial f}{\partial x_2\partial x_n} & ... & \frac{\partial f}{\partial x_n\partial x_n}
%                                 \end{pmatrix}\\
%                                 &= \nabla^2 f(x)                                                                                                                
%             \end{align*}
%         \end{itemize}
%     \item $f(\alpha v)$, $\alpha\in \R$, $v\in \R^n$, $f:\R^n\rightarrow \R$
%     \begin{itemize}
%         \item[*] $\frac{\partial f}{\partial \alpha} = \frac{\partial f}{\partial x}\frac{\partial x}{\partial \alpha}=(\nabla f(x))^T\cdot v$, $x=\alpha v$. 
%         \item[*] $\frac{\partial^2 f}{\partial \alpha^2} = \frac{\frac{\partial f}{\partial x}v}{\partial \alpha}$, $\frac{}{}$
%     \end{itemize}
% \end{itemize}