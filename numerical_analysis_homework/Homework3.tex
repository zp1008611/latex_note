%\documentclass[13pt]{cctart}
 
%%%%%%%%%%%  以下文件头用于 Mac OS   中文 %%%%%%%%%%%
%\RequirePackage{amssymb} 
%\documentclass[nofonts,hyperref]{ctexart}
%\setCJKmainfont[BoldFont={STHeiti},ItalicFont={STKaiti}]{STSong}
%\setCJKsansfont{STHeiti}
%\setCJKmonofont{STFangsong}
%%%%%%%%%%%  以上文件头用于 Mac OS  %%%%%%%%%%%

%%%%%%%%%%  以下文件头用于 Mac OS   中文 %%%%%%%%%%%
\documentclass[12pt,twoside]{article}
\usepackage{amsfonts,amsmath,amssymb,mathrsfs}
\usepackage{epsf,graphicx,indentfirst,latexsym}
%%%%%%%%%%  以上文件头用于 Mac OS  %%%%%%%%%%%

 
\usepackage{flafter,graphicx,xcolor}
%\usepackage{trackchanges} 
%\usepackage{zhongmacroCtex} 
%\renewcommand{\initialsOne}{zhong}  
%\renewcommand{\initialsTwo}{liu}   %%
% Symbol

% Matlab
\usepackage{listings}
\lstset{language=Matlab}%代码语言使用的是matlab
\lstset{breaklines}%自动将长的代码行换行排版
\lstset{extendedchars=false}%解决代码跨页时,章节标题,页眉等汉字不显示的问题
\lstset{numbers=left, 
numberstyle= \tiny,keywordstyle= \color{ blue!70},commentstyle=\color{red!50!green!50!blue!50}, 
frame=shadowbox, rulesepcolor= \color{ red!20!green!20!blue!20}, 
escapeinside=``} 
%%%%%%%
\usepackage{multicol}
   
 \newtheorem{theorem}{Theorem}
 \newtheorem{program}{Program}[section]
 
\newcommand{\ep}{\hfill\rule{0.15cm}{0.35cm}\vskip 0.3cm}
 

%%%%%
 \makeatletter
\newenvironment{exercise}[1][{\color{blue}\bf Exercise}]%
%\newenvironment{xiti}[1][{\color{blue}\bf Exercise}]%
{%
 \begin{center}   \begin{lrbox}{\@tempboxa}%
    \begin{minipage}{\textwidth}%
  {\color{blue}\bfseries
#1}   }{%
    \end{minipage}%
    \end{lrbox}
    \colorbox{green}{\noindent\usebox{\@tempboxa}} \end{center}  
}
 
\newenvironment{analysis}[1][\color{blue}\bf Analysis]%
{%
 \begin{center}   \begin{lrbox}{\@tempboxa}%
    \begin{minipage}{\textwidth}
  {\color{blue}\bfseries
%#1} \\ \hspace*{2em} }{%
 #1}  }{%
    \end{minipage}%
    \end{lrbox}
    %\colorbox{gray}{\noindent\usebox{\@tempboxa}} \end{center}  %
    \fbox{\usebox{\@tempboxa}}\end{center}%
    %\colorbox{red}{\fbox{\noindent\usebox{\@tempboxa}} \end{center}
}
 
\newenvironment{solve}[1][\color{blue}\bf Solve]{\begin{trivlist}
\item[\hskip \labelsep {\color{blue}\bfseries
#1}]}{\hfill$\Box$\end{trivlist}}
 \makeatother
 
 
%%%%%%%%%
 
\renewcommand{\baselinestretch}{1.18}
 
 \pagestyle{plain} 
 
 \textwidth  168 true mm \textheight 230 true mm \topmargin=-1.3cm
\oddsidemargin   0pt \evensidemargin  20pt \marginparwidth  10pt

 
\begin{document}
%%%================  标  题   ================

\begin{center}
{\bf  THE SOUTH CHINA NORMAL UNIVERSITY\vspace{0.08cm}

School of Mathematical Sciences\vspace{0.08cm}
 
Numerical Analysis ( 2022--2023 The Second Term) \vspace{0.18cm}

{\Large Homework 3}\vspace{0.18cm}

Due Date: \underline{March 26, 2024 (Tuesday)} }
\end{center}\vspace{0.06cm}

\begin{center}
Name:\ \underline{\hspace{4cm}}\hspace{0.298cm}  
% 
Student No.:\ \underline{\hspace{4cm}} 
%
Date:\ \ \underline{March 21, 2024} 
 \end{center}

 
%%%================  正  文   ================

\begin{center}  \bf \Large
{\S 2.4 Exercises for Newton-Raphson and Secant Methods}
\end{center} 
  
\begin{exercise}1. \quad 
 Let $f(x) = x^2-x-3$. 
\begin{description}
\item[(a)] Find the Newton-Raphson formula $p_k = g (p_{k-1})$. 
\item[(b)] Start with $p_0 = 1.6$ and find $p_1, p_2$, and $p_3$. 
\item[(c)] Start with $p_0 = 0.0$ and find $p_1, p_2, p_3$, and $p_4$. What do you conjecture about this sequence?
\end{description}
 \end{exercise}
  
 
 
 
\begin{exercise}2. \quad
 Let $f(x) = x^3-3x-2$. 
\begin{description}
\item[(a)] Find the Newton-Raphson formula $p_k = g (p_{k-1})$. 
\item[(b)] Start with $p_0 = 2.1$ and find $p_1, p_2, p_3$, and $p_4$. 
\item[(c)] Is the sequence converging quadratically or linearly?
\end{description}
\end{exercise}
 
 \end{document}

 
 