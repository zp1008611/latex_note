%%%%%%%%%%%%%%%%%%%%%%%%%%%%%%%%%%%%%%%%%%%%%%%%%%
%%%%%%%%%%%%%%%%%%%%% preamble %%%%%%%%%%%%%%%%%%%
%%%%%%%%%%%%%%%%%%%%%%%%%%%%%%%%%%%%%%%%%%%%%%%%%%

\usepackage[mono=false]{libertine} % new linux font, ignore mono

\usepackage{luatex85}

%\renewcommand{\baselinestretch}{1.05}
\usepackage{amsmath,amsthm,amssymb,mathrsfs,amsfonts,dsfont}
\usepackage{epsfig,graphicx}
\usepackage{tabularx}
\usepackage{blkarray}
\usepackage{slashed}
\usepackage{color}
\usepackage{listings}
\usepackage{caption}
% \usepackage{fullpage}
\usepackage{lipsum} % provides dummy text for testing
\usepackage[toc,title,titletoc,header]{appendix}
\usepackage{minitoc}
\usepackage{color}
\usepackage{multicol} % two-col ToC
\usepackage{bm}
\usepackage{imakeidx} % before hyperref
\usepackage{hyperref}
\usepackage{indentfirst}
\setlength{\parindent}{2em}


% link colors settings
\hypersetup{
    colorlinks=true,
    citecolor=magenta,
    linkcolor=blue,
    filecolor=green,      
    urlcolor=cyan,
    % hypertexnames=false,
}
\usepackage[capitalise]{cleveref}
\usepackage{subcaption}
\usepackage{enumitem}
\usepackage{mathtools}
\usepackage{physics}
\usepackage[linesnumbered,ruled,vlined,algosection]{algorithm2e}
\SetCommentSty{textsf}
\usepackage{epigraph}
\epigraphwidth=1.0\linewidth
\epigraphrule=0pt

% adjust margin
\usepackage[margin=2.3cm]{geometry}
\headheight13.6pt


\usepackage{graphicx}
\usepackage[justification=centering]{caption} % 图注居中
\usepackage{setspace}
\usepackage{geometry}
\usepackage{float}
\usepackage{hyperref}
\usepackage[utf8]{inputenc}
\usepackage[english]{babel}
\usepackage{framed}


\newcommand{\HRule}[1]{\rule{\linewidth}{#1}}





\setstretch{1.2}
% \geometry{
%     textheight=9in,
%     textwidth=5.5in,
%     top=1in,
%     headheight=12pt,
%     headsep=25pt,
%     footskip=30pt
% }





%%%%%%%%%%%%%%%% thmtools %%%%%%%%%%%%%%%%%%%%%

\usepackage{thmtools}
\usepackage[dvipsnames]{xcolor}
\usepackage[most]{tcolorbox}
\usepackage{enumerate}

\colorlet{LightBlue}{Blue!15}
\colorlet{LightOrange}{Orange!15}
\colorlet{LightGreen}{Green!15}
\colorlet{LightRed}{Red!50}

\newtcbtheorem[
  number within = chapter % 按每个 chapter 分别编号
]{definition% 环境名
}{Definition% 这个参数可以设成“定理”“引理”“推论”等,编号就会变成“定理 1.1”“引理 1.1”“推论 1.1”等
}{
  attach title to upper = \par\vspace{1ex}, % 不要单独的标题栏,定理名完了之后分段,加上适量空白
  separator sign = \quad, % 定理编号和定理名字之间用什么分隔;默认是冒号
  sharp corners, % 直角;默认是圆角
  enhanced jigsaw, frame hidden, % 隐藏 tcb 边框
  colback = LightGreen, % 背景色
  coltitle = blue!20!cyan!80!black, % 标题(定理编号和名字)的颜色
  fonttitle = \sffamily\small, % 标题(定理编号和名字)的字体
  description font = \normalsize, % 定理名字的字体
  fontupper = \normalfont, % box 内的字体
}{def% label 前缀
}


\newtcbtheorem[
  number within = chapter % 按每个 chapter 分别编号
]{theorem% 环境名
}{Theorem% 这个参数可以设成“定理”“引理”“推论”等,编号就会变成“定理 1.1”“引理 1.1”“推论 1.1”等
}{
  attach title to upper = \par\vspace{1ex}, % 不要单独的标题栏,定理名完了之后分段,加上适量空白
  separator sign = \quad, % 定理编号和定理名字之间用什么分隔;默认是冒号
  sharp corners, % 直角;默认是圆角
  enhanced jigsaw, frame hidden, % 隐藏 tcb 边框
  colback = LightBlue, % 背景色
  coltitle = blue!20!cyan!80!black, % 标题(定理编号和名字)的颜色
  fonttitle = \sffamily\small, % 标题(定理编号和名字)的字体
  description font = \normalsize, % 定理名字的字体
  fontupper = \normalfont, % box 内的字体
}{thm% label 前缀
}


% \declaretheorem[numberwithin=chapter,shaded={rulecolor=LightGreen,
% rulewidth=2pt,bgcolor=LightGreen,
% textwidth=12em}]{definition}

\usepackage{changepage}
\newenvironment{remark}{\underline{\textbf{Remark.}}}{\par}

\newenvironment{proofsolution}
    {\renewcommand\qedsymbol{$\square$}\color{blue}\begin{adjustwidth}{0em}{2em}\begin{proof}[\textit Proof.~]}
    {\end{proof}\end{adjustwidth}}


%%%%%%%%%%%%%%%% index %%%%%%%%%%%%%%%%%%%%%
\begin{filecontents}{index.ist}
% https://tex.stackexchange.com/questions/65247/index-with-an-initial-letter-of-the-group
headings_flag 1
heading_prefix "{\\centering\\large \\textbf{"
heading_suffix "}}\\nopagebreak\n"
delim_0 "\\nobreak\\dotfill"
\end{filecontents}
\newcommand{\myindex}[1]{\index{#1} \emph{#1}}
\makeindex[columns=3, intoc, title=Alphabetical Index, options= -s index.ist]
%%%%%%%%%%%%%%%% index %%%%%%%%%%%%%%%%%%%%%

%%%%%%%%%%%%%%%% ToC %%%%%%%%%%%%%%%%%%%%%
% Link Chapter title to ToC: https://tex.stackexchange.com/questions/32495/linking-the-section-text-to-the-toc
\usepackage[explicit]{titlesec}
\titleformat{\chapter}[display]
  {\normalfont\huge\bfseries}{\chaptertitlename\ {\thechapter}}{20pt}{\hyperlink{chap-\thechapter}{\Huge#1}
\addtocontents{toc}{\protect\hypertarget{chap-\thechapter}{}}}
\titleformat{name=\chapter,numberless}
  {\normalfont\huge\bfseries}{}{-20pt}{\Huge#1}

%%%%%%%%%%%%%%%%%%% fancyhdr %%%%%%%%%%%%%%%%%
\usepackage{fancyhdr}
\pagestyle{fancy} % enable fancy page style
\renewcommand{\headrulewidth}{0.0pt} % comment if you want the rule
\fancyhf{} % clear header and footer
\fancyhead[lo,le]{\leftmark}
\fancyhead[re,ro]{\rightmark}
\fancyfoot[CE,CO]{\hyperref[toc-contents]{\thepage}}

% https://tex.stackexchange.com/questions/550520/making-each-page-number-link-back-to-beginning-of-chapter-or-section
\makeatletter
\def\chaptermark#1{\markboth{\protect\hyper@linkstart{link}{\@currentHref}{Chapter \thechapter ~ #1}\protect\hyper@linkend}{}}
\def\sectionmark#1{\markright{\protect\hyper@linkstart{link}{\@currentHref}{\thesection ~ #1}\protect\hyper@linkend}}
\makeatother
%%%%%%%%%%%%%%%%%%% fancyhdr %%%%%%%%%%%%%%%%%


%%%%%%%%%%%%%%%%%%% biblatex %%%%%%%%%%%%%%%%%
\usepackage[doi=false,url=false,isbn=false,style=alphabetic,backend=biber,backref=true]{biblatex}
\addbibresource{bib.bib}

\newbibmacro{string+doiurlisbn}[1]{%
  \iffieldundef{doi}{%
    \iffieldundef{url}{%
      \iffieldundef{isbn}{%
        \iffieldundef{issn}{%
          #1%
        }{%
          \href{http://books.google.com/books?vid=ISSN\thefield{issn}}{#1}%
        }%
      }{%
        \href{http://books.google.com/books?vid=ISBN\thefield{isbn}}{#1}%
      }%
    }{%
      \href{\thefield{url}}{#1}%
    }%
  }{%
    \href{http://dx.doi.org/\thefield{doi}}{#1}%
  }%
}

% https://tex.stackexchange.com/questions/94089/remove-quotes-from-inbook-reference-title-with-biblatex
\DeclareFieldFormat[article,incollection,inproceedings,book,misc]{title}{\usebibmacro{string+doiurlisbn}{\mkbibemph{#1}}}
% https://tex.stackexchange.com/questions/454672/biblatex-journal-name-non-italic
\DeclareFieldFormat{journaltitle}{#1\isdot}
\DeclareFieldFormat{booktitle}{#1\isdot}
% https://tex.stackexchange.com/questions/10682/suppress-in-biblatex
\renewbibmacro{in:}{}
% add video field: https://tex.stackexchange.com/questions/111846/biblatex-2-custom-fields-only-one-is-working
\DeclareSourcemap{
    \maps[datatype=bibtex]{
      \map{
        \step[fieldsource=video]
        \step[fieldset=usera,origfieldval]
    }
  }
}
\DeclareFieldFormat{usera}{\href{#1}{\textsc{Online video}}}
\AtEveryBibitem{
    \csappto{blx@bbx@\thefield{entrytype}}{% put at end of entry
        \iffieldundef{usera}{}{\space \printfield{usera}}
    }
}


%%%%%%%%%%%%%%%%%%%%%%%notations%%%%%%%%%%%%%%%%%%%%%%%%%%%%%%
\newcommand{\C}{\ensuremath{\mathbb{C}}} 
\newcommand{\R}{\ensuremath{\mathbb{R}}}
\newcommand{\J}{\ensuremath{\mathbb{J}}}
\newcommand{\Q}{\ensuremath{\mathbb{Q}}}
\newcommand{\Z}{\ensuremath{\mathbb{Z}}}
\newcommand{\N}{\ensuremath{\mathbb{N}}}
\newcommand{\K}{\ensuremath{\mathbb{K}}}
\newcommand{\Zo}{\ensuremath{\mathbb{Z}_{\geqslant 0}}} % 非负整数集
\newcommand{\Zi}{\ensuremath{\mathbb{Z}_{\geqslant 1}}} % 正整数集
\newcommand{\id}{\mathrm{id}}
\newcommand{\im}{\mathrm{im}\,}                         % 映射的像
