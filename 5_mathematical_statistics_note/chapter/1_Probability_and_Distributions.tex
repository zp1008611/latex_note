\chapter{Probability and Distributions}

\section{Introduction}
\begin{definition}{}{}
    If an experiment can be repeated under the same conditions 
    it is a random experiment. The set of every possible outcome 
    of an experiment is the sample space, denoted $\mathcal{C}$.
\end{definition}
\begin{remark}
    For an experiment, the sample space is not unique.
    For example, When talking about the temperature in an area, 
    we can define the sample space as $\mathcal{C}=(-\infty,\infty)$ or $\mathcal{C}=[a,b]$.
    For a specific random experiment, we can use different sample spaces to describe it. 
    However, it is worth studying how to describe it with an appropriate sample space.
\end{remark}
\par
\textbf{Note/Definition}.
Notationally, we denote the elements of the sample space with
lower case letters such as $a,b,c$. Subsets of the sample space are \textit{events}
and we denote them with upper case letters such as $A,B,C$.

\begin{definition}{}{}
    If an experiment is performed $N$ times and a specific event occurs $f$ times,
    then $f$ is the frequency of the event and $f/N$ is the relative frequency of the event.
\end{definition}


\section{Sets}

\section{The Probability Set Function}

We need to define a set function that assigns a probability to the events (subsets of sample space $\mathcal{C}$).
We denote the colletion of events as $\mathcal{B}$.
If $\mathcal{C}$ is finite set, then we hope to assign a 
probability to all events (that is, to define a probability set function on the power set of $\mathcal{C}$).
More generally, we require that $\mathcal{B}$ (the colletion of events) to satisfy:
(1) the sample space $\mathcal{C}$ itself is an event,
(2) the complement of every event is again an event, and
(3) every countable union of events is again an event.
Symbolically, this means 
(1) $\mathcal{C}\in\mathcal{B}$,
(2) if $A\in\mathcal{B}$ then $A^c\in \mathcal{B}$, and
(3) if $A_1,A_2,...\in \mathcal{B}$ then $\cup_{n=1}^{\infty} A_n\in \mathcal{B}$.
Combining $(2)$ and $(3)$, we see by DeMorgan's Law (for countable unions)
that if $A_1,A_2,...\in\mathcal{B}$ then $\cap_{n=1}^{\infty}A_n\in \mathcal{B}$.
So the collection of events $\mathcal{B}$ is closed under complements, countable unions, 
and countable intersections.
Such a collection of sets form a \textit{$\sigma$-algebra}.


\begin{definition}{}{}
    A collection of events $\{A_n|n\in I\}$ (where $I$ is some indexing set)
    such that $A_i\cap A_j=\O$ is a mutually exclusive collection of events.
\end{definition}


\begin{definition}{}{}
    Let $\mathcal{C}$ be a sample space and let $\mathcal{B}$ be the set of all events (thus, $\mathcal{B}$ is a $\sigma$-field).
    Let $P$ be a real-valued function defined on $\mathcal{B}$.
    Then $P$ is a probability set function if $P$ satisfies the following three conditions:\\
    (1) $P(A)\geqs 0$ for $A\in \mathcal{B}$.\\
    (2) $P(\mathcal{C})=1$.\\
    (3) If $\{A_n\}$ is a mutually exclusive collection of events, then $P(\cup_{n=1}^{+\infty}A_n)=\sum\limits_{n=1}^{+\infty}P(C_n)$.
\end{definition}

\begin{theorem}{}{}
    For each event $A\in\mathcal{B}$, $P(A)=1-P(A^c)$.
\end{theorem}

\begin{theorem}{}{}
    The probability of the null set is zero; that is, $P(\O)=0$.
\end{theorem}

\begin{theorem}{}{}
    If $A$ and $B$ are events such that $A\subset B$,
    then $P(A)\leqs P(B)$.
\end{theorem}

\begin{theorem}{}{}
    For each event $A\in\mathcal{B}$ we have $0\leqs P(A)\leqs 1$.
\end{theorem}

\begin{theorem}{}{}
    If $A$ and $B$ are events in $\mathcal{C}$,
    then $P(A\cup B)=P(A)+P(B)-P(A\cap B)$.
\end{theorem}

\begin{theorem}{}{}
    Let $\{A_n\}$ be a nondecreasing sequence of events (ie. $A_n\subseteq A_{n+1}$). Then
    \begin{align*}
        \lim_{n\rightarrow \infty} P(A_n)=P(\lim_{n\rightarrow \infty}A_n)=P(\cup_{n=1}^{\infty}A_n).
    \end{align*}
    Let $\{A_n\}$ be a nonincreasing sequence of events (ie. $A_n\supseteq A_{n+1}$). Then
    \begin{align*}
        \lim_{n\rightarrow \infty} P(A_n)=P(\lim_{n\rightarrow \infty}A_n)=P(\cap_{n=1}^{\infty}A_n).
    \end{align*}
\end{theorem}

\begin{theorem}{}{}
    Let $\{A_n\}$ be an arbitrary sequence of events. Then
    \begin{align*}
        P(\cup_{n=1}^{\infty}A_n)\leqs \sum\limits_{n=1}^{\infty}P(A_n).
    \end{align*}
\end{theorem}



\section{Homework}

\begin{exercise}{}{}
    Show that the moment generating function of the random variable $X$
    having the pdf $f(x)=\frac{1}{3}$, $-1<x<2$, zero elsewhere, is
    \begin{align*}
        M(t) = \left\{\begin{matrix}
            \frac{e^{2t}-e^{-t}}{3t} & t\neq 0\\
            1 & t=0.
           \end{matrix}\right.
    \end{align*}
\end{exercise}

\section{Reference}
\begin{itemize}
    \item \href{https://faculty.etsu.edu/gardnerr/4047/notes-Hogg-McKean-Craig.htm}{lecture note}
    \item \href{https://ccrgpages.rit.edu/~whelan/courses/2013_3fa_STAT_405/notes01.pdf}{Probability and Distributions}
    \item \href{https://www.zhihu.com/question/28624845}{Sample space is unique?}
    \item \href{https://faculty.etsu.edu/gardnerr/4047/Beamer-Hogg-McKean-Craig/Proofs-HMC-1-3.pdf}{proof of 1.3}
\end{itemize}
