\chapter{Probability and Distributions}

\section{Introduction}
\begin{definition}{}{}
    If an experiment can be repeated under the same conditions 
    it is a random experiment. The set of every possible outcome 
    of an experiment is the sample space, denoted $\mathcal{C}$.
\end{definition}
\begin{remark}
    For an experiment, the sample space is not unique.
    For example, When talking about the temperature in an area, 
    we can define the sample space as $\mathcal{C}=(-\infty,\infty)$ or $\mathcal{C}=[a,b]$.
    For a specific random experiment, we can use different sample spaces to describe it. 
    However, it is worth studying how to describe it with an appropriate sample space.
\end{remark}
\par
\textbf{Note/Definition}.
Notationally, we denote the elements of the sample space with
lower case letters such as $a,b,c$. Subsets of the sample space are \textit{events}
and we denote them with upper case letters such as $A,B,C$.

\begin{definition}{}{}
    If an experiment is performed $N$ times and a specific event occurs $f$ times,
    then $f$ is the frequency of the event and $f/N$ is the relative frequency of the event.
\end{definition}


\section{Sets}

\section{The Probability Set Function}

We need to define a set function that assigns a probability to the events (subsets of sample space $\mathcal{C}$).
We denote the colletion of events as $\mathcal{B}$.
If $\mathcal{C}$ is finite set, then we hope to assign a 
probability to all events (that is, to define a probability set function on the power set of $\mathcal{C}$).
More generally, we require that $\mathcal{B}$ (the colletion of events) to satisfy:
(1) the sample space $\mathcal{C}$ itself is an event,
(2) the complement of every event is again an event, and
(3) every countable union of events is again an event.
Symbolically, this means 
(1) $\mathcal{C}\in\mathcal{B}$,
(2) if $A\in\mathcal{B}$ then $A^c\in \mathcal{B}$, and
(3) if $A_1,A_2,...\in \mathcal{B}$ then $\cup_{n=1}^{\infty} A_n\in \mathcal{B}$.
Combining $(2)$ and $(3)$, we see by DeMorgan's Law (for countable unions)
that if $A_1,A_2,...\in\mathcal{B}$ then $\cap_{n=1}^{\infty}A_n\in \mathcal{B}$.
So the collection of events $\mathcal{B}$ is closed under complements, countable unions, 
and countable intersections.
Such a collection of sets form a \textit{$\sigma$-algebra}.


\begin{definition}{}{}
    A collection of events $\{A_n|n\in I\}$ (where $I$ is some indexing set)
    such that $A_i\cap A_j=\O$ is a mutually exclusive collection of events.
\end{definition}


\begin{definition}{}{}
    Let $\mathcal{C}$ be a sample space and let $\mathcal{B}$ be the set of all events (thus, $\mathcal{B}$ is a $\sigma$-field).
    Let $P$ be a real-valued function defined on $\mathcal{B}$.
    Then $P$ is a probability set function if $P$ satisfies the following three conditions:\\
    (1) $P(A)\geqs 0$ for $A\in \mathcal{B}$.\\
    (2) $P(\mathcal{C})=1$.\\
    (3) If $\{A_n\}$ is a mutually exclusive collection of events, then $P(\cup_{n=1}^{+\infty}A_n)=\sum\limits_{n=1}^{+\infty}P(C_n)$.
\end{definition}

\begin{theorem}{}{}
    For each event $A\in\mathcal{B}$, $P(A)=1-P(A^c)$.
\end{theorem}

\begin{theorem}{}{}
    The probability of the null set is zero; that is, $P(\O)=0$.
\end{theorem}

\begin{theorem}{}{}
    If $A$ and $B$ are events such that $A\subset B$,
    then $P(A)\leqs P(B)$.
\end{theorem}

\begin{theorem}{}{}
    For each event $A\in\mathcal{B}$ we have $0\leqs P(A)\leqs 1$.
\end{theorem}

\begin{theorem}{}{}
    If $A$ and $B$ are events in $\mathcal{C}$,
    then $P(A\cup B)=P(A)+P(B)-P(A\cap B)$.
\end{theorem}

\begin{theorem}{}{}
    Let $\{A_n\}$ be a nondecreasing sequence of events (ie. $A_n\subseteq A_{n+1}$). Then
    \begin{align*}
        \lim_{n\rightarrow \infty} P(A_n)=P(\lim_{n\rightarrow \infty}A_n)=P(\cup_{n=1}^{\infty}A_n).
    \end{align*}
    Let $\{A_n\}$ be a nonincreasing sequence of events (ie. $A_n\supseteq A_{n+1}$). Then
    \begin{align*}
        \lim_{n\rightarrow \infty} P(A_n)=P(\lim_{n\rightarrow \infty}A_n)=P(\cap_{n=1}^{\infty}A_n).
    \end{align*}
\end{theorem}

\begin{theorem}{}{}
    Let $\{A_n\}$ be an arbitrary sequence of events. Then
    \begin{align*}
        P(\cup_{n=1}^{\infty}A_n)\leqs \sum\limits_{n=1}^{\infty}P(A_n).
    \end{align*}
\end{theorem}

\section{Conditional Probability and Independence}
The idea behind conditional probability is that the initial sample space $\mathcal{C}$
has been replaced with some subset $A\subset \mathcal{C}$.

\begin{definition}{}{conditional probability}
    Let $B$ and $A$ be events with $P(A)>0$.
    Then the conditional probability of $B$ given $A$ as $P(B|A)=\frac{P(A\cap B)}{P(A)}$.
\end{definition}

\textbf{Note/Definition}. If $A$ and $B$ are events where $P(A)>0$ then 
$P(A\cap B)=P(A)P(B|A)$ by Definition \ref{def:conditional probability}.
This is called the multiplication rule also.

\begin{definition}{}{}
    Let $A$ and $B$ be two events. Then $A$ and $B$ are Independent is $P(A\cap B)=P(A)P(B)$.
\end{definition}

\section{Random variables}

\begin{definition}{}{}
    Consider a random experiment with a sample space $\mathcal{C}$.
    A function $X$ which assigns to each $c\in\mathcal{C}$ one and only one real number $X(c)=x$ is
    a random variable. The space (or range) of $X$ is the set of real numbers $\mathcal{D}=\{x|x=X(c) \text{ for some } c\in \mathcal{C}\}$.
    If $\mathcal{D}$ is a countable set then $X$ is a discrete random variable and 
    if $\mathcal{D}$ is an interval of real numbers then $X$ is a continuous random variable.
\end{definition}

\begin{definition}{}{}
    Let X be a random variable. 
    Then its cumulative distribution function (cdf) $F:\R\rightarrow [0,1]$
    is defined as follows:
    \begin{align*}
        F(x) = P(X\leqs x).
    \end{align*}
\end{definition}

\begin{theorem}{}{}
    
\end{theorem}

\section{Discrete Random Variables}

\section{Continuous Random Variables}

\section{Expectation of a Random Variable}

\section{Some Special Expectations}
\subsection{The Moment Generating Function}
Recall ethe McLaurin series
\begin{align*}
    f(\alpha)=e^{\alpha}=\sum\limits_{m=0}^{\infty}\frac{\alpha^m}{m!},
\end{align*}
if we write the random variable
\begin{align*}
    e^{tX}=\sum\limits_{m=0}^{\infty}\frac{t^m}{m!}X^m,
\end{align*}
then its expectation value defines something called the moment generating function (mgf)
\begin{align*}
    M(t)=E(e^{tX})=\sum\limits_{m=0}^{\infty} \frac{t^m}{m!}E(X^m).
\end{align*}
If we take the $m$th derivative of the mgf,
evaluated at $t=0$, we get the $m$th ($m\geqs 1$) moment:
\begin{align*}
    M^{m}(0)=E(X^m).
\end{align*}
For this to work, the mgf has to be defined in a neighborhood of the origin,
i.e., for $-h<t<h$ where $h>0$
is some positive number. 

\begin{definition}{}{}
    Let $X$ be a random variable such that for some $h>0$,
    the expectation of $e^{tX}$ exists for $-h<t<h$.
    The moment generating function (or mgf) of $X$ is the function $M(t)=E(e^{tX})$ for $-h<t<h$.
\end{definition}
\begin{remark}
    When a moment generating function exists, we must have for $t=0$
    that $M(0)=E(1)=1$.
\end{remark}

\section{Homework}

\begin{exercise}{}{}
    Show that the moment generating function of the random variable $X$
    having the pdf $f(x)=\frac{1}{3}$, $-1<x<2$, zero elsewhere, is
    \begin{align*}
        M(t) = \left\{\begin{matrix}
            \frac{e^{2t}-e^{-t}}{3t} & t\neq 0\\
            1 & t=0.
           \end{matrix}\right.
    \end{align*}
\end{exercise}

\begin{solve}
    For $t\neq 0$, 
    \begin{align*}
        M(t)=E(e^{tX})=\int_{-\infty}^{+\infty} e^{tx}f(x)dx = \int_{-1}^{2}\frac{1}{3}e^{tx}dx = \frac{1}{3}\frac{e^{tx}}{t}|_{x=-1}^{x=2}=\frac{e^{2t}-e^{-t}}{3t}.
    \end{align*}
    And $M(0) = 1$ when a moment generating function
    exists and so the result follows.
\end{solve}

\section{Reference}
\begin{itemize}
    \item \href{https://faculty.etsu.edu/gardnerr/4047/notes-Hogg-McKean-Craig.htm}{lecture note}
    \item \href{https://ccrgpages.rit.edu/~whelan/courses/2013_3fa_STAT_405/notes01.pdf}{Probability and Distributions}
    \item \href{https://www.zhihu.com/question/28624845}{Sample space is unique?}
    \item \href{https://faculty.etsu.edu/gardnerr/4047/Beamer-Hogg-McKean-Craig/Proofs-HMC-1-3.pdf}{proof of 1.3}
\end{itemize}
