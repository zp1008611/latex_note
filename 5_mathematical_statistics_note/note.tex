% \documentclass[11pt,twoside]{book} %纸质版用twoside
\documentclass[12pt,oneside]{book} %电子版用oneside
\usepackage{setspace}

% \documentclass{article}

%%%%%%%%%%%%%%%%%%%%%%%%%%%%%%%%%%%%%%%%%%%%%%%%%%
%%%%%%%%%%%%%%%%%%%%% preamble %%%%%%%%%%%%%%%%%%%
%%%%%%%%%%%%%%%%%%%%%%%%%%%%%%%%%%%%%%%%%%%%%%%%%%

\usepackage[mono=false]{libertine} % new linux font, ignore mono

\usepackage{luatex85}

%\renewcommand{\baselinestretch}{1.05}
\usepackage{amsmath,amsthm,amssymb,mathrsfs,amsfonts,dsfont}
\usepackage{epsfig,graphicx}
\usepackage{tabularx}
\usepackage{blkarray}
\usepackage{slashed}
\usepackage{color}
\usepackage{listings}
\usepackage{caption}
% \usepackage{fullpage}
\usepackage{lipsum} % provides dummy text for testing
\usepackage[toc,title,titletoc,header]{appendix}
\usepackage{minitoc}
\usepackage{color}
\usepackage{multicol} % two-col ToC
\usepackage{bm}
\usepackage{imakeidx} % before hyperref
\usepackage{hyperref}
\usepackage{indentfirst}
\setlength{\parindent}{2em}


% link colors settings
\hypersetup{
    colorlinks=true,
    citecolor=magenta,
    linkcolor=blue,
    filecolor=green,      
    urlcolor=cyan,
    % hypertexnames=false,
}
\usepackage[capitalise]{cleveref}
\usepackage{subcaption}
\usepackage{enumitem}
\usepackage{mathtools}
\usepackage{physics}
\usepackage[linesnumbered,ruled,vlined,algosection]{algorithm2e}
\SetCommentSty{textsf}
\usepackage{epigraph}
\epigraphwidth=1.0\linewidth
\epigraphrule=0pt

% adjust margin
\usepackage[margin=2.3cm]{geometry}
\headheight13.6pt


\usepackage{graphicx}
\usepackage[justification=centering]{caption} % 图注居中
\usepackage{setspace}
\usepackage{geometry}
\usepackage{float}
\usepackage{hyperref}
\usepackage[utf8]{inputenc}
\usepackage[english]{babel}
\usepackage{framed}


\newcommand{\HRule}[1]{\rule{\linewidth}{#1}}





\setstretch{1.2}
% \geometry{
%     textheight=9in,
%     textwidth=5.5in,
%     top=1in,
%     headheight=12pt,
%     headsep=25pt,
%     footskip=30pt
% }





%%%%%%%%%%%%%%%% thmtools %%%%%%%%%%%%%%%%%%%%%

\usepackage{thmtools}
\usepackage[dvipsnames]{xcolor}
\usepackage[most]{tcolorbox}
\usepackage{enumerate}

\colorlet{LightGreen}{Green!15} %def
\colorlet{LightBlue}{Blue!15} %thm
\colorlet{LightOrange}{Orange!15} %lem
\colorlet{LightGray}{Gray!15}  %prop
\colorlet{LightRed}{Red!40} %cor
\colorlet{LightYellow}{Yellow!15} %exa


% \newtcbtheorem[
%   number within = chapter % 按每个 chapter 分别编号
% ]{definition% 环境名
% }{Definition% 这个参数可以设成“定理”“引理”“推论”等,编号就会变成“定理 1.1”“引理 1.1”“推论 1.1”等
% }{
%   attach title to upper = \par\vspace{1ex}, % 不要单独的标题栏,定理名完了之后分段,加上适量空白
%   separator sign = \quad, % 定理编号和定理名字之间用什么分隔;默认是冒号
%   sharp corners, % 直角;默认是圆角
%   enhanced jigsaw, frame hidden, % 隐藏 tcb 边框
%   colback = LightGreen, % 背景色
%   coltitle = blue!20!cyan!80!black, % 标题(定理编号和名字)的颜色
%   fonttitle = \sffamily\small, % 标题(定理编号和名字)的字体
%   description font = \normalsize, % 定理名字的字体
%   fontupper = \normalfont, % box 内的字体
% }{def% label 前缀
% }

\newtcbtheorem[
  auto counter,number within = chapter % 按每个 chapter 分别编号
]{definition% 环境名
}{Definition% 这个参数可以设成“定理”“引理”“推论”等,编号就会变成“定理 1.1”“引理 1.1”“推论 1.1”等
}{
  sharp corners, % 直角;默认是圆角
  colback=Green!5,
  colframe=Green!50!black,
  fonttitle=\sffamily\small
}{def% label 前缀
}


% 计数器设置
\makeatletter
\renewcommand\theHtcb@cnt@definition{\thechapter.\arabic{tcb@cnt@definition}}
\makeatother

\newtcbtheorem[
  auto counter,number within = chapter % 按每个 chapter 分别编号
]{theorem% 环境名
}{Theorem% 这个参数可以设成“定理”“引理”“推论”等,编号就会变成“定理 1.1”“引理 1.1”“推论 1.1”等
}{
  sharp corners, % 直角;默认是圆角
  colback=yellow!10,
  colframe=yellow!50!black,
  fonttitle=\sffamily\small
}{thm% label 前缀
}
% 计数器设置
\makeatletter
\renewcommand\theHtcb@cnt@theorem{\thechapter.\arabic{tcb@cnt@theorem}}
\makeatother

\newtcbtheorem[
  auto counter,number within = chapter % 按每个 chapter 分别编号
]{proposition% 环境名
}{Proposition% 这个参数可以设成“定理”“引理”“推论”等,编号就会变成“定理 1.1”“引理 1.1”“推论 1.1”等
}{
  sharp corners, % 直角;默认是圆角
  colback=Red!5,
  colframe=Red!50!black,
  fonttitle=\sffamily\small
}{prop% label 前缀
}
% 计数器设置
\makeatletter
\renewcommand\theHtcb@cnt@proposition{\thechapter.\arabic{tcb@cnt@proposition}}
\makeatother

\newtcbtheorem[
  auto counter,number within = chapter % 按每个 chapter 分别编号
]{corollary% 环境名
}{Corollary% 这个参数可以设成“定理”“引理”“推论”等,编号就会变成“定理 1.1”“引理 1.1”“推论 1.1”等
}{
  sharp corners, % 直角;默认是圆角
  colback=Blue!5,
  colframe=Blue!50!black,
  fonttitle=\sffamily\small
}{cor% label 前缀
}

% 计数器设置
\makeatletter
\renewcommand\theHtcb@cnt@corollary{\thechapter.\arabic{tcb@cnt@corollary}}
\makeatother

\newtcbtheorem[
  auto counter,number within = chapter % 按每个 chapter 分别编号
]{lemma% 环境名
}{Lemma% 这个参数可以设成“定理”“引理”“推论”等,编号就会变成“定理 1.1”“引理 1.1”“推论 1.1”等
}{
  sharp corners, % 直角;默认是圆角
  colback=Gray!10,
  colframe=Gray!50!black,
  fonttitle=\sffamily\small
}{lem% label 前缀
}

% 计数器设置
\makeatletter
\renewcommand\theHtcb@cnt@lemma{\thechapter.\arabic{tcb@cnt@lemma}}
\makeatother


\newtcbtheorem[
  auto counter,number within = chapter % 按每个 chapter 分别编号
]{example}
{Example}%
  {
    enhanced, breakable,
    colback = white, colframe = purple, colbacktitle = purple,
    attach boxed title to top left = {yshift = -2mm, xshift = 5mm},
    boxed title style = {sharp corners},
    fonttitle=\sffamily\small
  }
{exa}

% 计数器设置
\makeatletter
\renewcommand\theHtcb@cnt@example{\thechapter.\arabic{tcb@cnt@example}}
\makeatother


\newtcbtheorem[
  auto counter,number within = chapter % 按每个 chapter 分别编号
]{exercise}
{Exercise}%
  {
    enhanced, breakable,
    colback = white, colframe = cyan, colbacktitle = cyan,
    attach boxed title to top left = {yshift = -2mm, xshift = 5mm},
    boxed title style = {sharp corners},
    fonttitle=\sffamily\small
  }
{exer}

% 计数器设置
\makeatletter
\renewcommand\theHtcb@cnt@exercise{\thechapter.\arabic{tcb@cnt@exercise}}
\makeatother


% \declaretheorem[numberwithin=chapter,shaded={rulecolor=LightGreen,
% rulewidth=2pt,bgcolor=LightGreen,
% textwidth=12em}]{definition}

\usepackage{changepage}
\newenvironment{remark}{\underline{\textbf{Remark.}}}{\par}

\newenvironment{proofsolution}
    {\renewcommand\qedsymbol{$\square$}\color{blue}\begin{adjustwidth}{0em}{2em}\begin{proof}[\textit Proof.~]}
    {\end{proof}\end{adjustwidth}}


%%%%%%%%%%%%%%%% index %%%%%%%%%%%%%%%%%%%%%
\begin{filecontents}{index.ist}
% https://tex.stackexchange.com/questions/65247/index-with-an-initial-letter-of-the-group
headings_flag 1
heading_prefix "{\\centering\\large \\textbf{"
heading_suffix "}}\\nopagebreak\n"
delim_0 "\\nobreak\\dotfill"
\end{filecontents}
\newcommand{\myindex}[1]{\index{#1} \emph{#1}}
\makeindex[columns=3, intoc, title=Alphabetical Index, options= -s index.ist]
%%%%%%%%%%%%%%%% index %%%%%%%%%%%%%%%%%%%%%

%%%%%%%%%%%%%%%% ToC %%%%%%%%%%%%%%%%%%%%%
% Link Chapter title to ToC: https://tex.stackexchange.com/questions/32495/linking-the-section-text-to-the-toc
\usepackage[explicit]{titlesec}
\titleformat{\chapter}[display]
  {\normalfont\huge\bfseries}{\chaptertitlename\ {\thechapter}}{20pt}{\hyperlink{chap-\thechapter}{\Huge#1}
\addtocontents{toc}{\protect\hypertarget{chap-\thechapter}{}}}
\titleformat{name=\chapter,numberless}
  {\normalfont\huge\bfseries}{}{-20pt}{\Huge#1}

%%%%%%%%%%%%%%%%%%% fancyhdr %%%%%%%%%%%%%%%%%
\usepackage{fancyhdr}
\pagestyle{fancy} % enable fancy page style
\renewcommand{\headrulewidth}{0.0pt} % comment if you want the rule
\fancyhf{} % clear header and footer
\fancyhead[lo,le]{\leftmark}
\fancyhead[re,ro]{\rightmark}
\fancyfoot[CE,CO]{\hyperref[toc-contents]{\thepage}}

% https://tex.stackexchange.com/questions/550520/making-each-page-number-link-back-to-beginning-of-chapter-or-section
\makeatletter
\def\chaptermark#1{\markboth{\protect\hyper@linkstart{link}{\@currentHref}{Chapter \thechapter ~ #1}\protect\hyper@linkend}{}}
\def\sectionmark#1{\markright{\protect\hyper@linkstart{link}{\@currentHref}{\thesection ~ #1}\protect\hyper@linkend}}
\makeatother
%%%%%%%%%%%%%%%%%%% fancyhdr %%%%%%%%%%%%%%%%%


%%%%%%%%%%%%%%%%%%% biblatex %%%%%%%%%%%%%%%%%
\usepackage[doi=false,url=false,isbn=false,style=alphabetic,backend=biber,backref=true]{biblatex}
\addbibresource{bib.bib}

\newbibmacro{string+doiurlisbn}[1]{%
  \iffieldundef{doi}{%
    \iffieldundef{url}{%
      \iffieldundef{isbn}{%
        \iffieldundef{issn}{%
          #1%
        }{%
          \href{http://books.google.com/books?vid=ISSN\thefield{issn}}{#1}%
        }%
      }{%
        \href{http://books.google.com/books?vid=ISBN\thefield{isbn}}{#1}%
      }%
    }{%
      \href{\thefield{url}}{#1}%
    }%
  }{%
    \href{http://dx.doi.org/\thefield{doi}}{#1}%
  }%
}

% https://tex.stackexchange.com/questions/94089/remove-quotes-from-inbook-reference-title-with-biblatex
\DeclareFieldFormat[article,incollection,inproceedings,book,misc]{title}{\usebibmacro{string+doiurlisbn}{\mkbibemph{#1}}}
% https://tex.stackexchange.com/questions/454672/biblatex-journal-name-non-italic
\DeclareFieldFormat{journaltitle}{#1\isdot}
\DeclareFieldFormat{booktitle}{#1\isdot}
% https://tex.stackexchange.com/questions/10682/suppress-in-biblatex
\renewbibmacro{in:}{}
% add video field: https://tex.stackexchange.com/questions/111846/biblatex-2-custom-fields-only-one-is-working
\DeclareSourcemap{
    \maps[datatype=bibtex]{
      \map{
        \step[fieldsource=video]
        \step[fieldset=usera,origfieldval]
    }
  }
}
\DeclareFieldFormat{usera}{\href{#1}{\textsc{Online video}}}
\AtEveryBibitem{
    \csappto{blx@bbx@\thefield{entrytype}}{% put at end of entry
        \iffieldundef{usera}{}{\space \printfield{usera}}
    }
}


%%%%%%%%%%%%%%%%%%%%%%%notations%%%%%%%%%%%%%%%%%%%%%%%%%%%%%%
\newcommand{\F}{\ensuremath{\mathbb{F}}}
\newcommand{\C}{\ensuremath{\mathbb{C}}} 
\newcommand{\R}{\ensuremath{\mathbb{R}}}
\newcommand{\J}{\ensuremath{\mathbb{J}}}
\newcommand{\Q}{\ensuremath{\mathbb{Q}}}
\newcommand{\Z}{\ensuremath{\mathbb{Z}}}
\newcommand{\N}{\ensuremath{\mathbb{N}}}
\newcommand{\K}{\ensuremath{\mathbb{K}}}
\newcommand{\Zo}{\ensuremath{\mathbb{Z}_{\geqslant 0}}} % 非负整数集
\newcommand{\Zi}{\ensuremath{\mathbb{Z}_{\geqslant 1}}} % 正整数集
\newcommand{\id}{\mathrm{id}}
\newcommand{\im}{\mathrm{im}\,}                         % 映射的像
\newcommand{\leqs}{\leqslant}
\newcommand{\geqs}{\geqslant}
\newcommand{\ci}{\mathrm{i}}
\newcommand{\hH}{\mathscr{H}}
\newcommand{\hK}{\mathscr{K}}
\newcommand{\inner}[2]{\langle#1,#2\rangle}

%%%%%%%%%%%%%%%%%%% biblatex %%%%%%%%%%%%%%%%%

%%%%%%%%%%%%%%%%%%%%% glossaries %%%%%%%%%%%%%%%%%
% !TEX root = ./notes_template.tex
% \usepackage[style=super]{glossaries}
% https://www.overleaf.com/learn/latex/Glossaries
\usepackage[style=super,toc,acronym]{glossaries}
\setlength{\glsdescwidth}{1\linewidth}
\makeglossaries

\renewcommand\glossaryname{List of Abbreviations and Symbols}

\newglossaryentry{Q2}{name={$Q_2(f)$},
%sort=Q2,
description={Two-side (bounded) error quantum query complexity}}

\newglossaryentry{real_number}{name={$\mathbb{R}$},description={Real number}}

% \newglossaryentry{gcd}{name={gcd},description={greatest common divisor}}

\newacronym{gcd}{GCD}{Greatest Common Divisor}


\newglossaryentry{svm}{name={SVM},description={Support Vector Machine}}

\newglossaryentry{gd}{name={GD},description={Gradient Descent}}

\newglossaryentry{qft}{name={QFT},description={Quantum Field Theory}}

\newglossaryentry{qm}{name={QM},description={Quantum Mechanics}}

\newglossaryentry{v}{name={$\vec{v}$},description={a vector}}

% physics
\newglossaryentry{hamiltonian}{name={$\hat{H}$},description={Hamiltonian}}

\newglossaryentry{lagrangian}{name={$L$},description={Lagrangian}}
%%%%%%%%%%%%%%%%%%%%% glossaries %%%%%%%%%%%%%%%%%

%%%%%%%%%%%%%%%%%%%%% glossaries-extra %%%%%%%%%%%%%%%%%
% \usepackage[record,abbreviations,symbols,stylemods={list,tree,mcols}]{glossaries-extra}
%%%%%%%%%%%%%%%%%%%%% glossaries-extra %%%%%%%%%%%%%%%%%


% !TEX root = ./notes_template.tex

%%%%%%%%%%%%%%%%%%%%%%%%%%%%%%%%%%%%
%%%%%%%%%%%%%%%%%%%%%%%%%%%%%%%%%%%%
% math
\let\iff\relax
\newcommand{\iff}{\text{ iff }}
\newcommand{\OPT}{\textup{OPT}}

% physics
\newcommand{\acreation}{a^\dagger}



%%%%%%%%%%%%%%%%%%%%%%%%%%%%%%%%%%%%%%%%%%%%%%%%%%
%%%%%%%%%%%%%%%% begin of document %%%%%%%%%%%%%%%
%%%%%%%%%%%%%%%%%%%%%%%%%%%%%%%%%%%%%%%%%%%%%%%%%%

\begin{document}

\title{\bf \huge Study Notes of Matrix and Tensor}
% \title{\bf \huge Homework of Functional Analysis}
\author{Pei Zhong}
% \date{Update on \today}

\maketitle

% \newpage
% \let\cleardoublepag\clearpage

\tableofcontents

\begin{spacing}{1}

%%%%%%%%%%%%%%update progress%%%%%%%%%%



%%%%%%%%%%%%%%update progress end%%%%%%%%




%%%%%%%%%%%%%%%preface%%%%%%%%%%%%%
\chapter*{Preface}

Notes mainly refer to following materials:


\begin{itemize}
    \item[*] Machine learning
    \begin{itemize}
        \item \href{https://www.cs.cornell.edu/courses/cs4780/2023sp/}{lecture notes from cornell}
        \item \href{https://www.cs.cmu.edu/~hn1/documents/machine-learning/notes.pdf}{lecture notes from cmu}
        \item \href{https://cs229.stanford.edu/main_notes.pdf}{lecture notes of CS229}
    \end{itemize}
    \item[*] Deep learning
    \begin{itemize}
        \item \href{https://udlbook.github.io/udlbook/}{understanding deep learning}
        \item \href{https://www.bilibili.com/video/BV1Wv411h7kN/?spm_id_from=333.337.search-card.all.click}{lecture video from Hongyi Lee}
        \item \href{https://cs231n.github.io/}{lecture notes from Stanford}
    \end{itemize}
    \item[*] Reinforcement learning
    \begin{itemize}
        \item \href{https://web.stanford.edu/class/cs234/modules.html}{lecture notes from stanford}
        \item \href{https://people.cs.umass.edu/~bsilva/courses/CMPSCI_687/Fall2022/Lecture_Notes_v1.0_687_F22.pdf}{lecture notes from umass}
    \end{itemize}
\end{itemize}







%%%%%%%%%%%%%preface end%%%%%%%%%%%%%

\chapter{Probability and Distributions}

\section{Introduction}
\begin{definition}{}{}
    If an experiment can be repeated under the same conditions 
    it is a random experiment. The set of every possible outcome 
    of an experiment is the sample space, denoted $\mathcal{C}$.
\end{definition}
\begin{remark}
    For an experiment, the sample space is not unique.
    For example, When talking about the temperature in an area, 
    we can define the sample space as $\mathcal{C}=(-\infty,\infty)$ or $\mathcal{C}=[a,b]$.
    For a specific random experiment, we can use different sample spaces to describe it. 
    However, it is worth studying how to describe it with an appropriate sample space.
\end{remark}
\par
\textbf{Note/Definition}.
Notationally, we denote the elements of the sample space with
lower case letters such as $a,b,c$. Subsets of the sample space are \textit{events}
and we denote them with upper case letters such as $A,B,C$.

\begin{definition}{}{}
    If an experiment is performed $N$ times and a specific event occurs $f$ times,
    then $f$ is the frequency of the event and $f/N$ is the relative frequency of the event.
\end{definition}


\section{Sets}

\section{The Probability Set Function}

We need to define a set function that assigns a probability to the events (subsets of sample space $\mathcal{C}$).
We denote the colletion of events as $\mathcal{B}$.
If $\mathcal{C}$ is finite set, then we hope to assign a 
probability to all events (that is, to define a probability set function on the power set of $\mathcal{C}$).
More generally, we require that $\mathcal{B}$ (the colletion of events) to satisfy:
(1) the sample space $\mathcal{C}$ itself is an event,
(2) the complement of every event is again an event, and
(3) every countable union of events is again an event.
Symbolically, this means 
(1) $\mathcal{C}\in\mathcal{B}$,
(2) if $A\in\mathcal{B}$ then $A^c\in \mathcal{B}$, and
(3) if $A_1,A_2,...\in \mathcal{B}$ then $\cup_{n=1}^{\infty} A_n\in \mathcal{B}$.
Combining $(2)$ and $(3)$, we see by DeMorgan's Law (for countable unions)
that if $A_1,A_2,...\in\mathcal{B}$ then $\cap_{n=1}^{\infty}A_n\in \mathcal{B}$.
So the collection of events $\mathcal{B}$ is closed under complements, countable unions, 
and countable intersections.
Such a collection of sets form a \textit{$\sigma$-algebra}.


\begin{definition}{}{}
    A collection of events $\{A_n|n\in I\}$ (where $I$ is some indexing set)
    such that $A_i\cap A_j=\O$ is a mutually exclusive collection of events.
\end{definition}


\begin{definition}{}{}
    Let $\mathcal{C}$ be a sample space and let $\mathcal{B}$ be the set of all events (thus, $\mathcal{B}$ is a $\sigma$-field).
    Let $P$ be a real-valued function defined on $\mathcal{B}$.
    Then $P$ is a probability set function if $P$ satisfies the following three conditions:\\
    (1) $P(A)\geqs 0$ for $A\in \mathcal{B}$.\\
    (2) $P(\mathcal{C})=1$.\\
    (3) If $\{A_n\}$ is a mutually exclusive collection of events, then $P(\cup_{n=1}^{+\infty}A_n)=\sum\limits_{n=1}^{+\infty}P(C_n)$.
\end{definition}

\begin{theorem}{}{}
    For each event $A\in\mathcal{B}$, $P(A)=1-P(A^c)$.
\end{theorem}

\begin{theorem}{}{}
    The probability of the null set is zero; that is, $P(\O)=0$.
\end{theorem}

\begin{theorem}{}{}
    If $A$ and $B$ are events such that $A\subset B$,
    then $P(A)\leqs P(B)$.
\end{theorem}

\begin{theorem}{}{}
    For each event $A\in\mathcal{B}$ we have $0\leqs P(A)\leqs 1$.
\end{theorem}

\begin{theorem}{}{}
    If $A$ and $B$ are events in $\mathcal{C}$,
    then $P(A\cup B)=P(A)+P(B)-P(A\cap B)$.
\end{theorem}

\begin{theorem}{}{}
    Let $\{A_n\}$ be a nondecreasing sequence of events (ie. $A_n\subseteq A_{n+1}$). Then
    \begin{align*}
        \lim_{n\rightarrow \infty} P(A_n)=P(\lim_{n\rightarrow \infty}A_n)=P(\cup_{n=1}^{\infty}A_n).
    \end{align*}
    Let $\{A_n\}$ be a nonincreasing sequence of events (ie. $A_n\supseteq A_{n+1}$). Then
    \begin{align*}
        \lim_{n\rightarrow \infty} P(A_n)=P(\lim_{n\rightarrow \infty}A_n)=P(\cap_{n=1}^{\infty}A_n).
    \end{align*}
\end{theorem}

\begin{theorem}{}{}
    Let $\{A_n\}$ be an arbitrary sequence of events. Then
    \begin{align*}
        P(\cup_{n=1}^{\infty}A_n)\leqs \sum\limits_{n=1}^{\infty}P(A_n).
    \end{align*}
\end{theorem}

\section{Conditional Probability and Independence}
The idea behind conditional probability is that the initial sample space $\mathcal{C}$
has been replaced with some subset $A\subset \mathcal{C}$.

\begin{definition}{}{conditional probability}
    Let $B$ and $A$ be events with $P(A)>0$.
    Then the conditional probability of $B$ given $A$ as $P(B|A)=\frac{P(A\cap B)}{P(A)}$.
\end{definition}

\textbf{Note/Definition}. If $A$ and $B$ are events where $P(A)>0$ then 
$P(A\cap B)=P(A)P(B|A)$ by Definition \ref{def:conditional probability}.
This is called the multiplication rule also.

\begin{definition}{}{}
    Let $A$ and $B$ be two events. Then $A$ and $B$ are Independent is $P(A\cap B)=P(A)P(B)$.
\end{definition}

\section{Random variables}

\begin{definition}{}{}
    Consider a random experiment with a sample space $\mathcal{C}$.
    A function $X$ which assigns to each $c\in\mathcal{C}$ one and only one real number $X(c)=x$ is
    a random variable. The space (or range) of $X$ is the set of real numbers $\mathcal{D}=\{x|x=X(c) \text{ for some } c\in \mathcal{C}\}$.
    If $\mathcal{D}$ is a countable set then $X$ is a discrete random variable and 
    if $\mathcal{D}$ is an interval of real numbers then $X$ is a continuous random variable.
\end{definition}

\begin{definition}{}{}
    Let X be a random variable. 
    Then its cumulative distribution function (cdf) $F:\R\rightarrow [0,1]$
    is defined as follows:
    \begin{align*}
        F(x) = P(X\leqs x).
    \end{align*}
\end{definition}

\begin{theorem}{}{}
    
\end{theorem}

\section{Discrete Random Variables}

\section{Continuous Random Variables}

\section{Expectation of a Random Variable}

\section{Some Special Expectations}
\subsection{The Moment Generating Function}
Recall ethe McLaurin series
\begin{align*}
    f(\alpha)=e^{\alpha}=\sum\limits_{m=0}^{\infty}\frac{\alpha^m}{m!},
\end{align*}
if we write the random variable
\begin{align*}
    e^{tX}=\sum\limits_{m=0}^{\infty}\frac{t^m}{m!}X^m,
\end{align*}
then its expectation value defines something called the moment generating function (mgf)
\begin{align*}
    M(t)=E(e^{tX})=\sum\limits_{m=0}^{\infty} \frac{t^m}{m!}E(X^m).
\end{align*}
If we take the $m$th derivative of the mgf,
evaluated at $t=0$, we get the $m$th ($m\geqs 1$) moment:
\begin{align*}
    M^{m}(0)=E(X^m).
\end{align*}
For this to work, the mgf has to be defined in a neighborhood of the origin,
i.e., for $-h<t<h$ where $h>0$
is some positive number. 

\begin{definition}{}{}
    Let $X$ be a random variable such that for some $h>0$,
    the expectation of $e^{tX}$ exists for $-h<t<h$.
    The moment generating function (or mgf) of $X$ is the function $M(t)=E(e^{tX})$ for $-h<t<h$.
\end{definition}
\begin{remark}
    When a moment generating function exists, we must have for $t=0$
    that $M(0)=E(1)=1$.
\end{remark}

\section{Homework}

\begin{exercise}{1.9.7}{}
    Show that the moment generating function of the random variable $X$
    having the pdf $f(x)=\frac{1}{3}$, $-1<x<2$, zero elsewhere, is
    \begin{align*}
        M(t) = \left\{\begin{matrix}
            \frac{e^{2t}-e^{-t}}{3t} & t\neq 0\\
            1 & t=0.
           \end{matrix}\right.
    \end{align*}
\end{exercise}

\begin{solve}
    For $t\neq 0$, 
    \begin{align*}
        M(t)=E(e^{tX})=\int_{-\infty}^{+\infty} e^{tx}f(x)dx = \int_{-1}^{2}\frac{1}{3}e^{tx}dx = \frac{1}{3}\frac{e^{tx}}{t}|_{x=-1}^{x=2}=\frac{e^{2t}-e^{-t}}{3t}.
    \end{align*}
    And $M(0) = 1$ when a moment generating function
    exists and so the result follows.
\end{solve}

\section{Reference}
\begin{itemize}
    \item \href{https://faculty.etsu.edu/gardnerr/4047/notes-Hogg-McKean-Craig.htm}{lecture note}
    \item \href{https://ccrgpages.rit.edu/~whelan/courses/2013_3fa_STAT_405/notes01.pdf}{Probability and Distributions}
    \item \href{https://www.zhihu.com/question/28624845}{Sample space is unique?}
    \item \href{https://faculty.etsu.edu/gardnerr/4047/Beamer-Hogg-McKean-Craig/Proofs-HMC-1-3.pdf}{proof of 1.3}
\end{itemize}

\chapter{Multivariate Distributions}

\section{Distributions of Two Random Variables}

\begin{definition}{}{}
    Given a random experiment with a sample space $\mathcal{C}$,
    consider two random variables $X_1$ and $X_2$ which assign to each element $c$
    of $\mathcal{C}$ one and only one ordered pair of numbers $(X_1,X_2)$ is a random vector.
    The space of $(X_1,X_2)$ is the set of ordered pairs $\mathcal{D}=\{(x_1,x_2)|x_1=X_1(c),x_2=X_2(c),x\in\mathcal{C}\}$.
\end{definition}

\begin{definition}{}{}
    Let $\mathcal{D}$ be the space 
    associated with the random vectors ($X_1$,$X_2$).
    For $A\subset \mathcal{D}$ we call $A$ an event.
    The cumulative distribution function (cdf) for ($X_1$,$X_2$) is 
    \begin{align}
        F_{X_1,X_2}(x_1,x_2)=P(\{X_1\leqs x_1\}\cap \{X_2\leqs x_2\})
    \end{align}
    for $(x_1,x_2)\in \R^2$. This is the \textit{joint cumulative distribution function} of $(X_1,X_2)$.
    If $F_{X_1,X_2}$ is continuous then random variable $(X_1,X_2)$ is said to be continuous.
\end{definition}

\begin{definition}{}{}
    A random vector $(X_1,X_2)$ is a discrete random vector if its space 
    $\mathcal{D}$ is finite or countable. (Hence $X_1$ and $X_2$ both must be discrete.)
    The joint probability mass function of $(X_1,X_2)$ is $p_{X_1,X_2}(x_1,x_2)=P(X_1=x_1,X_2=x_2)$
    for all $(x_1,x_2)\in\mathcal{D}$.
\end{definition}

\begin{definition}{}{}
    If for random vector $(X_1,X_2)$ with cumulative distribution function $F_{X_1,X_2}$,
    there is a function $f_{X_1,X_2}:\R^2\rightarrow \R$ such that 
    \begin{align*}
        F_{X_1,X_2}(x_1,x_2)=\int_{-\infty}^{x_1}\int_{\infty}^{x_2} f_{X_1,X_2}(w_1,w_2)dw_1dw_2.
    \end{align*}
    Then $f_{X_1,X_2}$ is the joint probability density function (pdf) of $(X_1,X_2)$.
    The support of $(X_1,X_2)$ is the set of all points $(x_1,x_2)$ for which $f_{X_1,X_2}(x_1,x_2)>0$,
    denoted $\mathcal{S}$.
\end{definition}

\begin{remark}
    In this course, continuous random vectors will have joint probability
    density functions that determine the cumulative distribution function. By the
    Fundamental Theorem of Calculus (applied twice)
    \begin{align*}
        \frac{\partial^2 F_{X_1,X_2}(x_1,x_2)}{\partial x_1\partial x_2}=f_{X_1,X_2}(x_1,x_2).
    \end{align*}
    For event $A\in\mathcal{D}$, we have 
    \begin{align*}
        P((X_1,X_2)\in A)=\int\int_{A} f_{X_1,X_2}(x_1,x_2)dx_1dx_2.
    \end{align*}
\end{remark}



\begin{remark}
    We can find the distribution of random variable $X_1$ and $X_2$ 
    (called marginal distribution) based on the joint distribution of $(X_1,X_2)$.
    We have
    \begin{align*}
        \{X\leqs x_1\} = \{X_1\leqs x_1\}\cap \{-\infty<X_2<\infty\},
    \end{align*}
    so with $F_{x_1}$, the cumulative distribution function of $X_1$ we get for $x_1\in\R$
    \begin{align*}
        F_{X_1}(x_1) &= P(X\leqs x_1)=P(X_1\leqs x_1,-\infty<X_2<\infty)\\
                    &= \lim_{x_2\rightarrow \infty} F_{X_1,X_2}(x_1,x_2).
    \end{align*}
    We can similarly find the marginal distribution $F_{X_2}$ in terms of $F_{X_1,X_2}$.
    In the continuous case,
    \begin{align*}
        f_{X_1}(x_1)&=\int_{-\infty}^{\infty} f_{X_1,X_2}(x_1,x_2)dx_2,\\
        f_{X_2}(x_2)&=\int_{-\infty}^{\infty} f_{X_1,X_2}(x_1,x_2)dx_1.
    \end{align*}
\end{remark}


\section{Transformations: Bivariate Random Variables}

\section{Conditional Distributions and Expectations}



\section{Independent Random Variables}

\section{The Correlation Coefficient}

\section{Homework}

\begin{exercise}{}{}
    Let the joint pdf of $X$ and $Y$ be given by
    \begin{align*}
        f(x,y) = \left\{\begin{matrix}
            \frac{2}{(1+x+y)^3} & 0<x<\infty, 0<y<\infty\\
            0 & \text{elsewhere}.
           \end{matrix}\right.
    \end{align*}
    (a) Compute the marginal pdf of $X$ and the conditional pdf of $Y$, given $X=x$.\\
    (b) For a fixed $X=x$, compute $E(1+x+Y|x)$ and use the result to compute $E(Y|x)$.
\end{exercise}

\begin{solve}
    (a) By the definition of marginal probability density function:
    \begin{align*}
        f_X(x)&= \int_{-\infty}^{\infty}f(x,y)dy=\int_{0}^{\infty} \frac{2}{(1+x+y)^3} dy\stackrel{t=1+x+y}{=}\int_{1+x}^{\infty}\frac{2}{t^3}dt\\
              &= -t^{-2}|_{t=1+x}^{t=\infty} = 0-(-(1+x)^{-2})=\frac{1}{(1+x)^2}, \text{ for } 0<x<\infty.\\
        f_Y(y)&= \int_{-\infty}^{\infty}f(x,y)dx=\int_{0}^{\infty} \frac{2}{(1+x+y)^3} dx\\
              &= \frac{1}{(1+y)^2}, \text{ for } 0<y<\infty.
    \end{align*}
    Hence, $f_X(x)=\left\{\begin{matrix}
       \frac{1}{(1+x)^2} & 0<x<\infty \\
       0 & \text{elsewhere}
       \end{matrix}\right.$
    and $f_Y(y)=\left\{\begin{matrix}
        \frac{1}{(1+y)^2} & 0<y<\infty \\
        0 & \text{elsewhere}
        \end{matrix}\right.$.
    \par
    The conditional probability density function of $Y$ given $X=x$ is 
    \begin{align*}
        f_{Y|X}(y|x)=\frac{f_{X,Y}(x,y)}{f_X(x)}=\frac{\frac{2}{(1+x+y)^3}}{\frac{1}{(1+x)^2}}=\frac{2(1+x)^2}{(1+x+y)^3}, \text{ for } 0<x<\infty.
    \end{align*}
    Hence, $f_{Y|X}(y|x)=\left\{\begin{matrix}
        \frac{2(1+x)^2}{(1+x+y)^3} & 0<y<\infty \\
        0 & \text{elsewhere}.
        \end{matrix}\right.$
    \par
    (b) The conditional expectation of $g(Y)=1+X+Y$ given $X=x$ is
    \begin{align*}
        E(1+x+Y|x) &= \int_{-\infty}^{\infty} g(y) f_{Y|X}(y|x) dy\\
                   &= \int_{0}^{\infty} (1+x+y)\frac{2(1+x)^2}{(1+x+y)^2} dy\\
                   \stackrel{t=1+x+y}{=} &\int_{1+x}^{\infty} \frac{2(1+x)^2}{t^2}dt=-\frac{2(1+x)^2}{t}|_{t=1+x}^{t=\infty}=2(1+x).
    \end{align*}
    Since $E(1+x+Y|x)=1+x+E(Y|x)$, $E(Y|x)=2(1+x)-(1+x)=(1+x)$.
\end{solve}





\begin{exercise}{}{}
    Let $X_1,X_2,X_3$ be iid with common pdf $f(x)=\exp(-x)$, $0<x<\infty$,
    zero elsewhere. Evaluste:\\
    (a) $P(X_1<X_2|X_1<2X_2)$.\\
    (b) $P(X_1<X_2<X_3|X_3<1)$.
\end{exercise}
\begin{solve}
    The joint common pdf of $X_1,X_2$ is
    \begin{align*}
        f_{X_1,X_2}(x_1,x_2)= \left\{\begin{matrix}
            e^{-(x_1+x_2)} & 0<x_1<\infty,0<x_2<\infty\\
            0 & \text{elsewhere}
            \end{matrix}\right.
    \end{align*}
    The joint common pdf of $X_1,X_2,X_3$ is 
    \begin{align*}
        f_{X_1,X_2,X_3}(x_1,x_2,x_3)= \left\{\begin{matrix}
            e^{-(x_1+x_2+x_3)} & 0<x_1<\infty,0<x_2<\infty,0<x_3<\infty \\
            0 & \text{elsewhere}
            \end{matrix}\right.
    \end{align*}
    (a) Since
    \begin{align*}
        P(X_1<X_2,X_1<2X_2)&=\int_{0}^{\infty}dx_1\int_{x_1}^{\infty}e^{-(x_1+x_2)} dx_2 = \int_{0}^{\infty}-e^{-x_1}e^{-x_2}|_{x_2=x_1}^{x_2=\infty}dx_1\\
                           &= \int_{0}^{\infty}0-(-e^{-2x_1})dx_1\\
                           &= -\frac{1}{2}e^{-2x_1}|_{x_1=0}^{x_1=\infty}\\
                           &= \frac{1}{2}
    \end{align*}
    and 
    \begin{align*}
        P(X_1<2X_2)&=\int_{0}^{\infty}dx_1\int_{\frac{x_1}{2}}^{\infty}e^{-(x_1+x_2)} dx_2 = \int_{0}^{\infty}-e^{-x_1}e^{-x_2}|_{x_2=\frac{x_1}{2}}^{x_2=\infty}dx_1\\
                           &= \int_{0}^{\infty}0-(-e^{-x_1}e^{-\frac{x_1}{2}})dx_1\\
                           &= -\frac{2}{3}e^{-\frac{3}{2}x_1}|_{x_1=0}^{x_1=\infty}\\
                           &= \frac{2}{3},
    \end{align*}
    $P(X_1<X_2|X_1<2X_2)=\frac{P(X_1<X_2,X_1<2X_2)}{P(X_1<2X_2)}=\frac{\frac{1}{2}}{\frac{2}{3}}=\frac{3}{4}$.
    (b) Since
    \begin{align*}
        P(X_1<X_2<X_3)=\int_{-\infty}^{\infty}
    \end{align*}
\end{solve}



\section{Reference}
\begin{itemize}
    \item \href{https://ccrgpages.rit.edu/~whelan/courses/2013_3fa_STAT_405/notes02.pdf}{chapter 2}
    \item \href{https://faculty.etsu.edu/gardnerr/4047/notes-Hogg-McKean-Craig/Hogg-McKean-Craig-2-1.pdf}{2.1}
    \item \href{https://faculty.etsu.edu/gardnerr/4047/notes-Hogg-McKean-Craig/Hogg-McKean-Craig-2-3.pdf}{2.3}
\end{itemize}







%%%%%%%%%%%%%%%Content%%%%%%%%%%%%%%%
% % \mainmatter % separat the number of toc and mainmatter
% \chapter*{Preface}

Notes mainly refer to following materials:


\begin{itemize}
    \item[*] Machine learning
    \begin{itemize}
        \item \href{https://www.cs.cornell.edu/courses/cs4780/2023sp/}{lecture notes from cornell}
        \item \href{https://www.cs.cmu.edu/~hn1/documents/machine-learning/notes.pdf}{lecture notes from cmu}
        \item \href{https://cs229.stanford.edu/main_notes.pdf}{lecture notes of CS229}
    \end{itemize}
    \item[*] Deep learning
    \begin{itemize}
        \item \href{https://udlbook.github.io/udlbook/}{understanding deep learning}
        \item \href{https://www.bilibili.com/video/BV1Wv411h7kN/?spm_id_from=333.337.search-card.all.click}{lecture video from Hongyi Lee}
        \item \href{https://cs231n.github.io/}{lecture notes from Stanford}
    \end{itemize}
    \item[*] Reinforcement learning
    \begin{itemize}
        \item \href{https://web.stanford.edu/class/cs234/modules.html}{lecture notes from stanford}
        \item \href{https://people.cs.umass.edu/~bsilva/courses/CMPSCI_687/Fall2022/Lecture_Notes_v1.0_687_F22.pdf}{lecture notes from umass}
    \end{itemize}
\end{itemize}







% \part{Mathematics}

% % !TEX root = ../notes_template.tex
\chapter{Discrete Math}\label{chp:discrete_math}


\section{Proof}

\begin{theorem}
\end{theorem}

\begin{solution}
By induction:
\end{solution}



\section{Quantifier}
\lipsum % dummy text - remove from real document

\section{Graph}
\citetitle{babaiGraphIsomorphismQuasipolynomial2016}
\cite{babaiGraphIsomorphismQuasipolynomial2016}

\section{Number theory}
Figure example
\begin{figure}[!ht]
    \centering
    \includegraphics[width=1\linewidth]{./figure/elliptic_curves.pdf}
    \caption{Elliptic curves \cite{childsUniversalComputationQuantum2009} }
\end{figure}


\section{Algorithm}
% \begin{center}
% \begin{minipage}{.9\linewidth}
% algorithm2e
% https://www.overleaf.com/learn/latex/Algorithms#The_algorithm2e_package
\begin{algorithm}[H]
    \SetKwInOut{Input}{input}
    \SetKwInOut{Output}{output}
    \Input{Integer $N$ and parameter $1^t$}
    \Output{A decision as to whether $N$ is prime or composite}
    \BlankLine
    \For{ $i = 1,2, \ldots, t$} {
        $a\leftarrow \qty{1,\dots,N_1}$\;
        \If{$a^{N-1} \neq 1 \mod{N}$}
    {\Return "composite"}
    }
    \Return "prime"
    \caption{Primality testing - first attempt}
    \label{alg:miller_rabin}
\end{algorithm}
% \end{minipage}
% \end{center}

% \part{Computer Science}
% % \input{./chapter/complexity.tex}
% % !TEX root = ../notes_template.tex
\chapter{Machine Learning}\label{chp:machine_learning}
\minitoc

\section{Regression}
% \gls{algorithm};
\subsection{Gradient descent}\label{sec:gradient_descent}
\gls{gd};
% \glsxtrshort{gd}

\section{Support Vector Machine}
\gls{svm};
% % \input{./chapter/algorithms.tex}

% \part{Physics}
% % !TEX root = ../notes_template.tex
\chapter{Quantum Mechanics}\label{chp:quantum_mechanics}
\minitoc

\section{Hamiltonian}
\gls{hamiltonian};
% \glsxtrshort{qm};

\section{Path Integral}
\gls{lagrangian}

\section{Quantum Field Theory}
\gls{qft};
% % \input{./chapter/quantum_field_theory.tex}

% % \begin{appendices}
% % % !TEX root = ../notes_template.tex
\chapter{Formulas}

\section{Gaussian distribution}\label{sec:gaussian_distribution}
\begin{Definition}[Gaussian distribution]\label{def:gaussian_distribution}
    \myindex{Gaussian distribution}
\end{Definition}

\begin{theorem}[Central limit theorem]\label{thm:central_limit_theorem}
\end{theorem}
% % \end{appendices}

% \backmatter

% %%%%%%%%%%%%%%% Reference %%%%%%%%%%%%%%%

% \printbibliography[heading=bibintoc]
% \printindex
\end{spacing}
\end{document}