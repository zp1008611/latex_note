\chapter{Elementary Properties and Examples}\label{chp:3_1}

\begin{definition}
    If $\mathscr{X}$ is a vector space over $\F$,
    a seminorm is a function $p:\mathscr{X}\rightarrow [0,\infty)$ having the properties:\\
    (1) $p(x+y)\leqs p(x)+p(y)$ for all $x,y$ in $\mathscr{X}$.\\
    (2) $p(\alpha x)=|\alpha|p(x)$ for all $\alpha$ in $\F$ and $x$ in $\mathscr{X}$.\\
    
\end{definition}


\begin{exercise}{III1 T4}{}
    If $1\leqs p\leqs \infty$ and $(x_1,x_2)\in\R^2$, 
    define $||x||_p\equiv (|x_1|^p+|x_2|^p)^{1/p}$ and $||x||_{\infty}\equiv \sup\{|x_1|,|x_2|\}$,
    graph $\{x\in\R^2:||x||_p=1\}$. Note that if $1<p<\infty$, $||x||_p=||y||_p=1$ and $x\neq y$, 
    then for $0<t<1$, $||tx+(1-t)y||_p<1$.
    The same cannot be said for $p=1,\infty$.
\end{exercise}

\begin{proof}
    By Minkowski inequality, $||tx+(1-t)y||_p\leqs ||tx||_p+||(1-t)y||_p=|t|||x||_p+|1-t|||y||_p=t+1-t=1$.
    If $p=1$, for $x=(-1,0),y=(0,1),t=\frac{1}{2}$, 
    then $||tx+(1-t)y||_1=||(-t,0)+(0,1-t)||_1=||(-\frac{1}{2},0)+(0,\frac{1}{2})||_1=||(-\frac{1}{2},\frac{1}{2})||_1=1$.
    If $p=\infty$, for $x=(-1,1),y=(1,1),t=\frac{1}{2}$, 
    then $||tx+(1-t)y||_{\infty}=||(-t,t)+(1-t,1-t)||_{\infty}=||(1-2t,1)||_{\infty}=||(0,1)||_{\infty}=1$.
\end{proof}

\begin{exercise}{III1 T5}{}
    Let $c=$ the set of all sequences $\{\alpha_n\}_{1}^{\infty}$, 
    $\alpha_n$ in $\F$, such that lim$\alpha_n$ exists. 
    Show that $c$ is closed subspace of $l^{\infty}$ and hence is a Banach space.
\end{exercise}

\begin{proof}
    We know that $c=\{(\alpha_n)_{n\in \N_+}|\alpha_n\in \C \text{ and the sequences converges}\}$ and 
    $l^{\infty}=\{(\alpha_n)_{n\in\N_+}$ $|\alpha_n\in\C,\sup_{n} |\alpha_n|<\infty\}$.
    Then, for $(\alpha_n)$ in $c$, $\exists M>0$ s.t. $a_n\leqs M$ for all $n$.
    Then, $c$ is a subspace of $l^{\infty}$.
    Now, we show that $c$ is closed.
    Let $x=(\alpha_1,\alpha_2,...)\in l^{\infty}$ and $\{x^n\}$ is a sequences in $c$ convering to $x$ in the $|\cdot||_{\infty}$ norm.
    For $n\in \N$, write $x^{n}=(x_1^n,x_2^n,...)$ so that $x_i^n$ is the  i
    $i$-th term of the sequence $x^n$ in $c$. We claim that $x\in c$.
    Let $\epsilon>0$. As $x^n\underset{n\rightarrow \infty}{\longrightarrow} x$, 
    there exists $N\in \N$ such that $||x^{N}-x||_{\infty}<\frac{\epsilon}{3}$.
    As $x^{N}$ is in $c$, it follows that $x^{N}$ is Cauchy, so there exists $K\in \N$
    such that $|x_i^{N}-x_j^{N}|<\frac{\epsilon}{3}$ for all $i,j\geqs K$. Then for such $i,j$, we have
    \begin{align*}
        |x_i-x_j|&\leqs |x_i-x_i^{N}|+|x_i^N-x_j^N|+|x_j^N-x_j|\\
                &\leqs ||x-x^N||_{\infty}+|x_i^N-x_j^N|+||x^{N}-x||_{\infty}\\
                &<\frac{\epsilon}{3} + \frac{\epsilon}{3} + \frac{\epsilon}{3}.
    \end{align*}
    Then $x$ is Cauchy and so converges. Hence, $x$ is in $c$.
    So $c$ is a closed subspace of $l^{\infty}$ and so a Banach space.
\end{proof}

\section{Reference}

\begin{itemize}
    \item \href{https://math.stackexchange.com/questions/80139/why-is-the-l-p-norm-strictly-convex-for-1p-infty}{lp norm strictly convex when $1<p<\infty$}
    \item \href{https://math.stackexchange.com/questions/1258414/proving-that-c-is-a-banach-space?rq=1}{$c$ is a closed subspace of $l^{\infty}$}
\end{itemize}