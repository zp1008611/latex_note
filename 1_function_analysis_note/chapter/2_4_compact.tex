\chapter{Compact Operators}\label{chp:2_4}

Let ball $\mathscr{H}$ denote the closed unit ball $\{h\in\mathscr{H}:||h||\leqs 1\}$ 
in $\mathscr{H}$.
Now let me explain the definition of a few symbols that appear in this chapter.

\begin{tabular}{|c|c|c|}% 通过添加 | 来表示是否需要绘制竖线
    \hline  % 在表格最上方绘制横线
    \textbf{Symbol}&\textbf{Name}&\textbf{Definition}\\
    \hline  %在第一行和第二行之间绘制横线
    $\mathscr{B}(\mathscr{H},\mathscr{K})$&Bounded Operators&$||T||<\infty$ i.e. $T$(ball $\mathscr{H}$) is bounded\\
    \hline % 在表格最下方绘制横线
    $\mathscr{B}_0(\mathscr{H},\mathscr{K})$&Compact Operators& cl[ $T$(ball $\mathscr{H}$)] is compact\\
    \hline 
    $\mathscr{B}_{00}(\mathscr{H},\mathscr{K})$&Finite rank Operators& ran($T$) is finite dimensional\\
    \hline
\end{tabular}

\begin{proposition}{}{}
   In a normed space $\mathscr{X}$ and $S\subseteq \mathscr{X}$, the following
“compactness” are all equivalent:\\
(1) $S$ is compact.\\
(2) $S$ is sequentially compact. \\
(3) $S$ is complete and totally bounded.
\end{proposition}

\begin{proposition}{}{}
    $T\in\mathscr{B}(\mathscr{H},\mathscr{K})$ is compact iff for every bounded sequence $\{h_n\}$ in $\mathscr{H}$, 
    $\{T h_n\}$ has a convergent subsequence in $\mathscr{K}$.
\end{proposition}

\begin{proof}
    1
\end{proof}


\begin{proposition}{}{}
    (1) $\mathscr{B}_0(\mathscr{H},\mathscr{K})\subseteq \mathscr{B}(\mathscr{H},\mathscr{K})$.\\
    (2) $\mathscr{B}_0(\mathscr{H},\mathscr{K})$ is a linear space and if $\{T_n\}\subseteq \mathscr{B}_0(\mathscr{H},\mathscr{K})$ and $T\in\mathscr{B}(\mathscr{H},\mathscr{K})$
    such that $||T_n-T||\rightarrow 0$, then $T\in\mathscr{B}_0(\mathscr{H},\mathscr{K})$ ($\mathscr{B}_0(\mathscr{H},\mathscr{K})$ is closed).\\
    (3) If $A\in\mathscr{H},B\in\mathscr{K}$ and $T\in\mathscr{B}_0(\mathscr{H},\mathscr{K})$,
    then $TA$ and $BT\in\mathscr{B}_0(\mathscr{H},\mathscr{K})$.
\end{proposition}

\begin{proof}
    (1) For $T\in\mathscr{B}_0(\mathscr{H},\mathscr{K})$,
    cl[$T$(ball$\mathscr{H}$)] is compact, then cl[$T$(ball$\mathscr{H}$)] is totally bounded.
    Hence, $||T||=\sup_{||h||=1}||Th||=\sup_{h\in \text{ball }\mathscr{H} }||Th||<\infty$ and so $T$ is bounded. \\
    (2) To show $T\in\mathscr{B}_0(\mathscr{H},\mathscr{K})$, we should show that cl[$T$(ball $\mathscr{H}$)] is compact.
    Since $\mathscr{K}$ is complete, it follows that we should show that $T$(ball $\mathscr{H}$) is totally bounded and so cl[$T$(ball $\mathscr{H}$)] is totally bounded.
    Since $||T_n-T||\rightarrow 0$, for $\epsilon>0$, $\exists n\in\N$ such that $||T_n-T||<\frac{\epsilon}{3}$.
    Since $T_n$ is compact, there are vectors $h_1,...,h_m$ in ball $\mathscr{H}$
    such that $T_n$(ball $\mathscr{H}$) $\subseteq$ $\cup_{j=1}^{m} B(T_nh_j,\frac{\epsilon}{3})$.
    So if $h\in$ ball $\mathscr{H}$, there exists an $h_j$ such that $||T_nh_j-T_nh||<\frac{\epsilon}{3}$. Thus:
    \begin{align*}
        ||Th_j-Th||&\leqs ||Th_j-T_nh_j||+||T_nh_j-T_nh||+||T_nh-Th||\\
                  < ||T-T_n||+\frac{\epsilon}{3} + ||T-T_n||\\
                  < \epsilon.
    \end{align*}
    Hence, $T$(ball $\mathscr{H}$) $\subseteq \cup_{i=1}^{m} B(Th_j,\epsilon)$. Hence, $T$ (ball $\mathscr{H}$) is totally bounded.\\
    (3)
\end{proof}

\begin{proposition}{}{compact operator is the norm limit of finite rank operator sequence}
    If $T\in \mathscr{B}(\mathscr{H},\mathscr{K})$. Then 
    $T$ is compact iff there is a sequence $\{T_n\}$ of operators of finite rank such that $||T-T_n||\rightarrow 0$.
\end{proposition}

\begin{lemma}{}{}
    $T\in \mathscr{B}(\mathscr{H},\mathscr{K})$, 
    then dim(ran$T$)=dim(ran$T^*$). In particular, $T$ is finite rank iff $T^*$ is finite rank.
\end{lemma}

\begin{proposition}{}{}
    If $T\in\mathscr{B}(\mathscr{H},\mathscr{K})$, then $T$ is compact iff $T^*$ is compact.
\end{proposition}

\begin{proof}
    ($\Rightarrow$):
    If $T$ is compact, then there is sequence of $\{T_n\}$ of operators of finite rank
    such that $||T-T_n||\rightarrow 0$.
    Since $||(T_n-T)^*||=||T_n-T||$, it follows that $||T_n^*-T^*||\rightarrow 0$.
    Since $T_n^*\in\mathscr{B}_{00}(\mathscr{H},\mathscr{K})$ and $T^*\in\mathscr{B}(\mathscr{H},\mathscr{K})$,
    by proposition\ref{prop:compact operator is the norm limit of finite rank operator sequence}, $T^*\in\mathscr{B}_0(\mathscr{H},\mathscr{K})$.\\
    ($\Leftarrow$):
    Since $T=(T^*)^*$, result is clear by the same proof above. 
\end{proof}

\begin{definition}{}{}
    If $A\in\mathscr{B}(\mathscr{H})$, a scalar $\alpha$ is an eigenvalue of $A$ if $\text{ker}(A-\alpha)\neq (0)$.
    If $h$ is a nonzero vector in $\text{ker}(A-\alpha)$,
    $h$ is called an eigenvector for $\alpha$; thus $Ah=\alpha h$.
    Let $\sigma_p(A)$ denote the set of eigenvalues of $A$.
\end{definition}

\begin{proposition}{}{compact operator has non-zero eigenvalue then eigenspace is finite dimensional}
    If $T\in\mathscr{B}_0(\mathscr{H})$, $\lambda \in\sigma_p(A)$ and $\lambda\neq 0$, then the eigenspave $\text{ker}(T-\lambda)$ is finite dimensional.
\end{proposition}



\section{Homework}

\begin{exercise}{II4 T4}{}
    Show that an idempotent is compact if and only if it has finite rank.
\end{exercise}

\begin{proof}
    $(\Rightarrow)$:
    We need to show that $\text{ran}(E)$ is finite dimensional.\\
    If $\text{ran}(E)=\{0\}$, obviously $\text{ran}E$ is finite dimensional.
    If $\text{ran}(E)\neq \{0\}$, for $0\neq h\in\text{ran}E$, $h=Eh$ since $E$ is idempotent.
    Then $1\in\sigma_p(A)$, since $E$ is compact, by proposition\ref{prop:compact operator has non-zero eigenvalue then eigenspace is finite dimensional},
    $\text{ker}(E-I)=\text{ran}E$ is finite dimensional. \\
    ($\Leftarrow$):
    Since $E$ is finite rank, it follows that $\text{ran}E$ is finite dimensional.
    Since cl[$E$(ball $\mathscr{H}$)] is a closed subset of $\text{ran}E$ 
    and $\text{ran}E$ is Hausdorff,
    it follows that cl[$E$(ball $\mathscr{H}$)] is compact. Hence $E$ is compact.
\end{proof}

\begin{exercise}{II4 T8}{}
    If $h,g\in\mathscr{H}$,
    define $T:\mathscr{H}\rightarrow \mathscr{H}$ by $Tf=\inner{f}{h}g$.\\
    (1) Show that $T$ has rank $1$ [that is, dim(ran$T$)$=1$].\\
    (2) Moreover, every rank $1$ operator can be so represented.\\
    (3) Show that if $T$ is a finite rank operator, then there are orthonormal vectors
    $e_1,...,e_n$ and vectors $g_1,...,g_n$ such that $Th=\sum\limits_{j=1}^{n}\inner{h}{e_j}g_j$
    for all $h$ in $\mathscr{H}$. \\
    (4)  In this case show that
    $T$ is normal if $g_j=\lambda_je_j$ for some scalars $\lambda_1,...,\lambda_n$.\\
    (5) Find $\sigma_p(T)$.
\end{exercise}

\begin{proof}
    (1) If $h,g\neq 0\in\mathscr{H}$, then $Th=\inner{h}{h}g\neq 0$. Hence, dim(ran$T$)$\geqs 1$.
    For $Tf_1,Tf_2\neq 0\in$ ran$T$, $\inner{f_2}{h}Tf_1-\inner{f_1}{h}Tf_2=\inner{f_2}{h}\inner{f_1}{h}g-\inner{f_1}{h}\inner{f_2}{h}g=0$.
    Then $Tf_1$ and $Tf_2$ is linear dependent and so dim(ran$T$)$\leqs 1$.
    Hence, dim(ran$T$)$=1$.\\
    (2) If dim(ran$T$)$=1$, let $g$ be a basis of ran($T$) and $||g||=1$, then for $f\in\mathscr{H}$,
    assume $Tf=\alpha g$, then $\inner{Tf}{g}=\inner{\alpha g}{g}=\alpha$. Then $Tf=\inner{Tf}{g}g$. 
    Since $Tf\in\mathscr{H}$, let $Tf=h$ and $Tf=\inner{h}{g}g$.\\
    (3) 
\end{proof}


\section{Reference}
\begin{itemize}
    \item \href{https://tqft.net/web/teaching/current/Analysis3/LectureNotes/03.Compact.operators.pdf}{lectures notes from tqft}
    \item \href{https://sites.math.washington.edu//~hart/m556/notes1.pdf}{lecture notes from washington}
    \item \href{https://ocw.mit.edu/courses/18-102-introduction-to-functional-analysis-spring-2021/resources/mit18_102s21_lec19/}{lecture notes from mit}
    \item \href{https://www.lakeheadu.ca/sites/default/files/uploads/77/Nielsen.pdf}{Compact Operators}
\end{itemize}