% \usepackage{xr}
% \externaldocument{1_2_orthogonality}
\chapter{The Riesz Representation Theorem}\label{chp:1_3}


\begin{definition}{Linear Functional}{}
    (1) A linear functional on a vector space $\mathscr{X}$ is a linear mapping $L:\mathscr{X}\rightarrow \L$:
    \begin{align*}
        L(\alpha x+\beta y)= \alpha L(x)+\beta L(y), \forall x,y\in \mathscr{X}, \alpha,\beta \in \L.
    \end{align*}
    (2) A linear functional $L$ on a normed space $(\mathscr{X},||\cdot||)$ is called a bouned linear functional if
    there exists $C\geqs 0$ such that $|L(x)|\leqs C||x||$ for each $x\in \mathscr{X}$.      
\end{definition}

\begin{proposition}{}{bouded operator equivalent}
    Let $\hH$ be a Hilbert space and $L:\hH\rightarrow \L$ a linear functional. The following statements are equivalent.\\
    (1) $L$ is continuous at $h=0$.\\
    (2) $L$ is continuous.\\
    (3) $L$ is continuous at some points.\\
    (3) $L$ is bounded.\\
    (4) Ker $L$ is closed in $\hH$.
\end{proposition}


Note that the bounded linear functionals forms a vector space $\mathscr{H}^*$: $0\in \mathscr{H}^*$, if $f_i\in \mathscr{H}^*$ and $\alpha_i\in \L$ then
$\alpha_1f_1+\alpha_2f_2\in \mathscr{H}^*$. We will now explain how to define a norm on $\mathscr{H}^*$.

\begin{definition}{}{norm bounded linear funcional}
    For a bounded linear functional $L\in \mathscr{H}^*$, its norm is defined as 
    \begin{align*}
        ||L||_{\mathscr{H}^*} = \sup_{||h||_{\mathscr{H}}\leqs 1}|L(x)|.
    \end{align*}
\end{definition}

For convenience, the following content will follow the convention: $||L||=||L||_{\mathscr{H}^*}, ||x||=||x||_{\mathscr{H}}$.
Let's check three properties fo the norm:\\
(1) $||L||\geqs0$; $||L||=0\Leftrightarrow L=0$\\
(2) $||\alpha L|| =|\alpha| ||L||$\\
(3) \begin{align*}
    ||f_1+f_2|| &= \sup_{||h||\leqs 1}|(f_1+f_2)(h)|\\
                &= \sup_{||h||\leqs 1}|f_1(h)+f_2(h)|\\
                &\leqs \sup_{||h||\leqs 1}|f_1(h)| + \sup_{||h||\leqs 1}|f_2(h)|\\
                &= ||f_1||+||f_2||
\end{align*}
    
\begin{proposition}{}{}
    If $L$ is a bounded linear functional, then\\
    (1) $|L(h)|\leqs ||L||\cdot ||h||$ for every $h\in \mathscr{H}$.\\
    (2) \begin{align*}
        ||L|| &= \sup_{||h||=1}|L(x)|\\
              &= \sup_{h\in \hH \setminus\{0\}} \frac{|L(x)|}{||x||}\\
              &= \inf\{c>0:|L(h)|\leqs c||h||,h\in \hH\}. 
    \end{align*}
\end{proposition}

\begin{proposition}{}{element determines a linear funcional}
    If $h_0\in \hH$, then $L_{h_0}:\mathscr{H}\rightarrow \F$ given by 
    \begin{align}
        L_{h_0}(h)=\inner{h}{h_0}
        \label{eq:inner product functional}
    \end{align} 
    is a bounded linear funcional on $\hH$, with $||L_{h_0}||=||h_0||$.
\end{proposition}
\begin{proof}
    For $h_1,h_2\in\hH$, $\alpha,\beta\in\F$, 
    \begin{align*}
        L_{h_0}(\alpha h_1+\beta h_2)=\inner{\alpha h_1+\beta h_2}{h_0}= \alpha\inner{h_1}{h_0}+\beta\inner{h_2}{h_0}=\alpha L_{h_0}(h_1)+\beta L_{h_0}(h_2).
    \end{align*}
    Hence, $L_{h_0}$ is linear. 
    By the Cauchy-Schwarz inequality, for $h\in\hH$,
    \begin{align*}
        |L_{h_0}(h)|=|\inner{h}{h_0}|\leqs ||h||\cdot||h_0||. 
    \end{align*}
    Hence, $L_{h_0}$ is bounded and $||L_{h_0}||=\sup_{h\in \hH \setminus\{0\}} \frac{|L_{h_0}(h)|}{||h||}\leqs ||h_0||$. 
    Then $||L_{h_0}|| = \sup_{||h||=1}|L(h)|\leqs ||h_0||$. 
    Since $||\frac{h_0}{||h_0||}||=1$ and $L_{h_0}(\frac{h_0}{||h_0||})=\inner{\frac{h_0}{||h_0||}}{h_0}=||h_0||$,
    it follows that $||L_{h_0}||\geqs ||h_0||$.
    Hence, $||L_{h_0}||=||h_0||$. 
\end{proof}

The following theorem shows that all bounded linear
functionals in Hilbert space are of the form (\ref{eq:inner product functional}). In other words, 
every bounded linear functional on $\hH$ can be identified with a unique point in the space itself.
% Fix $y_0\in \hH$, then $x\mapsto <x,y_0>$ defines a linear functional $L$. By Cauchy Schwarz inequality, for all $x\in \hH\setminus \{0\}$, 
% Now we introduce the self-duality property of the Hilbert space.

\begin{theorem}{}{}
    If $L$ is a bounded linear functional on $\hH$, then there is a unique vector $h_0\in\hH$ such that
    \begin{align}
        L(h) = \inner{h}{h_0}, \forall h\in \hH.
    \end{align}
    Moreover, $||L||=||h_0||$
\end{theorem}

\begin{proofsolution}
    By proposition\ref{prop:bouded operator equivalent}, $\mathscr{M} = \text{Ker} L$ is closed in $\hH$. If $\mathscr{M}=\hH$, $L(h)=0$, $h_0=0$ is desired requested.
    If $\mathscr{M}\neq \hH$, then $\mathscr{M}^{\perp}\neq (0)$ and 
    so there exists some $u\in\mathscr{M}^{\perp}$ such that $L(u)\neq 0$, 
    and we take $f_0=\frac{f_0}{L(f_0)}\in\mathscr{M}^{\perp}$ so that $L(f_0)=1$. 
    Then $L(h-L(h)f_0)=L(h)-L(h)L(f_0)=L(h)-L(h)=0$ for all $h\in\mathscr{H}$.
    Hence, $h-L(h)f_0\in\mathscr{M}$ and so we have
    $0=\inner{h-L(h)f_0}{f_0}$ and so $\inner{h}{f_0}=L(h)||f_0||^2$.
    Let $h_0 = f_0/||f_0||^2$ now seals the deal.\\
    Uniquness follows because if $L(h) = \inner{h}{h_0} = \inner{h}{h_0'}$
    then $h_0 - h_0'\perp \mathscr{H}$ and so $h_0 - h_0'\in\mathscr{H}^{\perp}=\{0\}$.
    $||L||=||h_0||$ is shown in proposition\ref{prop:element determines a linear funcional}
\end{proofsolution}


\section{Homework}

\begin{exercise}{I3 T3}{}
    Let $\mathscr{H}=l^2(\N\cup \{0\})$. \\
    (a) Show that if $\{\alpha_n\}\in \mathscr{H}$, then the power series $\sum\limits_{n=0}^{\infty} \alpha_n z^n$ 
    has radius of convergence $\geqs 1$.\\
    (b) If $|\lambda|<1$ and $L:\mathscr{H}\rightarrow \F$ is defined by $L(\{\alpha_n\})=\sum\limits_{n=0}^{\infty}\alpha_n\lambda^n$, 
    find the vector $h_0$ in $\mathscr{H}$ such that $L(h)=\inner{h}{h_0}$ for every $h$ in $\mathscr{H}$.\\
    (c) What is the norm of the linear functional $L$ defined in $(b)$?
\end{exercise}

\begin{proof}
    (a) If $\{\alpha_n\}\in\mathscr{H}$, then $\sum\limits_{n}|\alpha_n|^2<\infty$, i.e. 
    $\sum\limits_{n}|\alpha_n|^2$ is absolutely convergent. Then by root test, $\lim_{n\rightarrow \infty}\sqrt[n]{|\alpha_n|^2}\leqs 1$.
    Then $\lim_{n\rightarrow \infty}\sqrt[n]{|\alpha_n|}\leqs 1$. 
    Hence, the radius of convergence is $\frac{1}{\lim_{n\rightarrow \infty}\sqrt[n]{|\alpha_n|}}\geqs 1$.
    \\
    (b)(c) Since $|\lambda|<1$, then for $h=\{\alpha_n\}\in \mathscr{H}$, $|L(h)|=\sum\limits_{n}\lambda^n \alpha_n$ converges and so is bounded.
    Then $L$ is a bounded linear functional on $\mathscr{H}$, so by the Riesz representation theorem,
    there exists a vector $h_0$ such that $L(h)=\inner{h}{h_0}$ for every $h$ in $\mathscr{H}$.
    Assume $h_0=\{\beta_n\}$, then $\inner{h}{h_0}=\sum\limits_{n}\alpha_n\overline{\beta_n}=L(h)=\sum\limits_{n}\lambda^n\alpha_n=\sum\limits_{n}\alpha_n\overline{(\overline{\lambda})^n}$.
    Hence, we can let $h_0=(1,\bar{\lambda}, (\bar{\lambda})^2,...)$. 
    Since $||h_0||_{l^2}^2=\sum\limits_{n}|(\bar{\lambda})^n|^2=\sum\limits_{n}|\lambda|^{2n}=\lim_{n\rightarrow \infty}\frac{1(1-(|\lambda|^2)^n)}{1-|\lambda|^2}=\frac{1}{1-|\lambda|^2}$,
    it follows that $||h_0||=\frac{1}{\sqrt{1-|\lambda|^2}}$ and by Riesz representation theorem $||L||=||h_0||=\frac{1}{\sqrt{1-|\lambda|^2}}$.
\end{proof}

\begin{exercise}{I3 T5}{}
    Let $\mathscr{H}=$ the collection of all absolutely continuous functions $f:[0,1]\rightarrow \F$ such that $f(0)=0$, $f'\in L^2(0,1)$ 
    and for $f,g\in \mathscr{H}$, $\inner{f}{g}=\int_{0}^{1} f'(t)\overline{g'(t)}dt$. By Example 1.8, we know that $\mathscr{H}$ is a Hilbert space.
    If $0<t\leqs 1$, define $L:\mathscr{H}\rightarrow \F$ by $L(h)=h(t)$.
    Show that $L$ is a bounded linear funcional, find $||L||$, 
    and find the vector $h_0$ in $\mathscr{H}$ such that $L(h)=\inner{h}{h_0}$ for all $h$ in $\mathscr{H}$. 
\end{exercise}
\begin{proof}
    Firstly, we show that $L$ is linear.
    For $\alpha,\beta\in\F$, $h_1,h_2\in\mathscr{H}$, then $L(\alpha h_1+ \beta h_2)=(\alpha h_1+\beta h_2)(t)=\alpha h_1(t) +\beta h_2(t)=\alpha L(h_1)+\beta L(h_2)$.
    Hence, $L$ is linear. Then, we show that $L$ is continuous. For a sequence $\{h_n\}\in \mathscr{H}$ such that $h_n\underset{n\rightarrow \infty}{\longrightarrow} h$,
    if $0<t\leqs 1$ we have
    \begin{align*}
        \lim_{n\rightarrow \infty}L(h_n)=\lim_{n\rightarrow \infty}h_n(t)\underset{h_n \text{ is absolutely continuous}}{=} h(t)=L(h).
    \end{align*}
    Hence, $L$ is a continuous linear functional and so a bounded linear funcional.
    Then by Riesz representation theorem, there exists a vector $h_0\in \mathscr{H}$ such that $L(h)=\inner{h}{h_0}$ for all $h$ in $\mathscr{H}$.
    and $||L||=||h_0||$. Then $\inner{h}{h_0}=\int_{0}^{1}h'(t)\overline{h_0'(t)}dt=L(h)=h(t)=\int_{0}^{t}h'(x)dx$.
    Hence, we can let $\overline{h_0'(t)}=\left\{\begin{matrix}
        1& 0< x\leqs t \\
        0& t<x\leqs 1.
      \end{matrix}\right.$ 
      and so $h_0(t)=\left\{\begin{matrix}
        x& 0< x\leqs t \\
        0& t<x\leqs 1.
    \end{matrix}\right.$
    Then $||L||=||h_0||=\sqrt{\inner{h_0}{h_0}}=\sqrt{\int_{0}^{1}h_0'(x)\overline{h_0'(x)}dx}=\sqrt{\int_{0}^{t}1\cdot 1dx}=\sqrt{t}$.
\end{proof}



\section{Reference}
\begin{itemize}
    \item \href{https://web.mat.bham.ac.uk/~malevao/MSM3P21/l15.pdf}{lecture notes from brmh}
    \item \href{https://math.stackexchange.com/questions/3699482/riesz-representation-theorem-geometric-intuition}{Riesz Representation Theorem geometric intuition}
    \item \href{https://users.math.msu.edu/users/banelson/teaching/920/chI_notes.pdf}{lecture notes from msu}
    \item \href{https://ocw.mit.edu/courses/18-102-introduction-to-functional-analysis-spring-2021/resources/mit18_102s21_lec17/}{lecture notes from mit}
\end{itemize}