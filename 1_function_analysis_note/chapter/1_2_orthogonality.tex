\chapter{Orthogonality}\label{chp:1_2}


\begin{definition}{}{}
    If $\mathscr{H}$ is a Hilbert space, we say $x,y\in \mathscr{H}$ are orthogonality if $<x,y>=0$, 
    in which case we write $x\perp y$.  We say subsets $A,B$ are orthogonal if $x\perp y$ 
    for all $x\in A$ and $y\in B$, and we write $A\perp B$. 
\end{definition}

\begin{theorem}{The Pythagorean Theorem}{}
    If $x_1,x_2,...,x_n\in\hH$ and $x_j\perp x_k$ for $j\neq k$, then
    \begin{align*}
        ||\sum\limits_{j=1}^{n}x_j||^2=\sum\limits_{j=1}^{n}||x_j||^2.
    \end{align*}
\end{theorem}
\begin{proofsolution}
    When $n=2$, expanding the left side.
    Then use induction.
\end{proofsolution}

\begin{remark}
    The pythagorean theorem is strongly related to orthogonality.
\end{remark}


\begin{definition}{}{}
    For any subset $A\subset \hH$, define $A^{\perp}=\{x\in \hH|\inner{y}{x}=0,\forall y\in A\}$.  
\end{definition}


\begin{definition}{Linear Span, Closed Linear Span}{}
    If $A\subset \hH$, let $span(A):=$ the intersetion of all linear subspaces of $\hH$ that contain $A$.
    $span(A)$ is called the linear span of $A$.
    Let $\overline{span(A)}:=$ the intersetion of all closed linear subspaces of $\hH$ that contain $A$.
    $\overline{span(A)}$ is called the closed linear span of $A$.
\end{definition}

\begin{proposition}{}{}
    (1) $span(A)$ is the smallest linear subspace of $\hH$ that contains $A$.\\
    (2) $span(A)=\{\sum\limits_{k=1}^{n}\alpha_kx_k|n\geqs 1,\alpha_k\in \F,x_k\in A\}$.
\end{proposition}

For convenience, we use the notation $A\leqs \hH$ to signify  that $A$ is a closed linear subspace of $\hH$.
and $\overline{span(A)}$ is denoted by $\vee A$.
\begin{proposition}{}{}
    (1) $\vee A\leqs \hH$\\
    (2) $\vee A$ is the smallest closed linear subspace of $\hH$ that contains $A$.\\
    (3) $\vee A=$ the closure of $\{\sum\limits_{k=1}^{n}\alpha_kx_k|n\geqs 1,\alpha_k\in \F,x_k\in A\}$.
\end{proposition}



\begin{definition}{Direct Sum}{}
    $\mathscr{M}_1,\mathscr{M}_2\leqs \hH$. We write 
    \begin{align*}
        \hH=\mathscr{M}_1 \bigoplus \mathscr{M}_2, 
    \end{align*} 
    if $\mathscr{M}_1\cap \mathscr{M}_2=\{0\}$ and for every $x\in\hH$ there exist $v_i\in \mathscr{M}_i$ with $v_1+v_2=x$.
\end{definition}

\begin{proposition}{}{}
If $\mathscr{M}\leqs \hH$, then
    \begin{align*}
        \hH = \mathscr{M}\bigoplus \mathscr{M}^{\perp}
    \end{align*}
\end{proposition}

\begin{definition}{Dense Subset}{}
    Let $(\mathscr{X},||\cdot||)$ be a normed space and $A\subset \mathscr{X}$. Then $A$ is dense in $\mathscr{X}$ if $\overline{A}=\mathscr{X}$. 
\end{definition}

In Conway textbook, non-closed subspaces are called "linear manifolds".

\begin{proposition}{}{}
    Let $\mathscr{Y}$ be a linear manifold in $\hH$. Then $\mathscr{Y}$ is dense iff $\mathscr{H}^{\perp}=\{0\}$.
\end{proposition}


\begin{definition}{Convex Sets}{convex set}
    A subset $A$ of a vector space $\mathscr{X}$ is called convex if 
    \begin{align*}
        \forall x,y\in A, t\in [0,1], tx+(1-t)y\in A.
    \end{align*}
\end{definition}

\begin{proposition}{}{}
    (1) Every subspace is convex.\\
    (2) In a normed linear space, the open ball $B(x,\epsilon)$ is convex.\\
    (3) If $A\subset\mathscr{X}$ is convex and $x\in \mathscr{X}$, then $A+x:=\{y+x|y\in A\}$ is convex.
\end{proposition}
\begin{proof}
    (3) $A$ is convex, then $\forall a,b\in A, t\in[0,1]$, $ta+(1-t)b\in A$. Then $t(a+x)+(1-t)(b+x)=ta+(1-t)b+x\in A+x$.
    Hence $A+x$ is convex.
\end{proof}
\begin{theorem}{}{closed convex smallest norm}
    Every none-empty closed convex subset $A$ of $\hH$ has a unique element of smallest norm, i.e.
    \begin{align*}
        \exists ! a\in A, ||a|| = \inf_{b\in A}||b||.
    \end{align*}
\end{theorem}

\begin{proofsolution}
    Let $\delta=\inf\{||x|||x\in A\}$. Since $A$ is convex, $x+y\in A(\forall x,y\in A$). Then, by the parallelogram law,
    \begin{align*}
        ||x-y||^2=2(||x||^2+||y||^2)-||x+y||^2\leqs 2(||x||^2+||y||^2)-4\delta^2.
    \end{align*}
    Existence follows: by the definition of infimum, we can choose $\{y_n\}_{n=1}^{\infty}\subset A$ for which $||y_n||\rightarrow \delta$. 
    Then when $n,m\rightarrow 0$,
    \begin{align*}
        ||y_n-y_m||^2\leqs 2(||y_n||^2+||y_m||^2)-4\delta^2\rightarrow 0.
    \end{align*}
    Hence, $\{y_n\}$ is Cauchy. By completeness, $\exists y\in \hH$ for which $y_n\rightarrow y$, and since $A$ is closed, $y\in A$. 
    Also $||y||=\lim_{n\rightarrow \infty} ||y_n||=\delta$. \\
    Uniqueness follows: if $||x||=||y||=\delta$, then $||x-y||^2\leqs 4\delta^2-4\delta^2=0$, so $x=y$.
\end{proofsolution}

\begin{corollary}{}{closed convex closet point}
    If $A$ is a nonempty closed convex set in $\hH$ and $x\in \hH$, then there exists a unique closest element of $A$ to $x$.
\end{corollary}
\begin{proof}
    Since $A$ is closed and convex, $A-x$ is closed and convex. By thm\ref{thm:closed convex smallest norm}, let $z_0$ be the unique element of smallest norm in $A-x$
    and let $y_0=z_0+x$. Then $y_0\in A$ and $||y_0-x||=||z_0||=\inf_{z\in A-x}||z|| = \inf_{y\in A} ||y-x||$. 
    Hence, $y$ is the unique closest element of $A$ to $x$. 
\end{proof}

Since the closet point from closed convex set to point can be obtained, we can define the distance about these two object.

\begin{definition}{}{}
    If $A$ is a nonempty closed convex set in $\hH$ and $x\in \hH$, we define the distance from $A$ to $x$:
    \begin{align*}
        \text{dist}(x,A):= \inf_{y\in A}||y-x||.
    \end{align*}
\end{definition}


If the convex set in cor\ref{cor:closed convex closet point} is in fact a closed linear subspace 
of $\hH$, more can be said.

\begin{theorem}{}{orthogonal distance}
    If $\mathscr{M}$ is a closed linear subspace of $\hH$, $x\in \hH$, and $y_0$ is the unique element of $\mathscr{M}$ such that $||y_0-x|| =\text{dist}(x,\mathscr{M})$ , then $x-y_0\perp \mathscr{M}$. 
    Conversely, if $y_0\in\mathscr{M}$ such that $x-y_0\perp\mathscr{M}$, then $||x-y_0||=\text{dist}(x,\mathscr{M})$.
\end{theorem}

\begin{proofsolution}
    You can refer to the lecture notes from brmh.
    % Suppose that $\text{dist}(x,\mathscr{M})=||x-y_0||$. Let $y\in \mathscr{M}$ be an arbitrary vector with $||y||=1$; put $\alpha:=<x-y_0,y>$. Then 
\end{proofsolution}



Note that Theorem\ref{thm:orthogonal distance}, together with the uniqueness statement in 
corollary\ref{cor:closed convex closet point}, shows that if $\mathscr{M}$ is a closed linear subspace of $\hH$ and $x\in \hH$, 
then there is a unique element $y_0\in A$ such that $x -y_0\perp \mathscr{M}$. Thus a function 
$P: \hH\rightarrow \mathscr{M}$ can be defined by $Px = y_0$.


\begin{definition}{}{projection}
    If $\mathscr{M}\leqs \hH$ and $P: \hH\rightarrow \mathscr{M}$ given by $Px = y_0$, 
    where $y_0$ is the unique element in $\mathscr{M}$ such that $x -y_0\perp \mathscr{M}$.
    Then $P$ is called the orthogonal projection of $\hH$ onto $\mathscr{M}$. If we wish to show this dependence of $P$ on $\mathscr{M}$, 
we will denote the orthogonal projection of $\hH$ onto $\mathscr{M}$ by $P_{\mathscr{M}}$.
\end{definition}
\begin{remark}
    $P$ is onto.
\end{remark}


\begin{theorem}{}{}
    If $P$ is the orthogonal projection of $\hH$ onto $\mathscr{M}$, then\\
    (1) $P$ is a linear transformation on $\hH$,\\
    (2) $||Px||\leqs ||x||$ for every $x$ in $\hH$,\\
    (3) $P^2=P$,\\
    (4) $N(P) = \mathscr{M}^{\perp}$ and $R(P)=\mathscr{M}$.
\end{theorem}

\begin{corollary}{}{}
    If $\mathscr{M}\leqs \hH$, then $(\mathscr{M}^{\perp})^{\perp}=\mathscr{M}$. 
\end{corollary}

\begin{proof}
    Let $P=P_{\mathscr{M}}$, then $I-P$ is the orthogonal projection onto $\mathscr{M}^{\perp}$,
    then $\text{ker}(I-P)=(\mathscr{M}^{\perp})^{\perp}$, then $(\mathscr{M}^{\perp})^{\perp}=\text{ker}(I-P)=\text{ran}P=\mathscr{M}$.
\end{proof}

\begin{proposition}{}{}
    For any $A\subset \hH$,\\
    (1) $A^{\perp}\leqs\hH$\\
    (2) $(span(A))^{\perp}=A^{\perp}$\\
    (3) $\overline{A}^{\perp} = A^{\perp}$\\    
    (4) $A^{\perp\perp}=\vee A$.
\end{proposition}

\begin{proof}
    (4) Since $A\subseteq ((A)^{\perp})^{\perp}$ and $\vee A$ is the smallest closed linear subspace containing $A$,
    it follows that $\vee A\subseteq (A^{\perp})^{\perp}$. On the other hand, 
    since $C\subseteq D\Rightarrow D^{\perp}\subseteq C^{\perp}$ and $A\subseteq \vee A$, 
    it follows that $A\subseteq \vee A\Rightarrow (\vee A)^{\perp}\subseteq A^{\perp}\Rightarrow A^{\perp\perp}\subseteq (\vee A)^{\perp\perp}$.
    Since $\vee A$ is closed, it follows that $(\vee A)^{\perp\perp}=\vee A$. Then $A^{\perp\perp}\subseteq \vee A$.
    Hence, $\vee A=A ^{\perp\perp}$.
\end{proof}

\section{Homework}

\begin{exercise}{I2 T2}{}
    If $\mathscr{M}\leqs \mathscr{H}$ and $P=P_{\mathscr{M}}$. 
    show that $I-P$ is the orthogonal projection of $\mathscr{H}$ onto $\mathscr{M}^{\perp}$.
\end{exercise}
\begin{proof}
    Since $\mathscr{M}\leqs \mathscr{H}$, it follows that $\mathscr{M}=(\mathscr{M}^{\perp})^{\perp}$.
    As $P=P_{\mathscr{M}}$, for $h\in\mathscr{H}$, $Ph\in \mathscr{M}$ and $h-ph=(I-P)h\in \mathscr{M}^{\perp}$.
    Also $h-(h-ph)=ph\perp \mathscr{M}^{\perp}$. 
    Since $Ph$ is unique, it follows that $(I-P)h$ is the unique element in $\mathscr{M}^{\perp}$ such that $h-(I-p)h=ph\perp \mathscr{M}^{\perp}$.
    Hence, $I-P$ is the orthogonal projection of $\mathscr{H}$ onto $\mathscr{M}^{\perp}$.
\end{proof}

\begin{exercise}{I2 T3}{}
    (a) If $\mathscr{M}\leqs \mathscr{H}$, show that $\mathscr{M}\cap \mathscr{M}^{\perp}=(0)$ 
    and every $h$ in $\mathscr{H}$ can be written as $h=f+g$ where $f\in \mathscr{M}$ and $g\in \mathscr{M}^{\perp}$. 
    \\
    (b) If $\mathscr{M}+\mathscr{M}^{\perp}\equiv \{(f,g):f\in\mathscr{M},g\in\mathscr{M}^{\perp}\}$ and $T:\mathscr{M}+\mathscr{M}^{\perp}\rightarrow \mathscr{H}$
    is defined by $T(f,g)=f+g$, show that $T$ is a linear bijection and a homeomorphism
    if $\mathscr{M}+\mathscr{M}^{\perp}$ is given the product topology.
    (This is usually phrased by stating the $\mathscr{M}$ and $\mathscr{M}^{\perp}$ are toplogically complementary in $\mathscr{H}$.) 
\end{exercise}

\begin{proof}
    (a) $\forall h\in \mathscr{M}\cap \mathscr{M}^{\perp}$, we have $h\perp h$, then $\inner{h}{h}=0$ and so $h=0$.
    Hence, $\mathscr{M}\cap \mathscr{M}^{\perp}=(0)$.\\
    Let $P=P_{\mathscr{M}}$, then $ph$ is the unique element in $\mathscr{M}$ such that $h-ph\perp \mathscr{M}$, then $h-ph\in \mathscr{M}^{\perp}$.
    Let $f=ph,g=h-ph$, then $f\in \mathscr{M},g\in\mathscr{M}^{\perp}$ and $h=f+g$.
    \\
    (b) 
    Firstly, we show that $T$ is linear. For $c\in\F$ and $(f_1,g_1),(f_2,g_2)\in \mathscr{M}+\mathscr{M}^{\perp}$,
    $T(f_1+f_2,g_1+g_2)=f_1+f_2+g_1+g_2=f_1+g_1+f_2+g_2=T(f_1,g_1)+T(f_2,g_2)$ and $T(cf_1,cg_1)=cf_1+cg_1=cT(f_1,g_1)$.
    Hence, $T$ is linear.
    \par
    Next, we show that $T$ is bijective.
    We know that $T$ is well-defined as $(f_1,g_1)=(f_2,g_2)$, $T(f_1,g_1)=f_1+g_1=f_2+g_2=T(f_2,g_2)$.
    By (a), we know that $T$ is surjective. Now, we show that $T$ is injective.  
    If $T(f_1,g_1)=T(f_2,g_2)$, then $f_1+g_1=f_2+g_2$ and so $f_1-f_2=g_1-g_2$.
    Hence, $f_1-f_2=g_1-g_2\in \mathscr{M}\cap \mathscr{M}^{\perp}=(0)$. 
    Then $f_1=f_2,g_1=g_2$ and so $(f_1,g_1)=(f_2,g_2)$. 
    Hence, $T$ is a bijection.
    \par
    Now we show that $T$ is homeomorphism. It suffices to show that $T$ and $T^{-1}$ is continuous.
    For a sequence $\{(f_n,g_n)\}\in \mathscr{M}+\mathscr{M}^{\perp}$ such that $(f_n,g_n)\underset{n\rightarrow \infty}{\longrightarrow} (f,g)$, 
    we have $f_n\underset{n\rightarrow \infty}{\longrightarrow} f$, $g_n\underset{n\rightarrow \infty}{\longrightarrow}g$ and so $T(f_n,g_n)=f_n+g_n\underset{n\rightarrow \infty}{\longrightarrow} f+g$. Hence, $T$ is continuous.
    For a sequence $\{h_n\}\in \mathscr{H}$ such that $h_n\underset{n\rightarrow \infty}{\longrightarrow} h$, 
    there are sequences $\{f_n\}\subseteq \mathscr{M}$, $\{g_n\}\subseteq \mathscr{M}^{\perp}$ and $f\in\mathscr{M},g\in\mathscr{M}^{\perp}$ 
    such that $h_n=f_n+g_n\underset{n\rightarrow \infty}{\longrightarrow} h=f+g$. 
    Since $f_n\perp g_n$, $f_n\underset{n\rightarrow \infty}{\longrightarrow} f$ and $g_n\underset{n\rightarrow \infty}{\longrightarrow} g$.
    Then $T^{-1}(h_n)=T^{-1}(f_n+g_n)=(f_n,g_n)\underset{n\rightarrow \infty}{\longrightarrow} (f,g)=T^{-1}(f+g)=T^{-1}(h)$.
    Hence, $T^{-1}$ is continuous.
    Hence, $T$ is a homeomorphism.
\end{proof}


\section{Reference}
\begin{itemize}
    \item \href{https://sites.math.washington.edu/~burke/crs/555/555_notes/hilbert.pdf}{lecture notes from washington university}
    \item \href{https://web.mat.bham.ac.uk/~malevao/MSM3P21/l12.pdf}{lecture notes from brmh}
    \item \href{https://web.eecs.umich.edu/~fessler/course/600/l/l03.pdf}{lecture notes from umich}
    \item \href{https://users.math.msu.edu/users/banelson/teaching/920/chI_notes.pdf}{lecture notes from msu}
\end{itemize}


