\chapter{Isomorphic Hilbert Spaces and the Fourier 
Transform for the Circle}\label{chp:1_5}

\begin{definition}{}{isomorphism}
    If $\hH$ and $\hK$ are Hilbert spaces, an isomorphism between $\hH$ and $\hK$
    is linear surjection $U:\hH\rightarrow \hK$ such that
    \begin{align*}
        \inner{Uh}{Ug}=\inner{h}{g}
    \end{align*}
    for all $h,g\in \hH$. In this case $\hH$ and $\hK$ are said to be isomorphic.
\end{definition}

\begin{remark}
    Many call what we call an isomorphism a unitary operator
\end{remark}

% Isomorphism defined above is an equivalent relation:\\
% (1) $\inner{id h}{id g}= <h,g>$\\
% (2) $<h,g>=<U^{-1}Uh,U^{-1}Ug>=<Uh,Ug>$\\
% (3) $<h,g>=<Uh,Ug>=<VUh,VUg>$


from the previous definition, we know that isomorphism preserves inner product.
Now we claim that isomorphism preserves distance and completeness. 

\begin{proposition}{}{}
    If $V:\hH\rightarrow \hK$ is a linear map between Hilbert space, then $V$ is an isometry iff $<Vh,Vg>=<h,g>$ for all $h,g\in \hH$.
\end{proposition}


\begin{exercise}{I5 T6}{}
    Let $\mathscr{C}=\{f\in C[0,2\pi]: f(0)=f(2\pi)\}$ and show that $\mathscr{C}$ is dense in $L^2[0,2\pi]$.
\end{exercise}

\begin{proof}
    In a Hilbert space $\mathscr{H}$, $A\subset \mathscr{H}$ is dense if for $h\in\mathscr{H}$, there exists
    a sequence $(a_n)\subseteq A$ such that $a_n\underset{n\rightarrow \infty}{\longrightarrow} h$.
    Since $C[0,2\pi]$ is dense in $L^2[0,2\pi]$ and $\mathscr{C}\subset C[0,2\pi]$, it suffices to show that $\mathscr{C}$ is dense in $C[0,2\pi]$.
    For $f\in C[0,2\pi]$, then there exists $M>0$ such that $|f(x)|\leqs M$, $\forall x\in [0,2\pi]$.
    Let  $f_n(x)=\left\{\begin{matrix}
       f(x) ,& 0\leqs x\leqs 2\pi-\frac{1}{n} \\
       f(2\pi-\frac{1}{n}) + \frac{f(0)-f(2\pi-\frac{1}{n})}{\frac{1}{n}}(x-2\pi+\frac{1}{n}) ,& 2\pi-\frac{1}{n}<x\leqs 2\pi
      \end{matrix}\right.$.
    Then $f_n(0)=f(0),f_n(2\pi)=f(2\pi-\frac{1}{n}) + \frac{f(0)-f(2\pi-\frac{1}{n})}{\frac{1}{n}}(2\pi-2\pi+\frac{1}{n})=f(0)$.
    Hence, $f_n\in\mathscr{C}$. 
    Since $|f|,|f_n|\leqs M$, it follows that $||f-f_n||=(\int_{[0,2\pi]}|f-f_n|^2 dx)^{1/2}=(\int_{[2\pi-\frac{1}{n},2\pi]}|f-f_n|^2)^{1/2}\leqs (\frac{(2M)^2}{n})^{1/2}=\frac{2M}{\sqrt{n}}$$\underset{n\rightarrow \infty}{\longrightarrow}0$.
    Then $\lim_{n\rightarrow \infty}f_n=f$. Hence, $\mathscr{C}$ is dense in $C[0,2\pi]$ and so is dense in $L^2[0,1]$.
\end{proof}


\begin{exercise}{I5 T9}{}
    If $\mathscr{H}$ and $\mathscr{K}$ are Hilbert spaces and $U:\mathscr{H}\rightarrow \mathscr{K}$ 
    is surjective function such that $\inner{Uf}{Ug}=\inner{f}{g}$ for all vectors $f$ and $g$
    in $\mathscr{H}$, then $U$ is linear.
\end{exercise}

\begin{proof}
    It suffices to show that $U(f+g)=U(f)+U(g)$ and $U(\alpha f)=\alpha U(f)$ for $\alpha\in\F$.
    For $f,g\in \mathscr{H}$, let $a:=U(f+g),b:=U(f)+U(g)$. Then $\inner{a-b}{a-b}=\inner{a}{a}-\inner{a}{b}-\inner{b}{a}+\inner{b}{b}$. \\
    (1) $\inner{a}{a}=\inner{f+g}{f+g}=\inner{f}{f}+\inner{f}{g}+\inner{g}{f}+\inner{g}{g}=\inner{U(f)}{U(f)}+\inner{U(f)}{U(g)}+\inner{U(g)}{U(f)}+\inner{U(g)}{U(g)}=\inner{U(f)+U(g)}{U(f)+U(g)}=\inner{b}{b}$.\\
    (2) $\inner{a}{b}=\inner{U(f+g)}{U(f)+U(g)}=\inner{f+g}{f}+\inner{f+g}{g}=\inner{f}{f}+\inner{f}{g}+\inner{g}{f}+\inner{g}{g}=\inner{U(f)+U(g)}{U(f)+U(g)}=\inner{b}{b}$.\\
    (3) $\inner{b}{a}=\inner{U(f)+U(g)}{U(f+g)}=\inner{f}{f+g}+\inner{g}{f+g}=\inner{f}{f}+\inner{f}{g}+\inner{g}{f}+\inner{g}{g}=\inner{U(f)+U(g)}{U(f)+U(g)}=\inner{b}{b}$.
    Hence, $\inner{a-b}{a-b}=0$ and so $a-b=0$. Then $U(f+g)=U(f)=U(g)$.\\
    And $\inner{U(\alpha f)-\alpha U(f)}{U(\alpha f)-\alpha U(f)}=\inner{U(\alpha f)}{U(\alpha f)}-\inner{U(\alpha f)}{\alpha U(f)}-\inner{\alpha U(f)}{U(\alpha f)}+\inner{\alpha U(f)}{\alpha U(f)}$
    $=\inner{\alpha f}{\alpha f}-\overline{\alpha}\inner{\alpha f}{f}-\alpha\inner{f}{\alpha f}+\alpha\overline{\alpha}\inner{f}{f}=0$. Hence, $U(\alpha f)=\alpha U(f)$.
    Hence, $U$ is linear.
\end{proof}

\section{Reference}

\begin{itemize}
    \item \href{https://personal.math.ubc.ca/~malabika/teaching/ubc/spring18/math421-510/HW3-Solution.pdf}{C([0, 1]) is dense in L2([0, 1]); P5}
\end{itemize}