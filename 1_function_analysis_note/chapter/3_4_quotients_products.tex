\chapter{Quotients and Products of Normed Spaces}\label{chp:3_4}

\begin{exercise}{III4 T1}{}
    Let $\mathscr{X}$ be a normed space, let $\mathscr{M}$ be a linear manifold in $\mathscr{X}$, 
    and let $Q:\mathscr{X}\rightarrow \mathscr{X}/\mathscr{M}$ be the natural map $Qx=x+\mathscr{M}$.
    show $||x+\mathscr{M}||=\inf\{||x+y||:y\in \mathscr{M}\}$ is a norm in $\mathscr{X}/\mathscr{M}$.
    
\end{exercise}

\begin{proof}
    Firstly, we show that the $||x+\mathscr{M}||$ is well defined i.e.
    if $x,x'\in\mathscr{X}$ such that $Q(x)=Q(x')$ then $||x+\mathscr{M}||=||x'+\mathscr{M}||$.
    If $Q(x)=Q(x')$, by $Q$ is linear transformation, $Q(x-x')=0$ and so $x-x'\in \text{ker}Q=\mathscr{M}$. 
    So, $x-x'+\mathscr{M}=\mathscr{M}$.
    Then
    \begin{align*}
        ||x+\mathscr{M}||&=\inf\{||x+y||:y\in\mathscr{M}\}\\
                        &=\inf\{||x+(z-(x-x'))||:z\in x-x'+\mathscr{M}\}\\
                        &=\inf\{||x'+z||:z\in \mathscr{M}\}\\
                        &=||x'+\mathscr{M}||.
    \end{align*}
    Secondly, we show that the $||x+\mathscr{M}||$ satisfys norm axioms.\\
    (1) $||x+\mathscr{M}||\geqs 0$ and $||x+\mathscr{M}||=0\Leftrightarrow x+\mathscr{M}=0_{x/\mathscr{M}}=\mathscr{M}$. 
    In fact, $||x+\mathscr{M}||\geqs 0$ is clear and 
    \begin{align*}
        x+\mathscr{M}=0_{\mathscr{X}/\mathscr{M}}=\mathscr{M}&\Rightarrow ||x+\mathscr{M}||=||\mathscr{M}||\\
        &= \inf_{y\in\mathscr{M}} ||y||\\
        & =0 & (0\in\mathscr{M},||0||=0).
    \end{align*}
        If $||x+\mathscr{M}||=\inf_{y\in\mathscr{M}}||x+y||\overset{z=-y}{=}\inf_{z\in \mathscr{M}}||x-z||=0$, 
        then For each  $n\in\N$, $\exists z_n\in\mathscr{M}$ s.t.
        $0\leqs||x-z_n||\leqs \frac{1}{n}$. Then $\lim_{n\rightarrow \infty}||x-z_n||=0$ and so $z_n\underset{n\rightarrow \infty}{\longrightarrow} x$.
        Since $\mathscr{M}$ is closed, it follows that $x\in\mathscr{M}$. Then $x+\mathscr{M}=Q(x)=0_{\mathscr{X}/\mathscr{M}}=\mathscr{M}$.\\
    (2) For $\alpha\in\F$, $||\alpha (x+\mathscr{M})||=|\alpha|\cdot||x+\mathscr{M}||$. In fact, equality clearly holds when $\alpha=0$, if $\alpha\neq 0$
    \begin{align*}
        ||\alpha(x+\mathscr{M})||&=||\alpha x+\mathscr{M}||=\inf_{y\in\mathscr{M}}||\alpha x+y||\overset{z=\frac{y}{\alpha}}{=}\inf_{z\in\mathscr{M}}||\alpha x+\alpha z||\\
                                &=|\alpha|\inf_{z\in\mathscr{M}} ||x+z|| =|\alpha|\cdot||x+\mathscr{M}||
    \end{align*}
    (3) For $x+\mathscr{M},y+\mathscr{M}$, $||x+y+\mathscr{M}||\leqs ||x+\mathscr{M}||+||y+\mathscr{M}||$. In fact,
    \begin{align*}
        ||x+y+\mathscr{M}||&=\inf_{z\in \mathscr{M}}||x+y+z||\overset{z=z_1+z_2}{=}\inf_{z_1,z_2\in \mathscr{M}}||x+z_1+y+z_2||\\
                        &\leqs \inf_{z_1,z_2\in \mathscr{M}} (||x+z_1||+||y+z_1||)\\
                        &=\inf_{z_1\in\mathscr{M}}||x+z_1|| + \inf_{z_2\in\mathscr{M}}||y+z_1||\\
                        &= ||x+\mathscr{M}||+||y+\mathscr{M}||.
    \end{align*}


\end{proof}

\begin{exercise}{III4 T3}{}
    Show that if $(X,d)$ is a metric space and $\{x_n\}$ is a Cauchy sequence such that
    there is a subsequence $\{x_{n_k}\}$ that converges to $x_0$, then $x_n\rightarrow x_0$.
\end{exercise}
\begin{proof}
    Since the subsequence $\{x_{n_k}\}$ converges to $x\in X$, we know that for all $\epsilon>0$,
    there exists $N_1\in\N$ such that if $n_k\geqs N_1$ then $d(x_{n_k},x)<\frac{\epsilon}{2}$.
    Moreover, since $\{x_n\}$ is Cauchy, for all $\epsilon>0$,
    there exists $N_2\in\N$ such that if $n,m\geqs N_2$, then $d(x_n,x_m)<\frac{\epsilon}{2}$.
    Choose $s\in\N$ such that $n_s>\max \{N_1,N_2\}$.
    Then, if $n\geqs \max \{N_1,N_2\}$, we have that 
    $d(x_n,x)\leqs d(x_n,x_{n_s})+d(x_{n_s},x)<\frac{\epsilon}{2}+\frac{\epsilon}{2}=\epsilon$,
    that is, $x_n$ converges to $x$.
\end{proof}


\section{Reference}

\begin{itemize}
    \item \href{https://proofwiki.org/wiki/Quotient_Norm_is_Norm}{Quotient Norm is Norm}
    \item \href{https://www.lehman.edu/faculty/rbettiol/lehman_teaching/2020mat320/HW3sol.pdf}{III4 T3}
\end{itemize}

