\chapter{Convex Sets and Convex Functions}

\noindent{This chapter mainly introduces} \\
(1) The definition of convex set and convex function \\
(2) The continuous property of convex function \\
(3) Methods for testing convex functions\\


\section{Convex Sets and Convex Functions}

\begin{definition}{Convex Sets and Functions: The Epi-graphical Perspective}{}
    (1) A set $C\subset \R^n$ is said to be a convex set if for every $x,y\in C$ and $\lambda\in [0,1]$, one has
    \begin{align}
        (1-\lambda)x+\lambda y \in C.
    \end{align}
    (2) Let $C$ be a convex subset of $\R^n$. 
    (2) Given an extended real-valued function $f:\R^n\rightarrow \R_e:=\R\cup {\pm \infty}$,
    the epi-graph and domain of $f$ are given by
    \begin{align}
        \text{epi}(f) := \{(x,r)\in \R^n\times \R|f(x)\leqs r\} \\
        \text{dom}(f) := \{x\in \R^n|f(x)\leqs +\infty\},
    \end{align}
    respectively.\\
    (3) The function $f:\R^n\rightarrow \R_e$ is said to be convex if $\text{epi}(f)$ is convex.
\end{definition}

\begin{lemma}{Convexity and Secant Lines}{}
    The function $f:\R^n\rightarrow \overline{\R}:=\R\cup \{+\infty\}$ is convex if and only if,
    for every $x,y\in \text{dom} f$ and $\lambda\in [0,1]$, we have
    \begin{align}
        f((1-\lambda)x+\lambda y) \leqs (1-\lambda) f(x) +\lambda f(y).
        \label{eq:Convexity and Secant Lines}
    \end{align}
    That is, the secant line connecting $(x,f(x))$ and $(y,f(y))$ lies above the graph of $f$.
\end{lemma}

\begin{proof}
    ($\Rightarrow$): If $f$ is convex, then $\text{epi}(f)$ is convex. That is, 
    for $(x,f(x)),(y,f(y))\in \text{epi}(f)$ and $\lambda\in [0,1]$, 
    \begin{align*}
        (1-\lambda) (x,f(x)) +\lambda (y,f(y))\in \text{epi}(f).
    \end{align*}
    That is,
    \begin{align*}
        f((1-\lambda)x+\lambda y)\leqs (1-\lambda)f(x)+\lambda f(y).
    \end{align*}
    ($\Leftarrow$):
    Let $(x,r_1),(y,r_2)\in \text{epi}(f)$ and $\lambda\in [0,1]$. If (\ref{eq:Convexity and Secant Lines}) holds, then 
    \begin{align*}
        f((1-\lambda)x+\lambda y)\leqs (1-\lambda)f(x)+\lambda f(x)\leqs (1-\lambda)r_1+\lambda r_2.
    \end{align*}
    Therefore, $((1-\lambda)x+\lambda y, (1-\lambda)r_1+\lambda r_2)\in \text{epi}(f)$. Then 
    $(1-\lambda) (x,r_1) + \lambda (y,r_2)\in \text{epi}(f)$. So $\text{epi}(f)$ is convex and $f$ is convex.
\end{proof}

\section{Local Lipschitz Continuity of Convex Functions}

Recall that a function $F:\R^n\rightarrow \R^m$ is said to be Lipschitz continuous on a set $S\subset \R^n$
if there is a constant $L>0$ such that 
\begin{align*}
    ||F(x)-F(y)||\leqs L||x-y|| \quad \forall x,y\in S.
\end{align*}
The function $F$ is said to be locally Lipschitz on an open set $V\subset \R^n$ if for every $\overline{x}\in V$ there is an $\epsilon>0$ and $L>0$ such that 
\begin{align*}
    ||F(x)-F(y)||\leqs L||x-y|| \quad \forall x,y\in \overline{x}+\epsilon \mathbb{B}\subset V,
\end{align*}
where $\mathbb{B}:=\{x:||x||\leqs 1\}$. In this section we establish the remarkable fact that a convex function is locally Lipschitz continuous
on the interior of its domain. It is possible to generalize this results to convex functions whose domains
have no interior. But this requires an understanding of the relative topology of convex sets. We begin by
establishing the local boundedness of a convex function on the interior of its domain.

\section{Tests for Convexity}

\section{Reference}
\begin{itemize}
    \item \href{https://www.math.cuhk.edu.hk/course_builder/1920/math4230/}{convex analysis cuhk}
    \item \href{https://ocw.mit.edu/courses/6-253-convex-analysis-and-optimization-spring-2012/resources/mit6_253s12_lec02/}{Convex Analysis and Optimization, Lecture 2}
    \item \href{https://link.springer.com/content/pdf/10.1007/0-387-31082-7_1.pdf}{Convex sets and convex functions:
    the fundamentals}
    \item \href{https://sites.math.washington.edu/~burke/crs/408/notes/Math408_W2020/math408text.pdf}{nonlinear optimization 5.1}
    \item \href{https://sites.math.washington.edu/~burke/crs/408/notes/Math408_W2020/math408text.pdf}{nonlinear optimization 5.3}
    \item \href{https://sites.math.washington.edu/~burke/crs/408/notes/Math408_W2020/math408text.pdf}{nonlinear optimization 5.4}
\end{itemize}
