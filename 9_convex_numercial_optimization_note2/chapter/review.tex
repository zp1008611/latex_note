\chapter{Knowledge review}

\section{Taylor Expansion}
\begin{theorem}{taylor formula for Peano-type remainder of a binary function}{taylor formula for Peano-type remainder of a binary function}
    $f(x,y)$ is a function which is continuously derivable up to order $n$ in some neighborhood of point $(x_0,y_0)$. 
    Then as $\rho = \sqrt{h^2+k^2}\rightarrow 0$, 
    \begin{equation}
        \begin{aligned}
            f(x_0+h,y_0+k)&=f(x_0,y_0)+(h\frac{\partial}{\partial x}+k\frac{\partial}{\partial y})f(x_0,y_0)
            +\frac{1}{2!}(h\frac{\partial}{\partial x}+k\frac{\partial}{\partial y})^2f(x_0,y_0)\\
            &\quad +\dots+
            \frac{1}{n!}(h\frac{\partial}{\partial x}+k\frac{\partial}{\partial y})^n f(x_0,y_0) + o(\rho^n)
        \end{aligned}
        \label{eq:taylor formula for Peano-type remainder of a binary function}
    \end{equation}
    holds. (\ref{eq:taylor formula for Peano-type remainder of a binary function}) 
    is called the taylor formula of order $n$ with Peano-type remainder of $f$ at $(x_0,y_0)$.
\end{theorem}


\begin{proposition}{}{}
    Let $f:X\rightarrow \R^n$, where $X\subset \R^n$ is open. 
    $f$ is differentiable at $\overline{x}$, then there exists a vector $\nabla f(\overline{x})$ 
    and a function $\alpha(\overline{x},y):\R^n\rightarrow \R$ satisfying $\lim_{y\rightarrow 0}\alpha(\overline{x},y)=0$, such that for each $x\in X$
    \begin{align}
        f(x) = f(\overline{x}) +\nabla f(\overline{x})^T(x-\overline{x}) + ||x-\overline{x}||\alpha(\overline{x},x-\overline{x}).
        \label{eq:differentiable taylor}
    \end{align} 
\end{proposition}
\begin{proof}
    Extending Theorem \ref{thm:taylor formula for Peano-type remainder of a binary function} to $n$ dimensions,
    we can get 
    \begin{align*}
        f(x) = f(\overline{x}) + (\frac{\partial f(\overline{x})}{\partial x_1},\frac{\partial f(\overline{x})}{\partial x_2},\dots,\frac{\partial f(\overline{x})}{\partial x_n}) (x-\overline{x})+o(||x-\overline{x}||).
    \end{align*}
    Since $o(||x-\overline{x}||)=||x-\overline{x}||o(1)(x\rightarrow \overline{x})$, let $\alpha(\overline{x},y)=o(1)(y=x-\overline{x}\rightarrow 0)$,
    $\lim_{y\rightarrow 0}\alpha(\overline{x},y)=0$. And let $\nabla f(x)^T = (\frac{\partial f(\overline{x})}{\partial x_1},\frac{\partial f(\overline{x})}{\partial x_2},\dots,\frac{\partial f(\overline{x})}{\partial x_n})$, 
    then (\ref{eq:differentiable taylor}) holds.
\end{proof}

\begin{proposition}{}{}
    Let $f:X\rightarrow \R^n$, where $X\subset \R^n$ is open. 
    $f$ is twice differentiable at $\overline{x}$, then there exists a vector $\nabla f(\overline{x})$ 
    , an $n\times n$ symmetric matrix $H(\overline{x})$ and a function $\alpha(\overline{x},y):\R^n\rightarrow \R$ satisfying $\lim_{y\rightarrow 0}\alpha(\overline{x},y)=0$, such that for each $x\in X$
    \begin{align}
        f(x) = f(\overline{x}) +\nabla f(\overline{x})^T(x-\overline{x}) + \frac{1}{2}(x-\overline{x})^TH(\overline{x})(x-\overline{x}) + ||x-\overline{x}||^2\alpha(\overline{x},x-\overline{x}).
        \label{eq:differentiable taylor}
    \end{align} 
\end{proposition}

\section{Row space, column space and nullspace of matrix}


\section{Reference}
\begin{itemize}
    \item \href{}{mathematical analysis SCNU version}
    \item \href{}{IOE 511/Math 652: Continuous Optimization Methods ch2}
\end{itemize}