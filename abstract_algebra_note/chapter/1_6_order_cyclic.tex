\chapter{Order of Element and Cyclic Group}\label{chp:1_6}

\section{Order of element and cyclic group}

\begin{definition}{}{}
    Let $G$ be a group and $X$ a subset of $G$.
    Let $\{H_i:i\in I\}$ be the set of all subgroups of $G$
    which contain $X$. Then $\cap_{i\in I}H_i$ is the subgroup of $G$
    generated by the set $X$, denoted $\geg{X}$.
\end{definition}

\begin{definition}{}{}
    For group $G$. the elements of $X\subset G$
    are called generators of subgroup $\geg{X}$.
    If $G=\geg{a_1,a_2,...,a_n}$ (notice the set brackets are dropped by convention)
    then $G$ is finitely generated.
\end{definition}

\begin{theorem}{}{}
    If $G$ is group and $X$ is a nonempty subset of $G$,
    then
    \begin{align*}
        \geg{X}=\{x_1^{m_1}x_2^{m_2}\cdots x_k^{m_k}:k\in \N_+, x_1,...,x_k\in X \text{ and } m_1,...,m_k\in\Z\}.
    \end{align*}
    In particular, for every $a\in G$, $\geg{a}=\{a^m:m\in\Z\}$.
\end{theorem}

\begin{definition}{}{}
    Let $G$ be a group. Then 
    $G$ is cyclic if $\exists a\in G$
    such that $G=\geg{a}=\{a^m:m\in\Z\}$.
\end{definition}

\begin{definition}{}{}
    Let $G$ be a group and $a\in G$.
    The order of $a$ is the least positive integer such that $a^n=e$,
    denoted $o(a)$. If such positive integer does not exist, $o(a)=\infty$.
\end{definition}

\begin{proposition}{}{}
    $o(a)=|\geg{a}|.$
\end{proposition}


We now explore the properties of elements of finite and infinite order.

\begin{proposition}{}{}
    Let $G$ be a group and $a\in G$.
    If $a$ has infinite order then\\
    (1) $a^k=e$ if and only if $k=0$.\\
    (2) the elements $a^k$ are all distinct as the values of $k$ range over $\Z$.\\
    If $a$ has finite order $m>0$ then \\
    (3) $a^k=e$ if and only if $m|k$.\\
    (4) $a^r=a^s$ if and only if $r\equiv s\pmod m$.\\
    (5) $\geg{a}$ consists of the distinct elements $a,a^2,...,a^{m-1},a^m=e$.\\
    (6) for each $k\in \Z$, $o(a^k)=\frac{m}{(m,k)}$, where $(m,k)=\text{gcd}(m,k)$. 
\end{proposition}

\begin{proposition}{}{}
    Let $G=\geg{a}$ be a cyclic group.
    If $G$ is infinite, then $a$ and $a^{-1}$ are the only generators of $G$.
    If $G$ is finite of order $m$, then $a^k$ is a generator of $G$ if and only if $(k,m)=1$ (i.e. $k$ and $m$ are coprime). 
\end{proposition}

\begin{proposition}{}{}
    The subgroup of cyclic group is cyclic.
\end{proposition}

\begin{proposition}{}{}
    Subgroup of infinite cyclic group is infinite cyclic group.
\end{proposition}


\begin{proposition}{}{}
    A finite cyclic group of order $n$ contains a subgroup of order m for each positive integer m which
divides n.
\end{proposition}

\begin{proposition}{}{}
    Every infinite cyclic group is isomorphic to the additive group $\Z$
    and every finite cyclic group of order $m$ 
    is isomorphic to the additive group $\Z_m$.
\end{proposition}

\begin{proposition}{}{}
    If a group $G$ has order $p^n$ where $p$
    is a prime, then $G$
    has a element of order $p$.
\end{proposition}

\begin{corollary}{}{}
    A group of order $p$ where $p$ is a prime number is cyclic.
\end{corollary}

\section{Application}
\begin{proposition}{}{}
    Let $G$ be a group of order $n$, then for $a\in G$, $a^n=e$.
\end{proposition}

\begin{definition}{}{}
    Euler's totient function counts the positive integers up to a given integer $n$ that are relatively prime to $n$. 
    It is written using $\varphi(n)$. 
\end{definition}

\begin{proposition}{}{}
    For $m\in \N_+$, $U_m:=\{\bar{a}=a+m\Z: a\in \Z \text{ and } \text{gcd}(a,m)=1\}$,
    then $U_m$ is a commutative semigroup with respect to multiplication 
    and $|U_m|=\varphi(m)$. If $U_m$ is finite, then $U_m$ is a abelian group.
\end{proposition}

\begin{theorem}{}{}
    If $m$ is a positive integer and $a$ is an integer such that $(a,m)=1$, then
    \begin{align*}
        a^{\varphi(m)}\equiv 1\pmod m.
    \end{align*}
\end{theorem}

\section{Reference}
\begin{itemize}
    \item \href{https://sites.millersville.edu/bikenaga/abstract-algebra-1/cyclic-groups/cyclic-groups.pdf}{cyclic group}
    \item \href{https://faculty.etsu.edu/gardnerr/5410/notes/I-3.pdf}{cyclic group}
\end{itemize}