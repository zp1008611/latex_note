\chapter{Preliminary Knowledge}\label{chp:pre1}

\section{Congruence and Congruence Classes}

\begin{definition}{}{}
    An equivalence relation "$\sim$" on a set $S$ is a rule such that \\
    (1) $a\sim a, \forall a\in S$\\
    (2) $a\sim b\Rightarrow b\sim a$\\
    (3) $a\sim b$ and $b\sim c\Rightarrow a\sim c$.   
\end{definition}

You should think of an equivalence relation as a generalization of the notion of equality. Indeed, the usual
notion of equality among the set of integers is an example of an equivalence relation. The next definition
yields another example of an equivalence relation.


\begin{definition}{}{}
    Let $a,b,n\in\Z$ with $n>0$. Then $a$ is congruent to $b$ modulo $n$, denoted by, 
    \begin{align*}
        a \equiv b \pmod n
    \end{align*}
    if $n$ divides $a-b$.
\end{definition}

\begin{example}{}{}
    $17\equiv 5\pmod 6$.
\end{example}
    $(17-5)= 2\cdot 6$.

The following theorem tells us that the notion of congruence defined above is an equivalence relation on the
set of integers.

\begin{theorem}{}{}
    Let $n$ be a positive integer. For all $a,b,c\in\Z$\\
    (1) $a\equiv a\pmod n$\\
    (2) $a\equiv b\pmod n\Rightarrow b\equiv a\pmod n$\\
    (3) $a\equiv b\pmod n$ and $b\equiv c\pmod n\Rightarrow a\equiv c\pmod n$.  
\end{theorem}


\begin{theorem}{}{}
    If $a\equiv b\pmod n$ and $c\equiv d\pmod n$, then\\
    (1) $a+c\equiv b+d\pmod n$\\
    (2) $ac\equiv bd\pmod n$.
\end{theorem}
\begin{proof}
    By the definition of congruence, there are $s,t\in \Z$ such that 
    \begin{align*}
        a-b=sn,c-d=tn.
    \end{align*}
    (1) $a+c-(b+d)=sn-tn=(s-t)n$. Hence, $a+c\equiv b+d\pmod n$.\\
    (2) Using the fact that $-bc+bc=0$, we have
    \begin{align*}
        ac-bd &= ac+0-bd\\
              &= ac + (-bc+bc) - bd\\
              &= c(a-b) + b(c-d)\\
              &= c(sn) + b(tn)\\
              &= n(cs+bt).
    \end{align*}
    Hence, $ac\equiv bd\pmod n$.
\end{proof}

\begin{definition}{}{congruence class definition}
    Let $a$ and $n$ be integers with $n>0$. The congruence class of $a$ modulo $n$, denoted $\overline{a}_n$, 
    is the set of all integers that are congruent to $a$ modulo $n$, i.e.
    \begin{align*}
        \overline{a}_n=\{z\in\Z: a-z=kn \text{ for some }k\in\Z\}.
    \end{align*}  
\end{definition}
\begin{example}{}{}
    In congruence modulo $2$ , we have
    \begin{align*}
        \overline{0}_2&=\{0,\pm 2,\pm 4,...\}\\
        \overline{1}_2 &=\{\pm 1, \pm 3,\pm 5,...\}.
    \end{align*}
\end{example}

\begin{theorem}{}{congruence classes equality}
    $a\equiv c\pmod n$ iff $\overline{a}_n=\overline{c}_n$.
\end{theorem}
\begin{proof}
    ($\Rightarrow$): Assume $a\equiv c\pmod n$. Let $b\in \overline{a}_n$, then by definition\ref{def:congruence class definition}, $b\equiv a\pmod n$. Then
    \begin{align*}
        & a\equiv b\pmod n \text{ and } a\equiv c\pmod n\\
        &\Rightarrow b\equiv c\pmod n\\
        &\Rightarrow c\equiv b\pmod n.
    \end{align*}
So, $b\in \overline{c}_n$. Thus, $\overline{a}_n\subseteq \overline{c}_n$. 
Reversing the roles of $a$ and $c$ in the
argument above we similarly conclude that $\overline{c}_n\subseteq \overline{a}_n$.
Hence, $\overline{a}_n=\overline{c}_n$.

($\Leftarrow$): Suppose $\overline{a}=\overline{c}$. Since $a\equiv a\pmod n$, we have $a\in\overline{a}$ and so, by hypothesis $\overline{a}=\overline{c}$, $a\in\overline{c}$. Hence, 
$a\equiv c\pmod n$.
\end{proof}

\begin{corollary}{}{}
    Two congruence classes modulo $n$ are either disjoint or identical.
\end{corollary}
\begin{proof}
    If $\overline{a}_n$ and $\overline{b}_n$ are disjoint there is nothing to prove. Suppose that $\overline{a}\cap \overline{b}\neq \O$. 
    Then there is an integer $c$ such that $c\in\overline{a}_n$ and $c \in \overline{b}_n$. So $c\equiv a\pmod n$ and $c\equiv b\pmod n$. Then we have
    $a\equiv b\pmod n$.
    Hence, $\overline{a}_n=\overline{b}_n$ by theorem\ref{thm:congruence classes equality}.
\end{proof}

\begin{corollary}{}{}
    There are exactly $n$ distinct congruence classes modulo $n$; namely, 
    $\overline{0}_n,\overline{1}_n,\overline{2}_n,...,\overline{n-1}_n$.
\end{corollary}
\begin{proof}
    We first show that $\overline{0}_n,\overline{1}_n,\overline{2}_n,...,\overline{n-1}_n$
    are all distinct. it is sufficient to show that  no two of $0, 1, 2, . . . , n-1$ are congruent modulo $n$. 
    To see this, suppose that $0\leqs s<t<n$. 
    Then $t-s$ is a positive integer and $t-s<n$. Thus, $n\nmid (t-s)$ and so $t$ is not congruent to $s$ modulo $n$. 
    Then by theorem\ref{thm:congruence classes equality}, the classes
    $\overline{0}_n,\overline{1}_n,\overline{2}_n,...,\overline{n-1}_n$ are all distinct. 

    To complete the proof we need to show that every congruence class is one of these classes.
    Let $a\in\Z$, by the division algorithm, 
    \begin{align}
        a=nq+r, q\in\Z,0\leqs r<n.
        \label{eq:division algorithm}
    \end{align}
    The condition on $r$ implies $r\in\{0,1,2,...,n-1\}$. If we write (\ref{eq:division algorithm}) as
    \begin{align*}
        a -r=nq,
    \end{align*}
    it is clear that $a\equiv r\pmod n$. Thus, by theorem\ref{thm:congruence classes equality}, $\overline{a}=\overline{r}$ for some $r\in\{0,1,2,...,n-1\}$.
\end{proof}

\begin{definition}{}{}
    The set of all congruence classes modulo $n$ is denoted $\Z_n$. Thus
    \begin{align*}
        \Z_n=\{\overline{0},\overline{1},...,\overline{n-1}\}.
    \end{align*}
\end{definition}


\begin{proposition}{}{}
    $\forall \overline{a},\overline{b}\in \Z_n, \overline{a}+\overline{b}=\overline{a+b}$, $\overline{a}\cdot \overline{b} = \overline{ab}$.
\end{proposition}


\section{Chinese Remainder Theorem}

The Chinese remainder theorem arises from the following congruence equations: 
\begin{align*}
    \left\{\begin{matrix}
       x &\equiv a_1\pmod {m_1}\\
       ......\\
       x &\equiv a_n\pmod {m_n},
    \end{matrix}\right.
\end{align*}
where, 
$m_1,...,m_n\in \N$ are pairwise coprime and $a_1,...,a_n\in\Z$. 

How to get the solution of these eqautions? Let's think about it from the view of algebra.
Let us concentrate on the case of $n=3$. The case of $n\neq 3$ is similar.
If we can find a map from $\Z_{abc}\rightarrow \Z_a\times \Z_b\times Z_c$ is bijective, 
and then if we can find the inverse of this map, we can get the solution of these equations by the inverse mapping.

\begin{theorem}{}{}
    Let $a,b,c$ be pairwise relatively prime positive integers. Then the map $f$
    \begin{align*}
        \Z_{abc}\rightarrow \Z_a\times \Z_b\times \Z_c.
    \end{align*}
    defined by the rule $\overline{x}_{abc}\mapsto (\overline{x}_a,\overline{x}_b,\overline{x}_c)$ 
    is a bijection.
    The inverse of $f$ is given by the formula
    \begin{align*}
        f^{-1}(\overline{x}_a,\overline{y}_b,\overline{z}_c) = \overline{xe_1+ye_2+ze_3}_{abc}
    \end{align*}
    where $e_1,e_2$
\end{theorem}




\section{Phi Function}

\section{Reference}
\begin{itemize}
    \item \href{https://math.okstate.edu/people/binegar/3613/3613-l11.pdf}{leture notes from okstate}
    \item \href{https://www.mathi.uni-heidelberg.de/~flemmermeyer/pell/bfc03.pdf}{Residue Class Rings}
    \item \href{https://public.csusm.edu/aitken_html/m422/Handout6.pdf}{THE CHINESE REMAINDER THEOREM AND THE PHI FUNCTION}
\end{itemize}