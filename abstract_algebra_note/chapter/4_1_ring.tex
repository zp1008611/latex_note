\chapter{The Concept and Basic Properties of Rings}\label{chp:4_1}

\begin{definition}{}{}
    A ring is a set $R$ endowed with addition and multiplication, usually denote $"+"$ and $"\cdot"$, satisfying (1)-(3):
    \\
    (1) $R$ is an abelian group with respect to addition : addition is associative and commutative, there is an additive identity $0_R$
    such that $0_R + a=a+0_R = a$ for all $a\in R$, and every element has an additive inverse.\\
    (2) $R$ is an semigroup with respect to multiplication : Multiplication is associative.
    \\
    (3) Addition and multiplication satisfy distributivity: for all $a, b, c \in R$, we have
    \begin{align*}
        a\cdot(b+c)=a\cdot b + a\cdot c, \ (b+c)\cdot a = b\cdot a+c\cdot a.
    \end{align*}
    Most often we will also impose some additional conditions on our rings, as follows:\\

    (4) There exists an element, denoted $1$, which has the property that $a \cdot 1 = 1 \cdot a = a$ for all $a$ in
    $R$, $1$ is called the unity of $R$.\\
    A ring satisfying (4) is called a ring with unity (or sometimes a unital ring).
    \\

    (5) multiplication is commutative : $a\cdot b=b\cdot a$ for all $a,b\in R$. \\
    A ring satisfying (5) is called a commutative ring.    
\end{definition}



\begin{remark}
    We always denote $a\cdot b$ by $ab$.
\end{remark}

\begin{remark}
    As usual we use exponents to denote compounded multiplication; associativity guarantees that the
usual rules for exponents apply. However, with rings (as opposed to multiplicative groups), we must use
a little caution, since $a^k$ may not make sense for $k < 0$, as $a$ is not guaranteed to have a multiplicative
inverse.
\end{remark}

\begin{definition}{}{}
    Let $a, b$ be in a ring $R$. If $a\neq 0$ and $b\neq0$ but $ab = 0$, then
we say that $a$ and $b$ are zero divisors. A commutative unital ring without zero divisors is called integral domain.  
\end{definition}

There are many familiar examples of rings:
\begin{example}{The ring of integers}{The ring of integers}
    $\Z$: the integers $... , -2, -1, 0, 1, 2, ...,$ with usual addition and multiplication, form a ring.
\end{example}

\begin{example}{The ring of residue classes modulo $n$}{The ring of residue classes modulo $n$}
    $\Z/n\Z$: The integers mod n. These are equivalence classes of the integers under the equivalence
relation “congruence mod n”. If we just think about addition, this is exactly the
cyclic group of order n. However, when we call it a ring, it means
we are also using the operation of multiplication.
\end{example}

$+: \Z/n\Z \times \Z/nZ\rightarrow \Z/nZ$ is given by $\overline{a}+\overline{b}=\overline{a+b}$. 
$\cdot: \Z/n\Z \times \Z/nZ\rightarrow \Z/nZ$ is given by $\overline{a}\overline{b}=\overline{ab}$.

\begin{example}{The ring of integer polynomials}{The ring of integer polynomials}
    $\Z[x]$: this is the set of polynomials whose coefficients are integers. It is an “extension” of $\Z$ in the
sense that we allow all the integers, plus an “extra symbol” $x$, which we are allowed to multiply
and add, giving rise to $x^2$, $x^3$, etc., as well as $2x, 3x$, etc.
Adding up various combinations of these gives all the possible integer polynomials.
\end{example}

\begin{example}{The ring of matrices}{The ring of matrices}
    $M_n(\R)$ (non-commutative): the set of $n \times n$ matrices with entries in $\R$. These form a ring, since
we can add, subtract, and multiply square matrices. This is the first example we've seen where
the order of multiplication matters: $AB$ is not always equal to $BA$ (usually it's not).
\end{example}


\begin{remark}
    Similar with example\ref{exa:The ring of integers} and example\ref{exa:The ring of integer polynomials}, for ring $R$, we can define $R[x]$ as 
    the set of polynomials whose coefficients are in the field of $R$ and $M_n(R)$ the set of $n\times n$ matrices with entries in the field of $R$.
     Since, $R[x]$ an “extension” of $R$, they have many similar properties:
    \begin{itemize}
        \item If $R$ is unital ring, then $R[x]$ is unital ring.
        \item If $R$ is commutative ring, then $R[x]$ is commutative ring. 
        \item If $R$ is integer domain, then $R[x]$ is integer domain. 
    \end{itemize}
    However, there are many difference between $R$ and $M_n(R)$.
    \begin{itemize}
        \item If $R$ is unital ring, then $M_n(R)$ is unital ring.
        \item If $R$ is a commutative ring, $M_n(R)$ may not be a commutative ring. 
        \item If $R$ is an integer domain, $M_n(R)$ may not be an  integer domain. 
    \end{itemize}
\end{remark}


\begin
