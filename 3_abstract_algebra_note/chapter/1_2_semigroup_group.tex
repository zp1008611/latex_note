\chapter{Semi-Group and Group}\label{chp:1_2}

\begin{definition}{}{}
    If $G$ is a nonempty set, 
    then a binary operation on $G$ is a function
    from $G \times G$ to $G$. 
    If the binary operation is denoted $\circ$, 
    then we use the notation
    $a \circ b = c$ 
    if $(a, b) \in G \times G$ is mapped to $c \in G$ 
    under the binary operation.
\end{definition}

\begin{remark}
    We will consider both "additive" and "multiplicative" binary operations.
    The only difference between these operations is notational, really. 
    When considering a binary operation, we denote the image of $(a, b)$ as $a + b$. 
    When using multiplicative notation, we denote the image of $(a, b)$ as $ab$ 
    (called the product of $a$ and $b$). 
    Throughout the group theory we use multiplicative notation.
\end{remark}

\begin{definition}{}{definition of group}
    For a multiplicative binary operation on $G\times G$,
    we define the following properties:\\
    (1) Multiplication is associative if $a(bc)=(ab)c$ for all $a,b,c\in G$.\\
    (2) Element $e\in G$ is a identity if $ae=ea=a$ for all $a\in G$.\\
    (3) Element $a\in G$ has a inversee $b$ 
    if for some $b\in G$ we have $ab=ba=e$.\\
    A semigroup is a nonempty set $G$ with an associative binary operation.\\
    A monoid is a semigroup with an identity.\\
    A group is a monoid such that each $a\in G$ has an inverse.\\
    In a semigroup, we define the property:\\
    (4) Semigroup $G$ is abelian or commutative if $ab=ba$ for all $a,b\in G$.\\
    The order of a semigroup/monoid/group is the cardinality of set $G$, denoted $|G|$.
    If $|G| < \infty$, then the semigroup/monoid/group is said to be finite.
\end{definition}

\begin{remark}
    If we define a binary algebraic structure as a set with a binary operation on
    it, then we have the following schematic:\\

    \hspace{2cm}(Binary Algebraic Structures) $\supseteq$ (Semigroups) $\supseteq$ (Monoids) $\supseteq$ (Groups).
\end{remark}

\begin{proposition}{Generalized Associative Law}{}
    Let $G$ be a semigroup. For any $a_1,...,a_n\in G$, 
    the value of $a_1\cdots a_n$
    is independent of how the expression is bracketed.
\end{proposition}

\begin{proposition}{Generalized Commutative Law}{Generalized Commutative Law}
    If $G$ is a commutative semigroup and $a_1, a_2, . . . , a_n \in G$ 
    then for any permutation
    $i_1, i_2, . . . , i_n$ of $i = 1, 2, . . . , n$
     we have the products $a_1a_2 \cdots a_n = a_{i_1} a_{i_2}\cdots a_{i_n}$.
\end{proposition}

\begin{definition}{}{}
    Let $G$ be a (multiplicative) semigroup, $a \in G$, and $n \in \N$. 
    The element $a^n$ is defined as 
    the standard $n$ product $a^n =\prod_{i=1}^{n}a_i$
    where $a = a_i$ for $1 \leqs i \leqs n$. 
    If $G$ is a monoid, $a^0$ is defined to be the identity element of $G$, 
    $a^0=e$. If $G$ is a group, 
    then for each $n \in \N$, $a^{-n}$
    is defined to be $a^{-n} = (a^{-1})^n\in G$.
    If $G$ is an additive group we replace $a^n, a^0, a^{-n}$, and $e$ 
    with $na, 0a, -na$, and $0$, respectively.
\end{definition}

\begin{proposition}{}{}
    If $G$ is a group (respectively, semigroup, monoid) and $a \in G$,
    then for all $m, n \in \Z$ (respectively, $\N$, $\N \cup \{0\}$):
    (1) $a^ma^n=a^{m+n}$ (or $ma + na = (m + n)a$ in additive notation);
    \\
    (2) $(a^m)^n=a^{mn}$(or $n(ma) = (mn)a$ in additive notation).
\end{proposition}

\begin{proposition}{}{}
    If G is a monoid, then the identity element $e$ is unique. If $G$ is a group then:\\
    (1) If $c \in G$ and $cc = c$ then $c = e$.\\
    (2) For all $a,b\in G$, if $ab=ac$ then $b=c$, 
    and if $ba=ca$ then $b=c$ (these properties
    of a group is called left cancellation and right cancellation, respectively).\\
    (3) For $a\in G$, the inverse of $a$ is unique.\\
    (4) For all $a\in G$, we have $(a^{-1})^{-1}=a$.\\
    (5) For all $a,b\in G$, we have $(ab)^{-1}=b^{-1}a^{-1}$.\\
    (6) For all $a,b\in G$, the equations $ax=b$ and $ya=b$ have unique solutions in $G$,
    namely $x=a^{-1}b$ and $y=ba^{-1}$, respectively.
\end{proposition}

\begin{remark}
    We can weaken the two-sided identity and inverse properties used in Definition \ref{def:definition of group} 
    of group. It turns out that if we simply assume right inverses and a right
    identity (or just left inverses and a left identity) then this implies the existence of
    left inverses and a left identity (and conversely), as shown in the following theorem.
\end{remark}

\begin{proposition}{}{}
    Let $G$ be a semigroup. 
    Then $G$ is a group if and only if the
    following conditions hold.\\
    (1) There exists an element $e \in G$ such that $ea = a$ for all $a \in G$ (called a left
    identity in G).
    (ii) For each $a \in G$, there exists an element $a^{-1}\in G$ such that $a^{-1}a=e$
    ($a^{-1}$ is called a left inverse of a).
\end{proposition}

The following proposition gives another condition by which a semigroup is
a group.

\begin{proposition}{}{}
    (1) Let $G$ be a semigroup. Then $G$ is a group if and only if for
    all $a, b \in G$ the equations $ax = b$ and $ya = b$ have solutions in $G$.\\
    (2) Let $G$ be a finite semigroup, if $G$ satisfys cancellation law, then $G$ is a group.
\end{proposition}

\section{Reference}

\begin{itemize}
    \item \href{https://faculty.etsu.edu/gardnerr/5410/notes/I-1.pdf}{Semigroups, Monoids, and Groups}
\end{itemize}