\chapter{The Direct Sum of Ring and Chinese Remainder Theorem}\label{chp:4_3}


\begin{definition}
    Given rings $R_1,...,R_n$, the outer direct sum of $R_1,...,R_n$ (denoted by $R_1\oplus\cdot\cdot\cdot\oplus R_n$) is
    \begin{align*}
        \{(r_1,...,r_n):r_i\in R_i\},
    \end{align*}
    with addtion and multiplication defined by 
    \begin{align*}
        (r_1,...,r_n) + (s_1,...,s_n) = (r_1+s_1,...,r_n+s_n)
    \end{align*}
    and 
    \begin{align*}
        (r_1,...,r_n)\times (s_1,...,s_n)=(r_1s_1,...,r_ns_n),
    \end{align*}
    where the operation in the $i$-th coordinate position is the relevant operation in $R_i$.
\end{definition}
It can be check that $R_1\oplus\cdot\cdot\cdot\oplus R_n$ is a ring.

\begin{theorem}{}{}
    $R = R_1\oplus ... \oplus R_n$. Put
    \begin{align*}
        R_i^* &= \{(x_1,...,x_n)\in R:x_i\in R_i\text{ and }x_j=e_j, \forall j\neq i\}
    \end{align*}
    and define $\sigma_i : R\rightarrow R_i$ by $\sigma_i(x_1,...,x_n)=x_i$. Then\\
    (1) $\sigma_i$ is a homomorphism, $\Im \sigma_i = R_i$ and $\Ker\sigma_i = R_i^*$\\
    (2) $R_i^*\unlhd R$.\\
    And more, \\
    (3) $\forall i, R_i\cong R_i^*$\\
    (4) $R_1^*\cdot\cdot\cdot R_n^*=G$\\
    (5) $\forall i, R_i^*\cap \prod_{j\neq i}R_j^*=\{0\}$.
\end{theorem}

\begin{proposition}{}{}
    If $I_1,...,I_m$ are ideals of $R$, $R$ is called the inner direct sum of $I_1,..,I_m$ if\\
    (1) $I_1+..+I_m=R$, and\\
    (2) $\forall i, I_i\cap \sum\limits_{j\neq i}^{}I_j=\{0\}$
\end{proposition}
\begin{remark}
    Since $(I,+)$ is a normal subgroup of $(R,+)$, the definition makes sense.
\end{remark}

\begin{theorem}{}{}
    Let $R$ be a ring and let $I_1,...,I_m$ be ideals of $R$. 
    If $R$ is the inner direct sum of $I_1,...,I_m$, then $R \cong I_1\oplus ... \oplus I_m$.
\end{theorem}


\begin{definition}{}{}
    Let $R$ be a ring and $I,J\subseteq R$ be ideals. The sum and product of $I$ and $J$ are the ideals
    \begin{align*}
        I+J&=\gi{I\cap J},\\
        IJ&=\gi{I\cdot J},
    \end{align*}
    where $I\cdot J=\{ab:a\in I,b\in J\}$.
\end{definition}

\begin{lemma}{}{}
    Let $R$ be a ring with ideals $I$ and $J$. Then
    \begin{align*}
        I+J &= \{a+b:a\in I,b\in J\},\\
        IJ &= \{\sum\limits_{i=1}^{n}a_ib_i:a_i\in I,b_i\in J,n\in\N\}. 
    \end{align*}
\end{lemma}

The distributive law holds for ideals $I,J,K$,
\begin{itemize}
    \item $I(J+K)=IJ+IK$
    \item $(I+J)K=IK+JK$.
\end{itemize}
If a product is replaced by an intersection, a partial distributive law holds: $I\cap(J+K)\supset I\cap J+I\cap K$.



\begin{proposition}{}{}
    Let $I,J$ be ideals in a ring $R$. 
    Then $IJ\subseteq R$ and $IJ\subseteq I$.
\end{proposition}
\begin{proof}
    Because $I$ and $J$ are ideals, 
    the elements of $I \cdot J$ all belong to both $I$ and $J$.
    Thus, $I\cdot J \subseteq I\cap J$, which implies $IJ \subseteq I \cap J$.
\end{proof}
In number theory, two integers $a$ and $b$ are coprime 
iff there exists $m, n \in Z$, such that 
$ma + nb = 1$. 
This identity is called B\'ezout's identity. 
By this identity, we can define coprime about ideals.



\begin{definition}{}{}
    Two ideals $I$ and $J$ in a ring $R$ are called coprime if $A+B=R$.
\end{definition}

\begin{proposition}{}{}
    Let $R$ be an unital ring. Then two ideals $I$ and $J$ in $R$
    are coprime iff $1\in I+J$.
\end{proposition}

\begin{proposition}{}{}
    Let $R$ be a ring with unity and $I,J$ are ideals in $R$. Then
    \begin{align*}
        (I\cap J)(I+J)\subseteq IJ+JI.
    \end{align*}
\end{proposition}
\begin{proof}
    \begin{align*}
        (I\cap J)(I+J)&= (I\cap J)I + (I\cap J)J &(I\cap J\subseteq J, I\cap J\subseteq I)\\
                      &\subseteq JI+IJ 
    \end{align*}
\end{proof}

\begin{lemma}{}{}
    Let $R$ be an unital ring. Two ideals $I$ and $J$ in $R$ are coprime, then
    \begin{align*}
        IJ+JI=I\cap J.
    \end{align*}  
    In particular, if $R$ is commutative, $IJ=I\cap J$.
\end{lemma}

\begin{proof}
    Since $IJ,JI\subseteq I\cap J$ and $I\cap J$ is a ring, $IJ+JI\subseteq I\cap J$.
    Also, $I+J=R$ and $(I\cap J)R=I\cap J$, then $(I\cap J)\subset IJ+JI$.
\end{proof}

\begin{lemma}{}{}
    Let $R$ be an unital ring and $I,J,K$ are ideals in $R$. If $I+J=R$ and $I+K=R$, 
    then $I$ and $JK$ are coprime.
\end{lemma}

\begin{proof}
    It suffices to show that $1\in I+JK$. Since $I+J=R$, there exists $i\in I$ and $j\in J$
    such that $i+j=1$. Since $I+K=R$, there exists $i'\in I, k\in K$ such that $i'+k=1$.
    Hence in $R$,
    \begin{align*}
        1=1\cdot 1=(i+j)(i'+k)=ii'+ik+ji'+jk=(ii'+ik+ji')+jk\in I+JK.
    \end{align*}
\end{proof}


\section{Reference}
\begin{itemize}
    \item \href{https://webspace.maths.qmul.ac.uk/r.a.bailey/MAS305/algnotes14.pdf}{lecture notes from queen mary}
    \item \href{https://www.youtube.com/watch?v=_1Ts5TcHyQ4}{Product of two ideals}
    \item \href{http://ramanujan.math.trinity.edu/rdaileda/teach/f19/m4363/arith2.pdf}{The Multiplicative Arithmetic of Ideals}
    \item \href{https://public.csusm.edu/aitken_html/m422/Handout6.pdf}{THE CHINESE REMAINDER THEOREM AND THE PHI FUNCTION}
\end{itemize}