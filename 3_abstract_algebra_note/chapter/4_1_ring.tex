\chapter{The Concept and Basic Properties of Rings}\label{chp:4_1}

\begin{definition}{}{}
    A ring is a set $R$ endowed with addition and multiplication, usually denote $"+"$ and $"\cdot"$, satisfying (1)-(3) :
    \\
    (1) $R$ is an abelian group with respect to addition : addition is associative and commutative, there is an additive identity $0_R$
    such that $0_R + a=a+0_R = a$ for all $a\in R$, and every element has an additive inverse.\\
    (2) $R$ is an semigroup with respect to multiplication : Multiplication is associative.
    \\
    (3) Addition and multiplication satisfy distributivity: for all $a, b, c \in R$, we have
    \begin{align*}
        a\cdot(b+c)=a\cdot b + a\cdot c, \ (b+c)\cdot a = b\cdot a+c\cdot a.
    \end{align*}
    Most often we will also impose some additional conditions on our rings, as follows:\\

    (4) There exists an element, denoted $1$, which has the property that $a \cdot 1 = 1 \cdot a = a$ for all $a$ in
    $R$, $1$ is called the unity of $R$.\\
    A ring satisfying (4) is called a ring with unity (or sometimes a unital ring).
    \\

    (5) multiplication is commutative : $a\cdot b=b\cdot a$ for all $a,b\in R$. \\
    A ring satisfying (5) is called a commutative ring.    
\end{definition}



\begin{remark}
    We always denote $a\cdot b$ by $ab$.
\end{remark}

\begin{remark}
    As usual we use exponents to denote compounded multiplication; associativity guarantees that the
usual rules for exponents apply. However, with rings (as opposed to multiplicative groups), we must use
a little caution, since $a^k$ may not make sense for $k < 0$, as $a$ is not guaranteed to have a multiplicative
inverse.
\end{remark}


\begin{definition}{}{}
    Let $R$ be a ring. The additive identity in $R$ is called the zero in $R$. 
\end{definition}

\begin{proposition}{}{}
    For any element $a$ in a ring $R$, one has $a0 = 0 = 0a$ (zero is an absorbing element with respect to multiplication). 
\end{proposition}

\begin{proposition}{}{}
    For elements $a_1,...,a_m,b_1,...,b_n$, one has
    \begin{align*}
        (a_1+...+a_m)(b_1+...+b_n)=\sum\limits_{i=1}^{m}\sum\limits_{j=1}^{n}a_ib_j.
    \end{align*}
\end{proposition}

\begin{definition}{}{}
    For any element $a$ in a ring $R$ and $n\in\N$.
    \begin{align*}
        na &:= \underbrace{a+...+a}_{n \text{ elements}},\\
        (-n)a=n(-a) &:= \underbrace{-a-...-a}_{n \text{ elements}}. 
    \end{align*}
\end{definition}

\begin{proposition}{}{}
    For $a,b$ in a ring $R$ and $m\in\Z$, one has 
    \begin{align*}
        (ma)b=a(mb)=m(ab).
    \end{align*}
\end{proposition}


\begin{definition}{}{}
    Let $a, b$ be in a ring $R$. If $a\neq 0$ and $b\neq0$ but $ab = 0$, then
we say that $a$ and $b$ are zero divisors. A commutative ring with unity but without zero divisors is called integral domain.  
\end{definition}
A natural question: under what conditions, the ring has no zero divisors. THe answer in the next proposition.
\begin{proposition}{}{}
    Let $R$ be a ring, then $R$ have no zero divisors iff
    $R$ satisfy cancellation law:
    \begin{align*}
        \text{ if } a,b,c\in R \text{ and } a\neq 0, \text{ then } ab=ac \text{ implies } b=c \text{ and } ba=ca \text{ implies } b=c.
    \end{align*}
\end{proposition}
\begin{proof}
    ($\Leftarrow$): Suppose $R$ satisfys cancellation law. Assume $R$ has zero divisors i.e. 
    there exists $a,b\in R\setminus \{0\}$ such that $ab=0$. Since $a0=0=ab$, we have $b=0$ by the cancellation law. 
    This is a contradiction as $b\neq 0$. Hence, $R$ has no zero divisors and so is a integer domain.
    ($\Rightarrow$): Suppose $R$ is a integral domain. Then for $a,b,c\in R$ and $a\neq 0$, we have
    \begin{align*}
        ab=ac\Rightarrow a(b-c)=ab-ac=0\Rightarrow b-c=0\Rightarrow b=c.
    \end{align*}
    Similarly, $ba=ca\Rightarrow (b-c)a=0\Rightarrow b=c$.
\end{proof}

\begin{theorem}{}{}
    Let $R$ be a commutative ring, then for any $a,b\in R$ and $n\in\N$, one has
    \begin{align*}
        (a+b)^n = a^n + \sum\limits_{0<k<n}\frac{n!}{k!(n-k)!}a^kb^{n-k} + b^n.
    \end{align*}
\end{theorem}

There are many familiar examples of rings:
\begin{example}{The ring of integers}{the ring of integers}
    $\Z$: the integers $... , -2, -1, 0, 1, 2, ...,$ with usual addition and multiplication, form a ring.
\end{example}

\begin{example}{The ring of residue classes modulo $n$}{the ring of residue classes modulo $n$}
    $\Z/m\Z$: The integers mod $m$. These are equivalence classes of the integers under the equivalence
relation “congruence mod $m$”. If we just think about addition, this is exactly the
cyclic group of order $m$. However, when we call it a ring, it means
we are also using the operation of multiplication.
\end{example}

$+: \Z/m\Z \times \Z/mZ\rightarrow \Z/mZ$ is given by $\overline{a}+\overline{b}=\overline{a+b}$. 
$\cdot: \Z/m\Z \times \Z/mZ\rightarrow \Z/mZ$ is given by $\overline{a}\overline{b}=\overline{ab}$.

\begin{example}{The ring of integer polynomials}{the ring of integer polynomials}
    $\Z[x]$: this is the set of polynomials whose coefficients are integers. It is an “extension” of $\Z$ in the
sense that we allow all the integers, plus an “extra symbol” $x$, which we are allowed to multiply
and add, giving rise to $x^2$, $x^3$, etc., as well as $2x, 3x$, etc.
Adding up various combinations of these gives all the possible integer polynomials.
\end{example}

\begin{example}{The ring of matrices}{the ring of matrices}
    $M_n(\R)$ (non-commutative): the set of $n \times n$ matrices with entries in $\R$. These form a ring, since
we can add, subtract, and multiply square matrices. This is the first example we've seen where
the order of multiplication matters: $AB$ is not always equal to $BA$ (usually it's not).
\end{example}


\begin{remark}
    Similar with example\ref{exa:the ring of integer polynomials} and example\ref{exa:the ring of matrices}, for ring $R$, we can define $R[x]$ as 
    the set of polynomials whose coefficients are in the field of $R$ and $M_n(R)$ the set of $n\times n$ matrices with entries in the field of $R$.
     Since, $R[x]$ an “extension” of $R$, they have many similar properties:
    \begin{itemize}
        \item If $R$ is unital ring, then $R[x]$ is unital ring.
        \item If $R$ is commutative ring, then $R[x]$ is commutative ring. 
        \item If $R$ is integer domain, then $R[x]$ is integer domain. 
    \end{itemize}
    However, there are many difference between $R$ and $M_n(R)$.
    \begin{itemize}
        \item If $R$ is unital ring, then $M_n(R)$ is unital ring.
        \item If $R$ is a commutative ring, $M_n(R)$ may be not a commutative ring. 
        \item If $R$ is an integer domain, $M_n(R)$ may be not an  integer domain. 
    \end{itemize}
\end{remark}

\begin{definition}{}{}
    If $R$ is a ring, and $S$ is a non-empty subset of $R$, we will say that $S$ is a subring of $R$ if \\
    (1) $S$ is a subgroup of $R$ under $+$. (closed under addition and subtraction)\\
    (2) $S$ is closed under multiplication. 
\end{definition}

\begin{remark}
    The minimum subring in $R$ is zero ring $O=\{0\}$, the maximum subring in $R$ is $R$, 
    which is similar to the minimum subgroup ${e}$ in group $G$ and the maximum subgroup $G$ in $G$. 
\end{remark}

\begin{example}{}{}
    $\Z[i]=\{a+bi:a,b\in\Z\}$ is a ring with respect to the addition and multiplication in $\C$, which is called Gaussian integers.
    Integer ring is a subring of Gaussian integers.  
\end{example}

\begin{definition}{}{}
    Let $a$ be an element of a unital $R$. We say that $a$ is a unit if $a$ has a multiplicative inverse, i.e. 
    if there exists an element $b$ in $R$ such that $ab=ba=1_R$. $b$ is denoted by $a^{-1}$.
    The set of units in $R$ is a group with respect to multiplication, which is called the unit group of $R$ and denoted by $U(R)$.
\end{definition}
\begin{remark}
    $R$ must have unity.
\end{remark}
\begin{remark}
    If $u,v$ are units in $R$, then $uv$ is a unit in $R$ as $(uv)(v^{-1}u^{-1})=1_R$.
\end{remark}

\begin{example}{}{}
    $U(\Z)=\{1,-1\}$.
\end{example}

$1\cdot 1=1,(-1)\cdot (-1)=1$


\begin{example}{}{}
    $U(\Z/mZ) = \{\overline{a}=a+m\Z: a\in\Z, gcd(a,m)=1\}$
\end{example}

$\overline{a}\overline{b}=\overline{ab}=1\Leftrightarrow ab\equiv 1\pmod m\Leftrightarrow ab+my=1\Leftrightarrow \text{gcd}(a,m)=1$.

We also can prove that $U(\Z/mZ)$ (some textbooks denote it by $U_m$ or $\Z_m^*$) is a group with multiplication.

\begin{proposition}{}{identity is zero when zeroring}
    Let $R$ be a ring, then 
    $R=\{0\}$ iff $0_R=1_R$.
\end{proposition}

\begin{proof}
    ($\Rightarrow$): $0\cdot 0 = 0\cdot 0=0$, then $1_R=0_R$. \\
    ($\Leftarrow$): If $0_R=1_R$, then $\forall a\in R$, $a=a\cdot 1_R=a\cdot 0_R=0$. Hence, $R=\{0\}$.  
\end{proof}

By proposition\ref{prop:identity is zero when zeroring}, 
we know that zero is not equal to identity for a general ring.
Hence, zero always has no inverse. So if every non-zero element in a ring has inverse, 
we can get more.

\begin{definition}{}{}
    Let $R$ be a unital ring. $R$ is called skew field if $U(R)=R\setminus \{0\}$. 
\end{definition}


\begin{definition}{}{}
    Let $R$ be a skew field. $R$ is called a field if $R\setminus \{0\}$ is abelian group with respect ro multiplication.
\end{definition}

\begin{proposition}{}{}
    Let $R$ be a field, then $R$ is a integer domain.
\end{proposition}
\begin{proof}
    Suppose $R$ is not a integer domain, i.e. there exists $0 \neq a,b\in R, ab=0$.
    But
    \begin{align*}
        ab = 0\Rightarrow a^{-1}ab=0\Rightarrow b=0.
    \end{align*}
    This is a contradiction as $b\neq 0$.
    Hence, $R$ is a integral domain.
\end{proof}

\begin{example}{}{}
    $\Q,\R,\C$ are field.
\end{example}

\begin{example}{}{modulo prime is a field}
    Let $p$ be a prime, then $\Z/p\Z$ is a field.
\end{example}

For $a,b,c\in \Z/p\Z\setminus \{0\}$, $\overline{a}\cdot\overline{b}=\overline{a}\cdot\overline{c}\Rightarrow ab-ac=np\Rightarrow a(b-c)=np\Rightarrow p|a(b-c) \overset{\text{ gcd(a,p)=1}}{\Longrightarrow} p|b-c\Rightarrow \overline{b}=\overline{c}$, 
then $\Z/p\Z\setminus \{0\}$ satisfys cancellation law.
Hence, $\Z/p\Z$ is a group and commutative is clear.


Let us construct the smallest and also most famous example of skew field. 
Take $1,i,j,k$ to be basis vectors for a 4-dimensional vector space over $\R$, 
and define multiplication by
\begin{align*}
    i^2=j^2=k^2=-1,ij=k,jk=i,ki=j,ji=-ij,kj=-jk,ik=-ki.
\end{align*}

Then 
\begin{align*}
    \mathbb H = \{a+bi+cj+dk:a,b,c,d\in\R\}
\end{align*}
forms a skew field, called the Hamilton's quaternions. So far, we have
only seen the ring structure. Let us now discuss the fact that every nonzero element is invertible.  
Define the conjugate of an element $h=a+bi+cj+dk\in\mathbb{H}$ to be $\overline{h}=a-bi-cj-dk$(yes, exactly the same way you did it for complex numbers). 
And define the norm of $h$ to be $|h|=\sqrt{a^2+b^2+c^2+d^2}$.
Then
\begin{align*}
    h\overline{h} = (a+bi+cj+dk)(a-bi-cj-dk)=&a^2-abi-acj-adk\\
                                             &+bai-b^2i^2-bcij-bdik\\
                                             &+caj-cbji-c^2j^2+cdjk\\
                                             &+dak-dbki-dckj-d^2k^2\\
                                            =& a^2+b^2+c^2+d^2=|h|^2.
\end{align*}

If $h\neq 0$, then $h\overline{h}=\overline{h}h\neq 0$, take $h^{-1}$ to be
\begin{align*}
    h^{-1} = \frac{\overline{h}}{h\overline{h}}.
\end{align*}
Clearly $hh^{-1}=h^{-1}h=1$. Hence, $U(\mathbb{H})=\mathbb{H}\setminus \{0\}$. 
Then $\mathbb{H}$ is a skew field.

\section{Reference}
\begin{itemize}
    \item \href{https://feog.github.io/ring.pdf}{lecture notes by feog}
\end{itemize}
