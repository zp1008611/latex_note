\chapter{The Degree of Field Extension}\label{chp:6_2}

\section{Field extensions}


\begin{definition}{}{}
    Let $R$ be a ring with unity. A (Left) R-module is an additive abelian group $V$ together with a function mapping
    $R\times V\rightarrow V$(the image of $(a,x)$ being denoted $ax$) such that for all $a,b\in R$ and $x,y\in V$:\\
    (1) $1_Rx=x$\\
    (2) $(ab)x=a(bx)$\\
    (3) $(a+b)x=ax+bx$\\
    (4) $a(x+y)=ax+ay$ 
\end{definition}

\begin{remark}
    If $R$ is a field, then R-module is called a (left) vector space.
\end{remark}

\begin{definition}{}{}
    A field $L$ is an extension field of field $K$ if $K$ is a subfield of $L$. 
    And we denote the corresponding field extension by $L/K$.
\end{definition}

\begin{remark}
    with $R=K$ (the field of "scalars") and $V=L$ (the additive abelian group of "vectors"), we see that $L$ is a vector space over $K$.
    It then makes sense to speak of the dimension of $L$ over $K$.
\end{remark}


\begin{definition}{}{}
    Let $L/K$ be a field extension. The dimension of $L$ as a vector space over $L$
    is called the degree of the extension, written $[L:K]$.
    If $[L:K]<\infty$, we say that $L$ is a finite extension of $K$, or that the extension $L/K$ is finite.
\end{definition}

\begin{proposition}{}{}
    Let $L/M$ and $M/K$ be finite field extension, then
    \begin{align*}
        [L:K]=[L:M][M:K].
    \end{align*}
\end{proposition}

\begin{definition}{}{}
    For field $L$ and $\O\neq X\subset L$, the subfield (respectively, subring) generated by $X$
    is the intersection of all subfields (respectively, subrings) of $L$ that contain $X$.  
    i.e., the smallest subfield (resp. subring) of
    $F$ containing $X$.
\end{definition}

\begin{definition}{}{}
    Let $L/K$ be a field extension and $\O\neq X\subset L$. Then the subfield (respectively, subring)
    generated by $K\cup X$ is the subfield(respectively, subring) generated by $X$ over $K$ and is denoted $K(X)$ 
    (or, respectively, in the case of rings, $K[X]$).
    i.e., the smallest subfield (resp. subring) of $F$ containing $K\cup X$.
\end{definition}

\begin{definition}{}{}
    If $X=\{\alpha_1,\alpha_2,...,\alpha_n\}$, then the subfield $K(X)$ (respectively, subring $K[X]$)
    of $F$ is denoted $K(\alpha_1,\alpha_2,...,\alpha_n)$ (respectively, $K[\alpha_1,...,\alpha_n]$). 
    The field $K(\alpha_1,...,\alpha_n)$
    is a finitely generated extension of $K$. If $X=\{\alpha\}$ then $K(\alpha)$ is a simple extension of $K$.
\end{definition}

\begin{remark}
    The field $K(\alpha_1,...,\alpha_n)$ is a finitely generated extension of $K$ but it may not be a finite dimensional extension over $K$.
\end{remark}

\begin{theorem}{}{}
    Let $L/K$ be a field extension and $X\subseteq L$ and $\alpha$, $\alpha_i\in L$. Then\\
    (1) the subring $K[\alpha]$ consists of all elements of the form $f(\alpha)$ where $f$ is a polynomial with coefficients in $K$ (that is, $f\in K[x]$), i.e. 
    $K[\alpha]=\{f(\alpha):f(x)\in K[x]\}$. \\
    (2) the subring $K[\alpha_1, \alpha_2, . . . , \alpha_n]$ consists of all elements of the form $f(\alpha_1, \alpha_2, . . . , \alpha_n)$,
    where $f$ is a polynomial in $n$ indeterminates with coefficients in $K$ (that is,
    $f \in K[x_1, x_2, . . . , x_n]$); i.e. 
    $K[\alpha_1,...\alpha_n]=\{f(\alpha_1,...,\alpha_n):f(x_1,...,x_n)\in K[x_1,...,x_n]\}$. \\
    (3) $K(\alpha)=\{\frac{f(\alpha)}{g(\alpha)}:f(x),g(x)\in K[x] \text{ and } g(\alpha)\neq 0\}$. \\
    (4) $K(\alpha_1,...,\alpha_n)=\{\frac{f(\alpha_1,...,\alpha_n)}{g(\alpha_1,...,\alpha_n)}:f,g\in K[x_1,...,x_n]\text{ and } g(\alpha_1,...,\alpha_n)\neq 0\}$.
\end{theorem}

\begin{proof}
    (2) Let $S=\{f(\alpha_1,...,\alpha_n):f(x_1,...,x_n)\in K[x_1,...,x_n]\}$ and $X=\{\alpha_1,...,\alpha_n\}$. 
    Then $S\subset K[\alpha_1,...,\alpha_n]$,   
    as $K[\alpha_1,...,\alpha_n]$ is a subring of $L$ containing $K\cup X$ and $S$ is the collection of linear combination of elements of $K\cup X$.
    Conversely, if $f_1\in K[x_1,...,x_m]$ and $f_2\in K[x_1,...,x_n]$, as $f_1\pm f_2, f_1\cdot f_2 \in K[x_1,...,x_n]$, 
    then for $\alpha_1,...,\alpha_n$, 
    \begin{align*}
        f_1(\alpha_1,...\alpha_n) \pm f_2(\alpha_1,...\alpha_n) , f_1(\alpha_1,...\alpha_n) \cdot f_2(\alpha_1,...\alpha_n)\in S.
    \end{align*}
    Therefore, $S$ is an subring of $L$ and so a ring.
    Since $X\subset S$ ($\forall i$ , let $f(x_1,...,x_n)=x_i$ then $f(\alpha_1,...,\alpha_n)=\alpha_i$) and $K\subset S$ ($\forall k\in K$, Let $f(x_1,...,x_n)=k$ then $f(\alpha_1,...,\alpha_n)=k$)
    and $K[\alpha_1,...,\alpha_n]$ is the intersection of subring containing $K\cup X$, 
    $K[\alpha_1,...,\alpha_n]\subset S$. Hence, $K[\alpha_1,...,\alpha_n]=S$. 
\end{proof}


We now distinguish between two types of elements of an extension field.
This is fundamental to all that follows.

\begin{definition}{}{}
    Let $L/K$ be a field extension. 
    An element $\alpha\in L$ is algebraic over $K$ if $\alpha$ is a root of some nonzero polynomial $f\in K[x]$.
    If $\alpha$ is not a root of any nonzero $f\in K[x]$ then $\alpha$ is transcendental over $K$.
    $L$ is an algebraic extension of $K$ if every element of $L$ is algebraic over $K$.
    $L$ is a transcendental extension if at least one element of $L$ is transcendental over $K$.
\end{definition}


Let $L/K$ be a field extension and $\alpha\in L$ is algebraic over $K$.
We claim that $I=\{g(x)\in K[x]:g(\alpha)=0\}$ is a ideal in $K[x]$. 
In fact, for $f(x),g(x)\in I, h(x)\in K[x]$, $f(\alpha)\pm g(\alpha)=0$ and $h(\alpha)f(\alpha)=0$.
Since $K[x]$ is principal ideal domain, there exists unique monic polynomial $f(x)\in K[x]$
such that $I=(f(x))$. If $g(x)\in I=(f(x))$, then $f(x)|g(x)$. Hence, the degree of $f(x)$ is smallest in $I$. 
Now we prove $f(x)$ is irreducible in $K$. Suppose $f = gh$. 
Since $f(\alpha)=g(\alpha)h(\alpha)=0$ and $K$ is a field(hence also an integer domain), either $g(\alpha)=0$ or $h(\alpha)=0$. 
But this is a contradiction with the minimal degree on $f$, so $f$ must be irreducible.
such monic $f(x)$ is called the minimal polynomial of algebraic $\alpha$ over $K$.
The degree of $\alpha$ over $K$ is deg($f$).

\begin{example}{}{}
    The element $\sqrt[3]{3}\in\R$ is algebraic over $\Q$ since it is a root of $x^3-3\in\Q[x]$.
    Since $x^3-3$ is irreducible over $\Q$, it is the minimal polynomial of $\sqrt[3]{3}$ over $\Q$,
    and hence $\sqrt[3]{3}$ has degree $3$ over $\Q$.
\end{example}

\begin{example}{}{}
    Then element $i=\sqrt{-1}\in\C$ is algebraic over the subfield $\R$ of $\C$, since it is a root of the polynomial $x^2+1\in\R[x]$.
    Since $x^2+1$ is irreducible over $\R$, it is the minimal polynomial of $i$ over $\R$,
    and hence $i$ has degree $2$ over $\R$.
\end{example}

\begin{proposition}{}{}
    Every finite extension of $K$ is algebraic over $K$.
\end{proposition}
\begin{proof}
    Let $K$ be a finite extension of $K$ and let $[L:K]=m$.
    For $\alpha\in L$, then $m+1$ elements $1,\alpha,...,\alpha^m$ must be linearly dependent over $K$, i.e. must satisfy $a_0+a_1\alpha+\cdots +a_m\alpha^m=0$
    for some $a_i\in K$ (not all zero). Then $\alpha$ is algebraic over $K$.
\end{proof}


In the next two theorems, we classify simple extensions (first, extending by
a transcendental and second extending by an algebraic).


\begin{theorem}{}{transcendental extension basis}
    If $L/K$ is a field extension and $\alpha\in L$ is transcendental over $K$,
    then there is an isomorphism of fields $K(\alpha)\cong K(x)$ which is the identity when restricted to $K$, 
    where $K(\alpha)=\{\frac{f(\alpha)}{g(\alpha)}:f(x),g(x)\in K[x] \text{ and } g(\alpha)\neq 0\}$
    and $K(x)=\{\frac{f(x)}{g(x)}:f(x),g(x)\in K[x] \text{ and } g(x)\neq 0\}$
\end{theorem}

\begin{proof}
    Since $\alpha$ is transcendental then 
    $f(\alpha)\neq 0,g(\alpha)\neq 0$ for all nonzero $f,g\in K[x]$.
    Define $\varphi :K(x)\rightarrow L$ as $\frac{f}{g}\mapsto \frac{f(u)}{g(u)}$. 
    Since $\varphi(\frac{f_1}{g_1}+\frac{f_2}{g_2})=(\frac{f_1}{g_1}+\frac{f_2}{g_2})(u)=\frac{f_1}{g_1}(u)+\frac{f_2}{g_2}(u)=\varphi(\frac{f_1}{g_1})+\varphi(\frac{f_2}{g_2})$
    and $\varphi(\frac{f_1}{g_1}\cdot \frac{f_2}{g_2})=\frac{f_1}{g_1}\cdot \frac{f_2}{g_2}(u)= \frac{f_1}{g_1}(u)\cdot \frac{f_2}{g_2}(u)=\varphi(\frac{f_1}{g_1})\varphi(\frac{f_2}{g_2})$, 
    $\varphi$ is a homomorphism. Now for $\frac{f_1}{g_1}\neq \frac{f_2}{g_2}$, then $f_1g_2\neq f_2g_1$ and $f_1g_2-f_2g_1\neq 0$ (not the $0$ polynomial, that is). 
    Now $f_1(\alpha)g_2(\alpha)-f_2(\alpha)g_1(\alpha)\neq 0$ (or else $\alpha$ is algebraic over $K$), and so
    $\varphi(\frac{f_1}{g_1})=\frac{f_1(u)}{g_1(u)}\neq \frac{f_2(u)}{g_2(u)}=\varphi(\frac{f_2}{g_2})$. 
    Therefore, $\varphi$ is one to one. Also, $\varphi$ is the identity on $K$ (treating $K$ as a subfield of $K(x)$; think of $K$ as the constant rational functions in $L(x)$).
    So the image of $\varphi$ is $K(\alpha)$. So $\varphi$ is an isomorphism from $K(x)$ to $K(\alpha)$ which is the identity on $K$.
\end{proof}

\begin{remark}
    Theorem \ref{thm:transcendental extension basis} tells us what elements of the transcendental extension $K(\alpha)$ of $K$
    "look like":
    \begin{align*}
        \frac{f(\alpha)}{g(\alpha)}, \text{ where } f,g\in K[x] \text{ and } g(\alpha)\neq 0 
    \end{align*}
\end{remark}

\begin{theorem}{}{algebraic extension basis}
    If $L/K$ is a field extension and $\alpha\in L$ is algebraic over $K$, then\\
    (1) $K(\alpha)=K[\alpha]$;\\
    (2) $K(\alpha)\cong K[x]/f(x)$ where $f\in K[x]$ is the minimal polynomial of $\alpha$ with degree $n \geqs 1$ over $K$;\\
    (3) $[K(\alpha):K]=n$;\\
    (4) $\{1_K,\alpha,\alpha^2,...,\alpha^{n-1}\}$ is a basis of the vector space $K(\alpha)$ over $K$;\\
    (5) every element of $K(\alpha)$ can be written uniquely in the form $a_0+a_1\alpha+a_2\alpha^2+...+a_{n-1}\alpha^{n-1}$ where each $a_i\in K$.
\end{theorem}


\begin{proof}
    (1) and (2): 
    Define $\varphi: K[x]\rightarrow K[\alpha]$ as $g\mapsto g(\alpha)$. 
    Then clearly $\varphi$ is a ring homomorphism. By the form of elements of $K[x]$, $\varphi$ is onto. 
    Since $K$ is a field, $K[x]$ is a principal ideal domain.
    So $\Ker(\varphi)=(f)$ for some $f\in K[x]$ as $\Ker(\varphi)$ is an ideal. 
    Since $\alpha$ is algebraic and $\varphi(f)=f(\alpha)=0$, $\Ker(f)\neq \{0\}$.
\end{proof}



\begin{remark}
    Theorem \ref{thm:algebraic extension basis} tells us what elements of the algebraic extension $K(\alpha)$ of $K$
"look like". That is, there exists a fixed $n\in\N$ such that every element of $K(\alpha)$ is of the form $a_0+a_1\alpha+...+a_{n-1}\alpha^{n-1}$
for some $a_i\in K$.
\end{remark}

\begin{example}{}{}
    Consider the simple extension $\R(i)$ of $\R$. We saw earlier that $i$ has minimal polynomial $x^2+1$
    over $\R$. So $\R(i)\cong \R[x]/(x^2+1)$, and $\{1,i\}$ is a basis for $\R(i)$ over $\R$.
    So $\R(i)=\{a+bi:a,b\in\R\}=\C$.
\end{example}

\begin{example}{}{}
    Consider the simple extension $\Q(\sqrt[3]{3})$ of $\Q$. 
    We saw earlier that $\sqrt[3]{3}$ has minimal polynomial $x^3-3$ over $\Q$.
    So $\Q(\sqrt[3]{3})\cong \Q[x]/(x^3-3)$, and $\{1,\sqrt[3]{3},(\sqrt[3]{3})^2\}$ is a basis for $\Q(\sqrt[3]{3})$ over $\Q$.
    So $\Q(\sqrt[3]{3})=\{a+b\sqrt[3]{3}+c(\sqrt[3]{3})^2:a,b,c\in\Q\}$.
\end{example}

Note that we have been assuming that both $K$ and $\alpha$ are embedded in some larger field $F$. Next,
we will consider constructing a simple algebraic extension without reference to a previously given
larger field, i.e. “from the ground up”.
The next result, due to Kronecker, is one of the most fundamental results in the theory of fields:
it says that, given any non-constant polynomial over any field, there exists an extension field in
which the polynomial has a root.


\begin{theorem}{Kronecker's Theorem}{Kronecker's Theorem}
    If $K$ is a field and $f\in K[x]$ a monic irreducible polynomial of degree $n$, 
    then there exists a simple extension field $L=K(\alpha)$ of $K$ such that $\alpha\in L$ and $f$ is the minimal polynomial of $\alpha$. 
\end{theorem}



\begin{theorem}{}{}
    If $L$ is a finite dimensional extension field of $K$, then $L$ is 
    finitely generated and algebraic over $K$.
\end{theorem}


\begin{theorem}{}{}
    Let $L$ be an extension field of $K$. Then 
    $[L:K]<\infty$ iff there exists finite algebraic $\alpha_1,...,\alpha_n\in L$ over $K$
    such that $K(\alpha_1,...,\alpha_n)=L$.
\end{theorem}

\section{Splitting fields}

Given a polynomial, we now want an extension field which contains all its roots.

\begin{definition}{}{}
    Let $f\in K[x]$ be polynomial of positive degree and $F$ an extension field of $K$.
    Then we say that $f$ splits in $F$ if $f$ can be written as a product of linear factors in $F[x]$,
    i.e. if there exist elements $\alpha_1,\alpha_2,...,\alpha_n\in F$ such that
    \begin{align*}
        f=a(x-a_1)\cdots (x-\alpha_n)
    \end{align*}
    where $a$ is the leading coefficient of $f$.
    The field $F$ is called a splitting field of $f$ over $K$ if it splits in $F$ and $F=K(\alpha_1,...,\alpha_n)$. 
\end{definition}
\begin{remark}
    a splitting field $F$ of a polynomial $f$ over $K$ is an extension field containing all
    the roots of $f$, and is "smallest possible" in the sense that
    no subfield of $F$ contains all roots of $f$.
    The following
result answers the questions: can we always find a splitting field, and how many are there?
\end{remark}

\begin{theorem}{Existence and uniqueness of splitting field}{Existence and uniqueness of splitting field}
    (1) If $K$ is a field and $f$ any polynomial of positive degree in $K[x]$,
    then there exists a splitting field of $f$ over $K$.\\
    (2) Any two splitting fields of $f$ over $K$ are isomorphic under an isomorphism which keeps the elements of $K$ fixed and
    maps roots of $f$ into each other.
\end{theorem}

So, we may therefore talk of the splitting field of f over K. It is obtained by adjoining to $K$
finitely many elements algebraic over $K$, and so it is a finite extension.
of K.

\begin{example}{}{}
    Find the splitting field of the polynomial $f=x^2+2\in \Q[x]$ over $\Q$.
\end{example}

Up to now we have been saying 'a' splitting field. Theorem\ref{thm:Existence and uniqueness of splitting field} 
give us the right to speak of the splitting field of a given polynomial $f$
over a given field $K$. We write it as $\text{SF}_K ( f )$.

Splitting fields will be central to our characterization of finite fields, in the next chapter.

\section{Reference}
\begin{itemize}
    \item \href{https://faculty.etsu.edu/gardnerr/5410/notes/V-1.pdf}{Section V.1. Field Extensions}
    \item \href{https://mdu.ac.in/UpFiles/UpPdfFiles/2021/Jun/4_06-28-2021_11-45-05_Theory%20of%20Field%20Extensions_(20MAT22C1).pdf}{THEORY OF FIELD EXTENSIONS}
    \item \href{https://feog.github.io/chap3.pdf}{Field Theory}
    \item \href{https://mathstat.dal.ca/~yanghs/notes.php?name=Galois}{lecture notes by yanghs}
    \item \href{https://faculty.etsu.edu/gardnerr/5410/Beamer-Proofs/Proofs-V-1.pdf}{lecture notes by gardnerr}
    \item \href{https://www.math.rwth-aachen.de/~Max.Neunhoeffer/Teaching/ff/ffchap2.pdf}{Some field theory}
\end{itemize}