\newcommand{\gi}[1]{\langle #1 \rangle}
\chapter{Homomorphism and Ideals}\label{chp:4_2}

Just as with groups, when we study rings, we are only concerned with functions that “preserve the
structure” of a ring, and these are called ring homomorphisms. Maybe you can guess what the definition
should be, by analogy with the case of groups.

\begin{definition}{}{}
    Let $R$ and $S$ be rings, and $\sigma:R\rightarrow S$ be a function. We say $\sigma$ is a ring homomorphism if, for all $a,b\in R$, \\
    (1) $\sigma (a+b)=\sigma(a)+\sigma(b),$\\
    (2) $\sigma(a\cdot b)=\sigma(a)\cdot \sigma(b)$.
\end{definition}

\begin{remark}
    $\sigma(0) =\sigma(0+0)=\sigma(0)+\sigma(0)\Rightarrow \sigma(0)=0$.
\end{remark}

Just as for groups, bijective homomorphisms are called isomorphisms, and they tell us when two rings
"have the same structure".

\begin{definition}{}{}
    An isomorphism from a ring $R$ to another ring $S$ is a bijective homomorphism. If an isomorphism between $R$ and $S$ exists, 
    then we say $R$ and $S$ are isomorphic and we write $R\cong S$.
\end{definition}

There is an alternative way to characterize isomorphisms, using inverse functions.

\begin{proposition}{}{}
    Let $f:R\rightarrow S$ be a homomorphism. Then $f$ is an isomorphism iff there exists a homomorphism $g:S\rightarrow R$ such that $g\circ f$ is the identity map on $R$ 
    and $f\circ g$ is the identity map on $S$.
\end{proposition}

\begin{definition}{}{}
    Let $\sigma:R\rightarrow S$ be a homomorphism of rings. The kernel of $\sigma$, denoted $\text{Ker}(\sigma)$, 
    is the subset $\{r\in R:\sigma(r)=0\}$ of $R$. In other words, it's the pre-image of $0$ under $\sigma$.
    The image of $\sigma$ is the set $\text{Im}(\sigma)=\{s\in S:s=\sigma(r) \text{ for some } r\in R\}$.
\end{definition}

\begin{example}{}{kernel example}
    The kernel of $\sigma:\Z\rightarrow \Z/m\Z$ sengding $a$ to $\overline{a}$ is $\{km:k\in\Z\}$.
\end{example}

Just as for groups, the kernel and image detect injectivity and surjectivity: to be injective means to
have a trivial kernel, so maps with a large kernel can be thought of as “very un-injective”.

\begin{proposition}{}{}    Let $\sigma:R\rightarrow S$ be a homomorphism of rings. Then $\sigma$ is injective iff $\text{Ker}\sigma = \{0_R\}$;
    $\sigma$ is surjective iff $\text{Im}\sigma = S$.
\end{proposition}

Let's look back at example\ref{exa:kernel example}. 
What properties does $\text{Ker}\sigma$ have? For $am,bm\in \text{Ker}\sigma$, $am + bm= (a+b)m\in \text{Ker}\sigma$, $-am\in \text{Ker}\sigma$, $am\cdot bm=(abm)m\in\text{Ker}\sigma$.
Hence, $\text{Ker}\sigma$ is a subring of $\Z$.
But we find more:  if we take one integer which is in $\text{Ker}\sigma$, and another one which might
not even be in $\text{Ker}\sigma$, and multiply, we still stay in $\text{Ker}\sigma$ as $a\cdot bm = ab \cdot m$. 
So the kernel is closed even under multiplication when one of the factors need
not lie in the kernel. This is stronger than the usual closure under multiplication, and this movtivate the definition of ideals.

\begin{definition}{}{}
    Let $R$ be a ring. A non-empty subset $I$ of $R$ is called an ideal if it satisfies the following conditions:\\
    (1) $I$ is an additive subgroup of $R$ under $+$. (closed under addition and subtraction)\\
    (2) For any $a\in I$ and $r\in R$, $ra,ar\in I$. (closed under scaling)\\
    If $I$ is an ideal of a ring $R$, we write $I\unlhd R$.
\end{definition}
\begin{remark}
    $\forall a,b\in I\subset R$, $ab,ba\in I$, then $I$ is a subring of $R$.
\end{remark}




\begin{example}{}{zero ideal}
    Every ring has at least one ideal: the subset consisting of only $0$. It's an additive
subgroup, and closed under scaling because anything times zero is zero. 
\end{example}

\begin{example}{}{unit ideal}
    In any ring $R$, the entire ring $R$ is itself ideal, called the "unit ideal" . The reason for this name
is: if an ideal $I$ contains $1$, then it is equal to the entire ring, because if $1$ is in $I$, then for any $r\in R$,
$r = r \cdot 1$ is also in $I$ by closure under scaling. This is the largest ideal in $R$.
\end{example}

\begin{example}{}{}
    Let $I_1,I_2,...,I_n$ be ideals of $R$, then the sum of $I_1,...,I_n$
    \begin{align*}
        I_1+...+I_2:=\{a_1+...+a_n:a_1\in I_1,...,a_n\in I_n\}
    \end{align*}
    is a ideal of $R$.
\end{example}

\noindent$(a_1+...+a_n)+(b_1+...+b_n)=(a_1+b_1)+...+(a_n+b_n)\in I_1+...+I_n$.\\
$(a_1-...-a_n)+(b_1-...-b_n)=(a_1-b_1)+...+(a_n-b_n)\in I_1+...+I_n$.\\
$(a_1+...+a_n)\cdot r=a_1\cdot r +... +a_n\cdot r\in I_1+...+I_n$

\begin{example}{}{intersection ideal}
    Let $I_1,I_2,...,I_n$ be ideals of $R$, then $\cap_{i}^{n}I_i$
    is an ideal of $R$.
\end{example}

\noindent$a+b\in I_i, \forall I_i$\\
$a-b\in I_i, \forall I_i$\\
$a\cdot r\in I_i,\forall I_i$.

\begin{proposition}{}{}
    Let $R$ be an unital ring and $I$ be an ideal in $R$. $I$ contains a unit if and only if $I = R$.
\end{proposition}
\begin{proof}
    ($\Rightarrow$): If $I$ has a unit, then $1_R\in I$, then $\forall r\in R,r\cdot 1=r\in I$ and so $R\subseteq I$. Hence, $R=I$ as $\O\neq I\subseteq R$.\\
    ($\Leftarrow$): If $I=R$, then $1_R\in I$ and so $I$ have a unit $1_R$.
\end{proof}

\begin{proposition}{}{}
    Let $R$ be a ring and $I$ is an ideal in $R$. Then $(I,+)$ is a normal subgroup of $(R,+)$.
\end{proposition}
\begin{proof}
    By the definitino of ideals, $\O\neq I\leqs R$. Since $R$ is abelian group with addition, $(I,+)$ is abelian group and so normal.
\end{proof}

\begin{definition}{}{}
    Let $X$ be any subset of a ring $R$. The ideal generated by $X$ is the intersection of all ideals containing $X$. 
    Hence, this ideal is the smallest ideal containing $X$, and is denoted $\gi X$. 
    If $X$ is a finite set, say $X=\{a_1,a_2,...,a_n\}$, we will write $\gi{X}$ as $(a_1,...,a_n)$. 
    An ideal generated by a single element $a$ is called a principal
ideal, denoted by $(a)$. 
\end{definition}
\begin{remark}
    Generalizing the example\ref{exa:intersection ideal} to infinite, we can know $\cap_{i}^{\infty} I_i$ is a ideal. 
    Then the definition is well-defined.
\end{remark}

\begin{proposition}{}{}
    If $R$ is an unital ring and $\O\neq X\subseteq R$. Then
    \begin{align*}
        \gi X =\{\sum\limits_{i=1}^{n}r_ix_is_i: x_i\in X \text{ and } r_i,s_i\in R\}.
    \end{align*}
\end{proposition}

\begin{proof}
    hard!
\end{proof}

\begin{proposition}{}{element form of generated ring in commutative ring with unity}
    If $R$ is a commutative ring with unity and $\O\neq X\subseteq R$. Then
    \begin{align*}
        \gi X =\{\sum\limits_{i=1}^{n}r_ix_i: x_i\in X \text{ and } r_i\in R\}.
    \end{align*}
    Particularly, 
    \begin{align*}
        (a) = \{ra:r\in R\}.
    \end{align*}
    If $X=\{a_1,...,a_n\}$, then
    \begin{align*}
        (a_1,...,a_n)=(a_1)+...+(a_n).
    \end{align*}
\end{proposition}
\begin{remark}
    zero ideal can be denoted $(0)$ and unit ideal can be denote $(1)$.
\end{remark}


% \begin{example}{}{}
%     What are all possible ideals in $R=\Z/6\Z$? This is a classic example-style question. Let's work it out
%     - let $I$ be a "mystery" ideal. 
%     By example\ref{exa:zero ideal} and \ref{exa:unit ideal}, $(\overline{0})$ and $R$ are ideals of $R$.
%     Now let's assume there is at least one element $a$ in $I$, which is nonzero, and not equal to $overline{1}$. 
%     If $a=\overline{5}$, since $\overline{5}\cdot \overline{5}=\overline{25}=\overline{1}$, then $I$ is unit ideal.
%     If $a=\overline{4}$, since $4r$ mod $6$ are $2$ and $0$, then $I=(\overline{2})$.
%     If $a=\overline{3}$, since $3r$ mod $6$ are $3$ and $0$, then $I={0,3}$
% \end{example}


Let $I$ be a ideal of $R$. Since $I$ is an additive subgroup of $R$ by definition, it makes sense to speak of cosets $r+I$ of $I$, $r\in R$. 
Furthermore, a ring has a structure of abelian group of addition, 
so $I$ satisfies the definition of a normal subgroup. From group theory, we thus know that it makes sense to speak of the quotient group
\begin{align*}
    R/I = \{\overline{r} = r+I:r\in R\}.
\end{align*}
Since $R$ is abelian group , $r\in R$ and $I\subset R$, $R/I$ is an abelian group for addition.
\par
We now endow $R/I$ with a multiplication operation as follows. Define
\begin{align*}
    (r+I)(s+I) = rs + I.
\end{align*}
Let us make sure that this is well-defined, namely that it does not depend on the choice of the repersentative in each coset. Suppose that
\begin{align*}
    r+I=r'+I, s+I=s'+I.
\end{align*}
Then, $a=r'-r\in I$ and $b=s'-s\in I$. Now
\begin{align*}
    r's' = (a+r)(b+s) =ab+as+rb+rs\in rs+I
\end{align*}
This tells us $r's'$ is also in the coset $rs+I$ and thus multiplication does not depend on the choice of repersentatives.
The multiplication association and distributivity are easily to prove. 
Hence, $R/I$ is a ring and is called quotient ring.



\begin{theorem}{}{}
    Let $\sigma$ be a homomorphism from ring $R$ to ring $S$, then
    \begin{align*}
        \text{Ker}\sigma \unlhd R, \text{Im}\sigma \leqs R\text{ and } R/\text{Ker}\sigma \cong \text{Im}\sigma
    \end{align*}
\end{theorem}

\begin{proof}
    Firstly, we prove $\text{Ker}\sigma\unlhd R$. 
    Since $\sigma(0_R)=\sigma(0_R+0_R)$, $\sigma(0_R)=0_S$. Then, $0_R\in \text{Ker}\sigma$.
    Then $\O\neq \text{Ker}\sigma$. 
    For $a,b\in\text{Ker}\sigma$, 
    \begin{align*}
        \sigma(a\pm b) =\sigma(a)\pm\sigma(b)=0_S+0_S=0_S.
    \end{align*}
    Hence, $\text{Ker}\sigma$ is a addition subgroup of $R$.
    If $a\in \text{Ker}\sigma$ and $r\in R$, then
    \begin{align*}
        \sigma(ra) = \sigma(r)\cdot \sigma(a) = \sigma(r)\cdot 0_S = 0_S, \\
        \sigma(ar) = \sigma(a)\cdot \sigma(r) = 0_S\cdot\sigma(r) = 0_S.
    \end{align*}
    Hence, $ra,ar\in \text{Ker}\sigma$ and so $\text{Ker}\sigma$ is a ideal of $R$.

    \par
    Next, we prove $\text{Im}\sigma\leqs R$.
    Since $0_S=\sigma(0_R)\in \text{Im}\sigma$, $\text{Im}\sigma$ is non-empty.
    For $a,b\in R$,
    \begin{align*}
        \sigma(a) \pm \sigma(b) = \sigma(a\pm b)\in \Im\sigma,\\
        \sigma(a)\sigma(b) = \sigma(ab)\in \Im\sigma.
    \end{align*}
    Hence, $\Im\sigma$ is a subring of $R$.
    \par
    Finally, we prove that $R/\Ker\sigma\cong \Im\sigma$. Let $I=\Ker\sigma$.
    Define $\overline{\sigma}:R/I\cong \Im\sigma$ given by $\overline{r}=\sigma(r)$. 
    Since we're dealing with cosets, there may be many different representatives for the same
coset, which means we have to check it's well-defined.

\textcolor{Red}{Havn't done!}
\end{proof}

\section{Reference}
\begin{itemize}
    \item \href{https://math.berkeley.edu/~mcivor/math113su16/113ringnotes2016.pdf}{lecture notes from berkeley}
    \item \href{https://feog.github.io/ring.pdf}{lecture notes by feog}
\end{itemize}

