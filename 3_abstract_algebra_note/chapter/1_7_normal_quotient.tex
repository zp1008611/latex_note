\chapter{Normal Subgroup and Qutient Subgroup}\label{chp:1_7}

\begin{proposition}{}{normal equivalent}
    If $N$ is a subgroup of group $G$, then the following conditions are equivalent.\\
    (1) $aN=Na$ for all $a\in G$.\\
    (2) For all $a\in G$, $n\in N$, $ana^{-1}\in N$.\\
    (3) For all $a\in G$, $aNa^{-1}=N$.
\end{proposition}

\begin{definition}{}{}
    A subgroup $N$ of a group $G$ 
    which satisfies the equivalent
    conditions of proposition \ref{prop:normal equivalent} 
    is said to be normal in $G$ (or a normal subgroup of $G$),
    denoted $N\unlhd G$.
\end{definition}

\begin{proposition}{}{}
    Every subgroup of an abelian group is normal.
\end{proposition}

\begin{proposition}{}{}
    Any subgroup $N$ of index 2 in group $G$ is a normal subgroup.
\end{proposition}

\begin{proposition}{}{}
    The intersection of any collection of normal subgroups
    of group $G$ is itself a normal subgroup.
\end{proposition}

\begin{proposition}{}{}
    Let $G$ be a group and $H\leqs K\leqs M$.
    If $H\unlhd G$, then $H\unlhd K$.
\end{proposition}

\begin{remark}
    It may be the case that for group $G$, 
    we have subgroups of $G$ satisfying
    $H\unlhd K$ and $K\unlhd G$ but $H$ 
    is not a normal subgroup of $G$.
\end{remark}

The following result is a big one! 
It says that if $N$ is a normal subgroup of
$G$, then we can form a group out of the cosets of $N$ 
by defining a binary operation on the cosets using representatives of the cosets. 
In fact, this can be done only
when $N$ is a normal subgroup, 
the group of cosets is called a "factor group" or "quotient group." 
Quotient groups are at the backbone of modern algebra!

\begin{theorem}{}{quotient group well defined}
    If $N$ is a normal subgroup of a group $G$ and $G/N$ is the set of 
    all left cosets of $N$ in $G$, 
    then $G/N$ is a group of order $[G:N]$ 
    under the binary operation given by $(aN)(bN)=(ab)N$.
\end{theorem}

\begin{remark}{}{}
    By the definition of the binary operation in $G/N$,
    we see that the identity element is $eN=N$,
    and the inverse of $aN$ is $a^{-1}N$.
\end{remark}

\begin{definition}{}{}
    If $N$ is a normal subgroup of $G$, then the group $G/N$
    of theorem\ref{thm:quotient group well defined}
    is the quotient group of $G$ by $N$.
\end{definition}
\begin{remark}
    The relationship between quotient groups 
    and normal subgroups is a little
    deeper than Theorem \ref{thm:quotient group well defined} 
    implies. From Fraleigh, we have:
    Let $H$ be a subgroup of a group $G$. 
    Then left coset multiplication is well defined by 
    $(aH)(bH) = (ab)H$ 
    if and only if H is a normal subgroup of $G$.
    This result gives the real importance of normal subgroups.
\end{remark}

\section{Reference}

\begin{itemize}
    \item \href{https://faculty.etsu.edu/gardnerr/5410/notes/I-5.pdf}{Normality, Quotient Groups}
\end{itemize}
