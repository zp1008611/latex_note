\chapter{Finite Field}\label{chp:6_4}

We have seen, in the previous chapters, some examples of finite fields. For example, the quotient
ring $\Z/p\Z$ (when $p$ is a prime) forms a field with $p$ elements which may be identified with the
Galois field $\F_p$ of order $p$.

The fields $\F_p$ are important in field theory.
Since every finite field must have characteristic $p$, 
this helps us to classify finite fields.

\begin{lemma}{}{finite field cardinality lemma}
    Let $F$ be a finite field containing a subfield $K$ with $q$ elements. Then $F$ has $q^m$ elements,
    where $m=[F:K]$.
\end{lemma}
\begin{proof}
    $F$ is a vector space over $K$, finite-dimensional since $F$ is finite.
    Denote this dimension by $m$, then $F$ has a basis over $K$ consisting of $m$ elements,
    say $\alpha_1,...,\alpha_m$. Every element of $F$ can be uniquely represented in the form
    $k_1\alpha_1+...+k_m\alpha_m$ (where $k_1,...,k_m\in K$).
    Since each $k_i\in K$ can take $q$ values, $F$ must have exactly $q^m$ elements.
\end{proof}

We are now ready to answer the question: "What are the possible cardinalities for finite fields?"

\begin{proposition}{}{}
    Let $F$ be a finite field.
    Then $F$ has $p^n$ elements, where the prime $p$ is the characteristic of $F$ and 
    $n$ is the degree of $F$ over its prime subfield.
\end{proposition}
\begin{proof}
    Since $F$ is finite, it must have characteristic $p$ for some prime $p$ for some prime $p$.(by corollary\ref{cor:finite field prime characteristic})
    Thus, the prime subfield $K$ of $F$ is isomorphic to $\F_p$ (by proposition\ref{prop:prime subfield isomorphic to Q or Fp}), 
    and so contains $p$ elements. By lemma\ref{lem:finite field cardinality lemma}, $F$ has $p^n$ elements.
\end{proof}

So, all finite fields must have prime power order - 
there is no finite field with $6$ elements, for example.
We next ask: does there exist a finite field of order $p^n$
for every prime power $p^n$? How can such fields be constructed?
We can take the prime fields $\F_p$ and construct other finite
fields from them by adjoining roots of polynomials. If $f \in \F_p[x]$ is irreducible of degree $n$ 
over $\F_p$, by Kronecker's Theorem\ref{thm:Kronecker's Theorem}, 
then adjoining a root of $f$ to $\F_p$ yields a finite field of $p^n$
elements. However, it is not clear whether
we can find an irreducible polynomial in $\F_p[x]$ of degree $n$, 
for every integer $n$.

The following two lemmas will help us to characterize fields using root adjunction.


\section{Reference}

\begin{itemize}
    \item \href{https://www.math.rwth-aachen.de/~Max.Neunhoeffer/Teaching/ff/ffchap3.pdf}{Finite fields}
\end{itemize}