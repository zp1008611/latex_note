\chapter{Maximal Ideals and Prime Ideals}\label{chp:4_4}

Today we discuss two very important types of ideals, and learn how to use them to check whether
certain quotients are fields or integral domains.

\section{Prime Ideals}

\begin{proposition}{}{: divides corresponds to contains}
    Let $f,g\in \Z$. $f|g$ iff $g\in (f)$.
\end{proposition}
\begin{proof}
    $\exists h\in \Z$ \text{ s.t. } $g=fh$
\end{proof}

\begin{proposition}{}{}
    Let $p,a,b\in\Z$. 
    $p$ is prime iff $p|ab\Rightarrow p|a$ or $p|b$. 
\end{proposition}
\begin{proof}
    ($\Rightarrow$): Suppose $p\nmid a$. Since $p$ is prime, $\gcd(p,a)=1$. 
    Then there exists $m,n\in \Z$ such that $mp+na=1$ and then $(mb)p+(n)ab=b$.
    Since $p|ab$, $\exists t\in\Z$, $tp=ab$. Then $(mb+tn)p=b$. Hence, $p|b$.\\
    ($\Leftarrow$): Suppose $p$ is not prime, then $\exists k,s\in\Z$ ($k,s\neq 1,p$ and $k,s\leq p$) \text{s.t.} $ks=p$.
    Since $p|ks$, by the condition, $p|k$ or $p|s$. This is a contradiction as $p>k,s$.
    Hence, $p$ is prime. 
\end{proof}

By proposition\ref{prop:: divides corresponds to contains}, $p|ab$ means $ab\in (p)$, 
and similarly $p|a$ means $a\in (p)$.
So in terms of ideals, $p$ is prime number means that if $ab\in (p)$, 
then $a$ or $b$ (or both) must be in $(p)$. So for an ideal in $\Z$ generated by a prime number, 
we have the following slogn: if a product is in it, one of the factors must be, also. 
This is basically the definition of a prime ideal, and it makes sense in any ring.

\begin{definition}{}{}
    Let $I$ be a proper ideal in a commutative ring with unity. 
    We say $I$ is a prime ideal if whenever $ab$($a,b\in R$) in $I$,
    either $a$ or $b$ (or both) is in $I$.
\end{definition}

\begin{example}{}{prime and prime ideal}
    In $\Z$, an ideal $(n)$ is prime iff the integer $|n|$ is prime (being absolute value since primes are positive, but the generator may not be), or $n=0$.
\end{example}

If $|n|$ is  not prime

\begin{example}{}{}
    In the ring $\Z$, the zero ideal is prime, but in the ring $\Z/6\Z$, the zero ideal is not prime, 
    since $\overline{2}\cdot \overline{3}\in (0)$ but $\overline{2}\notin (0)$ and $\overline{3}\notin (0)$ 
\end{example}


\section{Maximal Ideals}
A maximal ideal is what the name suggests: the biggest possible ideal. But that would be the entire
ring, and there would be only one. So we require maximal ideals to be proper, and then it turns out
that there can be many of them.

\begin{definition}{}{}
    An ideal $I$ in a commutative ring with unity is called maximal if it is not the unit ideal and there are no other ideals $J$ such that $I\subset J \subset R$.
\end{definition}

\begin{proposition}{}{}
    Every nonzero subgroup of $(\Z, +)$ has the form $n\Z$, 
    then every nonzero ideals of $\Z$ has the form $(n)$.
\end{proposition}
\begin{proof}
    proof referring to \href{https://proofwiki.org/wiki/Subgroups_of_Additive_Group_of_Integers}{proof from subwiki}
\end{proof}

\begin{proposition}{}{}
    In $\Z$, if $n>1$ is prime, then $(n)$ is maximal.
\end{proposition}
\begin{proof}
    If $(n)$ is not maximal, then there exists $(d)$ such that $(n)\subset (d)$.
    Then $n=md$ for some $m\in\Z$. Since $n$ is prime, by example \ref{exa:prime and prime ideal}, $(n)$ is prime.
    Since $n=md\in (n)$, $m\in (n)$ or $d\in (n)$. The latter would imply $(d)\subseteq (n)$, a contradiction. 
    Hence, $m\in (n)$. Then $m=tn$ for some $t\in\Z$, hence $n=tnd$, implying that $td=1$. Thus $1\in (d)$ so $(d)=R$.
    Hence, $(n)$ is maximal.
\end{proof}
    
The relationship between prime and maximal ideals is as follows:

\begin{proposition}{}{}
    Any maximal ideal in a commutative ring with unity is prime.    
\end{proposition}

\begin{proof}
    Let $I$ be a maximal ideal. To show it's prime, assume $a,b\in R$, with $ab\in I$ and $a\notin I$. 
    We must show that $b\in I$. Since $a\notin I$, the ideal sum $(a)+I$ is strictly larger than $I$, and
    since $I$ is maximal, $(a)+I=R$. So $1\in (a)+I$, which means we can write $1=x+ra$ for some $x\in I$ and $r\in R$.
    Then $b=b\cdot 1= b(x+ra)=bx+bra$, and since both $x$ and $ab$ are in $I$, this shows that $b\in I$.
\end{proof}


\section{Relations between these ideals and their quotients}
\begin{proposition}{}{}
    Let $R$ be a commutative ring with unity. Then
    \begin{align*}
        I \text{ is prime } \Leftrightarrow R/I \text{ is an integral domain }.
    \end{align*} 
\end{proposition}

\begin{proof}
    Since $R$ is a commutative ring with unity, $R/I=\{\overline{a}=a+I:a\in R\}$ is a commutative unital ring.
    And for $a,b\in R$, $\overline{ab}=ab+ I$. Then $\overline{ab}=\overline{0}\Leftrightarrow ab+I = I\Leftrightarrow ab\in I$.
    Similarly, $\overline{a}=\overline{0}\Leftrightarrow a\in I$, $\overline{b}=\overline{0}\Leftrightarrow b\in I$.
    Hence,
    \begin{align*}
        I \text{ is prime} &\Leftrightarrow ab\in I \text{ implies } a\in I \text{ or } b\in I \\
        &\Leftrightarrow \overline{ab}=\overline{0} \text{ implies } \overline{a}=\overline{0} \text{ or }\overline{b}=\overline{0} \\
        &\Leftrightarrow R/I \text{ is a integer domain}.
    \end{align*}
\end{proof}

\begin{proposition}{}{field and ideals relation}
    Let $R$ be a commutative ring with unity. Then
    \begin{align*}
        R \text{ is a field}\Leftrightarrow \text{ the only ideals of } R \text{ are } R \text{ and } (0).
    \end{align*}
\end{proposition}
\begin{proof}
    ($\Rightarrow$): Let $I$ be a ideal of $R$. 
    Since $R$ is field, $R$ is a skew field and every non-zero element in $R$ is a unit.
    If $I\neq (0)$,
    then for $0\neq a\in I$, $\exists r\in R$ s.t. $ar=1\in I$. Then $I=(1)=R$. 
    And $I=(0)$ is a ideal for every ring. Hence, the only ideals of $R$ are $(0)$ and $R$.\\
    ($\Leftarrow$):
    Let $a\in R\setminus \{0\}$, then $(a)\neq (0)$. Since the only ideals of $R$ are $(0)$ and $R$, $(a)=R$.
    Thus $1\in (a)$. Then by proposition\ref{prop:element form of generated ring in commutative ring with unity}, 
    $\exists r\in R, ra=ar=1$. Therefore, $a$ is a unit of $R$. Then $U(R)=R\setminus \{0\}$ and $R\setminus \{0\}$
    is abelian group with respect to multiplication as $R$ is commutative. Hence, $R$ is a field. 
\end{proof}


\begin{proposition}{}{}
    Let $R$ be a commutative ring with unity and $I\neq R$ be a ideal of $R$. Then
    \begin{align*}
        I \text{ is maximal }\Leftrightarrow R/I \text{ is a field}.  
    \end{align*}
\end{proposition}

\begin{proof}
    $R$ is a commutative ring with unity, then $R/I$ is a commutative ring with unity. Then\\
    ($\Rightarrow$): Let $\overline{a}\neq \overline{0}$. 
    Then $a\notin I$. Since $I\subseteq I+(a)$ and $I$ is maximal, $I+(a)=R$.
    Since $1\notin I$, $1\in (a)$. Then $\exists r\in R$ such that $ra=ar=1\in I$. 
    Thus, $\overline{ar}=ar+I=\overline{1}$. Hence, every non-zero elements in $R/I$ has inverse.
    Then, $U(R/I)=R/I\setminus\{\overline{0}\}$ and $R/I$ is commutative. So, $R/I$ is a field.\\
    ($\Leftarrow$): Suppose $\exists \text{ ideal } J$ s.t. $I\subset J\subset R$. Let $a\in J\setminus I$, then $a\notin I$ and so $\overline{a}\neq \overline{0}$.
    Since $R/I$ is a field, $\exists \overline{b}\in R/I$ s.t. $\overline{a}\overline{b}=\overline{1}$. Then $ab-1\in I\subset I'$.
    Since $1=ab-(ab-1)$ and $ab\in J$ ($J$ is ideal, $a\in J$ and $b\in R$, then $ab\in J$), $1\in J$ and so $J=R$.
    Hence, $I$ is maximal. 
\end{proof}

\section{Zorn's Lemma and the existence of maximal ideals}

\section{Reference}

\begin{itemize}
    \item \href{https://math.berkeley.edu/~mcivor/math113su16/113ringnotes2016.pdf}{lecture notes from berkeley}
    \item \href{https://osebje.famnit.upr.si/~russ.woodroofe/wustl-notes/zorn.pdf}{lecture notes from upr}
    \item \href{https://crypto.stanford.edu/pbc/notes/commalg/factorization.html}{Let $R$ be a PID. Then every nonzero prime ideal is maximal.}
\end{itemize}