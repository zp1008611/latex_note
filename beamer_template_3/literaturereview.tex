\documentclass[10pt,cjk]{beamer}
%命令行输入texdoc beamer可得到beamer官方介绍文档!!!

\usepackage[backend=bibtex,sorting=none]{biblatex}
%\usepackage[backend=bibtex,style=verbose-trad2]{biblatex}
\addbibresource{example.bib} %BibTeX数据文件及位置
\setbeamerfont{footnote}{size=\tiny}

\usetheme{Madrid}

%\usepackage[UTF8]{ctex}
\usepackage{graphicx}%插入
\graphicspath{{picture/}}
\usepackage{amsmath}
\usepackage{color}

\useinnertheme{circles}%设置内部环境的序号形状
%\useoutertheme[height=0\textwidth,width=0.18\textwidth,hideothersubsections]{sidebar}$左侧目录

\usecolortheme{whale}%颜色主题设置

  
%\setbeamerfont{frametitle}{family=\ttfamily}%设置字体,{frametitle}设置的是页标题字体



\begin{document}
	\newcommand{\mx}[1]{\boldsymbol{#1}}
	\newcommand{\vr}[1]{\vec{\boldsymbol{#1}}}
	\newcommand{\qn}[1]{\boldsymbol{#1}}
	\newcommand{\dqn}[1]{\breve{\boldsymbol{#1}}}
	\newcommand{\m}{~\mathrm{m}}
	\newcommand{\rad}{~\mathrm{rad}}
	\newcommand{\opvec}{\mathrm{vec}}
	\newcommand{\opdqvec}{\mathrm{\breve{v}ec}}
	
	\newtheorem{Thm}{Theorem}[section]
	\newtheorem{Def}[Thm]{Definition}
	\newtheorem{Ass}[Thm]{Assumption}
	\newtheorem{Lem}[Thm]{Lemma}
	\newtheorem{Cor}[Thm]{Corollary}
	\newtheorem{Exm}[Thm]{Example}

	\title{Literature Review}
	\subtitle{}
	\author{Baoying Shi}
	%\institute{School of Mathematical Sciences,South China Normal University}
	\date{\today}
	\frame{\titlepage}
	
	\begin{frame}
		\tableofcontents  
	\end{frame}
	
	\begin{frame}{New Model}
		\begin{equation}\label{newmodel}
			\begin{aligned}
				\min ~&f\left(y\right)\\
				\mathrm{s.t.}~~ &Ax=y
			\end{aligned}
		\end{equation}
		where $f\left(y\right)=\sum\limits_{i=1}^{m}\Vert \dqn{r}_{i}\Vert_{s}+i_{\Omega}\left(\dqn{p}\right)$ and $A$ has full column rank.
	\end{frame}
	
	\begin{frame}{Assumption}
		\begin{block}{title}
			for $\forall k$,
			\begin{equation}\label{ass}
				\Vert A^{T}\left(Ax^{k}-y^{k}\right)\Vert\ge\xi\Vert Ax^{k}-y^{k}\Vert
			\end{equation}
		\end{block}
		
	\end{frame}
	\begin{frame}{Literature Review}
		\printbibliography
	\end{frame}
	\begin{frame}{Hand-eye Calibration Literature Review}
		\section{Literature Review}
		The first hand-eye calibration methods were published by
		Shiu and Ahmad\footfullcite{shiu1987calibration} and Tsai and Lenz\footfullcite{tsai1989new} who used  axis/angle representation.
		\\ \hspace*{\fill} \\ 
		Quaternions were used by
		Chou and Kamel\footfullcite{chou1991finding} and Horaud and Dornaika\footfullcite{horaud1995hand}.
		\\ \hspace*{\fill} \\ 
		Wang\footfullcite{wang1992extrinsic} used  canonical
		matrix representation. 
	\end{frame}
	
	\begin{frame}{Dual Quaternion Literature Review}
		Dual quaternions are an extension of the real quaternions by means of the dual numbers\footfullcite{study1891bewegungen}$^{,}$ \footfullcite{blaschke1960kinematik}, and were first introduced by Clifford\footfullcite{Clifford1871PreliminarySO}.
		\\ \hspace*{\fill} \\ 
		Daniilidis\footfullcite{daniilidis1999hand} first used dual quaternions to solve the hand-eye calibration problem.
	\end{frame}
	
	\begin{frame}{Data Selection Literature Review}
		
		
	\end{frame}
	
	
	\begin{frame}{Convergence Analysis Literature Review}
		
		Our paper needs to prove the convergence of the classic ADMM for our nonconvex optimization problems(\ref{newmodel}).A very important technique to prove the convergence for nonconvex optimization problems relies on the assumption that the objective functions satisfy the Kurdyka-Łojasiewicz inequality which our problem satisfies. 
		\\ \hspace*{\fill} \\ 
		
		These facts originate in the pioneering and fundamental works of Łojasiewicz\footfullcite{lojasiewicz1963propriete} and Kurdyka\footfullcite{kurdyka1998gradients}.
		Bolte	\footfullcite{bolte2007lojasiewicz}$^{,}$\footfullcite{bolte2007clarke} extended Kurdyka-Łojasiewicz inequality to nonsmooth functions.
	\end{frame}
	
	\begin{frame}{Convergence Analysis Literature Review}
		Hong et al.\footfullcite{hong2016convergence}, Wang et al\footfullcite{wang2014convergence}. and Li and Pong\footfullcite{li2015global} make the contribution on 
		ADMM algorithm used to solve nonconvex problems.
		
		
		Li and Pong proved the convergence analysis of ADMM  in nonconvex optimization problems which the objective functions is
		required to be twice continuously differentiable with bounded Hessian.
		\\ \hspace*{\fill} \\ 
		In Guo,Han and Wu\footfullcite{guo2017convergence},they proved the convergence analysis and rete of alternating direction mothod in nonconvex optimization problems using Kurdyka-Łojasiewicz inequality.
		\\ \hspace*{\fill} \\ 
		Bot and Nguyen\footfullcite{boct2020proximal} gave the convergence analysis and rete of proximal alternating direction mothod in nonconvex optimization problems using Kurdyka-Łojasiewicz inequality.
	\end{frame}
	

	
	\begin{frame}{Kurdyka-Łojasiewicz inequality}
		Let $f: \mathcal{R}^n \rightarrow \mathcal{R} \cup\{+\infty\}$ be a proper lower semicontinuous function. For $-\infty<\eta_1<\eta_2 \leq+\infty$, set
		$$
		\left[\eta_1<f<\eta_2\right]=\left\{x \in \mathcal{R}^n: \eta_1<f\left(x\right)<\eta_2\right\} .
		$$
		
		We say that function $f$ has the $\mathrm{KL}$ property at $x^* \in \operatorname{dom} \partial f$ if there exist $\eta \in(0,+\infty]$, a neighborhood $U$ of $x^*$, and a continuous concave function $\varphi:[0, \eta) \rightarrow \mathcal{R}_{+}$, such that
		
		(i). $\varphi\left(0\right)=0$;
		
		(ii). $\varphi$ is $C^1$ on $\left(0, \eta\right)$ and continuous at 0 ;
		
		(iii). $\varphi^{\prime}\left(s\right)>0, \forall s \in\left(0, \eta\right)$;
		
		(iv). for all $x$ in $U \cap\left[f\left(x^*\right)<f<f\left(x^*\right)+\eta\right]$, the Kurdyka-Łojasiewicz inequality holds:
		$$
		\varphi^{\prime}\left(f\left(x\right)-f\left(x^*\right)\right) d\left(0, \partial f\left(x\right)\right) \ge 1 .
		$$
	\end{frame}
	
	\begin{frame}{Uniformized KL property}
		Let $\Omega$ be a compact set and let $f: \mathcal{R}^n \rightarrow \mathcal{R} \cup\{+\infty\}$ be a proper and lower semicontinuous function. Assume that $f$ is constant on $\Omega$ and satisfies the KL property at each point of $\Omega$. Then, there exist $\epsilon>0, \eta>0$, and $\varphi \in \Phi_\eta$ such that for all $\bar{x} \in \Omega$ and for all $x$ in the following intersection
		$$
		\left\{x \in \mathcal{R}^n: d\left(x, \Omega\right)<\epsilon\right\} \cap\left[f\left(\bar{x}\right)<f<f\left(\bar{x}\right)+\eta\right]
		$$
		one has,
		$$
		\varphi^{\prime}\left(f\left(x\right)-f\left(\bar{x}\right)\right) d\left(0, \partial f\left(x\right)\right) \geq 1 .
		$$
	\end{frame}
	
	\begin{frame}{Convergence Analysis Lemma1}
		Let $\{\omega^{k}=\left(x^{k},y^{k},\lambda^{k}\right)\}$ be the sequence generated by ADMM.Define $x_{k+1}^{*}=\left(-2\sigma+2\mu\right)\left(x^{k+1}-x^{k}\right)+A^{T}\left(\lambda^{k+1}-\lambda^{k}\right)$,$x_{k}^{*}=-2\mu\left(x^{k+1}-x^{k}\right)$,$y_{k+1}^{*}=-\beta A\left(x^{k+1}-x^{k}\right)-\left(\lambda^{k+1}-\lambda^{k}\right)$ and $\lambda_{k+1}^{*}=\frac{1}{\beta}\left(\lambda^{k+1}-\lambda^{k}\right)$.Then we have $\left(x_{k+1}^{*},y_{k+1}^{*},\lambda_{k+1}^{*},x_{k}^{*}\right)\in \partial\Phi_{\beta,\mu}\left(\omega^{k+1},x^{k}\right)$.Moreover,we can get
		$$d\left(0,\partial\Phi_{\beta,\mu}\left(\omega^{k+1},x^{k}\right)\right)\le\left(\zeta_{1}+\frac{2\sigma\zeta_{2}}{\xi}\right)\Vert x^{k+1}-x^{k}\Vert+\frac{2\sigma\zeta_{2}}{\xi}\Vert x^{k}-x^{k-1}\Vert.$$
	\end{frame}
	
	\begin{frame}{Convergence Analysis Lemma2}
		Let $\{\omega^{k}=\left(x^{k},y^{k},\lambda^{k}\right)\}$ be the sequence generated by ADMM.Let $S\left(\omega^0\right)$ denote the set of its limit points of ${\left(x^{k},y^{k},\lambda^{k}\right)}$ and $S^{'}\left(\omega^{1},x^{0}\right)$ denote the set of its limit points of  ${\left(x^{k+1},y^{k+1},\lambda^{k+1},x^{k}\right)}.$ Then
		
		(i) $S^{'}\left(\omega^{1},x^{0}\right)$ is a nonempty compact set, and
		$$
		d\left(\left(\omega^k,x^k\right), S^{'}\left(\omega^{1},x^{0}\right)\right) \rightarrow 0, \text { as } k \rightarrow+\infty ;
		$$
		
		(ii) $S\left(\omega^0\right) \subset \operatorname{crit} \mathcal{L}_\beta$;
		
		(iii) $\Phi_{\beta,\mu}\left(\cdot\right)$ is finite and constant on $S\left(\omega^1,x^0\right)$, equal to $$\inf\limits_{k \in N} \Phi_{\beta,\mu}\left(\omega^{k+1},x^{k}\right)=\lim\limits_{k \rightarrow+\infty} \Phi_{\beta,\mu}\left(\omega^{k+1},x^{k}\right).$$
	\end{frame}
	
	
	\begin{frame}{Convergence Analysis}
		\section{Convergence Analysis}
		Let $\{\omega^{k}=\left(x^{k},y^{k},\lambda^{k}\right)\}$ be the sequence generated by ADMM. From that $f$ satisfies the KL property, then $\{\omega^{k+1}.x^{k}\}$ has finite length, that is
		$$
		\sum_{k=0}^{+\infty}\Vert \omega^{k+1}-\omega^k\Vert<+\infty,
		$$
		and as a consequence, $\{\omega^{k+1},x^{k}\}$ converges to a critical point of $\Phi_{\beta,\mu}\left(\cdot\right)$.
	\end{frame}
	
	
	\begin{frame}{Convergence Rate Lemma}
		Let $\triangle_{k}$ be a noninscreasing sequence in $\mathbb{R}$ converging 0.Assume futher that there exist $k_{0}$ such that for every $k>k_{0}$,
		$$\triangle_{k}\le\triangle_{k-2}-\triangle_{k}+\alpha\left(\triangle_{k-2}-\triangle_{k}\right)^{\frac{1-\theta}{\theta}},$$
		where $\alpha>0$ and $\theta\in[0,1)$,then following statements are true:
		
		(i)If $\theta\in(0,\frac{1}{2}]$,then there exist $c>0$ and $\tau\in[0,1)$,such that
		$$\triangle_{k}\le c\tau^{k}.$$
		
		(ii)If $\theta\in\left(\frac{1}{2},1\right)$,then there exists $c>0$,such that
		$$\triangle_{k}\le ck^{\frac{\theta-1}{2\theta-1}}.$$
	\end{frame}
	
	\begin{frame}{Convergence Rate}
		\section{Convergence Rate}
		Let $\{\omega^{k}=\left(x^{k},y^{k},\lambda^{k}\right)\}$ be the sequence generated by ADMM and converges to $\{\omega^{*}=\left(x^{*},y^{*},\lambda^{*}\right)\}.$ Assume that $\Phi_{\beta,\mu}\left(\cdot\right)$ has the KL property at $\left(\omega^{*},x^{*}\right)$ with $\varphi\left(s\right)=cs^{1-\theta},\theta\in\left(0,1\right),c>0$.Then the following estimations hold:
		
		(i)If $\theta=0$,then the sequence $\{\omega^{k}=\left(x^{k},y^{k},\lambda^{k}\right)\}$ converges in a finite number of steps.
		
		(ii)If $\theta\in(0,\frac{1}{2}]$,then there exist $c>0$ and $\tau\in[0,1)$,such that
		$$\Vert\left(x^{k},y^{k},\lambda^{k}\right)-\left(x^{*},y^{*},\lambda^{*}\right)\Vert\le c\tau^{k}.$$
		
		(iii)If $\theta\in\left(\frac{1}{2},1\right)$,then there exists $c>0$,such that
		$$\Vert\left(x^{k},y^{k},\lambda^{k}\right)-\left(x^{*},y^{*},\lambda^{*}\right)\Vert\le ck^{\frac{\theta-1}{2\theta-1}}.$$
	\end{frame}

\end{document}


