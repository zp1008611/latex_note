% !Mode:: "TeX:UTF-8"

\documentclass[UTF8,12pt,hyperref]{ctexart}

\usepackage{amsfonts,amsmath,amssymb,mathrsfs}
\usepackage{flafter,graphicx,xcolor}

\newtheorem{algorithm}{Algorithm}
 
 \newtheorem{theorem}{Theorem}
% Matlab
\usepackage{listings}
\lstset{language=Matlab}%代码语言使用的是matlab
\lstset{breaklines}%自动将长的代码行换行排版
\lstset{extendedchars=false}%解决代码跨页时,章节标题,页眉等汉字不显示的问题
\lstset{numbers=left, 
numberstyle= \tiny,keywordstyle= \color{ blue!70},commentstyle=\color{red!50!green!50!blue!50}, 
frame=shadowbox, rulesepcolor= \color{ red!20!green!20!blue!20}, 
escapeinside=``} 
%%%%%%%
\usepackage{multicol}
   
 
\newcommand{\ep}{\hfill\rule{0.15cm}{0.35cm}\vskip 0.3cm}
\newenvironment{proof}[1][Proof.]{\begin{trivlist}
\item[\hskip \labelsep {\bfseries #1}]}{\ep\end{trivlist}} 


%%%%%
 \makeatletter
\newenvironment{exercise}[1][{\color{blue}\bf Exercise}]%
%\newenvironment{xiti}[1][{\color{blue}\bf Exercise}]%
{%
 \begin{center}   \begin{lrbox}{\@tempboxa}%
    \begin{minipage}{\textwidth}%
  {\color{blue}\bfseries
#1}   }{%
    \end{minipage}%
    \end{lrbox}
    \colorbox{green}{\noindent\usebox{\@tempboxa}} \end{center}  
}
 
\newenvironment{analysis}[1][\color{blue}\bf Analysis]%
{%
 \begin{center}   \begin{lrbox}{\@tempboxa}%
    \begin{minipage}{\textwidth}
  {\color{blue}\bfseries
%#1} \\ \hspace*{2em} }{%
 #1}  }{%
    \end{minipage}%
    \end{lrbox}
    %\colorbox{gray}{\noindent\usebox{\@tempboxa}} \end{center}  %
    \fbox{\usebox{\@tempboxa}}\end{center}%
    %\colorbox{red}{\fbox{\noindent\usebox{\@tempboxa}} \end{center}
}
 
\newenvironment{solve}[1][\color{blue}\bf Solve]{\begin{trivlist}
\item[\hskip \labelsep {\color{blue}\bfseries
#1}]}{\hfill$\Box$\end{trivlist}}
 \makeatother
 
 
%%%%%%%%%
 
\renewcommand{\baselinestretch}{1.38}
 
 \pagestyle{plain} 
 
 \textwidth  168 true mm \textheight 230 true mm \topmargin=-1.3cm
\oddsidemargin   0pt \evensidemargin  20pt \marginparwidth  10pt

\CTEXoptions[today=old]

 
\begin{document}
%%%================  标  题   ================

\begin{center}
{\bf  THE SOUTH CHINA NORMAL UNIVERSITY\vspace{0.08cm}

School of Mathematical Sciences\vspace{0.08cm}
 
Numerical Analysis ( 2023-2024 The Second Term) \vspace{0.18cm}

{\Large Homework 11 }\vspace{0.18cm}

Due Date: \underline{June 5, 2024 (Wednesday)} }
\end{center}\vspace{-0.16cm}

\begin{center}
Name:\ \underline{\hspace{4cm}}\hspace{0.298cm}  
% 
Student No.:\ \underline{\hspace{4cm}} 
%
Date:\ \ \underline{May  30, 2024} 
 \end{center}
 
 
%%%================  正  文   ================  

\section*{\S 7.3  Recursive Rules and Romberg Integration}
 
  
\begin{exercise}1.    \quad %4   
\begin{description}
\item[(a)] Start with $T(0)=(h/2)(f(a) + f(b))$. Then a sequence of trapezoidal rules $\{T(J)\}$ is generated by the recursive formula  \vspace{-0.18cm} 
\begin{equation}\label{Eqn:7-3-2}
T(J)=\dfrac{T(J-1)}{2}+h\sum\limits_{k=1}^Mf(x_{2k-1}) \quad\text{for}\quad J=1, 2, \cdots ,   \vspace{-0.18cm}  
\end{equation}
where $h= (b- a)/2^J$, {$M=2^{J-1}$ } and $\{x_k=a+kh\}$. 

{\bf Show that the sequential trapezoidal rule converges to $L$ (i.e., $\lim_{J\to \infty}T(J) = L)$. }


\item[(b)] Suppose that $\{T(J)\}$ is the sequence of trapezoidal rules generated by \eqref{Eqn:7-3-2}. If $J\geqslant 1$ and $S(J)$ is Simpson's rule for $2^J$ subintervals of $[a, b]$, then $S(J)$ and the trapezoidal rules $T(J -1)$ and $T(J)$ obey the relationship  \vspace{-0.18cm}  
\begin{equation}\label{Eqn:7-3-7}
S(J)=\dfrac{4T(J)-T(J-1)}{3} \quad\text{for}\quad J=1, 2, \cdots . 
\end{equation}


 {\bf  Show that the sequential Simpson rule converges to $L$(i.e., $\lim_{J\to \infty}S(J) = L)$. }
 
 \end{description}  
  \end{exercise} 
  
 
\begin{solve}
  1

\end{solve}


\end{document}

 
 
 