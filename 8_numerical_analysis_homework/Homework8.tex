
 
 
% !Mode:: "TeX:UTF-8"

\documentclass[UTF8,12pt,hyperref]{ctexart}

\usepackage{amsfonts,amsmath,amssymb,mathrsfs}
\usepackage{flafter,graphicx,xcolor}

\newtheorem{algorithm}{Algorithm}
 
 \newtheorem{theorem}{Theorem}
% Matlab
\usepackage{listings}
\lstset{language=Matlab}%代码语言使用的是matlab
\lstset{breaklines}%自动将长的代码行换行排版
\lstset{extendedchars=false}%解决代码跨页时,章节标题,页眉等汉字不显示的问题
\lstset{numbers=left, 
numberstyle= \tiny,keywordstyle= \color{ blue!70},commentstyle=\color{red!50!green!50!blue!50}, 
frame=shadowbox, rulesepcolor= \color{ red!20!green!20!blue!20}, 
escapeinside=``} 
%%%%%%%
\usepackage{multicol}
   
 
\newcommand{\ep}{\hfill\rule{0.15cm}{0.35cm}\vskip 0.3cm}
\newenvironment{proof}[1][Proof.]{\begin{trivlist}
\item[\hskip \labelsep {\bfseries #1}]}{\ep\end{trivlist}} 


%%%%%
 \makeatletter
\newenvironment{exercise}[1][{\color{blue}\bf Exercise}]%
%\newenvironment{xiti}[1][{\color{blue}\bf Exercise}]%
{%
 \begin{center}   \begin{lrbox}{\@tempboxa}%
    \begin{minipage}{\textwidth}%
  {\color{blue}\bfseries
#1}   }{%
    \end{minipage}%
    \end{lrbox}
    \colorbox{green}{\noindent\usebox{\@tempboxa}} \end{center}  
}
 
\newenvironment{analysis}[1][\color{blue}\bf Analysis]%
{%
 \begin{center}   \begin{lrbox}{\@tempboxa}%
    \begin{minipage}{\textwidth}
  {\color{blue}\bfseries
%#1} \\ \hspace*{2em} }{%
 #1}  }{%
    \end{minipage}%
    \end{lrbox}
    %\colorbox{gray}{\noindent\usebox{\@tempboxa}} \end{center}  %
    \fbox{\usebox{\@tempboxa}}\end{center}%
    %\colorbox{red}{\fbox{\noindent\usebox{\@tempboxa}} \end{center}
}
 
\newenvironment{solve}[1][\color{blue}\bf Solve]{\begin{trivlist}
\item[\hskip \labelsep {\color{blue}\bfseries
#1}]}{\hfill$\Box$\end{trivlist}}
 \makeatother
 
 
%%%%%%%%%
 
\renewcommand{\baselinestretch}{1.38}
 
 \pagestyle{plain} 
 
 \textwidth  168 true mm \textheight 230 true mm \topmargin=-1.3cm
\oddsidemargin   0pt \evensidemargin  20pt \marginparwidth  10pt

\CTEXoptions[today=old]

 
\begin{document}
%%%================  标  题   ================

\begin{center}
{\bf  THE SOUTH CHINA NORMAL UNIVERSITY\vspace{0.08cm}

School of Mathematical Sciences\vspace{0.08cm}
 
Numerical Analysis ( 2023-2024 The Second Term) \vspace{0.18cm}

{\Large Homework 8: }\vspace{0.18cm}

Due Date: \underline{May 15, 2022 (Wednesday)} }
\end{center}\vspace{-0.16cm}

\begin{center}
  Name:\ \underline{\qquad 钟沛 \hspace{1cm}}\hspace{0.298cm}  
  % 
  Student No.:\ \underline{\qquad 2023021950\hspace{1cm}} 
%
Date:\ \ \underline{May 9, 2022} 
 \end{center}
 
 
%%%================  正  文   ================

\section*{\S 5.1 Exercises for Least-squares Line}
  
\begin{exercise}1.\qquad
 Derive the normal equations for finding the least-squares parabola 
  $$
 y=Ax^2+B.
 $$ 
 \end{exercise}   
  
\begin{solve}
  Suppose that $\{(x_k,y_k)\}_{k=1}^N$ are $N$ points, where the abscissas $\{x_k\}_{k=1}^N$ are distinct.
  The coefficients $A$ and $B$ will minimize the quantity:
  \begin{align*}
    E(A,B) = \sum\limits_{k=1}^N(Ax_k^2+B-y_k)^2.
  \end{align*}
  The partial derivatives $\partial E/\partial A$ and $\partial E/\partial B$ must all be zero.
  This results in 
  \begin{align}
    0&=\frac{\partial E(A,B)}{\partial A}= 2\sum\limits_{k=1}^N (Ax_k^2+B-y_k)x_k^2,\label{eq:partial A}\\
    0&=\frac{\partial E(A,B)}{\partial B}= 2\sum\limits_{k=1}^N (Ax_k^2+B-y_k).\label{eq:partial B}
  \end{align}
  Equations (\ref{eq:partial A}) and (\ref{eq:partial B}) can be rearranged in the standard form for a system
  and result in the normal equations:
  \begin{align}
    (\sum\limits_{k=1}^{N}x_k^4)A+(\sum\limits_{k=1}^Nx_k^2)B&=\sum\limits_{k=1}^N x_k^2y_k,\\
    (\sum\limits_{k=1}^{N}x_k^2)A+NB&=\sum\limits_{k=1}^N y_k.
  \end{align}
\end{solve}

%%%================  正  文   ================

\section*{\S  5.2 Exercises for Curve Fitting}
  
\begin{exercise}1.\qquad
Carry out the indicated change of variables in Table 5.6, and derive the Iinearized form for the following functions:
$$
\displaystyle y=\dfrac{D}{x+C}.
$$
\end{exercise} 

\begin{solve}
  If $C\neq 0$, then 
  \begin{align*}
    y=\frac{D}{x+C} \Leftrightarrow xy+Cy=D \Leftrightarrow y=-\frac{1}{C}xy + \frac{D}{C}.
  \end{align*}
  Let $X=xy$, $Y=y$, $A=-\frac{1}{C}$, $B=\frac{D}{C}$. This results in a linear relation between the new variables $X$ and $Y$:
    $Y=AX+B$. 
  If $C=0$, then
  \begin{align*}
    y=\frac{D}{x+C}=\frac{D}{x}.
  \end{align*}
  Let $X=\frac{1}{x}$, $Y=y$ , $A=D$, $B=0$. This results in a linear relation between the new variables $X$ and $Y$:
    $Y=AX+B$.
\end{solve}

\end{document}


 
 