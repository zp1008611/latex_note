% !Mode:: "TeX:UTF-8"

\documentclass[UTF8,12pt,hyperref]{ctexart}

\usepackage{amsfonts,amsmath,amssymb,mathrsfs}
\usepackage{flafter,graphicx,xcolor}

\newtheorem{algorithm}{Algorithm}

% Matlab
\usepackage{listings}
\lstset{language=Matlab}%代码语言使用的是matlab
\lstset{breaklines}%自动将长的代码行换行排版
\lstset{extendedchars=false}%解决代码跨页时,章节标题,页眉等汉字不显示的问题
\lstset{numbers=left, 
numberstyle= \tiny,keywordstyle= \color{ blue!70},commentstyle=\color{red!50!green!50!blue!50}, 
frame=shadowbox, rulesepcolor= \color{ red!20!green!20!blue!20}, 
escapeinside=``} 
%%%%%%%
\usepackage{multicol}
   
 
\newcommand{\ep}{\hfill\rule{0.15cm}{0.35cm}\vskip 0.3cm}
\newenvironment{proof}[1][Proof.]{\begin{trivlist}
\item[\hskip \labelsep {\bfseries #1}]}{\ep\end{trivlist}} 


%%%%%
 \makeatletter
\newenvironment{exercise}[1][{\color{blue}\bf Exercise}]%
%\newenvironment{xiti}[1][{\color{blue}\bf Exercise}]%
{%
 \begin{center}   \begin{lrbox}{\@tempboxa}%
    \begin{minipage}{\textwidth}%
  {\color{blue}\bfseries
#1}   }{%
    \end{minipage}%
    \end{lrbox}
    \colorbox{green}{\noindent\usebox{\@tempboxa}} \end{center}  
}
 
\newenvironment{analysis}[1][\color{blue}\bf Analysis]%
{%
 \begin{center}   \begin{lrbox}{\@tempboxa}%
    \begin{minipage}{\textwidth}
  {\color{blue}\bfseries
%#1} \\ \hspace*{2em} }{%
 #1}  }{%
    \end{minipage}%
    \end{lrbox}
    %\colorbox{gray}{\noindent\usebox{\@tempboxa}} \end{center}  %
    \fbox{\usebox{\@tempboxa}}\end{center}%
    %\colorbox{red}{\fbox{\noindent\usebox{\@tempboxa}} \end{center}
}
 
\newenvironment{solve}[1][\color{blue}\bf Solve]{\begin{trivlist}
\item[\hskip \labelsep {\color{blue}\bfseries
#1}]}{\hfill$\Box$\end{trivlist}}
 \makeatother
 
 
%%%%%%%%%
 
\renewcommand{\baselinestretch}{1.38}
 
 \pagestyle{plain} 
 
 \textwidth  168 true mm \textheight 230 true mm \topmargin=-1.3cm
\oddsidemargin   0pt \evensidemargin  20pt \marginparwidth  10pt

\CTEXoptions[today=old]

 
\begin{document}
%%%================  标  题   ================

\begin{center}
{\bf  THE SOUTH CHINA NORMAL UNIVERSITY\vspace{0.08cm}

School of Mathematical Sciences\vspace{0.08cm}
 
Numerical Analysis ( 2023-2024 The Second Term) \vspace{0.18cm}

{\Large Homework 5: }\vspace{0.18cm}

Due Date: \underline{April  24, 2024 (Wednesday)} }
\end{center}\vspace{-0.16cm}

\begin{center}
  Name:\ \underline{\qquad 钟沛 \hspace{1cm}}\hspace{0.298cm}  
  % 
  Student No.:\ \underline{\qquad 2023021950\hspace{1cm}}  
%
Date:\ \ \underline{April 17, 2024} 
 \end{center}
 
 
%%%================  正  文   ================

 \begin{center}  \bf \Large
{\S 3.4 Gaussian Elimination and Pivoting}
\end{center} 

\begin{exercise}1. \quad   
 Find the parabola $y=A+Bx+Cx^2$ that passes through $(1, 6),(2, 5)$, and $(3, 2)$. 
 \end{exercise}
 
 \begin{solve}
  For each point we obtain an equation relating the value of $x$ to the value of $y$.
  The result is the linear system
  \begin{equation}
    \begin{aligned}
      \begin{array}{ccccccccr}
        A  & + & B & + & C & = & 6 & \text{\quad at (1,6)}\\
        A  & + & 2B & + & 4C & = & 5 & \text{\quad at (2,5)}\\
        A  & + & 3B & + & 9C & = & 2 &  \text{\quad at (3,2)}. \\
        \end{array}
    \end{aligned}
  \end{equation}
The variable $A$ is eliminated from the second and third equations by
subtracting the first equation from them, and the resulting equivalent linear system is
\begin{equation}
  \begin{aligned}
    \begin{array}{ccccccccr}
      A  & + & B & + & C & = & 6 & \\
        &  & B & + & 3C & = & -1 & \\
        &  & 2B & + & 8C & = & -4. & \\
      \end{array}
  \end{aligned}
  \label{eq:array2}
\end{equation}
The variable $B$ is eliminated from the third equation in (\ref{eq:array2}) by subtracting
from it two times the second equation. We arrive at the equivalent
upper-triangular system:
\begin{equation}
  \begin{aligned}
    \begin{array}{ccccccccr}
      A  & + & B & + & C & = & 6 & \\
        &  & B & + & 3C & = & -1 & \\
        &  &  &  & 2C & = & -2. & \\
      \end{array}
  \end{aligned}
  \label{eq:array3}
\end{equation}
The back-substitution algorithm is now used to find the coefficients
$C=-1,B=2,A=5$, and the
equation of the parabola is $y=5+2x-x^2$.


\end{solve}
%%%%================
%%%%================
% 
  
 
\begin{center}  \bf \Large
{\S 3.5 Triangular Factorization }
\end{center} 
%%%================ 
\begin{exercise}1. \quad
\quad Find the triangular factorization $\boldsymbol{A}=\boldsymbol{LU}$ for the matrix
$$
\left[\begin{array}{rrrr}
1 & 1 & 0 & 4  \\
2 & -1 & 5 & 0 \\
5 & 2 & 1 & 2  \\
-3 & 0 & 2 & 6
\end{array}\right]. 
$$
\end{exercise}
%% 
\begin{solve}
  We've seen how to use elimination to convert a matrix $\boldsymbol{A}$ into an upper
triangular matrix $\boldsymbol{U}$. This leads to the factorization $\boldsymbol{A} = \boldsymbol{LU}$.
we can describe the elimination of the entries of matrix $\boldsymbol{A}$ in terms of multiplication by a succession of
elimination matrices $L_{i}$, so that $\boldsymbol{A}\longrightarrow \boldsymbol{L}_1\boldsymbol{A}\longrightarrow \boldsymbol{L}_2\boldsymbol{L}_{1}\boldsymbol{A}\longrightarrow \dots \boldsymbol{U}$.
The first step of Gaussian elimination looks like this: twice the first row is subtracted from
the second one,  five times the first row is subtracted from the third one, and three times 
first row is added to forth one. This can be written as:
\begin{align*}
  \boldsymbol{L}_1\boldsymbol{A} = \left[\begin{array}{rrrr}
    1 & 0 & 0 & 0  \\
    -2 & 1 & 0 & 0 \\
    -5 &0 & 1 & 0  \\
    3 & 0 & 0 & 1
    \end{array}\right]\boldsymbol{A}=\left[\begin{array}{rrrr}
      1 & 1 & 0 & 4  \\
      0 & -3 & 5 & -8 \\
      0 & -3 & 1 & -18  \\
      0 & 3 & 2 & 18
      \end{array}\right]
\end{align*}
Next we subtract the second row from the third and add it to the fourth row:
\begin{align*}
  \boldsymbol{L}_2\boldsymbol{L}_1\boldsymbol{A} =
  \left[\begin{array}{rrrr}
    1 & 0 & 0 & 0  \\
    0 & 1 & 0 & 0 \\
    0 & -1 & 1 & 0  \\
    0 & 1 & 0 & 1
    \end{array}\right]
    \left[\begin{array}{rrrr}
      1 & 1 & 0 & 4  \\
      0 & -3 & 5 & -8 \\
      0 & -3 & 1 & -18  \\
      0 & 3 & 2 & 18
      \end{array}\right]=\left[\begin{array}{rrrr}
      1 & 1 & 0 & 4  \\
      0 & -3 & 5 & -8 \\
      0 & 0 & -4 & -10  \\
      0 & 0 & 7 & 10
      \end{array}\right]
\end{align*}
Finally we eliminate the fourth row:
\begin{align*}
  \boldsymbol{L}_3\boldsymbol{L}_2\boldsymbol{L}_1\boldsymbol{A}=
  \left[\begin{array}{rrrr}
    1 & 0 & 0 & 0  \\
    0 & 1 & 0 & 0 \\
    0 & 0 & 1 & 0  \\
    0 & 0 & \frac{7}{4} & 1
    \end{array}\right]
  \left[\begin{array}{rrrr}
    1 & 1 & 0 & 4  \\
    0 & -3 & 5 & -8 \\
    0 & 0 & -4 & -10  \\
    0 & 0 & 7 & 10
    \end{array}\right]=
    \left[\begin{array}{rrrr}
      1 & 1 & 0 & 4  \\
      0 & -3 & 5 & -8 \\
      0 & 0 & -4 & -10  \\
      0 & 0 & 0 & -\frac{15}{2}
      \end{array}\right]
\end{align*}
To exhibit the full factorization $\boldsymbol{A}=\boldsymbol{L}\boldsymbol{U}$ we need to compute the product
$\boldsymbol{L}=\boldsymbol{L}_1^{-1}\boldsymbol{L}_2^{-1}\boldsymbol{L}_3^{-1}$.
And the inverse of $L_j$, $j=1,2,3$ is just $L_j$ itself, but with each entry below the diagonal negated:
\begin{align*}
  \boldsymbol{L}_1^{-1} = 
  \left[\begin{array}{rrrr}
    1 & 0 & 0 & 0  \\
    2 & 1 & 0 & 0 \\
    5 &0 & 1 & 0  \\
    -3 & 0 & 0 & 1
    \end{array}\right],
    \boldsymbol{L}_2^{-1} = 
    \left[\begin{array}{rrrr}
      1 & 0 & 0 & 0  \\
      0 & 1 & 0 & 0 \\
      0 & 1 & 1 & 0  \\
      0 & -1 & 0 & 1
      \end{array}\right],
    \boldsymbol{L}_3^{-1} = 
      \left[\begin{array}{rrrr}
        1 & 0 & 0 & 0  \\
        0 & 1 & 0 & 0 \\
        0 & 0 & 1 & 0  \\
        0 & 0 & -\frac{7}{4} & 1
        \end{array}\right]
\end{align*}

The product $\boldsymbol{L}_1^{-1}\boldsymbol{L}_2^{-1}\boldsymbol{L}_3^{-1}$
is just the unit lower-triangular matrix with the nonzero
subdiagonal entries of $\boldsymbol{L}_1^{-1},\boldsymbol{L}_2^{-1}$ and $\boldsymbol{L}_3^{-1}$ 
inserted in the appropriate places:
\begin{align*}
  \boldsymbol{L} = \left[\begin{array}{rrrr}
    1 & 0 & 0 & 0  \\
    2 & 1 & 0 & 0 \\
    5 & 1 & 1 & 0  \\
    -3 & -1 & -\frac{7}{4} & 1
    \end{array}\right]
\end{align*}
Together we have
\begin{align*}
  \boldsymbol{A} = 
  \left[\begin{array}{rrrr}
    1 & 0 & 0 & 0  \\
    2 & 1 & 0 & 0 \\
    5 & 1 & 1 & 0  \\
    -3 & -1 & -\frac{7}{4} & 1
    \end{array}\right]\left[\begin{array}{rrrr}
      1 & 1 & 0 & 4  \\
      0 & -3 & 5 & -8 \\
      0 & 0 & -4 & -10  \\
      0 & 0 & 0 & -\frac{15}{2}
      \end{array}\right]=\boldsymbol{L}\boldsymbol{U}.
\end{align*}

\end{solve}



\end{document}

 
 