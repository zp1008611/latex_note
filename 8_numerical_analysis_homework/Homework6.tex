% !Mode:: "TeX:UTF-8"

\documentclass[UTF8,12pt,hyperref]{ctexart}

\usepackage{amsfonts,amsmath,amssymb,mathrsfs}
\usepackage{flafter,graphicx,xcolor}

\newtheorem{algorithm}{Algorithm}

% Matlab
\usepackage{listings}
\lstset{language=Matlab}%代码语言使用的是matlab
\lstset{breaklines}%自动将长的代码行换行排版
\lstset{extendedchars=false}%解决代码跨页时,章节标题,页眉等汉字不显示的问题
\lstset{numbers=left, 
numberstyle= \tiny,keywordstyle= \color{ blue!70},commentstyle=\color{red!50!green!50!blue!50}, 
frame=shadowbox, rulesepcolor= \color{ red!20!green!20!blue!20}, 
escapeinside=``} 
%%%%%%%
\usepackage{multicol}
   
 
\newcommand{\ep}{\hfill\rule{0.15cm}{0.35cm}\vskip 0.3cm}
\newenvironment{proof}[1][Proof.]{\begin{trivlist}
\item[\hskip \labelsep {\bfseries #1}]}{\ep\end{trivlist}} 


%%%%%
 \makeatletter
\newenvironment{exercise}[1][{\color{blue}\bf Exercise}]%
%\newenvironment{xiti}[1][{\color{blue}\bf Exercise}]%
{%
 \begin{center}   \begin{lrbox}{\@tempboxa}%
    \begin{minipage}{\textwidth}%
  {\color{blue}\bfseries
#1}   }{%
    \end{minipage}%
    \end{lrbox}
    \colorbox{green}{\noindent\usebox{\@tempboxa}} \end{center}  
}
 
\newenvironment{analysis}[1][\color{blue}\bf Analysis]%
{%
 \begin{center}   \begin{lrbox}{\@tempboxa}%
    \begin{minipage}{\textwidth}
  {\color{blue}\bfseries
%#1} \\ \hspace*{2em} }{%
 #1}  }{%
    \end{minipage}%
    \end{lrbox}
    %\colorbox{gray}{\noindent\usebox{\@tempboxa}} \end{center}  %
    \fbox{\usebox{\@tempboxa}}\end{center}%
    %\colorbox{red}{\fbox{\noindent\usebox{\@tempboxa}} \end{center}
}
 
\newenvironment{solve}[1][\color{blue}\bf Solve]{\begin{trivlist}
\item[\hskip \labelsep {\color{blue}\bfseries
#1}]}{\hfill$\Box$\end{trivlist}}
 \makeatother
 
 
%%%%%%%%%
 
\renewcommand{\baselinestretch}{1.38}
 
 \pagestyle{plain} 
 
 \textwidth  168 true mm \textheight 230 true mm \topmargin=-1.3cm
\oddsidemargin   0pt \evensidemargin  20pt \marginparwidth  10pt

\CTEXoptions[today=old]

 
\begin{document}
%%%================  标  题   ================

\begin{center}
{\bf  THE SOUTH CHINA NORMAL UNIVERSITY\vspace{0.08cm}

School of Mathematical Sciences\vspace{0.08cm}
 
Numerical Analysis ( 2023-2024 The Second Term) \vspace{0.18cm}

{\Large Homework 6: }\vspace{0.18cm}

Due Date: \underline{April  27, 2024 (Wednesday)} }
\end{center}\vspace{-0.16cm}

\begin{center}
  Name:\ \underline{\qquad 钟沛 \hspace{1cm}}\hspace{0.298cm}  
  % 
  Student No.:\ \underline{\qquad 2023021950\hspace{1cm}} 
%
Date:\ \ \underline{April 25, 2024} 
 \end{center}
 
 
%%%================  正  文   ================

\begin{center}  \bf \Large
{\S  4.1 Exercises for Taylor Series and Calculation of Functions}
\end{center} 

\begin{exercise}1. \quad
\begin{description}
\item[(a)] Find a Taylor polynomial of degree $N = 5$ for $f(x) = x^{1/2}$ expanded about $x_0 =4$.
\item[(b)] Find a Taylor polynomial of degree $N = 5$ for $f(x) = x^{1/2}$ expanded about $x_0 =9$.
\item[(c)] Determine which of the polynomials in parts (a) and (b) best approximates $(6.5)^{1/2}$. 
\end{description} 
\end{exercise}

\begin{solve}
    A Taylor polynomial expanded at $x_0$ has the form:
    \begin{align*}
       T_n(x) =\sum\limits_{k=0}^{n}\frac{f^{(k)}(x_0)(x-x_0)^k}{k!}.
    \end{align*} 
    And the first five derivatives of $f(x)$ are
    \begin{align*}
      f^{(0)}&=f(x)=x^{\frac{1}{2}}\\
      f'&=\frac{1}{2}x^{-\frac{1}{2}}\\
      f''&=-\frac{1}{4}x^{-\frac{3}{2}}\\
      f'''&=\frac{3}{8}x^{-\frac{5}{2}}\\
      f^{(4)} &= -\frac{15}{16} x^{-\frac{7}{2}}\\
      f^{(5)} &= \frac{105}{32} x^{-\frac{9}{2}}.
    \end{align*}
    (a) Plugging in $x_0=4$, we get
    \begin{align*}
      T_5^1(x) &= 4^{\frac{1}{2}} + \frac{1}{2}4^{-\frac{1}{2}}(x-4)-\frac{1}{2!}\frac{1}{4}4^{-\frac{3}{2}}(x-4)^2 +
      \frac{1}{3!}\frac{3}{8}4^{-\frac{5}{2}}(x-4)^3-\frac{1}{4!}\frac{15}{16}4^{-\frac{7}{2}}(x-4)^4
      +\frac{1}{5!}\frac{105}{32}4^{-\frac{9}{2}}(x-4)^5\\
      &= 2+\frac{1}{4}(x-4)-\frac{1}{64}(x-4)^2 + \frac{1}{512}(x-4)^3-\frac{5}{16384}(x-4)^4 + \frac{7}{131072}(x-4)^5.
    \end{align*}
    (b) Plugging in $x_0=9$, we get
    \begin{align*}
      T_5^2(x) &= 9^{\frac{1}{2}} + \frac{1}{2}9^{-\frac{1}{2}}(x-9)-\frac{1}{2!}\frac{1}{4}9^{-\frac{3}{2}}(x-9)^2 +
      \frac{1}{3!}\frac{3}{8}9^{-\frac{5}{2}}(x-9)^3-\frac{1}{4!}\frac{15}{16}9^{-\frac{7}{2}}(x-9)^4
      +\frac{1}{5!}\frac{105}{32}9^{-\frac{9}{2}}(x-9)^5\\
      &= 3+\frac{1}{6}(x-9)-\frac{1}{216}(x-9)^2 + \frac{1}{3888}(x-9)^3-\frac{5}{279936}(x-9)^4 + \frac{7}{5038848}(x-9)^5.
    \end{align*}
    (c) The error term $E_N(x)$ has the form
    \begin{align*}
      E_N(x) =\frac{f^{N+1}(c)}{(N+1)!}(x-x_0)^{N+1}
    \end{align*}
    for some value $c = c(x)$ that lies between $x$ and $x_0$.
    Hence, the bound for the error term of $T^1_5(x)$ is 
    \begin{align*}
      |E_5^1(6.5)|=|\frac{f^{(6)}(c)}{6!}(6.5-4)^{6}| 
        \leqslant |\frac{\frac{105}{32}\cdot \frac{9}{2}\cdot 4^{-\frac{11}{2}}}{6!}2.5^6|
      = 0.002444721758365631
      \end{align*}
    and the bound for the error term of $T^2_5(x)$ is
    \begin{align*}
      |E_5^2(6.5)|=|\frac{f^{(6)}(c)}{6!}(6.5-9)^{6}| 
        \leqslant |\frac{\frac{105}{32}\cdot \frac{9}{2}\cdot 6.5^{-\frac{11}{2}}}{6!}3.5^6|
      = 0.001274395245867137.
    \end{align*}
    Since $|E^2_5(6.5)|<|E^1_5(6.5)|$, 
    polynomials in parts (b) best approximates $(6.5)^{1/2}$.


  \end{solve}

  
\end{document}

 
 