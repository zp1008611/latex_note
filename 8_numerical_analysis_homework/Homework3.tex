% !Mode:: "TeX:UTF-8"

\documentclass[UTF8,12pt,hyperref]{ctexart}

\usepackage{amsfonts,amsmath,amssymb,mathrsfs}
\usepackage{flafter,graphicx,xcolor}

\newtheorem{algorithm}{Algorithm}
\newtheorem{theorem}{Theorem}

% Matlab
\usepackage{listings}
\lstset{language=Matlab}%代码语言使用的是matlab
\lstset{breaklines}%自动将长的代码行换行排版
\lstset{extendedchars=false}%解决代码跨页时,章节标题,页眉等汉字不显示的问题
\lstset{numbers=left, 
numberstyle= \tiny,keywordstyle= \color{ blue!70},commentstyle=\color{red!50!green!50!blue!50}, 
frame=shadowbox, rulesepcolor= \color{ red!20!green!20!blue!20}, 
escapeinside=``} 
%%%%%%%
\usepackage{multicol}
\usepackage{caption}
\captionsetup[table]{name=Table} 
 
\newcommand{\ep}{\hfill\rule{0.15cm}{0.35cm}\vskip 0.3cm}
\newenvironment{proof}[1][Proof.]{\begin{trivlist}
\item[\hskip \labelsep {\bfseries #1}]}{\ep\end{trivlist}} 


%%%%%
 \makeatletter
\newenvironment{exercise}[1][{\color{blue}\bf Exercise}]%
%\newenvironment{xiti}[1][{\color{blue}\bf Exercise}]%
{%
 \begin{center}   \begin{lrbox}{\@tempboxa}%
    \begin{minipage}{\textwidth}%
  {\color{blue}\bfseries
#1}   }{%
    \end{minipage}%
    \end{lrbox}
    \colorbox{green}{\noindent\usebox{\@tempboxa}} \end{center}  
}
 
\newenvironment{analysis}[1][\color{blue}\bf Analysis]%
{%
 \begin{center}   \begin{lrbox}{\@tempboxa}%
    \begin{minipage}{\textwidth}
  {\color{blue}\bfseries
%#1} \\ \hspace*{2em} }{%
 #1}  }{%
    \end{minipage}%
    \end{lrbox}
    %\colorbox{gray}{\noindent\usebox{\@tempboxa}} \end{center}  %
    \fbox{\usebox{\@tempboxa}}\end{center}%
    %\colorbox{red}{\fbox{\noindent\usebox{\@tempboxa}} \end{center}
}
 
\newenvironment{solve}[1][\color{blue}\bf Solve]{\begin{trivlist}
\item[\hskip \labelsep {\color{blue}\bfseries
#1}]}{\hfill$\Box$\end{trivlist}}
 \makeatother
 
 
%%%%%%%%%
 
\renewcommand{\baselinestretch}{1.38}
 
 \pagestyle{plain} 
 
 \textwidth  168 true mm \textheight 230 true mm \topmargin=-1.3cm
\oddsidemargin   0pt \evensidemargin  20pt \marginparwidth  10pt

\CTEXoptions[today=old]

 
\begin{document}
%%%================  标  题   ================

\begin{center}
{\bf  THE SOUTH CHINA NORMAL UNIVERSITY\vspace{0.08cm}

School of Mathematical Sciences\vspace{0.08cm}
 
Numerical Analysis ( 2022--2023 The Second Term) \vspace{0.18cm}

{\Large Homework 3}\vspace{0.18cm}

Due Date: \underline{March 26, 2024 (Tuesday)} }
\end{center}\vspace{-0.16cm}

\begin{center}
  Name:\ \underline{\qquad 钟沛 \hspace{1cm}}\hspace{0.298cm}  
  % 
  Student No.:\ \underline{\qquad 2023021950\hspace{1cm}} 
  %
  Date:\ \ \underline{March 23, 2024} 
   \end{center}

 
%%%================  正  文   ================

\begin{center}  \bf \Large
{\S 2.4 Exercises for Newton-Raphson and Secant Methods}
\end{center} 
  
\begin{exercise}1. \quad 
 Let $f(x) = x^2-x-3$. 
\begin{description}
\item[(a)] Find the Newton-Raphson formula $p_k = g (p_{k-1})$. 
\item[(b)] Start with $p_0 = 1.6$ and find $p_1, p_2$, and $p_3$. 
\item[(c)] Start with $p_0 = 0.0$ and find $p_1, p_2, p_3$, and $p_4$. What do you conjecture about this sequence?
\end{description}
 \end{exercise}
  
\begin{solve}
 (a) In this case, the Newton-Raphson iterative function is
  \begin{align*}
    g(x)= x - \frac{f(x)}{f'(x)} = x - \frac{x^2-x-3}{2x-1} = \frac{(2x^2-x)-x^2+x+3}{2x-1}=\frac{x^2+3}{2x-1}.
  \end{align*}
  Hence, the Newton-Raphson formula is
  \begin{align}
    \label{eq:Newton-Raphson formula 1}
    p_k=\frac{P_{k-1}^2+3}{2p_{k-1}-1}.
  \end{align}
  \par
  (b) Start with $p_0=1.6$, one can find
  \begin{align*}
    p_1 &= \frac{1.6^2+3}{2\times 1.6-1}=2.52727,\\
    p_2 &= \frac{2.52727^2+3}{2\times 2.52727 -1}= 2.31521,\\
    p_3 &= \frac{2.31521^2+3}{2\times 2.31521 -1}= 2.30282.
  \end{align*}
  \par
  (c) Start with $p_0=0.0$, one can find
  \begin{align*}
    p_1 &= \frac{0.0^2+3}{2\times 1.6-1}=-3.0,\\
    p_2 &= \frac{(-3.0)^2+3}{2\times (-3.0) -1}= -1.71429,\\
    p_3 &= \frac{(-1.71429)^2+3}{2\times (-1.71429) -1}= -1.34101,\\
    p_4 &= \frac{(-1.34101)^2+3}{2\times (-1.34101) -1}= -1.30317.
  \end{align*}
  we can conjecture that the convergence of the sequence is affected by the selection of the initial point $p_0$.  
\end{solve}
 
 
\begin{exercise}2. \quad
 Let $f(x) = x^3-3x-2$. 
\begin{description}
\item[(a)] Find the Newton-Raphson formula $p_k = g (p_{k-1})$. 
\item[(b)] Start with $p_0 = 2.1$ and find $p_1, p_2, p_3$, and $p_4$. 
\item[(c)] Is the sequence converging quadratically or linearly?
\end{description}
\end{exercise}
 
\begin{solve}
  (a) In this case, the Newton-Raphson iterative function is
   \begin{align*}
     g(x)= x - \frac{f(x)}{f'(x)} = x - \frac{x^3-3x-2}{3x^2-3} = \frac{(3x^3-3x)-x^3+3x+2}{3x^2-3}=\frac{2x^3+2}{3x^2-3}.
   \end{align*}
   Hence, the Newton-Raphson formula is
   \begin{align}
     \label{eq:Newton-Raphson formula 2}
     p_k=\frac{2P_{k-1}^3+2}{3p_{k-1}^2-3}.
   \end{align}
   \par
   (b) Start with $p_0=2.1$, one can find
   \begin{align*}
     p_1 &= \frac{2\times 2.1^3+2}{3\times 2.1^2-3}=2.00606,\\
     p_2 &= \frac{2\times 2.00606^3+2}{3\times 2.00606^2-3}= 2.00002,\\
     p_3 &= \frac{2\times 2.00002^3+2}{3\times 2.00002^2-3}= 2.00000,\\
     p_4 &= \frac{2\times 2.00000^3+2}{3\times 2.00000^2-3}= 2.0.
   \end{align*}
   \par
   (c) Since $f(2) = 0,f'(2)=(3\times 2^2-3)=9$, $2$ is the simple root of $f$.
   Hence, the sequence converges quadratically.
\end{solve}

\end{document}

 
 