% !Mode:: "TeX:UTF-8"

\documentclass[UTF8,12pt,hyperref]{ctexart}

\usepackage{amsfonts,amsmath,amssymb,mathrsfs}
\usepackage{flafter,graphicx,xcolor}

\newtheorem{algorithm}{Algorithm}

% Matlab
\usepackage{listings}
\lstset{language=Matlab}%代码语言使用的是matlab
\lstset{breaklines}%自动将长的代码行换行排版
\lstset{extendedchars=false}%解决代码跨页时,章节标题,页眉等汉字不显示的问题
\lstset{numbers=left, 
numberstyle= \tiny,keywordstyle= \color{ blue!70},commentstyle=\color{red!50!green!50!blue!50}, 
frame=shadowbox, rulesepcolor= \color{ red!20!green!20!blue!20}, 
escapeinside=``} 
%%%%%%%
\usepackage{multicol}
   
 
\newcommand{\ep}{\hfill\rule{0.15cm}{0.35cm}\vskip 0.3cm}
\newenvironment{proof}[1][Proof.]{\begin{trivlist}
\item[\hskip \labelsep {\bfseries #1}]}{\ep\end{trivlist}} 


%%%%%
 \makeatletter
\newenvironment{exercise}[1][{\color{blue}\bf Exercise}]%
%\newenvironment{xiti}[1][{\color{blue}\bf Exercise}]%
{%
 \begin{center}   \begin{lrbox}{\@tempboxa}%
    \begin{minipage}{\textwidth}%
  {\color{blue}\bfseries
#1}   }{%
    \end{minipage}%
    \end{lrbox}
    \colorbox{green}{\noindent\usebox{\@tempboxa}} \end{center}  
}
 
\newenvironment{analysis}[1][\color{blue}\bf Analysis]%
{%
 \begin{center}   \begin{lrbox}{\@tempboxa}%
    \begin{minipage}{\textwidth}
  {\color{blue}\bfseries
%#1} \\ \hspace*{2em} }{%
 #1}  }{%
    \end{minipage}%
    \end{lrbox}
    %\colorbox{gray}{\noindent\usebox{\@tempboxa}} \end{center}  %
    \fbox{\usebox{\@tempboxa}}\end{center}%
    %\colorbox{red}{\fbox{\noindent\usebox{\@tempboxa}} \end{center}
}
 
\newenvironment{solve}[1][\color{blue}\bf Solve]{\begin{trivlist}
\item[\hskip \labelsep {\color{blue}\bfseries
#1}]}{\hfill$\Box$\end{trivlist}}
 \makeatother
 
 
%%%%%%%%%
 
\renewcommand{\baselinestretch}{1.38}
 
 \pagestyle{plain} 
 
 \textwidth  168 true mm \textheight 230 true mm \topmargin=-1.3cm
\oddsidemargin   0pt \evensidemargin  20pt \marginparwidth  10pt

\CTEXoptions[today=old]

 
\begin{document}
%%%================  标  题   ================

\begin{center}
{\bf  THE SOUTH CHINA NORMAL UNIVERSITY\vspace{0.08cm}

School of Mathematical Sciences\vspace{0.08cm}
 
Numerical Analysis ( 2023-2024 The Second Term) \vspace{0.18cm}

{\Large Homework 4: }\vspace{0.18cm}

Due Date: \underline{April  24, 2024 (Wednesday)} }
\end{center}\vspace{-0.16cm}

\begin{center}
  Name:\ \underline{\qquad 钟沛 \hspace{1cm}}\hspace{0.298cm}  
  % 
  Student No.:\ \underline{\qquad 2023021950\hspace{1cm}} 
%
Date:\ \ \underline{April 17, 2024} 
 \end{center}
 
 
%%%================  正  文   ================

\begin{center}  \bf \Large
{\S 3.3 Upper-triangular Linear Systems }
\end{center} 
  
\begin{exercise}1.\quad   
Solve the upper-triangular system and find the value of the determinant of the coefficient matrix.
$$
\begin{array}{ccccccccr}
3x_1  & - & 2x_2 & + & x_3 & - & x_4 & = & 8 \\
  &   &  4x_2 & - & x_3 & + & 2x_4 & = & -3 \\
 & &   &  & 2x_3 & + & 3x_4 & = & 11  \\
  & &   &  &  &  & 5x_4 & = & 15  
\end{array}$$
\end{exercise}

\begin{solve}
    Solving for $x_4$ in the last equation yields
    \begin{align*}
      x_4=\frac{15}{5}=3.
    \end{align*}
    Using $x_4=3$ in the third equation, we obtain
    \begin{align*}
      x_3=\frac{11-3\cdot 3}{2} = 1.
    \end{align*}
    Now $x_3=1$ and $x_4=3$ are used to find $x_2$ in the second equation: 
    \begin{align*}
      x_2=\frac{-3-2\cdot 3+1}{4}=-2.
    \end{align*}
    Finally, $x_1$ is obtained using the first equation:
    \begin{align*}
      x_1=\frac{8+3-1-4}{3}=2.
    \end{align*}
    And the determinant of the coefficient matrix is
    \begin{align*}
      \begin{vmatrix}
        3& -2 & 1 &-4 \\
        0&  4&  -1& 2\\
        0&  0&  2& 3\\
        0&  0&  0& 5
      \end{vmatrix} = 3\cdot 4\cdot 2\cdot 5 = 120.
    \end{align*}
\end{solve}
 
%%%================
%%%================

\begin{exercise}2.\quad   
 \begin{description}
\item[(a)] Consider the two upper-triangular matrices
$$
\boldsymbol{A}=\left[\begin{array}{ccc}
a_{11}  & a_{12} & a_{13} \\
 0 &  a_{22} & a_{23} \\
 0 & 0 & a_{33}
\end{array}\right] \quad \mbox{and} \quad
\boldsymbol{B}=\left[\begin{array}{ccc}
b_{11}  & b_{12} & b_{13} \\
 0 &  b_{22} & b_{23} \\
 0 & 0 & b_{33}
\end{array}\right]. 
$$
Show that their product $\boldsymbol{C}=\boldsymbol{A}\boldsymbol{B}$ is also upper triangular.

\item[(b)] Let $\boldsymbol{A}$ and $\boldsymbol{B}$ be two $N \times N$ upper-triangular matrices. Show that their product is also upper triangular. 
\end{description}
\end{exercise}
  
\begin{solve}
  (a) Since
  \begin{align*}
    \boldsymbol{C}=\boldsymbol{AB}=
    \begin{pmatrix}
      a_{11}b_{11}& a_{11}b_{12}+a_{12}b_{22}  & a_{11}b_{13}+a_{12}b_{23}+a_{13}b_{33} \\
      0&  a_{22}b_{22}& a_{22}b_{23}+a_{23}b_{33} \\
      0&  0& a_{33}b_{33}
    \end{pmatrix},
  \end{align*}
  the product of $\boldsymbol{A}$ and $\boldsymbol{B}$ is also upper triangular.
  \\
  (b) Consider the product $\boldsymbol{C}=(c_{ij})_{N\times N}$ of two upper-triangular matrices $\boldsymbol{A}=(a_{ij})_{N\times N}$ and
  $\boldsymbol{B}=(b_{ij})_{N\times N}$, $c_{ij}$ with $i>j$ is given by
  \begin{align*}
     \sum\limits_{k=1}^{n} a_{ik}b_{kj}=\sum\limits_{k=1}^{i-1}a_{ik}b_{kj} + \sum\limits_{k=i}^{n}a_{ik}b_{kj}=0,
  \end{align*}
  since $a_{ik}=0$ when $k\leqslant i-1$ and $b_{kj}=0$ when $k\geqslant i>j$.
  Hence $\boldsymbol{C}$ is an upper triangular.
\end{solve} 
\end{document}

 
 