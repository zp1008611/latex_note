% !Mode:: "TeX:UTF-8"

\documentclass[UTF8,12pt,hyperref]{ctexart}

\usepackage{amsfonts,amsmath,amssymb,mathrsfs}
\usepackage{flafter,graphicx,xcolor}

\newtheorem{algorithm}{Algorithm}
\newtheorem{theorem}{Theorem}

% Matlab
\usepackage{listings}
\lstset{language=Matlab}%代码语言使用的是matlab
\lstset{breaklines}%自动将长的代码行换行排版
\lstset{extendedchars=false}%解决代码跨页时,章节标题,页眉等汉字不显示的问题
\lstset{numbers=left, 
numberstyle= \tiny,keywordstyle= \color{ blue!70},commentstyle=\color{red!50!green!50!blue!50}, 
frame=shadowbox, rulesepcolor= \color{ red!20!green!20!blue!20}, 
escapeinside=``} 
%%%%%%%
\usepackage{multicol}
\usepackage{caption}
\captionsetup[table]{name=Table} 
 
\newcommand{\ep}{\hfill\rule{0.15cm}{0.35cm}\vskip 0.3cm}
\newenvironment{proof}[1][Proof.]{\begin{trivlist}
\item[\hskip \labelsep {\bfseries #1}]}{\ep\end{trivlist}} 


%%%%%
 \makeatletter
\newenvironment{exercise}[1][{\color{blue}\bf Exercise}]%
%\newenvironment{xiti}[1][{\color{blue}\bf Exercise}]%
{%
 \begin{center}   \begin{lrbox}{\@tempboxa}%
    \begin{minipage}{\textwidth}%
  {\color{blue}\bfseries
#1}   }{%
    \end{minipage}%
    \end{lrbox}
    \colorbox{green}{\noindent\usebox{\@tempboxa}} \end{center}  
}
 
\newenvironment{analysis}[1][\color{blue}\bf Analysis]%
{%
 \begin{center}   \begin{lrbox}{\@tempboxa}%
    \begin{minipage}{\textwidth}
  {\color{blue}\bfseries
%#1} \\ \hspace*{2em} }{%
 #1}  }{%
    \end{minipage}%
    \end{lrbox}
    %\colorbox{gray}{\noindent\usebox{\@tempboxa}} \end{center}  %
    \fbox{\usebox{\@tempboxa}}\end{center}%
    %\colorbox{red}{\fbox{\noindent\usebox{\@tempboxa}} \end{center}
}
 
\newenvironment{solve}[1][\color{blue}\bf Solve]{\begin{trivlist}
\item[\hskip \labelsep {\color{blue}\bfseries
#1}]}{\hfill$\Box$\end{trivlist}}
 \makeatother
 
 
%%%%%%%%%
 
\renewcommand{\baselinestretch}{1.38}
 
 \pagestyle{plain} 
 
 \textwidth  168 true mm \textheight 230 true mm \topmargin=-1.3cm
\oddsidemargin   0pt \evensidemargin  20pt \marginparwidth  10pt

\CTEXoptions[today=old]

 
\begin{document}
%%%================  标  题   ================

\begin{center}
{\bf  THE SOUTH CHINA NORMAL UNIVERSITY\vspace{0.08cm}

School of Mathematical Sciences\vspace{0.08cm}
 
Numerical Analysis ( 2022-2023 The Second Term) \vspace{0.18cm}

{\Large Homework 2: }\vspace{0.18cm}

Due Date: \underline{March 19, 2024 (Tuesday)} }
\end{center}\vspace{-0.16cm}

\begin{center}
  Name:\ \underline{\qquad 钟沛\hspace{1cm}}\hspace{0.298cm}  
  % 
  Student No.:\ \underline{\qquad 2023021950\hspace{1cm}} 
  %
  Date:\ \ \underline{March 14, 2024} 
   \end{center}

 
%%%================  正  文   ================

\section*{\S 2.1 Exercises for Iteration for solving $x=g(x)$}

%%%%%%%%%% Exercise 1.1
  
 

%%%%%%%%%% Exercise 1.1
  
\begin{exercise}1.\qquad  Let $g(x) = x\cos(x)$. Solve $x = g(x)$ and find all the fixed points of $g$ (there are infinitely many). Can fixed-point iteration be used to find the solution(s) to the equation $x = g(x)$? Why?
 \end{exercise}


 \begin{solve}
	\qquad For $x=g(x)$, if $x\neq 0$, then $x=x\cos(x)$ could be simplification to $\cos(x)=1$
  and so $x=2k\pi$ with $k=\pm 1,\pm 2,...$ are the solutions for the equation.
  If $x=0$, then the equation $x=x\cos(x)$ can be established and so $x=0$ is the solution for the equation. 
  Therefore, $x=0$ and $x=2k\pi$ with $k=\pm 1,\pm 2,...$ are the fixed points of $g$.
  \par
  The fixed-point iteration can not be used to find the solutions to the equation $x=g(x)$.
  Since for any interval $I$ containing $x=0$, $\max_{x\in I} |g'(x)| \geqslant |\cos(0)-0\sin(0)|=1$ and 
  for any interval $I_k$ containing $x=2k\pi$, $\max_{x\in I_k} |g'(x)| \geqslant |\cos(2k\pi) - 2k\pi \sin(2k\pi)=2k\pi>1$, 
  then the iterative sequence obtained by $x_{n+1}=g(x_n)$ diverges.
\end{solve}

 
\section*{\S 2.2 Exercises for Bracketing Methods}  
  
 
%%%%%%%%%% Exercise 2.1

\begin{exercise}1.\qquad  
 What will happen if the bisection method is used with the function $f (x) = \tan(x)$ and
 
  $
  \begin{array}{ll}
  \mbox{(a) the interval is}\ $[3, 4]$?  \qquad\qquad &\mbox{(b) the interval is}\ $[1, 3]$?
  \end{array}
  $
\end{exercise}

\begin{solve}
  \qquad
  The roots of $f(x)=\tan(x)$ in $\mathbb{R}$ are $k\pi$ with $k=0,\pm 1,\pm 2,\dots$.
  \par
  (a) For the interval $[3,4]$, there is only one solution within this range 
  for the function $f(x)=\tan(x)$. Therefore, using the bisection method in $[1,3]$
  can be effective in finding the solution.
  A sample iteration process is performed in table \ref{table1}.
  \begin{table}[htbp]
		\centering
		\caption{Bisection Method Solution of $f(x) = \tan(x)$ in interval is $[3,4]$}
		\label{table1}
		\begin{tabular}{c|c|c|c|c}
			k & Left point $a_{k}$ & Right point $b_{k}$ & Center point $c_{k}$ & Function value  $f(c_{k})$\\
			\hline
			0 & 3 & 4 & 3.5 & 0.374585640158595\\
			1 & 3 & 3.5 &  3.25  & 0.108834025513330\\
			2 & 3 & 3.25 &  3.125   & -0.016594176499358\\
			3 & 3.125 & 3.25 &  3.1875   & 0.045939623292659\\
			4 & 3.125 & 3.1875 &  3.15625  & 0.014658396151121\\
			5 & 3.125 & 3.15625 &  3.140625  & $-9.676538918152837\times 10^{-4}$\\
			6 & 3.140625  & 3.15625 &  3.1484375  & $0.006844953310281$\\
			7 & 3.140625  & 3.1484375  &  3.14453125  & 0.002938604868838 
		\end{tabular}
	\end{table}
  \par
  (b) For the interval $[1,3]$, the function $f(x)=\tan(x)$ has no solution within
  this range due to the periodic nature of the tangent function. This mean that using the 
  bisection method may not be effective as it may lead to no possible solution.
  A sample iteration process is performed in table \ref{table2}.
  \begin{table}[htbp]
		\centering
		\caption{Bisection Method Solution of $f(x) = \tan(x)$ in interval is $[1,3]$}
		\label{table2}
    \begin{tabular}{c|c|c|c|c}
			k & Left point $a_{k}$ & Right point $b_{k}$ & Center point $c_{k}$ & Function value  $f(c_{k})$\\
			\hline
			0 & 1 & 3 & 2 & -2.185039863261519 \\
			1 & 1 & 2 & 1.5  & 14.101419947171719 \\
			2 & 1.5 & 2 & 1.75 & -5.520379922509330 \\
			3 & 1.5 & 1.75 &  1.625 & -18.430862762369620 \\
			4 & 1.5 & 1.625 & 1.5625 & $1.205325057225426\times 10^2$\\
			5 & 1.5625 & 1.625 & 1.59375 & -43.558360406739730 \\
			6 & 1.5625  & 1.59375 &  1.578125  & $-1.364479038428448 \times 10^2$\\
			7 & 1.5625  & 1.578125  &  1.5703125  & $2.066855189746604 \times 10^3$
		\end{tabular}
	\end{table}

\end{solve}
 

%%%%%%%%%% Exercise 2.2

\begin{exercise}2.\qquad  
 Suppose that the bisection method is used to find a zero of $f(x)$ in the interval $[2, 7]$. How many times must this interval be bisected to guarantee that the approximation $c_N$ has an accuracy of $5 \times 10^{-9}$?
\end{exercise}
 
\begin{solve}
  Assume the root of $f(x)$ in $[a,b]=[2,7]$ is $x^*$. 
  To achive accuracy of $|c_N-x^*|\leqslant \epsilon=5\times 10^{-9}$,
  it suffices to take
  \begin{align*}
    N\geqslant \frac{\log(b-a)-\log\epsilon}{\log2}=\frac{\log 5-(\log 5-9\log 10)}{\log2}\approx 29.89735285398626.
  \end{align*}
  Hence, $30$ times must this interval be divided to guarantee
  that the approximation $c_N$ has an accuracy of $5\times 10^{-9}$.
\end{solve}
\end{document}

 
 