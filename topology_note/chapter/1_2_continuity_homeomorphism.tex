\chapter{Continuous Mappings and Homeomorphisms}\label{1_2}

\section{Continuous Mappings}
We are already familiar with the notion of a continuous function from $\R$ to $\R$.
\par
A function $f:\R\rightarrow \R$ is said to be continuous at $x_0\in \R$ iff each positive real number $\epsilon$, 
there exists a positive real number $\delta$ such that $|f(x)-f(x_0)|<\epsilon$ when $|x-x_0|<\delta$. 
\par
It is not all obvious how to generalize this definition to general topological spaces where we do not have 
"absolute value" or "subtraction". So we shall seek another(equivalent) definition of continuity which lends itself more to 
generalization. 

It is easily seen that: $f:\R\rightarrow \R$ is continuous at $x_0\in\R$ iff for each interval $(f(x_0)-\epsilon, f(x_0)+\epsilon)$, 
for $\epsilon>0$, there exists a $\delta>0$ such that $f(x)\in(f(x_0)-\epsilon, f(x_0)+\epsilon)$ for all $x\in(x_0-\delta,x_0+\delta)$.
\par
This definition is an improvement since it does not involve the concept "absolute value" but it still involves "substraction". 
The next definition shows how to avoid substraction.

\begin{definition}{}{continuity definition}
    Let $(X,\tau)$ and $(Y,\tau')$ be topological spaces and $f$ a function from $X$ into $Y$.
    Then $f$ is continuous at $x_0\in X$ iff
    for each $U\in \tau'$ containing $f(x_0)$, there exists $K\in\tau$ containing $x_0$, such that $f(K)\subseteq U$.
\end{definition}

\begin{definition}{}{}
    Let $(X,\tau)$ and $(Y,\tau')$ be topological spaces and $f$ a function from $X$ into $Y$.
    Then $f$ is continuous iff for each $x_0\in X$ and for each $U\in \tau'$ containing $f(x_0)$, 
    there exists $K\in\tau$ containing $x_0$, such that $f(K)\subseteq U$.
\end{definition}

As in analysis, continuity is a local concept. 
\begin{proposition}{}{}
    Let $(X,\tau)$ and $(Y,\tau')$ be topological spaces and $f$ a function from $X$ into $Y$ , $A$ a subset of $X$ and $x_0\in A$.
    We define the restriction of $f$ on $A$ as $f_A = f|A:A\rightarrow Y$, then\\
    (1) If $f$ is continuous at $x_0$, then $f_A$ is continuous at $x_0$.\\
    (2) When $A$ is open in $X$, if $f_A$ is continuous at $x_0$, then $f$ is continuous at $x_0$.
\end{proposition}

\begin{proof}
    (1) We need to prove for each $U\in \tau'$ with $f_A(x_0)\in U$, there exists $O\in \tau_A$ with $x_0\in O$, $f_A(O)\subseteq U$.
    $f$ is continuous at $x_0$ and $x_0\in A$, then for each $U\in \tau'$ with $f(x_0)=f_A(x_0)\in U$, there exists $K\in \tau$ with $x_0\in K$, $f(K)\subseteq U$.
    Since $A\cap K\in \tau_A$ with $x_0\in A\cap K$ and $f_A(A\cap K) = f(A\cap K)\subseteq f(A)\cap f(K)\subseteq Y\cap U= U$. Hence, $f_A$ is continuous at $x_0$.
    \par
    (2) $f_A$ is continuous at $x_0$, then for each $U\in\tau'$ containing $f_A(x_0)=f(x_0)$, there exists $(K\cap A)\in \tau_A$($K\in\tau$) containing $x_0$, 
    such that $f_A(K\cap A)\subseteq U$. Since $A\in \tau$, $K\cap A\in \tau$ and $f(A\cap K)=f_A(A\cap K)\subset U$. Hence, $f$ is continuous at $x_0$. 
\end{proof}


\begin{definition}{}{}
    Let $f$ be a function from a set $x$ into a set $Y$. If $S$ is any subset of $Y$, 
    then the set $f^{-1}(S)$ is defined by
    \begin{align*}
        f^{-1}(S) = \{x:x\in X \text{ and } f(x)\in S\}.
    \end{align*}
    Then subset $f^{-1}(S)$ of $X$ is said to be the inverse image of $S$.
\end{definition}
\begin{remark}
    Note that an inverse function of $f$ exists iff $f$ is bijective. 
    But the inverse image of any subset of $Y$ exists even if $f$ is neither one-to-one nor onto.
\end{remark}

\begin{proposition}{}{}
    Let $f$ be a mapping of a topological space $(X,\tau)$ into a topological space $(Y,\tau')$. Then the following conditions are equivalent:\\
    (1) $f$ is continuous;\\
    (2) for each $U\in\tau'$, $f^{-1}(U)\in \tau$;\\
    (3) for each closed set $V$ in $Y$, $f^{-1}(Y)$ is closed in $X$.
\end{proposition}

\begin{proof}
    
\end{proof}

In $(\R,\tau_e)$, we can use sequence convergence to characterize continuity, 
but in general topological space, we cannot do this.
\begin{proposition}{}{}
    $f:X\rightarrow Y$ is continuous at $x_0\in X$, then $x_n\rightarrow x_0$ implies $f(x_n)\rightarrow f(x_0)$.
\end{proposition}
\begin{proof}
    $f$ is continous at $x_0\in X$ and $x_n\rightarrow x_0$, 
    then $\forall U\in\tau_Y$ containing $f(x_0)$, 
    there exists $K\in\tau_X$ containing $x_0$ such that $f(K)\subseteq U$.
    Since $x_n\rightarrow x_0$, $\exists N\in\N$, $\forall n>N$, $x_n\in K$, then $f(x_n)\in f(K)\subseteq U$.
    So $f(x_n)\rightarrow f(x_0)$.
\end{proof}
However, the inverse proposition is not true. 
Let $f:X\rightarrow Y$ be injective, $X$ be a uncountable space with $\tau_c$ and $Y$ be a discrete space.
Then, by proposition\ref{prop:sequence convergence in cocountable topology}, 
When $x_n\rightarrow x_0$ in $X$, $\exists N\in N$, $\forall n>N$, $x_n=x$, 
then $f(x_n)=f(x)$ and so $f(x_n)\rightarrow f(x)$.
But $f$ is not continuous at $x_0$, because for $U=\{f(x_0)\}\in \tau_Y$ containing $f(x_0)$, 
$\forall K\in \tau_X$ containing $x_0$, $f(K)\supset \{f(x_0)\}$ as $f$ is injective and $K\supset \{x_0\}$.


\section{The properties of continuous mapping}

Firstly, we introduce some simple and common continuous mappings.
\begin{proposition}{}{}
    Identity mapping $\text{id}:X\rightarrow X$ is continuous.
\end{proposition}


\section{Exercise}

\begin{exercise}{}{}
    $f:X\rightarrow Y$ is called open(closed) mapping, if $f(X)$ is open(closed). 
    Illustrate that open mapping may not be closed mapping and vice versa.
\end{exercise}
\begin{proof}
    
\end{proof}

\begin{exercise}
    If $f:X\rightarrow Y$ is bijective, then
    \begin{align*}
        f \text{ is open mapping }\Leftrightarrow f\text{ is closed mapping }\Leftrightarrow f^{-1} \text{ is continuous}.
    \end{align*}
\end{exercise}

\begin{proof}
    $\forall U\in\tau_X$,
    \begin{align*}
        f(U)\in \tau_Y \Leftrightarrow & Y\setminus f(U) \text{ is closed }\\
                                \overset{f \text{ is bijective }}{\Leftrightarrow} & f(X\setminus U) \text{ is closed } \\
                                \Leftrightarrow & f \text{ is closed mapping}.
    \end{align*}
    Since $f$ is bijective, $f^{-1}$ exists. As $f$ is open mapping, $f^{-1}$ is continuous.
\end{proof}

\begin{remark}
    From this exercise, we can know if $f$ is bijective, continuous and open, then $f$ is homeomorphism.
\end{remark}
