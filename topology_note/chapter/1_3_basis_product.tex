\chapter{Topological basis and Product Space}\label{chp:1_3}

Let's recall the euclidean topology in $\R$, 
\begin{align*}
    \tau_e = \{U:U=U_i(a_i,b_i),a_i<b_i\in\R\}.
\end{align*}
It seems like the entire collection of sets
in $\tau_e$ can be specified by declaring that just the usual open intervals are open. Once these
“special sets” are known to be open, we get all the other sets for free by taking unions.
These special collections of sets are called bases of topologies.

\begin{definition}{}{}
    Let $(X,\tau)$ be a topological space. 
    A collection $\mathcal{B}$ of open subsets of $X$ is said to be a basis for the topology $\tau$
    if for each $U\in \tau$, $U=\cup_{i\in I} B_i$, where $I$ is a index set and $B_i\in\mathcal{B}$.
\end{definition}

\begin{remark}
    If $\mathcal{B}$ is a basis for a topology $\tau$ on a set $X$ then
    a subset $U$ of $X$ is in $\tau$ iff it is a union of members of $\mathcal{B}$.
    So $\mathcal{B}$ "generates" the topology $\tau$ in the following sense:
    if we told what sets are members of $\mathcal{B}$ then we can determine the members of $\tau$ -- 
    they are just all the sets which are unions of members of $\mathcal{B}$.
\end{remark}

\begin{example}{}{}
    $\mathcal{B}=\{(a,b):a,b\in\R,a<b\}$ is a basis for $(\R,\tau_e)$.
\end{example}

\begin{example}{}{}
    $\mathcal{B}=\{\{x\}:x\in X\}$ is a basis for $(X,\tau_d)$. 
\end{example}

