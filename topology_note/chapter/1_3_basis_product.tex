\newcommand{\ob}{\overline{\mathcal{B}}}
\chapter{Topological basis and Product Space}\label{chp:1_3}


\section{Topological basis}
Let's recall the euclidean topology in $\R$, 
\begin{align*}
    \tau_e = \{U : U=\cup_{(a,b)\in I} (a,b), a<b\in \R, I \text{ is a collection of open interval}\}.
\end{align*}
It seems like the entire collection of sets
in $\tau_e$ can be specified by declaring that just the usual open intervals are open. Once these
“special sets” are known to be open, we get all the other sets for free by taking unions.
These special collections of sets are called bases of topologies.

\begin{definition}{}{basis for given topology definition}
    Let $(X,\tau)$ be a topological space. 
    A collection $\mathcal{B}$ of subsets of $X$ is said to be a basis for the topology $\tau$
    if for each $U\in \tau$, $U=\cup_{B\in I} B$, where $I\subseteq \mathcal{B}$.
\end{definition}

If we construct a set $\ob=\{U_{B\in I}B: I \subseteq \mathcal{B}\}$, 
then we can get a improved definition. 

\begin{definition}{}{basis for given topology definition}
    Let $(X,\tau)$ be a topological space. 
    A collection $\mathcal{B}$ of subsets of $X$ is said to be a basis for the topology $\tau$
    if $\ob=\tau$.
\end{definition}

\begin{proposition}{}{}
    Let $(X, \tau)$ be a topological space.
     A collection $\mathcal{B} \subseteq \mathcal{P}(X)$ is a
basis for $\tau$ if and only if\\
(1) $\mathcal{B} \subseteq \tau$,\\
(2) for any $U \in \tau$ and any $x \in U$, there exists $B \in \mathcal{B}$ such that $x \in B \subseteq U$.
\end{proposition}

\begin{proof}
    ($\Rightarrow$): Suppose $\ob=\tau$, then $\mathcal{B}\subseteq \ob\subseteq \tau$. 
    Since $\tau\subseteq \ob$, then take $B=U$, $x\in B\subseteq U$.
    \\
    ($\Leftarrow$): (1) implies $\ob\subseteq \tau$, 
    (2) implies $U\subseteq \cup_{B_x}B_x\subseteq U$ and so $U=\cup_{B_x}B_x\in \ob$. 
    Hence, $\ob\subseteq \tau$ and so $\ob=\tau$.
\end{proof}

\begin{example}{}{}
    $\mathcal{B}=\{(a,b):a,b\in\R,a<b\}$ is a basis for $(\R,\tau_e)$.
\end{example}

\begin{example}{}{}
    $\mathcal{B}=\{\{x\}:x\in X\}$ is a basis for $(X,\tau_s)$. 
\end{example}

Observe that $\tau=\mathcal{P}(X)$ is also a basis for the discrete topology on $X$. 
Therefore, there can be many different bases for the same topology. indeed if $\mathcal{B}$ is a basis for a topology $\tau$ on a set $X$ and $\mathcal{B}_1$ is a collections of subsets of $X$ such that $\mathcal{B}\subseteq \mathcal{B}_1\subseteq \tau$, then $\mathcal{B}_1$ is also a basis for $\tau$.


The above content show us when $\mathcal{B}$ is basis for a given topology in $X$. 
Now we consider when $\mathcal{B}$ is basis for a topology in $X$.

\begin{proposition}{}{}
    Let $X$ be a non-empty set and $\ob$ be a collection of subsets of $X$. Then $\mathcal{B}$ is a basis for a topology on $X$ iff
    $\overline{B}$ is a topology.
\end{proposition}


\begin{proposition}{Basis for a topology}{}
    Let $X$ be a non-empty set and $\mathcal{B}$ be a collection of subsets of $X$. Then $\mathcal{B}$ is a basis for a topology on $X$ iff $\mathcal{B}$ has the following properties:\\
    (1) $X=\cup_{B\in \mathcal{B}}B$, and\\
    (2) for any $B_1,B_2\in \mathcal{B}$, $B_1\cap B_2=\cup_{B\in I}B$, where $I\subseteq\mathcal{B}$.
\end{proposition}

\begin{proof}
($\Rightarrow$): $\ob$ is a topology, then $X\in \ob$ and so $X=\cup_{B\in I}B$.
Since $B\in\mathcal{B}$ is contained in $X$, $X\subseteq \cup_{B\in\mathcal{B}}\subseteq X$. Hence, $X=\cup_{B\in\mathcal{B}}B$.
If $B_1,B_2\in\mathcal{B}$, then $B_1,B_2\in\ob$ and so $B_1,B_2$ is open. Then $B_1\cap B_2$ is open and so contained in $\ob$. 

($\Leftarrow$): We should show that $\ob$ is a topology. By (1), $X\in\ob$. 
$\O$ is the empty union of members in $\ob$. Hence, $\O\in\ob$. For $U_J=\{U\in J: J\subseteq\ob\}$, $U$ is the union of members in $\mathcal{B}$.
Then $\cup_{U\in U_J} U$ is the union of members in $\mathcal{B}$ and so belong to $\ob$. 
If $U_1,U_2\in \ob$, then $U_1=\cup_{B_1\in I_1}B_1, U_2=\cup_{B_2\in I_2}B_2$.
Then $U_1\cap U_2 = (\cup_{B_1\in I_1}B_1)\cap (\cup_{B_2\in I_2}B_2)=\cup_{B_1\in I_1}(B_1\cap (\cup_{B_2\in I_2}B_2))= \cup_{B_1\in I_1,B_2\in I_2}(B_1\cap B_2)$. 
By (2), $B_1\cap B_2$ is the union of members in $\mathcal{B}$, thus $U_1\cap U_2$ is the union of members in $\mathcal{B}$ and so contained in $\mathcal{B}$.
Hence, $\ob$ is a topology.
\end{proof}

\textcolor{red}{Haven't done! Add the content about basis for a given topology.}

\section{Product space}
