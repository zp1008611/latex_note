\chapter{Preliminary Knowledge}\label{chp:0_1}

\section{Countability}

\section{Inverse Image and Direct Image}

\begin{definition}{}{}
    Let $f:X\rightarrow Y$ be a function, and let $U\subset Y$ be a subset.
    The inverse image(or, preimage) of $U$ is the set $f^{-1}(U)\subset X$ consisting of all elements $a\in X$
    such that $f(a)\in U$.
\end{definition}

\begin{proposition}{}{}
    $f:X\rightarrow Y$, $A\subseteq X$, $B\subseteq Y$.\\
    (1) $A\subseteq f^{-1}(f(A))$ with equality if $f$ is injective.\\
    (2) $f(f^{-1}(B))\subseteq B$ with equality if $f$ is surjective.\\
    (3) $B\subseteq f(A)$, then $f^{-1}(B)\subseteq A$ only when $f$ is injective.\\
\end{proposition}

\begin{proof}
    (3) $\forall x\in f^{-1}(B), f(x)\in B$. Since $B\subseteq f(A)$, $\exists y\in f(A)$, $f(x)=y\in f(A)$.
    Then $x\in f^{-1}(f(A))$. If $f$ is injective, then $A=f^{-1}(f(A))$. Then $f^{-1}(B)\subseteq A$.
\end{proof}

The inverse image commutes with all set operations:


\section{Reference}
\begin{itemize}
    \item Countability: \href{https://www.math.toronto.edu/ivan/mat327/docs/notes/04-countability.pdf}{lecture notes from toronto}
    \item \href{https://web.northeastern.edu/suciu/U565/MATH4565-sp10-handout1.pdf}{Inverse images and direct images}
\end{itemize}
\section{Reference}
